\documentclass[letterpaper, 12pt]{article}

\usepackage{/Users/zhengz/Desktop/Math/Workspace/Homework1/homework}

%%%%%%%%%%%%%%%%%%%%%%%%%%%%%%%%%%%%%%%%%%%%%%%%%%%%%%%%%%%%%%%%%%%%%%%%%%%%%%%%%%%%%%%%%%%%%%%%%%%%%%%%%%%%%%%%%%%%%%%%%%%%%%%%%%%%%%%%
\begin{document}
%Header-Make sure you update this information!!!!
\noindent
%%%%%%%%%%%%%%%%%%%%%%%%%%%%%%%%%%%%%%%%%%%%%%%%%%%%%%%%%%%%%%%%%%%%%%%%%%%%%%%%%%%%%%%%%%%%%%%%%%%%%%%%%%%%%%%%%%%%%%%%%%%%%%%%%%%%%%%%
\large\textbf{Zhengdong Zhang} \hfill \textbf{Homework - Chapter 2 Exercises}   \\
Email: zhengz@uoregon.edu \hfill ID: 952091294 \\
\normalsize Course: MATH 681 - Algebraic Geometry I \hfill Term: Fall 2025 \\
Instructor: Professor Nick Addington \hfill Due Date: Oct 13th, 2025 \\
\noindent\rule{7in}{2.8pt}
\setstretch{1.1}
%%%%%%%%%%%%%%%%%%%%%%%%%%%%%%%%%%%%%%%%%%%%%%%%%%%%%%%%%%%%%%%%%%%%%%%%%%%%%%%%%%%%%%%%%%%%%%%%%%%%%%%%%%%%%%%%%%%%%%%%%%%%%%%%%%%%%%%%
% Exercise 2.3  
%%%%%%%%%%%%%%%%%%%%%%%%%%%%%%%%%%%%%%%%%%%%%%%%%%%%%%%%%%%%%%%%%%%%%%%%%%%%%%%%%%%%%%%%%%%%%%%%%%%%%%%%%%%%%%%%%%%%%%%%%%%%%%%%%%%%%%%%
\begin{problem}{2.3}
Show that any Hausdoff space if of dimension zero.
\end{problem}
\begin{solution}
Let \(X\) be a Hausdoff topological space and \(x\in X\) is a point. It is easy to see that \(\left\{ x \right\}\) is irreducible. If \(X=\left\{ x \right\}\), then obviously \(\dim X=0\) since we only have one point. For any \(y\in X\) that is not \(x\), we have an open set 
\[U_y\cap \left\{ x \right\}=\varnothing\]
because \(X\) is Hausdoff. Then we have 
\[X\setminus \left\{ x \right\}=\bigcup_{y\in X, y\neq x}U_y\]
is open in \(X\). This implies \(\left\{ x \right\}\) is closed and irreducible for any \(x\in X\). Thus, the only irreducible subsets in \(X\) are singletons. Thus, we can conclude that \(\dim X=0\).
\end{solution}

\noindent\rule{7in}{2.8pt}
%%%%%%%%%%%%%%%%%%%%%%%%%%%%%%%%%%%%%%%%%%%%%%%%%%%%%%%%%%%%%%%%%%%%%%%%%%%%%%%%%%%%%%%%%%%%%%%%%%%%%%%%%%%%%%%%%%%%%%%%%%%%%%%%%%%%%%%%
% Exercise 2.4
%%%%%%%%%%%%%%%%%%%%%%%%%%%%%%%%%%%%%%%%%%%%%%%%%%%%%%%%%%%%%%%%%%%%%%%%%%%%%%%%%%%%%%%%%%%%%%%%%%%%%%%%%%%%%%%%%%%%%%%%%%%%%%%%%%%%%%%%
\begin{problem}{2.4}
Assume that 
\[Y=Y_1\cup\cdots \cup Y_r\]
is the primary decomposition of the Noetherian space \(Y\) into irreducible components. Show that 
\[\dim Y=\max_{1\leq i\leq r} \dim Y_i.\]
\end{problem}
\begin{solution}
For \(1\leq i\leq r\), we know that \(Y_i\subseteq Y\), so \(\dim Y_i\leq \dim Y\). This implies that 
\[\max_{1\leq i\leq r}\dim Y_i\leq \dim Y.\]
On the other hand, suppose 
\[Z_0\subsetneq Z_1\subsetneq \cdots\subsetneq Z_k\]
is a maximal strictly increasing chains of irreducible subsets in \(Y\). Then \(Z_k=Y_i\) for some \(1\leq i\leq r\) because \(Y_1,\ldots,Y_r\) are irreducible components of \(Y\). Thus, this chain is also a strictly increasing chain of irreducible subsets in \(Y_i\). This proves 
\[\dim Y\leq \max_{1\leq i\leq r}\dim Y_i.\]
Hence, we can conclude that 
\[\dim Y=\max_{1\leq i\leq r}\dim Y_i.\]
\end{solution}

\noindent\rule{7in}{2.8pt}
%%%%%%%%%%%%%%%%%%%%%%%%%%%%%%%%%%%%%%%%%%%%%%%%%%%%%%%%%%%%%%%%%%%%%%%%%%%%%%%%%%%%%%%%%%%%%%%%%%%%%%%%%%%%%%%%%%%%%%%%%%%%%%%%%%%%%%%%
% Exercise 2.6
%%%%%%%%%%%%%%%%%%%%%%%%%%%%%%%%%%%%%%%%%%%%%%%%%%%%%%%%%%%%%%%%%%%%%%%%%%%%%%%%%%%%%%%%%%%%%%%%%%%%%%%%%%%%%%%%%%%%%%%%%%%%%%%%%%%%%%%%
\begin{problem}{2.6}
Let \(X=Z(zx,zy)\subseteq \mathbb{A}^2\). Describe \(X\) and determine \(\dim X\). Exhibit two maximal chains of irreducible subvarieties of different lengths. Exhibit a hypersurface \(Z\) so that \(Z\cap X\) is of dimension zero.
\end{problem}
\begin{solution}
The ideal \(I(X)=(zx,zy)\) has the primary decomposition
\[(zx,zy)=(z)\cap (x,y)\]
So \(X\) has two irreducible components \(X_1=Z(z)\) and \(X_2=Z(x,y)\). The coordinate ring 
\begin{align*}
     A(X_1)&=k[x,y,z]/(z)\cong k[x,y],\\ 
     A(X_2)&=k[x,y,z]/(x,y)\cong k[z].
\end{align*} 
So by proposition 2.46, we have 
\begin{align*}
     \dim X_1&=\dim A(X_1)=\dim k[x,y]=2,\\ 
     \dim X_2&=\dim A(X_2)=\dim k[z]=1.
\end{align*}
This implies that \(\dim X=\max_{i=1,2} \dim X_i=2\). We have the following two maximal chains of irreducible subvarieties 
\begin{align*}
     Z(x,y,z)&\subsetneq Z(y,z)\subsetneq Z(z), \\ 
     Z(x,y,z)&\subsetneq Z(x,y).
\end{align*}
Consider the hypersurface \(Z=Z(z-1)\subseteq \mathbb{A}^3\). The intersection 
\[Z\cap X=Z(z-1)\cap Z(zx,zy)=Z(z-1,zx,zy).\]
Then we have 
\begin{align*}
     \dim Z\cap X&=\dim A(Z\cap X)\\ 
                 &=\dim k[x,y,z]/(z-1,zx,zy)\\ 
                 &\cong \dim k[x,y,z]/(z-1,x,y)\\ 
                 &\cong \dim k\\ 
                 &=0.
\end{align*}

\end{solution}

\noindent\rule{7in}{2.8pt}
%%%%%%%%%%%%%%%%%%%%%%%%%%%%%%%%%%%%%%%%%%%%%%%%%%%%%%%%%%%%%%%%%%%%%%%%%%%%%%%%%%%%%%%%%%%%%%%%%%%%%%%%%%%%%%%%%%%%%%%%%%%%%%%%%%%%%%%%
% Exercise 2.11
%%%%%%%%%%%%%%%%%%%%%%%%%%%%%%%%%%%%%%%%%%%%%%%%%%%%%%%%%%%%%%%%%%%%%%%%%%%%%%%%%%%%%%%%%%%%%%%%%%%%%%%%%%%%%%%%%%%%%%%%%%%%%%%%%%%%%%%%
\begin{problem}{2.11}
Let \(\psi:\mathbb{A}^3\rightarrow \mathbb{A}^3\) be given as \((x,y,z)\mapsto (yz,xz,xy)\). Find all fibers of \(\psi\) and their ideals. 
\end{problem}
\begin{solution}
Let \((a,b,c)\) be a point in \(\mathbb{A}^3\). 

If \(a=b=c=0\), then the preimage 
\begin{align*}
     \psi^{-1}(0,0,0)&=\left\{ x=y=0 \right\}\cup \left\{ x=z=0 \right\}\cup \left\{ y=z=0 \right\}\\ 
                     &=Z(x,y)\cup Z(x,z)\cup Z(y,z).
\end{align*}
The fiber over \((0,0,0)\) is the union of three axes. Note that the maximal corresponding to \((0,0,0)\) is \(\mathfrak{m}_0=(x,y,z)\), so we have 
\[\psi^*\mathfrak{m}_0=(x,y)\cap (x,z)\cap (y,z)=(yz,xz,yz).\]

If only one of \(a,b,c\) equals \(0\), the other two are not zero, then the preimage \(\psi^{-1}(a,b,c)=\varnothing\), and 
\[\psi^*\mathfrak{m}_{(a,b,c)}=Z(\varnothing)=k[x,y,z].\]

If two of \(a,b,c\) equal \(0\), then the preimage 
\begin{align*}
     \psi^{-1}(0,0,c)&=Z(xy-c,z),\\ 
     \psi^{-1}(0,b,0)&=Z(xz-b,y),\\ 
     \psi^{-1}(a,0,0)&=Z(yz-a,x).
\end{align*}
The corresponding algebra map 
\begin{align*}
     \psi^*\mathfrak{m}_{(0,0,c)}&=(xy-c,z),\\
     \psi^*\mathfrak{m}_{(0,b,0)}&=(xz-b,y),\\
     \psi^*\mathfrak{m}_{(a,0,0)}&=(yz-a,x),\\
\end{align*}

If none of \(a,b,c\) equal \(0\), then the preimage of \((a,b,c)\) has two points:
\[\left\{ (\sqrt{\frac{bc}{a}},\sqrt{\frac{ac}{b}},\sqrt{\frac{ab}{c}}),(-\sqrt{\frac{bc}{a}},-\sqrt{\frac{ac}{b}},-\sqrt{\frac{ab}{c}}) \right\}.\]
Let \(\mathfrak{m}_1,\mathfrak{m}_2\) be the corresponding maximal ideals at these two points. We have 
\[\psi^*\mathfrak{m}_{(a,b,c)}=\mathfrak{m}_1\cap \mathfrak{m}_2.\] 

\end{solution}

\noindent\rule{7in}{2.8pt}


%%%%%%%%%%%%%%%%%%%%%%%%%%%%%%%%%%%%%%%%%%%%%%%%%%%%%%%%%%%%%%%%%%%%%%%%%%%%%%%%%%%%%%%%%%%%%%%%%%%%%%%%%%%%%%%%%%%%%%%%%%%%%%%%%%%%%%%%
% Exercise 2.28
%%%%%%%%%%%%%%%%%%%%%%%%%%%%%%%%%%%%%%%%%%%%%%%%%%%%%%%%%%%%%%%%%%%%%%%%%%%%%%%%%%%%%%%%%%%%%%%%%%%%%%%%%%%%%%%%%%%%%%%%%%%%%%%%%%%%%%%%
\begin{problem}{2.28}
Let \(f=y^2-x(x-1)(x-2)\) and \(g=y^2+(x-1)^2-1\). Show that \(Z(f,g)=\left\{ (0,0),(2,0) \right\}\). Determine the primary decomposition of \(Z(f,g)\).
\end{problem}
\begin{solution}
Consider the solutions to \(f=g=0\). We have 
\begin{align*}
     y^2-x(x-1)(x-2)&=y^2+(x-1)^2-1=0\\
     (x-1)^2-1+x(x-1)(x-2)&=0\\ 
     x^3-2x^2&=0\\
     x=0\ \ & \t{or} \ \ x=2
\end{align*}
This implies that \(x=0,y=0\) and \(x=2,y=0\) are the only solutions, so 
\[Z(f,g)=\left\{ (0,0),(2,0) \right\}.\]
Note that \(f=y^2-x^3+3x^2-2x\) and \(g=y^2+x^2-2x\). We claim that 
\[(f,g)=(y^2,x-2)\cap (y^2-2x, x^2).\]
First note that we can write 
\begin{align*}
     (f,g)&=(g-f,g)\\ 
          &=(x^2(x-2),y^2+x^2-2x)
\end{align*}
We claim that 
\[(x^2(x-2),y^2+x^2-2x)=(x^2,y^2+x^2-2x)\cap (x-2,y^2+x^2-2x).\]
One direction \(\subseteq\) is clear. Conversely, suppose we have an element 
\[f_1(y^2+x^2-2x)+g_1x^2=f_2(y^2+x^2-2x)+g_2(x-2)\]
for some \(f_1,f_2,g_1,g_2\in k[x,y]\). Then 
\[(f_1-f_2)y^2+g_1x^2=(f_2-f_1)(x^2-2x)+g_2(x-2).\]
The right-hand side is divisible by \(x-2\), so the left-hand side is also divisible by \(x-2\). This implies that \((x-2)|g_1\) and \((x-2)|(f_1-f_2)\). Similarly, we can prove that \(x^2|g_2\). Thus, the claim is proved. 

Rewrite 
\begin{align*}
     (y^2+x^2-2x,x-2)&=(y^2,x-2),\\
     (y^2+x^2-2x,x^2)&=(y^2-2x,x^2).
\end{align*}
Note that 
\begin{align*}
     \sqrt{(y^2,x-2)}&=(y,x-2),\\ 
     \sqrt{(y^2-2x,x^2)}&=(y,x).
\end{align*} 
are maximal, so both two ideals are primary and we obtain a primary decomposition 
\[(f,g)=(y^2,x-2)\cap (y^2-2x, x^2).\]
\end{solution}


\end{document}
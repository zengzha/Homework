\documentclass[letterpaper, 12pt]{article}

\usepackage{/Users/zhengz/Desktop/Math/Workspace/Homework1/homework}

%%%%%%%%%%%%%%%%%%%%%%%%%%%%%%%%%%%%%%%%%%%%%%%%%%%%%%%%%%%%%%%%%%%%%%%%%%%%%%%%%%%%%%%%%%%%%%%%%%%%%%%%%%%%%%%%%%%%%%%%%%%%%%%%%%%%%%%%
\begin{document}
%Header-Make sure you update this information!!!!
\noindent
%%%%%%%%%%%%%%%%%%%%%%%%%%%%%%%%%%%%%%%%%%%%%%%%%%%%%%%%%%%%%%%%%%%%%%%%%%%%%%%%%%%%%%%%%%%%%%%%%%%%%%%%%%%%%%%%%%%%%%%%%%%%%%%%%%%%%%%%
\large\textbf{Zhengdong Zhang} \hfill \textbf{Homework - Chapter 4 Exercises}   \\
Email: zhengz@uoregon.edu \hfill ID: 952091294 \\
\normalsize Course: MATH 681 - Algebraic Geometry I \hfill Term: Fall 2025 \\
Instructor: Professor Nick Addington \hfill Due Date: Oct 31st, 2025 \\
\noindent\rule{7in}{2.8pt}
\setstretch{1.1}
%%%%%%%%%%%%%%%%%%%%%%%%%%%%%%%%%%%%%%%%%%%%%%%%%%%%%%%%%%%%%%%%%%%%%%%%%%%%%%%%%%%%%%%%%%%%%%%%%%%%%%%%%%%%%%%%%%%%%%%%%%%%%%%%%%%%%%%%
% Exercise 4.3  
%%%%%%%%%%%%%%%%%%%%%%%%%%%%%%%%%%%%%%%%%%%%%%%%%%%%%%%%%%%%%%%%%%%%%%%%%%%%%%%%%%%%%%%%%%%%%%%%%%%%%%%%%%%%%%%%%%%%%%%%%%%%%%%%%%%%%%%%
\begin{problem}{4.3}
Let \((a_{10}:a_{11}:a_{12})\) and \((a_{20}:a_{21}:a_{22})\) be two different points in \(\mathbb{P}^2\). Show that the line through them has equation 
\[\det \begin{pmatrix}
  x_0 &x_1 &x_2\\ 
  a_{10}&a_{11}&a_{12}\\ 
  a_{20}&a_{21}&a_{22}
\end{pmatrix}=0.\]
Dually, if 
\begin{align*}
     \lambda_{10}x_0+\lambda_{11}x_1+\lambda_{12}x_2&=0,\\ 
     \lambda_{20}x_0+\lambda_{21}x_1+\lambda_{22}x_2&=0
\end{align*}
are equations for two different lines, show that the coordinates of the intersection point are the \(2\times 2\)-minors of the matrix
\[\begin{pmatrix}
  \lambda_{10}&\lambda_{11}&\lambda_{12}\\
  \lambda_{20}&\lambda_{21}&\lambda_{22}
\end{pmatrix}.\]
\end{problem}
\begin{solution}
The two points 
\[p=(a_{10}:a_{11}:a_{12}),q=(a_{20}:a_{21}:a_{22})\in \mathbb{P}^2\]
corresponding to the lines generated by the vector \((a_{10},a_{11},a_{12})\) and \((a_{20},a_{21},a_{22})\) respectively in \(\mathbb{A}^3\). Denote by \(\ell\) the line passing through the points \(p,q\in \mathbb{P}^2\). \(\ell\) corresponds to a plane passing through the origin in \(\mathbb{A}^3\), and any point \((x_0:x_1:x_2)\) in \(\mathbb{P}^2\) lies in this line \(\ell\) if and only if the vector \((x_0,x_1,x_2)\) generates a line through the origin in this plane. Then we know the row vector of the following matrix is linearly dependent,
\[\begin{pmatrix}
  x_0 &x_1 &x_2\\
  a_{10}&a_{11}&a_{12}\\
  a_{20}&a_{21}&a_{22}
\end{pmatrix}\]
And its determinant must be 0 since the rank of this matrix is smaller than 3. 

Dually, the equations for lines in the projective space \(\mathbb{P}^2\) are the equations for planes in \(\mathbb{A}^3\). Let \(M_1, M_2\subset \mathbb{A}^3\) be the two planes defined by these two equations respectively. The intersection \(M_1\cap M_2\) is a line \(\ell\) through the origin. The vector \((\lambda_{10},\lambda_{11},\lambda_{12})\) is a vector normal to the plane \(M_1\) and the vector \((\lambda_{20},\lambda_{21},\lambda_{22})\) is a vector normal to the plane \(M_2\). Note that \(\ell\) is normal to these two vectors and \(\ell\) passes through the origin, so the points on the line \(\ell\) are given by the coordinates 
\[(\lambda_{11}\lambda_{22}-\lambda_{12}\lambda_{21},-(\lambda_{10}\lambda_{22}-\lambda_{12}\lambda_{20}),\lambda_{10}\lambda_{21}-\lambda_{11}\lambda_{20})\]
up to the rescaling of \(t\in \mathbb{k}^*\). The coordinates of the intersection point are given by the \(2\times 2\)-minors of the matrix 
\[\begin{pmatrix}
    \lambda_{10} & \lambda_{11} & \lambda_{12} \\
    \lambda_{20} & \lambda_{21} & \lambda_{22}
  \end{pmatrix}.\]
\end{solution}

\noindent\rule{7in}{2.8pt}
%%%%%%%%%%%%%%%%%%%%%%%%%%%%%%%%%%%%%%%%%%%%%%%%%%%%%%%%%%%%%%%%%%%%%%%%%%%%%%%%%%%%%%%%%%%%%%%%%%%%%%%%%%%%%%%%%%%%%%%%%%%%%%%%%%%%%%%%
% Exercise 4.4
%%%%%%%%%%%%%%%%%%%%%%%%%%%%%%%%%%%%%%%%%%%%%%%%%%%%%%%%%%%%%%%%%%%%%%%%%%%%%%%%%%%%%%%%%%%%%%%%%%%%%%%%%%%%%%%%%%%%%%%%%%%%%%%%%%%%%%%%
\begin{problem}{4.4}
Show that \(n\) hyperplanes in \(\mathbb{P}^n\) always have a common point of intersection. Show that \(n\) linearly independent hyperplanes meet in exactly one point.
\end{problem}
\begin{solution}
The hyperplanes in \(\mathbb{P}^n\) can be viewed as hyperplanes through the origin in the affine space \(\mathbb{A}^{n+1}\), where lines through the origin in the affine space are points in the projective space. To prove \(n\) hyperplanes in \(\mathbb{P}^n\) always have a common point of intersection, it is enough to prove that \(n\) \(\mathbb{k}\)-subspace of dimension \(n\) in \(\mathbb{k}^{n+1}\) has a common \(1\)-dimentional subspace. Let \(V_1,V_2,\ldots,V_n\subset \mathbb{k}^{n+1}\) be subspaces of dimension \(n\), the dimension formula tells us that 
\[\dim(V_1\cap V_2)=\dim V_1+\dim V_2-\dim(V_1\cup V_2)\geq \dim V_1+\dim V_2-\dim \mathbb{k}^{n+1}=n-1.\]
Now use \(V_1\cap V_2\) to intersect \(V_3\), similarly, we obtain
\[\dim(V_1\cap V_2\cap V_3)\geq \dim(V_1\cap V_2)+\dim V_3-(n+1)=n-2.\]
Repeat this process and we get 
\[\dim(\bigcap_{i=1}^n V_i)\geq 1.\]
This is exactly what we need. 

Now assume the \(n\) hyperplanes are linearly independent. In \(\mathbb{A}^{n+1}\), note that a hyperplane is uniquely determined by its normal vector. \(n\) linearly independent hyperplanes give us \(n\) linearly independent vectors in \(\mathbb{A}^{n+1}\). Since the vectors in the intersection subspace is normal to every normal vector of these hyperplanes, so the intersection subspace can only be of dimension 1 because \(\mathbb{A}^{n+1}\) has dimension \(n+1\). This means in the projective space, these hyperplanes meet at exactly one point. 
\end{solution}

\noindent\rule{7in}{2.8pt}
%%%%%%%%%%%%%%%%%%%%%%%%%%%%%%%%%%%%%%%%%%%%%%%%%%%%%%%%%%%%%%%%%%%%%%%%%%%%%%%%%%%%%%%%%%%%%%%%%%%%%%%%%%%%%%%%%%%%%%%%%%%%%%%%%%%%%%%%
% Exercise 4.13
%%%%%%%%%%%%%%%%%%%%%%%%%%%%%%%%%%%%%%%%%%%%%%%%%%%%%%%%%%%%%%%%%%%%%%%%%%%%%%%%%%%%%%%%%%%%%%%%%%%%%%%%%%%%%%%%%%%%%%%%%%%%%%%%%%%%%%%%
\begin{problem}{4.13}
Show that each rational function in one variable defines a rational map \(\mathbb{P}^1\dashrightarrow \mathbb{P}^1\) which extends to a morphism \(\mathbb{P}^1\rightarrow \mathbb{P}^1\).
\end{problem}
\begin{solution}
Suppose we have a rational function in one variable \(\frac{P(t)}{Q(t)}\) defined on \(t\in \mathbb{A}^1\) for \(Q(t)\neq 0\). Let \(d=\max(\deg P,\deg Q)\). Consider the following map 
\begin{align*}
     \phi:\mathbb{P}^1&\dashrightarrow \mathbb{P}^1,\\
          (x_0:x_1)&\mapsto (x_0^d P(\frac{x_1}{x_0}):x_0^d Q(\frac{x_1}{x_0})).
\end{align*}
This is well-defined as the right-hand side is the homogenization of the polynomials \(P\) and \(Q\). We can write \(P\) and \(Q\) as product of linear terms and assume they do not have common factors. Otherwise, just cancel the common factors in the rational function \(\frac{P(t)}{Q(t)}\), and it gives us the same rational function. Thus, \(\phi\) can be extended to \(\mathbb{P}^1\) as \(P\) and \(Q\) have no common roots, so no point \((x_0:x_1)\) is mapped to \(0\) under \(\phi\). This implies \(\phi\) is a morphism \(\mathbb{P}^1\rightarrow \mathbb{P}^1\).
\end{solution}

\noindent\rule{7in}{2.8pt}
%%%%%%%%%%%%%%%%%%%%%%%%%%%%%%%%%%%%%%%%%%%%%%%%%%%%%%%%%%%%%%%%%%%%%%%%%%%%%%%%%%%%%%%%%%%%%%%%%%%%%%%%%%%%%%%%%%%%%%%%%%%%%%%%%%%%%%%%
% Exercise 4.14
%%%%%%%%%%%%%%%%%%%%%%%%%%%%%%%%%%%%%%%%%%%%%%%%%%%%%%%%%%%%%%%%%%%%%%%%%%%%%%%%%%%%%%%%%%%%%%%%%%%%%%%%%%%%%%%%%%%%%%%%%%%%%%%%%%%%%%%%
\begin{problem}{4.14}
Let the projection \(\mathbb{P}^3\) to \(\mathbb{P}^2\) be given by the assignment \((x:y:z:w)\mapsto (x:x+z:w+y)\). Determine the center and describe the projection of the twisted cubic parametrized as \((u:v)\mapsto (u^3:u^2v:uv^2:v^3)\).
\end{problem}
\begin{solution}
The center is \((0:1:0:-1)\) as this point is mapped to \((0,0,0)\) under the projection. The image of the twisted cubic curve can be parametrized as \((u:v)\mapsto (u^3:u(u^2+v^2):v(u^2+v^2))\). Write \((X:Y:Z)\) as homogeneous coordinates in \(\mathbb{P}^2\). The parametrization \((u:v)\mapsto (u^3:u(u^2+v^2):v(u^2+v^2))\) gives a curve \(C\) which is the zero locus of the ideal \((XY^2-Y^3+XZ^2)\). On the affine patch \(\left\{ X=1 \right\}\), this is the rational nodal curve \(Z^2=Y^3-Y^2\).
\end{solution}

\end{document}
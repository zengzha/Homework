\documentclass[letterpaper, 12pt]{article}

\usepackage{/Users/zhengz/Desktop/Math/Workspace/Homework1/homework}

%%%%%%%%%%%%%%%%%%%%%%%%%%%%%%%%%%%%%%%%%%%%%%%%%%%%%%%%%%%%%%%%%%%%%%%%%%%%%%%%%%%%%%%%%%%%%%%%%%%%%%%%%%%%%%%%%%%%%%%%%%%%%%%%%%%%%%%%
\begin{document}
%Header-Make sure you update this information!!!!
\noindent
%%%%%%%%%%%%%%%%%%%%%%%%%%%%%%%%%%%%%%%%%%%%%%%%%%%%%%%%%%%%%%%%%%%%%%%%%%%%%%%%%%%%%%%%%%%%%%%%%%%%%%%%%%%%%%%%%%%%%%%%%%%%%%%%%%%%%%%%
\large\textbf{Zhengdong Zhang} \hfill \textbf{Homework - Chapter 2 Exercises}   \\
Email: zhengz@uoregon.edu \hfill ID: 952091294 \\
\normalsize Course: MATH 682 - Algebraic Geometry II \hfill Term: Winter 2026 \\
Instructor: Professor Nick Addington \hfill Due Date: Jan 16, 2026 \\
\noindent\rule{7in}{2.8pt}
\setstretch{1.1}
%%%%%%%%%%%%%%%%%%%%%%%%%%%%%%%%%%%%%%%%%%%%%%%%%%%%%%%%%%%%%%%%%%%%%%%%%%%%%%%%%%%%%%%%%%%%%%%%%%%%%%%%%%%%%%%%%%%%%%%%%%%%%%%%%%%%%%%%
% Exercise 2.7.2
%%%%%%%%%%%%%%%%%%%%%%%%%%%%%%%%%%%%%%%%%%%%%%%%%%%%%%%%%%%%%%%%%%%%%%%%%%%%%%%%%%%%%%%%%%%%%%%%%%%%%%%%%%%%%%%%%%%%%%%%%%%%%%%%%%%%%%%%
\begin{problem}{2.7.2}
Describe \(\spec \mathbb{Z}[\frac{1}{18}]\).
\end{problem}
\begin{solution}
Note that we have a ring isomorphism \(\mathbb{Z}[\frac{1}{18}]\cong \mathbb{Z}[x]/(18x-1)\). Let \(R=\mathbb{Z}[x]/(18x-1)\). We need to describe \(\spec R\). Consider the ring homomorphism 
\[f:\mathbb{Z}[x]\rightarrow R.\]
given by the quotient map. We know that prime ideals in \(R\) corresponds to prime ideals in \(\mathbb{Z}[x]\) containing the ideal \((18x-1)\). It must be of the form \((p,18x-1)\) where \(p\in \mathbb{Z}\) is a prime number. If \(p=2\) or \(p=3\), then \((p,18x-1)=\mathbb{Z}[x]\), which is not an ideal. When \(p\neq 2,3\), the ideal \((p,18x-1)\) is a prime ideal in \(\mathbb{Z}[x]\), thus corresponds to a prime ideal of \(R\). So \(\spec R\) has closed points corresponds to the maximal ideal \((p,18x-1)\) in \(R\) where \(p\neq 2,3\), and a generic point corresponds to the zero ideal in \(R\) (or the ideal \((18x-1)\) in \(\mathbb{Z}[x]\)).
\end{solution}

\noindent\rule{7in}{2.8pt}
%\begin{comment}
%%%%%%%%%%%%%%%%%%%%%%%%%%%%%%%%%%%%%%%%%%%%%%%%%%%%%%%%%%%%%%%%%%%%%%%%%%%%%%%%%%%%%%%%%%%%%%%%%%%%%%%%%%%%%%%%%%%%%%%%%%%%%%%%%%%%%%%%
% Exercise 2.7.8
%%%%%%%%%%%%%%%%%%%%%%%%%%%%%%%%%%%%%%%%%%%%%%%%%%%%%%%%%%%%%%%%%%%%%%%%%%%%%%%%%%%%%%%%%%%%%%%%%%%%%%%%%%%%%%%%%%%%%%%%%%%%%%%%%%%%%%%%
\begin{problem}{2.7.8}
Let \(A\) be a Noetherian ring. Show that \(X=\spec A\) is a finite set, and the topology is the discrete topology if and only if \(A\) is an Artinian ring.
\end{problem}
\begin{solution}
\(X\) having the discrete topology means that for any \(x\in X\), the set \(\left\{ x \right\}\) is closed, so this is equivalent to that every prime ideal in \(A\) is maximal. We first prove a very useful claim.

\begin{claim}
  Suppose in a ring \(A\), the zero ideal \((0)\) can be written as the product of finitely many maximial ideals, then \(A\) is Noetherian if and only id \(A\) is Artinian. 
\end{claim}
\begin{claimproof}
  Suppose \((0)=\mathfrak{m}_1\cdots \mathfrak{m}_n\) where \(\mathfrak{m}_1,\ldots,\mathfrak{m}_n\) are maximal ideals. Consider the following finite sequence
  \[A\supset \mathfrak{m_1}\supset \mathfrak{m}_1 \mathfrak{m}_2\supset\cdots\supset \mathfrak{m}_1\cdots \mathfrak{m}_n=(0).\]
  For any \(1\leq t\leq n\), the factor \(\mathfrak{m}_1\cdots \mathfrak{m}_t/\mathfrak{m}_1 \cdots \mathfrak{m}_{t+1}\) is a vector space over the field \(A/\mathfrak{m}_{t+1}\). 
\end{claimproof}

\end{solution}

\noindent\rule{7in}{2.8pt}
%\end{comment}
%%%%%%%%%%%%%%%%%%%%%%%%%%%%%%%%%%%%%%%%%%%%%%%%%%%%%%%%%%%%%%%%%%%%%%%%%%%%%%%%%%%%%%%%%%%%%%%%%%%%%%%%%%%%%%%%%%%%%%%%%%%%%%%%%%%%%%%%
% Exercise 2.7.9
%%%%%%%%%%%%%%%%%%%%%%%%%%%%%%%%%%%%%%%%%%%%%%%%%%%%%%%%%%%%%%%%%%%%%%%%%%%%%%%%%%%%%%%%%%%%%%%%%%%%%%%%%%%%%%%%%%%%%%%%%%%%%%%%%%%%%%%%
\begin{problem}{2.7.9}
Show that \(D(f)=\varnothing\) if and only if \(f\) is nilpotent.
\end{problem}
\begin{solution}
Suppose \(f\) is nilpotent. Then there exists \(n\geq 1\) such that \(f^n=0\in \mathfrak{p}\) for any prime ideal \(\mathfrak{p}\subset A\). So 
\[D(f)=\left\{ \mathfrak{p}\ \  \mathrm{prime}\mid f\notin \mathfrak{p}\right\}=\varnothing.\]
Conversely, suppose \(D(f)=\varnothing\). This implies that \(f\in \mathfrak{p}\) for all prime ideal \(\mathfrak{p}\), so \(f\in \cap_{\mathfrak{p}\ \ \mathrm{prime}}\mathfrak{p}=\sqrt{(0)}\). There exists \(n\geq 1\) such that \(f^n=0\). So \(f\) is nilpotent.
\end{solution}

\noindent\rule{7in}{2.8pt}
%%%%%%%%%%%%%%%%%%%%%%%%%%%%%%%%%%%%%%%%%%%%%%%%%%%%%%%%%%%%%%%%%%%%%%%%%%%%%%%%%%%%%%%%%%%%%%%%%%%%%%%%%%%%%%%%%%%%%%%%%%%%%%%%%%%%%%%%
% Exercise 2.7.10
%%%%%%%%%%%%%%%%%%%%%%%%%%%%%%%%%%%%%%%%%%%%%%%%%%%%%%%%%%%%%%%%%%%%%%%%%%%%%%%%%%%%%%%%%%%%%%%%%%%%%%%%%%%%%%%%%%%%%%%%%%%%%%%%%%%%%%%%
\begin{problem}{2.7.10}
Show that the ideal \(\mathfrak{m}=(x,y-1)\) in \(A=\mathbb{R}[x,y]/(x^2+y^2-1)\) is not principal.
\end{problem}
\begin{solution}
We calculate that 
\[\mathfrak{m}^2=(x^2,(y-1)^2,x(y-1))=(1-y^2,x(y-1),(y-1)^2).\]
It is easy to see that \(\mathfrak{m}^2\subset (y-1)\). Conversely, note that 
\[(y^2-1)-(y-1)^2=(y-1)(y+1-y+1)=2(y-1).\]
so \(y-1\in \mathfrak{m}^2\). This implies that \(\mathfrak{m}^2=(y-1)\) is principal. The elements in \(A\) can be written as \(f(x)+g(x)y\). Suppose \(\mathfrak{m}=(f(x)+g(x)y)\) is principal. Then 
\[\mathfrak{m}^2=((f(x)+g(x)y)^2)=(y-1).\]
Thus, \(y-1\) and \((f(x)+g(x)y)^2\) differ by a unit in \(\mathbb{R}\). By choosing \(f(x)+g(x)y\) properly, we have 
\[(f(x)+g(x)y)^2=y-1.\]
By checking the norms \(N(f(x)+g(x)y)=f(x)^2+g(x)^2(x^2-1)\), we have 
\[N(y-1)=1+x^2-1=x^2\]
Note that norm \(N\) is multiplicative, so we need to find an element with norm \(x\). Such element does not exist. So \((x,y-1)\) is not principal. 
\end{solution}

\noindent\rule{7in}{2.8pt}

\end{document}
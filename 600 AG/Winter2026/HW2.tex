\documentclass[letterpaper, 12pt]{article}

\usepackage{/Users/zhengz/Desktop/Math/Workspace/Homework1/homework}

%%%%%%%%%%%%%%%%%%%%%%%%%%%%%%%%%%%%%%%%%%%%%%%%%%%%%%%%%%%%%%%%%%%%%%%%%%%%%%%%%%%%%%%%%%%%%%%%%%%%%%%%%%%%%%%%%%%%%%%%%%%%%%%%%%%%%%%%
\begin{document}
%Header-Make sure you update this information!!!!
\noindent
%%%%%%%%%%%%%%%%%%%%%%%%%%%%%%%%%%%%%%%%%%%%%%%%%%%%%%%%%%%%%%%%%%%%%%%%%%%%%%%%%%%%%%%%%%%%%%%%%%%%%%%%%%%%%%%%%%%%%%%%%%%%%%%%%%%%%%%%
\large\textbf{Zhengdong Zhang} \hfill \textbf{Homework - Chapter 2 Exercises}   \\
Email: zhengz@uoregon.edu \hfill ID: 952091294 \\
\normalsize Course: MATH 682 - Algebraic Geometry II \hfill Term: Winter 2026 \\
Instructor: Professor Nick Addington \hfill Due Date: Jan 16, 2026 \\
\noindent\rule{7in}{2.8pt}
\setstretch{1.1}
%%%%%%%%%%%%%%%%%%%%%%%%%%%%%%%%%%%%%%%%%%%%%%%%%%%%%%%%%%%%%%%%%%%%%%%%%%%%%%%%%%%%%%%%%%%%%%%%%%%%%%%%%%%%%%%%%%%%%%%%%%%%%%%%%%%%%%%%
% Exercise 2.7.2
%%%%%%%%%%%%%%%%%%%%%%%%%%%%%%%%%%%%%%%%%%%%%%%%%%%%%%%%%%%%%%%%%%%%%%%%%%%%%%%%%%%%%%%%%%%%%%%%%%%%%%%%%%%%%%%%%%%%%%%%%%%%%%%%%%%%%%%%
\begin{problem}{2.7.2}
Describe \(\spec \mathbb{Z}[\frac{1}{18}]\).
\end{problem}
\begin{solution}
Note that we have a ring isomorphism \(\mathbb{Z}[\frac{1}{18}]\cong \mathbb{Z}[x]/(18x-1)\). Let \(R=\mathbb{Z}[x]/(18x-1)\). We need to describe \(\spec R\). Consider the ring homomorphism 
\[f:\mathbb{Z}[x]\rightarrow R.\]
given by the quotient map. We know that prime ideals in \(R\) corresponds to prime ideals in \(\mathbb{Z}[x]\) containing the ideal \((18x-1)\). It must be of the form \((p,18x-1)\) where \(p\in \mathbb{Z}\) is a prime number. If \(p=2\) or \(p=3\), then \((p,18x-1)=\mathbb{Z}[x]\), which is not an ideal. When \(p\neq 2,3\), the ideal \((p,18x-1)\) is a prime ideal in \(\mathbb{Z}[x]\), thus corresponds to a prime ideal of \(R\). So \(\spec R\) has closed points corresponds to the maximal ideal \((p,18x-1)\) in \(R\) where \(p\neq 2,3\), and a generic point corresponds to the zero ideal in \(R\) (or the ideal \((18x-1)\) in \(\mathbb{Z}[x]\)).
\end{solution}

\noindent\rule{7in}{2.8pt}
%%%%%%%%%%%%%%%%%%%%%%%%%%%%%%%%%%%%%%%%%%%%%%%%%%%%%%%%%%%%%%%%%%%%%%%%%%%%%%%%%%%%%%%%%%%%%%%%%%%%%%%%%%%%%%%%%%%%%%%%%%%%%%%%%%%%%%%%
% Exercise 2.7.9
%%%%%%%%%%%%%%%%%%%%%%%%%%%%%%%%%%%%%%%%%%%%%%%%%%%%%%%%%%%%%%%%%%%%%%%%%%%%%%%%%%%%%%%%%%%%%%%%%%%%%%%%%%%%%%%%%%%%%%%%%%%%%%%%%%%%%%%%
\begin{problem}{2.7.9}
Show that \(D(f)=\varnothing\) if and only if \(f\) is nilpotent.
\end{problem}
\begin{solution}
Suppose \(f\) is nilpotent. Then there exists \(n\geq 1\) such that \(f^n=0\in \mathfrak{p}\) for any prime ideal \(\mathfrak{p}\subset A\). So 
\[D(f)=\left\{ \mathfrak{p}\ \  \mathrm{prime}\mid f\notin \mathfrak{p}\right\}=\varnothing.\]
Conversely, suppose \(D(f)=\varnothing\). This implies that \(f\in \mathfrak{p}\) for all prime ideal \(\mathfrak{p}\), so \(f\in \cap_{\mathfrak{p}\ \ \mathrm{prime}}\mathfrak{p}=\sqrt{(0)}\). There exists \(n\geq 1\) such that \(f^n=0\). So \(f\) is nilpotent.
\end{solution}

\noindent\rule{7in}{2.8pt}


\end{document}
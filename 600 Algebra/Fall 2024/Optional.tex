\documentclass[a4paper, 12pt]{article}

\usepackage{/Users/zhengz/Desktop/Math/Workspace/Homework1/homework}

%%%%%%%%%%%%%%%%%%%%%%%%%%%%%%%%%%%%%%%%%%%%%%%%%%%%%%%%%%%%%%%%%%%%%%%%%%%%%%%%%%%%%%%%%%%%%%%%%%%%%%%%%%%%%%%%%%%%%%%%%%%%%%%%%%%%%%%%
\begin{document}
%Header-Make sure you update this information!!!!
\noindent
%%%%%%%%%%%%%%%%%%%%%%%%%%%%%%%%%%%%%%%%%%%%%%%%%%%%%%%%%%%%%%%%%%%%%%%%%%%%%%%%%%%%%%%%%%%%%%%%%%%%%%%%%%%%%%%%%%%%%%%%%%%%%%%%%%%%%%%%
\large\textbf{Zhengdong Zhang} \hfill \textbf{Homework - Optional Problems}   \\
Email: zhengz@uoregon.edu \hfill ID: 952091294 \\
\normalsize Course: MATH 647 - Abstract Algebra  \hfill Term: Fall 2024\\
Instructor: Dr.Victor Ostrik \hfill Due Date: End of fall term, 2024 \\
\noindent\rule{7in}{2.8pt}
\setstretch{1.1}
%%%%%%%%%%%%%%%%%%%%%%%%%%%%%%%%%%%%%%%%%%%%%%%%%%%%%%%%%%%%%%%%%%%%%%%%%%%%%%%%%%%%%%%%%%%%%%%%%%%%%%%%%%%%%%%%%%%%%%%%%%%%%%%%%%%%%%%%
% Exercise 1.6.7
%%%%%%%%%%%%%%%%%%%%%%%%%%%%%%%%%%%%%%%%%%%%%%%%%%%%%%%%%%%%%%%%%%%%%%%%%%%%%%%%%%%%%%%%%%%%%%%%%%%%%%%%%%%%%%%%%%%%%%%%%%%%%%%%%%%%%%%%
\begin{problem}{1.6.7.}
Let \(V\) be an infinite dimensional vector space. Show that the linear map \(\iota_V:V\rightarrow V^{**}\) defined just before Exercise 1.2.8. is injective but not surjective.
\end{problem} 
\begin{solution}
We first prove that \(\iota_V\) is injective. Let \(v \in \text{ker}(\iota_V)\), we have \(f(v)=0\) for every \(f\in V^*\). We claim that \(v=0\). Assume the opposite. \(\left\{v\right\}\) is linearly independent and can 
be extended to a basis for \(V\). Define a linear functional \(g\in V^*\) which sends \(v\) to 1 and sends any other base vectors to 0. This contradicts that \(g(v)=0\). Thus, \(\text{ker}(\iota_V)=0\) and \(\iota_V\) is injective.
\par 
Now we are going to prove that \(\iota_V\) can never be surjective by show that \(\dim V^*\) is strictly larger than \(\dim V\) if \(V\) is infinite dimensional. Let \(X\) be a set of basis of \(V\) and since \(X\) is infinite, it must contain a 
countable subset, denoted by \(\left\{ e_n \right\}_{n\in \mathbb{N}}\). For each \(a\in \mathbb{F}\), we define a functional \(f_a:V\rightarrow \mathbb{F}\, ,\, f_a(e_n)=a^n \) for all \(n\in \mathbb{N}\) and \(f_a\) maps the basis in \(X\setminus \left\{ e_n \right\}_{n\in \mathbb{N}}\) to \(0\). 
\begin{claim}
The set \(\left\{ f_a \right\}_{a\in \mathbb{F}}\subset V^*\) is linearly independent.
\end{claim}
\begin{claimproof}
Assume the opposite. Then there exists different \(a_1,\ldots,a_n\in \mathbb{F}\) and \(c_1,\ldots,c^n\in \mathbb{F}\) such that 
\[c_1f_{a_1}+\cdots+c_n f_{a_n}=0\]
and \(c_1,\ldots,c_n\) are not all zero. Evaluate the above functional on \(e_0,e_1,\ldots,e_m\) and we have 
\[\begin{pmatrix}
	1   & 1   & \cdots & 1\\ 
	a_1 & a_2 & \cdots & a_n\\ 
	a_1^2 & a_2^2 & \cdots & a_n^2\\ 
	\vdots & \vdots & \ddots & \vdots \\ 
	a_1^m & a_2^m & \cdots & a_n^m
\end{pmatrix}
\begin{pmatrix}
	c_1\\ 
	c_2\\ 
	\vdots\\ 
	c_n
\end{pmatrix}=0\]
This function has a nonzero solution for \(c_1,\ldots,c_n\), so we know that the determinant of 
\[A=\begin{pmatrix}
	1   & 1   & \cdots & 1\\ 
	a_1 & a_2 & \cdots & a_n\\ 
	a_1^2 & a_2^2 & \cdots & a_n^2\\ 
	\vdots & \vdots & \ddots & \vdots \\ 
	a_1^m & a_2^m & \cdots & a_n^m
\end{pmatrix}\]
must be \(0\). But \(A\) is the transpose of Vandermonde matrix and \(0=\det A=\prod_{1\leq i<j\leq m}(a_j-a_i)\). Thus, there exist \(1\leq i<j\leq m \) such that \(a_i=a_j\). This contradicts our assumption 
that \(a_1,\ldots,a_n\) are different elements in \(\mathbb{F}\).
\end{claimproof}
We know that \(\left\{ f_a \right\}_{a\in \mathbb{F}}\) is linearly independent subset in \(V^*\) and can be extended to a basis of \(V^*\), therefore, we know that \(\dim V^*\geq |\mathbb{F}|\). By Exercise 1.6.5., 
\(|V^*|=\text{max}(|\mathbb{F}|,\dim V^*)=\dim V^*\). By Exercise 1.6.6., \(|V^*|=\dim V^*>\dim V\), so \(|V^*|>\text{max}(|\mathbb{F}|,\dim V)=|V|\), it is impossible to have a surjective map from \(V\) to \(V^*\).
\end{solution}

\noindent\rule{7in}{2.8pt}
%%%%%%%%%%%%%%%%%%%%%%%%%%%%%%%%%%%%%%%%%%%%%%%%%%%%%%%%%%%%%%%%%%%%%%%%%%%%%%%%%%%%%%%%%%%%%%%%%%%%%%%%%%%%%%%%%%%%%%%%%%%%%%%%%%%%%%%%
% Exercise 2.4.6
%%%%%%%%%%%%%%%%%%%%%%%%%%%%%%%%%%%%%%%%%%%%%%%%%%%%%%%%%%%%%%%%%%%%%%%%%%%%%%%%%%%%%%%%%%%%%%%%%%%%%%%%%%%%%%%%%%%%%%%%%%%%%%%%%%%%%%%%
\begin{problem}{2.4.6}
For a commutative ring \(R\), let \(GL_n(R)\) be the group of all invertible \(n\times n\) matrices with the entries in \(R\) with respect to the usual matrix multiplication. 
Given a homomorphism \(f:R\rightarrow S\) of commutative rings, show that the map \(GL_n(f):GL_n(R)\rightarrow GL_n(S)\) obtained by applying \(f\) to all of the entries of an \(n\times n\) matrix 
is actually a group homomorphism. Then verify that this defines a group scheme \(GL_n\).
\end{problem}
\begin{solution}
Let \(M,N\in GL_n(R)\) be matrices with entries in \(R\). Write \(M=(a_{ij})_{1\leq i,j\leq n}\) and \(N=(b_{kl})_{1\leq k,l\leq n}\). Then by matrices multiplication \((MN)_{ij}=\Sigma_{k=1}^n a_{ik}b_{kj}\). Apply \(GL_n(f)\) and we get 
\begin{align*}
(f(MN)) & =f((\Sigma_{k=1}^n a_{ik}b_{kj})_{1\leq i,j\leq n})\\ 
        & =(\Sigma_{k=1}^n f(a_{ik})f(b_{kj}))_{1\leq i,j\leq n}\\ 
        & =f((a_{ij})_{1\leq i,j\leq n})\cdot f((b_{kl})_{1\leq k,l\leq n})\\ 
        & =f(M)\cdot f(N).
\end{align*}
The middle equality is because \(f:R\rightarrow S\) is a ring homomorphism. This proves that \(GL_n(f)\) is actually a group homomorphism. 
Next, we are going to show that \(GL_n\) is compatible with morphisms composition in \textbf{CRings}. Suppose \(f:R\rightarrow S\) and \(g:S\rightarrow T\) are morphisms between commutative rings. 
Let \(M=(a_{ij})_{1\leq i,j\leq n}\) be a \(n\times n\) matrix with entries in \(R\). Then for each \(1\leq i,j\leq n\), we have 
$$(GL_n(g\circ f)(M))_{ij}=(g\circ f)(a_{ij})=g(f(a_{ij}))=(GL_n(g)\circ GL_n(f)(M))_{ij}.$$
Let \(id:R\rightarrow R\) be an identity morphism of a commutative ring \(R\). Then \(GL_n(id):GL_n(R)\rightarrow GL_n(R)\) is also the identity morphism since for each entry of the matrix, it is the identity. Thus we can conclude that 
\(GL_n\) is a functor from \textbf{CRings} to \textbf{Groups}, which means that it is a group scheme. 
\end{solution}

\noindent\rule{7in}{2.8pt}
%%%%%%%%%%%%%%%%%%%%%%%%%%%%%%%%%%%%%%%%%%%%%%%%%%%%%%%%%%%%%%%%%%%%%%%%%%%%%%%%%%%%%%%%%%%%%%%%%%%%%%%%%%%%%%%%%%%%%%%%%%%%%%%%%%%%%%%%
% Exercise 2.4.9
%%%%%%%%%%%%%%%%%%%%%%%%%%%%%%%%%%%%%%%%%%%%%%%%%%%%%%%%%%%%%%%%%%%%%%%%%%%%%%%%%%%%%%%%%%%%%%%%%%%%%%%%%%%%%%%%%%%%%%%%%%%%%%%%%%%%%%%%
\begin{problem}{2.4.9}
Let \(\mathbf{A}\),\(\mathbf{B}\) and \(\mathbf{C}\) be categories. Use the interchange law to show that there is a bifunctor 
$$\textbf{Func}(\mathbf{B},\mathbf{C})\times \textbf{Func}(\mathbf{A},\mathbf{B})\rightarrow \textbf{Func}(\mathbf{A},\mathbf{C})$$
mapping an object \((\mathcal{G},\mathcal{F})\) to \(\mathcal{G}\circ \mathcal{F}\) and a morphism \((\beta,\alpha)\) to \(\beta\star \alpha\).

\end{problem}
\begin{solution}
Write the bifunctor as \(T\). Let \(\mathcal{G}:\mathbf{B}\rightarrow \mathbf{C}\) and \(\mathcal{F}:\mathbf{A}\rightarrow\mathbf{B}\) be two functors. Write \(id_{\mathcal{F}}\) and \(id_{\mathcal{G}}\) as 
the identity natural transformation of \(\mathcal{F}\) and \(\mathcal{G}\). Then \(T(id_{\mathcal{G}},id_{\mathcal{F}})=id_{\mathcal{G}}\star id_{\mathcal{F}}\). For every object \(X\in \text{Ob}\, A\), by Exerise 2.4.7.(3), we have 
\((id_{\mathcal{G}}\star id_{\mathcal{F}})_X=(id_{\mathcal{G}}\mathcal{F})_X=\mathcal{G}\mathcal{F}X\). So \(T(id_{\mathcal{G}},id_{\mathcal{F}})=id_{\mathcal{G\circ F}}\). 
\par 
Let \(\mathcal{E},\mathcal{F},\mathcal{G}:\mathbf{A}\rightarrow \mathbf{B}\) and \(\mathcal{H},\mathcal{I},\mathcal{J}:\mathbf{B}\rightarrow \mathbf{C}\) be 
functors, and \(\alpha:\mathcal{E}\Rightarrow \mathcal{F}\), \(\beta:\mathcal{F}\Rightarrow \mathcal{G}\), \(\gamma:\mathcal{H}\Rightarrow \mathcal{I}\) and 
\(\delta:\mathcal{I}\Rightarrow \mathcal{J}\) be natural transformations as the following diagram:
$$\begin{tikzcd}
	{\mathbf{C}} && {\mathbf{B}} && {\mathbf{A}}
	\arrow[""{name=0, anchor=center, inner sep=0}, "{\mathcal{J}}"', curve={height=30pt}, from=1-3, to=1-1]
	\arrow[""{name=1, anchor=center, inner sep=0}, "{\mathcal{H}}", curve={height=-30pt}, from=1-3, to=1-1]
	\arrow[""{name=2, anchor=center, inner sep=0}, "{\mathcal{I}}"{description}, from=1-3, to=1-1]
	\arrow[""{name=3, anchor=center, inner sep=0}, "{\mathcal{G}}"', curve={height=30pt}, from=1-5, to=1-3]
	\arrow[""{name=4, anchor=center, inner sep=0}, "{\mathcal{E}}", curve={height=-30pt}, from=1-5, to=1-3]
	\arrow[""{name=5, anchor=center, inner sep=0}, "{\mathcal{F}}"{description}, from=1-5, to=1-3]
	\arrow["\delta"', shorten <=4pt, shorten >=4pt, Rightarrow, from=2, to=0]
	\arrow["\gamma"', shorten <=4pt, shorten >=4pt, Rightarrow, from=1, to=2]
	\arrow["\beta"', shorten <=4pt, shorten >=4pt, Rightarrow, from=5, to=3]
	\arrow["\alpha"', shorten <=4pt, shorten >=4pt, Rightarrow, from=4, to=5]
\end{tikzcd}$$
We know from Exercise 2.4.8. (The interchange law) that 
$$T((\delta\circ \gamma),(\beta\circ \alpha))=(\delta\circ \gamma)\star (\beta\circ \alpha)=(\delta\star \beta)\circ (\gamma\star \alpha)=T(\delta,\beta)\circ T(\gamma,\alpha).$$
This proves that \(T\) is a bifunctor. 
\end{solution}

\noindent\rule{7in}{2.8pt}
%%%%%%%%%%%%%%%%%%%%%%%%%%%%%%%%%%%%%%%%%%%%%%%%%%%%%%%%%%%%%%%%%%%%%%%%%%%%%%%%%%%%%%%%%%%%%%%%%%%%%%%%%%%%%%%%%%%%%%%%%%%%%%%%%%%%%%%%
% Exercise 3.3.6
%%%%%%%%%%%%%%%%%%%%%%%%%%%%%%%%%%%%%%%%%%%%%%%%%%%%%%%%%%%%%%%%%%%%%%%%%%%%%%%%%%%%%%%%%%%%%%%%%%%%%%%%%%%%%%%%%%%%%%%%%%%%%%%%%%%%%%%%
\begin{problem}{3.3.6}
If \(G\) is a finite group with an even number of elements, then the number of involutions in \(G\) is odd.
\end{problem}
\begin{solution}
To prove that the number of involutions in \(G\) is odd, it is the same as showing that the number of elements in \(G\) which are not involutions is odd. 
Let \(a\in G\) such that the order of \(a\) is larger than 2. We claim that \(a\neq a^{-1}\). Indeed, if \(a=a^{-1}\), then \(a^2=1\), which means that \(a\) has order 2. A contradiction. 
Moreover, both \(a\) and \(a^{-1}\) has the same order as \((a^{-1})^n=1\) if and only if \(a^n=1\). So the elements in \(G\) with order larger than 2 come in pairs, which means the number of them must be even. 
And the identity element has order 1. So the number of elements in \(G\) which are not involutions must be odd. 
\end{solution}

\noindent\rule{7in}{2.8pt}
%%%%%%%%%%%%%%%%%%%%%%%%%%%%%%%%%%%%%%%%%%%%%%%%%%%%%%%%%%%%%%%%%%%%%%%%%%%%%%%%%%%%%%%%%%%%%%%%%%%%%%%%%%%%%%%%%%%%%%%%%%%%%%%%%%%%%%%%
% Exercise 3.3.10
%%%%%%%%%%%%%%%%%%%%%%%%%%%%%%%%%%%%%%%%%%%%%%%%%%%%%%%%%%%%%%%%%%%%%%%%%%%%%%%%%%%%%%%%%%%%%%%%%%%%%%%%%%%%%%%%%%%%%%%%%%%%%%%%%%%%%%%%
\begin{problem}{3.3.10}
Let \(H\leq G\) and \(K\unlhd G\). Show that \(K\unlhd HK\leq G\) and that the map \(f:H\rightarrow HK/K,h\mapsto hK\) is surjective with the kernel \(H\cap K\). Hence it 
induces an isomorphism \(f:H/(H\cap K)\xrightarrow{\sim}HK/K\).
\end{problem}
\begin{solution}
\begin{enumerate}
\item \(K\unlhd HK\unlhd G\)\\ 
We know that \(HK=\left\{ hk\, |\, h\in H,\, k\in K\right\}\). Given \(k_1\in K\), for every \(h\in H\) and \(k\in K\), since \(K\) is normal in \(G\), we have \((hk)k_1(hk)^{-1}=hk\cdot k_1\cdot k^{-1}h^{-1}=h(kk_1k^{-1})h^{-1}\in K\). Thus, 
\(K\unlhd HK\). Given \(h_1k_1,h_2k_2\in HK\), we have \(h_1k_1h_2k_2=(h_1h_2)(h_2^{-1}k_1h_2)k_2\), where \(h_1h_2\in H\) and \((h_2^{-1}k_1h_2)k_2\in K\) as \(K\unlhd G\). This proves that \(h_1k_1h_2k_2\in HK\), meaning \(HK\) is a subgroup of \(G\). 
\item \(f\) is surjective and \(\bar{f}\) is an isomorphism.\\ 
We first show that for every \(h\in H\) and \(k_1,k_2\in K\), \(hk_1\) and \(hk_2\) are in the same coset. Indeed, \(hk_1(hk_2)^{-1}=hk_1k_2^{-1}h^{-1}\in K\). Therefore, for any coset \(hK\in HK/K\), its preimage under \(f\) must contain \(h\). This proves that 
\(f\) is surjective. Let \(a\in H\). We have \(f(a)=aK\). We know that \(aK=K\) if and only if \(a\in K\), which means \(a\in \ker f\) if and only if \(a\in H\cap K\). This proves that \(\ker f=H\cap K\). By the first isomorphism theorem (Exercise 3.3.9.), we know that 
\(\bar{f}:H/(H\cap K)\xrightarrow{\sim} HK/K\) is an isomorphism.
\end{enumerate}
\end{solution}

\noindent\rule{7in}{2.8pt}
%%%%%%%%%%%%%%%%%%%%%%%%%%%%%%%%%%%%%%%%%%%%%%%%%%%%%%%%%%%%%%%%%%%%%%%%%%%%%%%%%%%%%%%%%%%%%%%%%%%%%%%%%%%%%%%%%%%%%%%%%%%%%%%%%%%%%%%%
% Exercise 3.6.6
%%%%%%%%%%%%%%%%%%%%%%%%%%%%%%%%%%%%%%%%%%%%%%%%%%%%%%%%%%%%%%%%%%%%%%%%%%%%%%%%%%%%%%%%%%%%%%%%%%%%%%%%%%%%%%%%%%%%%%%%%%%%%%%%%%%%%%%%
\begin{problem}{3.6.6}
Show that \(\mathbb{F}[x_1,\ldots,x_n]\) is an integral domain.
	
\end{problem}
\begin{solution}
We prove this by induction on the number \(n\) of indeterminates. When \(n=1\), write the free \(\mathbb{F}\)-algebra as \(\mathbb{F}[x]\), where the elements are just polynomials. Assume 
\(f=a_0+a_1x+a_2x^2+\cdots+a_nx^n\) and \(g=b_0+b_1x+b_2x^2+\cdots+b_mx^m\) where the leading term \(a_n,b_m\) are nonzero and we have \(fg=0\) for some \(m,n\geq 0\). Then \(fg\) can be written as 
\[0=fg=a_0b_0+(a_1b_0+a_0b_1)x+\cdots+(\sum_{i=0}^{k}a_ib_{k-i})x^k+\cdots+(\sum_{i=0}^{m+n}a_ib_{m+n-i})x^{m+n}.\]
This implies \(\sum_{i=0}^{k}a_ib_{k-i}=0\) for \(k=0,1,\ldots,m+n\). A field is always an integral domain so we can see that 
\begin{align*}
	a_0b_0=0 & & \Rightarrow a_0=b_0=0  \\ 
	a_2b_0+a_1b_1+a_0b_2=0 & \Rightarrow a_1b_1=0 &\Rightarrow a_1=b_1=0\\ 
	\sum_{i=0}^{4}a_ib_{4-i}=0 &\Rightarrow a_2b_2=0 &\Rightarrow a_2=b_2=0\\ 
	\cdots 
\end{align*}
This proves that both \(f=g=0\).
\par 
Now assume \(n\geq 2\) and we have prove that \(\mathbb{F}[x_1,\ldots,x_{n-1}]\) is an integral domain. View the field \(\mathbb{F}[x_1,\ldots,x_n]\) as a free \(\mathbb{F}[x_1,\ldots,x_{n-1}]\)-algebra. 
Let \(f,g\in \mathbb{F}[x_1,\ldots,x_n]\). Write \(f=p_0+p_1x_n+p_2x_n^2+\cdots+p_kx_n^k\) and \(g=q_0+q_1x_n+q_2x_n^2+\cdots+q_lx_n^l\) where \(k,l\in \mathbb{N}\) and \(p_i,q_j\in \mathbb{F}[x_1,\ldots,x_{n-1}]\) for all 
\(i=0,1,\ldots,k\) and \(j=0,1,\ldots,l\). Use the assumption that \(\mathbb{F}[x_1,\ldots,x_{n-1}]\) is an integral domain and a similar argument in the case \(n=1\), we can show that \(f=g=0\). This prove that \(\mathbb{F}[x_1,\ldots,x_n]\) is also 
an integral domain. 
\end{solution}

\noindent\rule{7in}{2.8pt}
%%%%%%%%%%%%%%%%%%%%%%%%%%%%%%%%%%%%%%%%%%%%%%%%%%%%%%%%%%%%%%%%%%%%%%%%%%%%%%%%%%%%%%%%%%%%%%%%%%%%%%%%%%%%%%%%%%%%%%%%%%%%%%%%%%%%%%%%
% Exercise 3.6.10
%%%%%%%%%%%%%%%%%%%%%%%%%%%%%%%%%%%%%%%%%%%%%%%%%%%%%%%%%%%%%%%%%%%%%%%%%%%%%%%%%%%%%%%%%%%%%%%%%%%%%%%%%%%%%%%%%%%%%%%%%%%%%%%%%%%%%%%%
\begin{problem}{3.6.10(Polynomial Functions).}
Suppose \(\mathbb{F}\) is an infinite field.
\begin{enumerate}[(1)]
	\item Prove that polynomials \(f,g\in \mathbb{F}[x]\) are equal if and only if \(f(c)=g(c)\) for infinitely many \(c\in \mathbb{F}\). Hence, \(f\) and \(g\) are equal if and only if 
	      they define the same polynomial function.
	\item More generally, use induction on \(n\) to show that \(f,g\in \mathbb{F}[x_1,\ldots,x_n]\) are equal if and only if \(f(x_1,\ldots,x_n)=g(x_1,\ldots,x_n)\) as functions.
\end{enumerate}
\end{problem}
\begin{solution}
\begin{enumerate}[(1)]
\item If \(f(x)=g(x)\in \mathbb{F}[x]\) are equal, then it is easy to see that \(f(c)=g(c)\) for infinite many \(c\in \mathbb{F}\) since \(\mathbb{F}\) is an infinite field. Now assume there exist infinitely many \(c\in \mathbb{F}\) such that \(f(c)=g(c)\) for some 
\(f,g\in \mathbb{F}[x]\). Note that by Exercise 3.6.9., for every \(n\geq 1\), there exists different \(c_1,c_2,\ldots,c_n\in \mathbb{F}\) such that 
\[f(x)-g(x)=(x-c_1)(x-c_2)\cdots(x-c_n)q(x)\]
where \(q(x)\in \mathbb{F}[x]\) is a polynomial. But \(\deg (f-g)\) is finite, so it is only possible if \(f(x)=g(x)\). Thus, we can conclude that \(f\) and \(g\) are equal if and only if they define the same polynomial function.
\item We prove this by induction on the number \(n\) of indeterminates. When \(n=1\), this has been proved in (1). Now assume \(n\geq 2\) and this is true for \(\mathbb{F}[x_1,\ldots,x_{n-1}]\).

\begin{claim}
If \(R\) is a commutative ring and an integral domain, then \(c\in R\) is a root of \(f\in R[x]\) if and only if \(f\) can be written as \(f(x)=(x-c)q(x)\) where \(q(x)\in R[x]\) and \(\deg q(x)<\deg f(x)\).
\end{claim}
\begin{claimproof}
Write \(f(x)=a_nx^n+\cdots+a_1x+a_0\) for some \(a_n,\ldots,a_0\in R\). There exists a polynomial \(g(x)\in R[x]\) with leading term \(a_nx^{n-1}\) such that 
\[f(x)=(x-c)g(x)+r(x)\]
where \(r(x)\in R[x]\) with \(r(c)=0\) and \(\deg r(x)<\deg f(x)\). Repeat this process with \(r(x)\) and finally we will obtain a polynomial of degree 1 which has \(c\) as its root, so it can only be \(x-c\). This implies \(x-c|f(x)\). 
\end{claimproof}

View \(f,g\in \mathbb{F}[x_1,\ldots,x_n]\) as polynomials in \(R[x_n]\) with \(R=\mathbb{F}[x_1,\ldots,x_{n-1}]\). By our discussion in \((1)\), \(f=g\) if and only if \(f(x_n)=g(x_n)\) as functions. This implies they have the same coefficents in \(R\), by our assumption, 
\(f=g\) if and only if \(f(x_1,\ldots,x_n)=g(x_1,\ldots,x_n)\).
\end{enumerate}
\end{solution}

\noindent\rule{7in}{2.8pt}
%%%%%%%%%%%%%%%%%%%%%%%%%%%%%%%%%%%%%%%%%%%%%%%%%%%%%%%%%%%%%%%%%%%%%%%%%%%%%%%%%%%%%%%%%%%%%%%%%%%%%%%%%%%%%%%%%%%%%%%%%%%%%%%%%%%%%%%%
% Exercise 3.6.11
%%%%%%%%%%%%%%%%%%%%%%%%%%%%%%%%%%%%%%%%%%%%%%%%%%%%%%%%%%%%%%%%%%%%%%%%%%%%%%%%%%%%%%%%%%%%%%%%%%%%%%%%%%%%%%%%%%%%%%%%%%%%%%%%%%%%%%%%
\begin{problem}{3.6.11}
Suppose \(\mathbb{F}\) is a finite field with \(|\mathbb{F}|=q\). How many functions \(f:\mathbb{F}\rightarrow \mathbb{F}\) are there? How many polynomials \(f(x)\in \mathbb{F}[x]\) are there? Deduce that 
there are infinitely many different polynomials \(f(x)\in \mathbb{F}[x]\) such that \(f(c)=0\) for all \(c\in \mathbb{F}\). Give two examples of such polynomials.
\end{problem}
\begin{solution}
\(\mathbb{F}\) is a finite set with \(q\) elements. So there are \(q^2\) functions \(\mathbb{F}\rightarrow \mathbb{F}\). \(\mathbb{F}[x]\) can be viewed as an infinite dimensional \(\mathbb{F}\)-vector space with basis 
\(\left\{ 1,x,x^2,\ldots,x^n,\ldots \right\}\). By Exercise 1.6.5, we know that 
\[|\mathbb{F}[x]|=\text{max}(|\mathbb{F}|,\dim_{\mathbb{F}}\mathbb{F}[x])=\aleph_0.\]
So the cardinality of polynomials over \(\mathbb{F}\) is \(\aleph_0\). Every polynomial can be viewed as a function \(\mathbb{F}\rightarrow \mathbb{F}\), and since the number of functions is finite, we have infinite pairs of polynomials 
\((f,g)\) with \(f\neq g\) as polynomials in \(\mathbb{F}[x]\) but \(f(c)=g(c)\) for every \(c\in \mathbb{F}\). Each pair \((f,g)\) will give us a polynomial \(f-g\) with \((f-g)(c)=0\) for every \(c\in \mathbb{F}\). For example, consider 
\begin{align*}
    h_1(x)&=(x-c_1)(x-c_2)\cdots(x-c_q),\\
    h_2(x)&=(x-c_1)^2(x-c_2)^2\cdots(x-c_q)^2
\end{align*}
where \(c_1,\ldots,c_q\) are different elements in \(\mathbb{F}\). It is easy to see that \(h_1(x)\neq h_2(x)\) because \(\deg h_1=q\neq 2q=\deg h_2\), but for any \(c\in \mathbb{F}\), we have \(h_1(c)=h_2(c)=0\).
\end{solution}

\noindent\rule{7in}{2.8pt}
%%%%%%%%%%%%%%%%%%%%%%%%%%%%%%%%%%%%%%%%%%%%%%%%%%%%%%%%%%%%%%%%%%%%%%%%%%%%%%%%%%%%%%%%%%%%%%%%%%%%%%%%%%%%%%%%%%%%%%%%%%%%%%%%%%%%%%%%
% Exercise 3.7.19
%%%%%%%%%%%%%%%%%%%%%%%%%%%%%%%%%%%%%%%%%%%%%%%%%%%%%%%%%%%%%%%%%%%%%%%%%%%%%%%%%%%%%%%%%%%%%%%%%%%%%%%%%%%%%%%%%%%%%%%%%%%%%%%%%%%%%%%%
\begin{problem}{3.7.19}
Let \(X\) be a small set and \((X_i)_{i\in I}\)	be the collection of all finite subsets of \(X\). View \(I\) as a directed set so that \(i\leq j\Leftrightarrow X_i\subset X_j\); then 
\((X_i)_{i\in I}\) is a direct system with \(f_{i,j}:X_i\hookrightarrow X_j\) being the inclusion for all \(i\leq j\). Show that \((X,(\iota_i)_{i\in I})\) is a direct limit of \((X_i)_{i\in I}\) 
in the category \(\mathbf{Sets}\), where \(\iota_i:X_i\hookrightarrow X\) is the inclusion.
\end{problem}
\begin{solution}
We prove this by showing that \((X,(\iota_i)_{i\in I})\) satisfies the universal property. Let \(Y\) be a set and for every \(i\in I\), there exists a map \(f_i:X_i\rightarrow H\) such that if \(i\leq j\), we have a commutative diagram: 
\[\begin{tikzcd}
	{X_i} && {X_j} \\
	& Y
	\arrow["{\text{inclusion}}", hook, from=1-1, to=1-3]
	\arrow["{f_i}"', from=1-1, to=2-2]
	\arrow["{f_j}", from=1-3, to=2-2]
\end{tikzcd}\]
For every \(x\in X\), consider all the one element set \(\left\{ x \right\}\subset X\). Since it is finite, we have \(\left\{ x \right\}\in (X_i)_{i\in I}\). So there exists a map \(f_x:\left\{ x \right\}\rightarrow Y\). Define \(f:X\rightarrow Y\) by sending 
\(x\in X\) to \(f_x(x)\in Y\). This map is the unique map making the following diagram commutes:
\[\begin{tikzcd}
	{\left\{x\right\}} && {\left\{x\right\}} \\
	& X \\
	& Y
	\arrow["id", from=1-1, to=1-3]
	\arrow[hook, from=1-1, to=2-2]
	\arrow["{f_x}"', from=1-1, to=3-2]
	\arrow[hook', from=1-3, to=2-2]
	\arrow["{f_x}", from=1-3, to=3-2]
\end{tikzcd}\]
where \(\left\{ x \right\}\rightarrow X\) is the inclusion map. Moreover for any inclusion of finite set \(X_i\hookrightarrow X_j\), we have a commutative diagram:
\[\begin{tikzcd}
	{X_i} && {X_j} \\
	& X \\
	& Y
	\arrow["{f_{i,j}}", hook, from=1-1, to=1-3]
	\arrow["{\iota_i}", hook, from=1-1, to=2-2]
	\arrow["{f_i}"', from=1-1, to=3-2]
	\arrow["{\iota_j}"', hook', from=1-3, to=2-2]
	\arrow["{f_j}", from=1-3, to=3-2]
	\arrow["f", from=2-2, to=3-2]
\end{tikzcd}\]
The commutativity can be seen by the composition 
\[\left\{ x \right\}\hookrightarrow X_i\hookrightarrow X\]
for every \(x\in X_i\) and for every \(i\in I\). This proves that \((X,\iota_i)\) is the colimit of the direct system \((X_i)_{i\in I}\).	
\end{solution}

\noindent\rule{7in}{2.8pt}
%%%%%%%%%%%%%%%%%%%%%%%%%%%%%%%%%%%%%%%%%%%%%%%%%%%%%%%%%%%%%%%%%%%%%%%%%%%%%%%%%%%%%%%%%%%%%%%%%%%%%%%%%%%%%%%%%%%%%%%%%%%%%%%%%%%%%%%%
% Exercise 4.1.13
%%%%%%%%%%%%%%%%%%%%%%%%%%%%%%%%%%%%%%%%%%%%%%%%%%%%%%%%%%%%%%%%%%%%%%%%%%%%%%%%%%%%%%%%%%%%%%%%%%%%%%%%%%%%%%%%%%%%%%%%%%%%%%%%%%%%%%%%
\begin{problem}{4.1.13}
Let \(V\) be a two-dimensional vector space over \(\mathbb{F}\) with basis \(x,y\) and set 
\[t:2x\otimes x\otimes x+x\otimes y\otimes y+y\otimes x\otimes y+y\otimes y\otimes x.\]
\begin{enumerate}[(1)]
	\item For \(\mathbb{F}=\mathbb{R}\) check that 
	      \[t=(x+cy)\otimes (x+cy)\otimes (x+cy)+(x-cy)\otimes (x-cy)\otimes (x-cy)\]
		  where \(c:=1/\sqrt{2}\). Deduce that \(t\) has rank 2.
     \item For \(\mathbb{F}=\mathbb{Q}\) show that \(t\) has rank strictly greater than 2.
\end{enumerate}
\end{problem}
\begin{solution}
\begin{enumerate}[(1)]
\item We have 
\begin{align*}
    &(x+cy)\otimes (x+cy)\otimes (x+cy)\\ 
   =&x\otimes x\otimes x+cx\otimes y\otimes x+cx\otimes x\otimes y+c^2x\otimes y\otimes y\\ 
    &+cy\otimes x\otimes x+c^2y\otimes y\otimes x+c^2 y\otimes x\otimes y+c^3y\otimes y\otimes y 
\end{align*}
Note that \(c=-\frac{1}{\sqrt{2}}\) is negative, so we have 
\begin{align*}
    &(x+cy)\otimes (x+cy)\otimes (x+cy)+(x-cy)\otimes (x-cy)\otimes (x-cy)\\ 
   =&2x\otimes x\otimes x+2c^2x\otimes y\otimes y+2c^2y\otimes y\otimes x+2c^2y\otimes x\otimes y\\ 
   =&2x\otimes x\otimes x+x\otimes y\otimes y+y\otimes y\otimes x+y\otimes x\otimes y\\ 
   =&t 
\end{align*}
We can see that \(t\) has rank 2.
\item Assume the opposite. Suppose 
\[t=f_1(x,y)\otimes f_2(x,y)\otimes f_3(x,y)+g_1(x,y)\otimes g_2(x,y)\otimes g_3(x,y)\]
where \(f_i(x,y),g_i(x,y)\in \mathbb{Q}[x,y]\) for \(i=1,2,3\). If the degree of \(f_i(x,y)\) is larger than 1, than the corresponding \(g_i(x,y)\) must have the same leading term with opposite sign, so 
without loss of generality, we could assume every \(f_i\) and \(g_i\) are of degree 1 and have no constant terms. Write \(f_i(x,y)=a_ix+b_iy\) and \(g_i(x,y)=c_ix+d_iy\) where \(a_i,b_i\in \mathbb{Q}\) for \(i=1,2,3\), we have 
\begin{align*}
    &a_1a_2a_3+c_1c_2c_3=2,\\
    &
\end{align*} 
\end{enumerate}
\end{solution}

\noindent\rule{7in}{2.8pt}
%%%%%%%%%%%%%%%%%%%%%%%%%%%%%%%%%%%%%%%%%%%%%%%%%%%%%%%%%%%%%%%%%%%%%%%%%%%%%%%%%%%%%%%%%%%%%%%%%%%%%%%%%%%%%%%%%%%%%%%%%%%%%%%%%%%%%%%%
% Exercise 4.1.15
%%%%%%%%%%%%%%%%%%%%%%%%%%%%%%%%%%%%%%%%%%%%%%%%%%%%%%%%%%%%%%%%%%%%%%%%%%%%%%%%%%%%%%%%%%%%%%%%%%%%%%%%%%%%%%%%%%%%%%%%%%%%%%%%%%%%%%%%
\begin{problem}{4.1.15}
Assume that the ground field \(\mathbb{F}=\mathbb{R}\). Show that 
\[M_2(\mathbb{R})\otimes M_2(\mathbb{R})\cong M_4(\mathbb{R})\cong \mathbb{H}\otimes \mathbb{H}\]
as \(\mathbb{R}\)-algebras.
\end{problem}
\begin{solution}
Let \(A,B\in M_2(\mathbb{R})\) be two \(2\times 2\) matrices where \(A=\begin{pmatrix}
    a & b\\ 
    c&d
\end{pmatrix}\) and \(B=\begin{pmatrix}
    e&f\\ 
    g&h
\end{pmatrix}\). We define the following map 
\begin{align*}
    \phi:M_2(\mathbb{R})\otimes M_2(\mathbb{R})&\rightarrow M_4(\mathbb{R}),\\ 
             A\otimes B&\mapsto \begin{pmatrix}
                aB&bB\\ 
                cB&dB
             \end{pmatrix}=\begin{pmatrix}
                ae&af&be&bf\\ 
                ag&ah&bg&bh\\ 
                ce&cf&de&df\\ 
                cg&ch&dg&dh
             \end{pmatrix}
\end{align*}
To see that this is a well-defined map between \(\mathbb{R}\)-algebras, let \(A_1,B_1,A_2,B_2\in M_2(\mathbb{R})\). We have 
\begin{align*}
    &\phi((A_1\otimes B_1)(A_2\otimes B_2))\\[1em]
   =&\phi((A_1A_2)\otimes (B_1B_2))\\[1em]
   =&\phi(\begin{pmatrix}
    a_1a_2+b_1c_2&a_1c_2+b_1d_2\\
    a_2c_1+c_2d_1&c_1c_2+d_1d_2
   \end{pmatrix}\otimes B_1B_2 )\\[1em] 
   =&\begin{pmatrix}
    (a_1a_2+b_1c_2)B_1B_2&(a_1c_2+b_1d_2)B_1B_2\\ 
    (a_2c_1+c_2d_1)B_1B_2&(c_1c_2+d_1d_2)B_1B_2
   \end{pmatrix}\\[1em] 
   =&\begin{pmatrix}
    a_1B_1&b_1B_1\\ 
    c_1B_1&d_1B_1
   \end{pmatrix}\begin{pmatrix}
    a_2B_2&b_2B_2\\ 
    c_2B_2&d_2B_2
   \end{pmatrix}\\[1em]
   =&\phi(A_1\otimes B_1)\phi(A_2\otimes B_2)
\end{align*}
Moreover, \(\phi\) is injective. Indeed, suppose \(A\otimes B\in \ker \phi\). This means that 
\[\begin{pmatrix}
    aB &bB\\
    cB &dB
\end{pmatrix}=\begin{pmatrix}
    0&0\\ 
    0&0
\end{pmatrix}.\]
If \(A=0\), then \(A\otimes B=0\otimes B=0\). If there is at least one nonzero entry in \(A\), for example \(a\neq 0\), then \(aB=0\) being the zero matrix implies 
that \(B=0\), and we have \(A\otimes B=A\otimes 0=0\). Note that 
\[\dim (M_2(\mathbb{R})\otimes M_2(\mathbb{R}))=(\dim(M_2(\mathbb{R})))^2=16=\dim M_4(\mathbb{R}).\]
So \(\phi\) is an isomorphism. 

Recall that the quaternions \(\mathbb{H}\) over \(\mathbb{R}\) has a basis \(\left\{ 1,i,j,k \right\}\) with multiplication \(i^2=j^2=k^2=-1\) and \(ij=-ji=k\). View \(\mathbb{H}\) as a 4-dimensional 
\(\mathbb{R}\)-vector space. And we know that \(End_{\mathbb{R}}(\mathbb{H})\cong M_4(\mathbb{R})\). Define a bilinear map 
\begin{align*}
    \mathbb{H}\times \mathbb{H}&\rightarrow End_{\mathbb{R}}(\mathbb{H}),\\ 
    (a,b)&\mapsto (h\mapsto ah\bar{b}).
\end{align*}
where 
\[\bar{b}=\overline{b_1+b_2i+b_3j+b_4k}=b_1-b_2i-b_3j-b_4k.\]
This map induces a linear map \(\psi:\mathbb{H}\otimes \mathbb{H}\rightarrow End_{\mathbb{R}}(\mathbb{H})\). Let \((a,b)\in \ker \psi\). For every \(h\in H\), we have 
\(ah\bar{b}=0\). Note that \(\mathbb{R}\) has characteristic 2, so either \(a\) or \(\bar{b}\) must \(0\). Thus, \(a\otimes b=0\) and \(\psi\) is injective. Moreover, we know that 
\[\dim(\mathbb{H}\otimes \mathbb{H})=16=\dim (End_{\mathbb{R}}(\mathbb{H})).\]
So we have the isomorphisms 
\[\mathbb{H}\otimes \mathbb{H}\cong M_4(\mathbb{R})\cong M_2(\mathbb{R})\otimes M_2(\mathbb{R}).\]
\end{solution}

\noindent\rule{7in}{2.8pt}
%%%%%%%%%%%%%%%%%%%%%%%%%%%%%%%%%%%%%%%%%%%%%%%%%%%%%%%%%%%%%%%%%%%%%%%%%%%%%%%%%%%%%%%%%%%%%%%%%%%%%%%%%%%%%%%%%%%%%%%%%%%%%%%%%%%%%%%%
% Exercise 4.2.12
%%%%%%%%%%%%%%%%%%%%%%%%%%%%%%%%%%%%%%%%%%%%%%%%%%%%%%%%%%%%%%%%%%%%%%%%%%%%%%%%%%%%%%%%%%%%%%%%%%%%%%%%%%%%%%%%%%%%%%%%%%%%%%%%%%%%%%%%
\begin{problem}{4.2.12(Duailty of symmetric and divided powers).}
Let \(V\) be a finite dimensional vector space. From Example 4.1.3, we get a natural isomorphism \(T^n(V^*)\xrightarrow{\sim} T^n(V)^*\) mapping a pure tensor 
\(f_1\otimes \cdots\otimes f_n\in T^n(V^*)\) to the unique linear map \(f_1\bar{\otimes} \cdots\bar{\otimes}f_n:T^n(V)\rightarrow \mathbb{F} \) which sends \(v_1\otimes \cdots\otimes v_n\in T^n(V)\) to 
\(f(v_1)\cdots f(v_n)\). Composing the dual map \(\pi^*\) to the quotient map \(\pi:T^n(V)\rightarrow S^n(V)\) with this isomorphism gives a linear map \(\pi^*:S^n(V)^*\hookrightarrow T^n(V^*)\). Prove 
that \(\pi^*\) is an isomorphism between \(S^n(v)^*\) and \(\Gamma^n(V^*)\).
\end{problem}
\begin{solution}
	
\end{solution}

\noindent\rule{7in}{2.8pt}
%%%%%%%%%%%%%%%%%%%%%%%%%%%%%%%%%%%%%%%%%%%%%%%%%%%%%%%%%%%%%%%%%%%%%%%%%%%%%%%%%%%%%%%%%%%%%%%%%%%%%%%%%%%%%%%%%%%%%%%%%%%%%%%%%%%%%%%%
% Exercise 4.3.17
%%%%%%%%%%%%%%%%%%%%%%%%%%%%%%%%%%%%%%%%%%%%%%%%%%%%%%%%%%%%%%%%%%%%%%%%%%%%%%%%%%%%%%%%%%%%%%%%%%%%%%%%%%%%%%%%%%%%%%%%%%%%%%%%%%%%%%%%
\begin{problem}{4.3.17}
Let \(V\) be a vector space.
\begin{enumerate}[(1)]
\item Vectors \(v_1,\ldots,v_m\in V\) are linearly independent if and only if \(v_1\wedge \cdots\wedge v_m\neq 0\) in \(\bigwedge^m V\).
\item Let \(v_1,\ldots,v_m\) and \(w_1,\ldots,w_m\) be two linearly independent systems of vectors in \(V\). Show that \(\mathbb{F}v_1+\cdots+\mathbb{F}v_m=\mathbb{F}w_1+\cdots+\mathbb{F}w_m\) 
      if and only if \(v_1\wedge \cdots\wedge v_m\) is proportional to \(w_1\wedge \cdots\wedge w_m\) in \(\bigwedge^m V\).
\item Show that there is a well-defined embedding \(Gr_m(V)\hookrightarrow \mathbb{P}(\bigwedge^m V)\) sending a subspace with basis \(v_1,\ldots,v_m\) to the line spanned by \(v_1\wedge\cdots\wedge v_m\).
\item Give an example to show that the map from (3) is not surjective in general.
\end{enumerate}
\end{problem}
\begin{solution}
	
\end{solution}

\noindent\rule{7in}{2.8pt}

%%%%%%%%%%%%%%%%%%%%%%%%%%%%%%%%%%%%%%%%%%%%%%%%%%%%%%%%%%%%%%%%%%%%%%%%%%%%%%%%%%%%%%%%%%%%%%%%%%%%%%%%%%%%%%%%%%%%%%%%%%%%%%%%%%%%%%%%
% Exercise 4.4.4
%%%%%%%%%%%%%%%%%%%%%%%%%%%%%%%%%%%%%%%%%%%%%%%%%%%%%%%%%%%%%%%%%%%%%%%%%%%%%%%%%%%%%%%%%%%%%%%%%%%%%%%%%%%%%%%%%%%%%%%%%%%%%%%%%%%%%%%%
\begin{problem}{4.4.4}
Assume char\(\mathbb{F}=2\). Then a \textit{quadratic form} on a vector space \(V\) is a function \(Q:V\rightarrow \mathbb{F}\) such that \(Q(\lambda v)=\lambda^2 Q(v)\) and \(Q(v+w)=Q(v)+Q(w)+(v|w)\) for some 
(necessarily unique) bilinear form \((-|-):V\times V\rightarrow \mathbb{F}\). Show that the form \((-|-)\) is skew-symmetric. Convince yourself that you cannot recover \(Q\) from \((-|-)\).
\end{problem}
\begin{solution}
Because char\(\mathbb{F}=2\), for any \(v\in V\), we have 
\[(v|v)=Q(v+v)+Q(v)+Q(v)=Q(2v)+2Q(v)=0.\]
Thus, \((-|-)\) is skew-symmetric. \(Q\) cannot be recovered from \((-|-)\) since \(2\) is not invertible in \(\mathbb{F}\).
\end{solution}

\noindent\rule{7in}{2.8pt}
%%%%%%%%%%%%%%%%%%%%%%%%%%%%%%%%%%%%%%%%%%%%%%%%%%%%%%%%%%%%%%%%%%%%%%%%%%%%%%%%%%%%%%%%%%%%%%%%%%%%%%%%%%%%%%%%%%%%%%%%%%%%%%%%%%%%%%%%
% Exercise 4.4.9
%%%%%%%%%%%%%%%%%%%%%%%%%%%%%%%%%%%%%%%%%%%%%%%%%%%%%%%%%%%%%%%%%%%%%%%%%%%%%%%%%%%%%%%%%%%%%%%%%%%%%%%%%%%%%%%%%%%%%%%%%%%%%%%%%%%%%%%%
\begin{problem}{4.4.9}
Let \(V\) be a finite dimensional vector space equipped with a non-degenerate skew-symmetric bilinear form \((-|-)\). If \(V\subseteq V\) is an isotropic subspace with 
basis \(u_1,\ldots,u_m\), then there exists an isotropic subspace \(U'\), with basis \(u_1',\ldots,u_m'\) such that \(U\cap U'=0\) and \((u_i|u_j')=\delta_{i,j}\) for all \(1\leq i,j\leq m\).
\end{problem}
\begin{solution}
	
\end{solution}

\noindent\rule{7in}{2.8pt}


%%%%%%%%%%%%%%%%%%%%%%%%%%%%%%%%%%%%%%%%%%%%%%%%%%%%%%%%%%%%%%%%%%%%%%%%%%%%%%%%%%%%%%%%%%%%%%%%%%%%%%%%%%%%%%%%%%%%%%%%%%%%%%%%%%%%%%%%
% Exercise 4.4.10
%%%%%%%%%%%%%%%%%%%%%%%%%%%%%%%%%%%%%%%%%%%%%%%%%%%%%%%%%%%%%%%%%%%%%%%%%%%%%%%%%%%%%%%%%%%%%%%%%%%%%%%%%%%%%%%%%%%%%%%%%%%%%%%%%%%%%%%%
\begin{problem}{4.4.10(Witt's Theorem for skew-symmetric bilinear form).}
Let \(V\) be a finite dimensional vector space equipped with a non-degenerate skew-symmetric bilinear form. Let \(U\) be a subspace of \(V\) with the induced bilinear form. Prove that any isometric embedding 
\(f:U\hookrightarrow V\) of \(U\) into \(V\) can be extended to an isometry \(\hat{f}:V\rightarrow V\).
\end{problem}
\begin{solution}
	
\end{solution}

\noindent\rule{7in}{2.8pt}
%%%%%%%%%%%%%%%%%%%%%%%%%%%%%%%%%%%%%%%%%%%%%%%%%%%%%%%%%%%%%%%%%%%%%%%%%%%%%%%%%%%%%%%%%%%%%%%%%%%%%%%%%%%%%%%%%%%%%%%%%%%%%%%%%%%%%%%%
% Exercise 4.4.12
%%%%%%%%%%%%%%%%%%%%%%%%%%%%%%%%%%%%%%%%%%%%%%%%%%%%%%%%%%%%%%%%%%%%%%%%%%%%%%%%%%%%%%%%%%%%%%%%%%%%%%%%%%%%%%%%%%%%%%%%%%%%%%%%%%%%%%%%
\begin{problem}{4.4.12(Pfaffians).}
Assume that \(\mathbb{F}\) is of characteristic zero. Let \(n=2m\) be even and \(A=[a_{ij}]_{1\leq i,j\leq n}\) be an \(n\times n\) skew-symmetric matrix with entries in \(\mathbb{F}\). Let \(V\) be a vector space with basis \(v_1,\ldots,v_n\) and 
set \(a:=\sum_{1\leq i,j\leq n} a_{ij}v_i\wedge v_j\in \bigwedge^2(V)\).
\begin{enumerate}[(1)]
\item Prove that \(a^m=2^m m!(\text{Pf}\, A)v_1\wedge \cdots\wedge v_n\) (\(m\)th power taken in the exterior algebra \(\wedge(V)\).
\item For any matrix \(P=[p_{ij}]_{1\leq i,j\leq n}\), show that \(\text{Pf}(P^T AP)=(\det P)(\text{Pf}\, A)\).
\item Show that \(\det A=(\text{Pf}\, A)^2\).
\end{enumerate}
\end{problem}
\begin{solution}
	
\end{solution}

\noindent\rule{7in}{2.8pt}
%%%%%%%%%%%%%%%%%%%%%%%%%%%%%%%%%%%%%%%%%%%%%%%%%%%%%%%%%%%%%%%%%%%%%%%%%%%%%%%%%%%%%%%%%%%%%%%%%%%%%%%%%%%%%%%%%%%%%%%%%%%%%%%%%%%%%%%%
% Exercise 6.1.12
%%%%%%%%%%%%%%%%%%%%%%%%%%%%%%%%%%%%%%%%%%%%%%%%%%%%%%%%%%%%%%%%%%%%%%%%%%%%%%%%%%%%%%%%%%%%%%%%%%%%%%%%%%%%%%%%%%%%%%%%%%%%%%%%%%%%%%%%
\begin{problem}{6.1.12}
Let \(H\) be a characteristic subgroup of \(G\). Prove:
\begin{enumerate}[(1)]
\item If \(G\) is a characteristic subgroup of \(K\), then \(H\) is a characteristic subgroup of \(K\).
\item If \(G\unlhd K\), then \(H\unlhd K\).
\item If \(K\) is a characteristic subgroup of \(G\), then \(HK\) and \(H\cap K\) are characteristic subgroups of \(G\).
\end{enumerate}
\end{problem}
\begin{solution}
\begin{enumerate}[(1)]
\item Let \(\phi:K\rightarrow K\) be a group automorphism. We know that \(\phi(G)\subset G\) since \(G\) is a characteristic subgroup of \(K\). This means that \(\phi\) can be viewes as a group 
automorphism of \(G\). Thus, \(\phi(H)\subset H\) because \(H\) is a characteristic subgroup of \(G\). This proves that \(H\) is a characteristic subgroup of \(K\). 
\item For any \(k\in K\), we have \(kHk^{-1}\subset G\) since \(G\) is a normal subgroup of \(K\). Note that in this case we have a group automorphism:
\begin{align*}
    \phi:G&\rightarrow G,\\ 
    g&\mapsto kgk^{-1}.
\end{align*}
And because \(H\) is a characteristic subgroup of \(G\), we have \(\phi(H)\subset H\). This proves that \(kHk^{-1}=H\). \(H\) is a normal subgroup of \(K\). 
\item For any \(hk\in HK\) and any automorphism \(\phi:G\rightarrow G\), we have \(\phi(gh)=\phi(g)\phi(h)\in HK\) since both \(H\) and \(K\) are characteristic subgroups of \(G\). Similarly for any \(a\in H\cap K\), we have 
\(\phi(a)\in H\cap K\). So \(HK\) and \(H\cap K\) are characteristic subgroups of \(G\). 
\end{enumerate}
\end{solution}

\noindent\rule{7in}{2.8pt}
%%%%%%%%%%%%%%%%%%%%%%%%%%%%%%%%%%%%%%%%%%%%%%%%%%%%%%%%%%%%%%%%%%%%%%%%%%%%%%%%%%%%%%%%%%%%%%%%%%%%%%%%%%%%%%%%%%%%%%%%%%%%%%%%%%%%%%%%
% Exercise 6.2.17
%%%%%%%%%%%%%%%%%%%%%%%%%%%%%%%%%%%%%%%%%%%%%%%%%%%%%%%%%%%%%%%%%%%%%%%%%%%%%%%%%%%%%%%%%%%%%%%%%%%%%%%%%%%%%%%%%%%%%%%%%%%%%%%%%%%%%%%%
\begin{problem}{6.2.17}
Let \(p\) be a prime.
\begin{enumerate}[(1)]
\item Construct an isomorphism between \(C_{p^\infty}\) and the subgroup of \(\mathbb{C}^\times\) which consists of all \(p^n\)th roots of \(1\) for all \(n\in \mathbb{Z}_{\geq 0}\).
\item Explain why the map \(g\mapsto g^p\) yields an isomorphism \(C_{p^\infty}/C_p\cong C_{p^\infty}\).
\item \(C_{p^\infty}\) is not finitely generated.
\item Describe all subgroups of \(C_{p^\infty}\).
\item Any non-trivial quotient of \(C_{p^\infty}\) is isomorphic to \(C_{p^\infty}\).
\end{enumerate}
\end{problem}
\begin{solution}
    
\end{solution}

\noindent\rule{7in}{2.8pt}
%%%%%%%%%%%%%%%%%%%%%%%%%%%%%%%%%%%%%%%%%%%%%%%%%%%%%%%%%%%%%%%%%%%%%%%%%%%%%%%%%%%%%%%%%%%%%%%%%%%%%%%%%%%%%%%%%%%%%%%%%%%%%%%%%%%%%%%%
% Exercise 6.2.18
%%%%%%%%%%%%%%%%%%%%%%%%%%%%%%%%%%%%%%%%%%%%%%%%%%%%%%%%%%%%%%%%%%%%%%%%%%%%%%%%%%%%%%%%%%%%%%%%%%%%%%%%%%%%%%%%%%%%%%%%%%%%%%%%%%%%%%%%
\begin{problem}{6.2.18}
Let \(p\) be a prime and \(\mathbb{Q}_{(p)}\) be a subgroup of \((\mathbb{Q},+)\) which consists of all numbers of the form \(m/p^n\) for \(m,n\in \mathbb{Z}\).Use the map 
\begin{align*}
    \mathbb{Q}_{(p)}&\rightarrow C_{p^\infty},\\ 
    m/p^n&\mapsto e^{2\pi im/p^n}
\end{align*} 
to deduce an isomorphism \(\mathbb{Q}_{(p)}/\mathbb{Z}\cong C_{p^\infty}\).  
\end{problem}
\begin{solution}
    
\end{solution}

\noindent\rule{7in}{2.8pt}
%%%%%%%%%%%%%%%%%%%%%%%%%%%%%%%%%%%%%%%%%%%%%%%%%%%%%%%%%%%%%%%%%%%%%%%%%%%%%%%%%%%%%%%%%%%%%%%%%%%%%%%%%%%%%%%%%%%%%%%%%%%%%%%%%%%%%%%%
% Exercise 6.3.8
%%%%%%%%%%%%%%%%%%%%%%%%%%%%%%%%%%%%%%%%%%%%%%%%%%%%%%%%%%%%%%%%%%%%%%%%%%%%%%%%%%%%%%%%%%%%%%%%%%%%%%%%%%%%%%%%%%%%%%%%%%%%%%%%%%%%%%%%
\begin{problem}{6.3.8}
Let \(p\) be a prime, \(\sigma\) be any \(p\)-cycle in \(S_p\) and \(\tau\) be any transposition in \(S_p\). Prove that \(\la \sigma,\tau\ra=S_p\).
\end{problem}
\begin{solution}
    
\end{solution}

\noindent\rule{7in}{2.8pt}
%%%%%%%%%%%%%%%%%%%%%%%%%%%%%%%%%%%%%%%%%%%%%%%%%%%%%%%%%%%%%%%%%%%%%%%%%%%%%%%%%%%%%%%%%%%%%%%%%%%%%%%%%%%%%%%%%%%%%%%%%%%%%%%%%%%%%%%%
% Exercise 6.5.3
%%%%%%%%%%%%%%%%%%%%%%%%%%%%%%%%%%%%%%%%%%%%%%%%%%%%%%%%%%%%%%%%%%%%%%%%%%%%%%%%%%%%%%%%%%%%%%%%%%%%%%%%%%%%%%%%%%%%%%%%%%%%%%%%%%%%%%%%
\begin{problem}{6.5.3(Elements of \(O(2)\)).}
\begin{enumerate}[(a)]
\item The matrix \(\begin{pmatrix}
    a&b\\ 
    c&d
\end{pmatrix}\) is orthogonal if and only if \(a^2+c^2=b^2+d^2=1\) and \(ab+cd=0\).
\item Deduce that a matrix if orthogonal if and only if it looks like 
\begin{equation}
    \begin{pmatrix}
    \cos \alpha&-\sin \alpha\\ 
    \sin \alpha&\cos \alpha
\end{pmatrix}
\end{equation}
or
\begin{equation}
    \begin{pmatrix}
    \cos \alpha&\sin \alpha\\ 
    \sin \alpha&-\cos \alpha
\end{pmatrix}
\end{equation}
for some \(\alpha\in \mathbb{R}\).
\item Prove that a matrix is in \(SO(2)\) if and only if it is of the form (1) for some \(\alpha\in \mathbb{R}\).
\item The linear transformation whose matrix is of the form (1) is the rotation through the angle \(\alpha\); the linear transformation whose matrix is of the form (2) is 
the reflection through the line forming the angle \(\alpha/2\) with the \(x\)-axis.
\item The group \(O(2)\) is generated by reflections. 
\end{enumerate}
\end{problem}
\begin{solution}
\begin{enumerate}[(a)]
\item A matrix \(A=\begin{pmatrix}
    a&b\\ 
    c&d
\end{pmatrix}\) is orthogonal if and only if \(A^T A=I_2\), written as 
\[\begin{pmatrix}
    a&c\\ 
    b&d
\end{pmatrix}\begin{pmatrix}
    a&b\\ 
    c&d
\end{pmatrix}=\begin{pmatrix}
    1&0\\ 
    0&1
\end{pmatrix}.\]
This is the same as the following equations:
\begin{align*}
    a^2+c^2&=1,\\
    ab+cd&=0,\\ 
    b^2+d^2&=1. 
\end{align*}
\item It is easy to check directly that this is a sufficient condition. We prove that it is also necessary. Since \(a^2+c^2=1\), we know there exists some \(\alpha\in \mathbb{R}\) such that \(a=\cos \alpha\) and 
\(c=\sin \alpha\). If \(a=\cos \alpha=0\), then \(c^2=\sin^2 \alpha=1\), and \(ab+cd=0\) tells us that \(d=0\). Thus, \(b^2=1\). Since \(b^2=d^2=1\), if \(b=d\) then \(A\) has the form (2). If \(b=-d\), then \(A\) 
has the form (1). Now suppose \(a=\cos \alpha\neq 0\). We can write 
\[b=-\frac{d\sin \alpha}{\cos \alpha}.\]
Plug this into \(b^2+d^2=1\), and we have 
\[d^2(1+\frac{\sin^2 \alpha}{\cos^2 \alpha})=\frac{d^2}{\cos^2 \alpha}=1.\]
If \(d=\cos \alpha\), then \(A\) has the form (1). If \(d=-\cos \alpha\), then \(A\) has the form (2).
\item A matrix \(A\) is in \(SO(2)\) if and only if \(A\) is orthogonal and \(\det A=1\). From what we have proved in (b), \(A\) must be of the form (1). 
\item Given a point nonzero point \(v=(x,y)\in \mathbb{R}^2\) and a matrix \(A\) of the form (1). The linear transformation associated with \(A\) maps \(v\) to 
\[Av=(x\cos \alpha-y\sin \alpha,x\sin \alpha+y\cos \alpha).\]
We can see that 
\[|Av|^2=(x\cos \alpha-y\sin \alpha)^2+(x\sin \alpha+y\cos \alpha)^2=x^2+y^2=|v|^2.\]
And moreover, the angle \(\theta\) between the vector \(v\) and \(Av\) can be calculated as 
\[\cos \theta=\dfrac{v\cdot Av}{|v||Av|}=\dfrac{(x^2+y^2)\cos \alpha}{x^2+y^2}=\cos \alpha.\]
So \(A\) is the rotation through angle \(\alpha\). 

Now assume the linear transformation has the form (2). In this case, \(Av\) can be written as 
\[Av=(x\cos \alpha+y\sin \alpha,x\sin \alpha-y\cos \alpha).\]
We still have \(|Av|^2=x^2+y^2=|v|^2\) and consider the line represented by the vector \(w=(\cos \frac{\alpha}{2},\sin \frac{\alpha}{2})\). The angle \(\theta_1\) between \(v\) and \(w\) is 
\[\cos \theta_1=\dfrac{v\cdot w}{|v|}=\dfrac{x\cos \frac{\alpha}{2}+y\sin \frac{\alpha}{2}}{x^2+y^2}.\]
and the angle \(\theta_2\) between \(Av\) and \(w\) is 
\[\cos \theta_2=\dfrac{Av\cdot w}{|Av|}=\dfrac{x(\cos \alpha\cos \frac{\alpha}{2}+\sin \alpha\sin \frac{\alpha}{2})+y(\sin \alpha\cos \frac{\alpha}{2}-\cos \alpha\sin \frac{\alpha}{2})}{x^2+y^2}=\dfrac{x\cos \frac{\alpha}{2}+y\sin \frac{\alpha}{2}}{x^2+y^2}.\]
This shows that \(\theta_1=\theta_2\) and we can conclude that \(A\) of the form (2) is the reflection through the line forming the angle \(\frac{\alpha}{2}\) with the \(x\)-axis.
\item We prove that the matrices of the form (1) can be generated by the matrices of the form (2), namely reflections. Write \(A=R(\alpha)\) is of the form (1) (rotation) for some \(\alpha\in \mathbb{R}\) and \(A=F(\beta)\) is of the form (2) (reflection) for some \(\beta\in \mathbb{R}\). 
\begin{claim}
For any \(\alpha\in \mathbb{R}\), we have \(R(\alpha)=F(\pi)F(\pi-\alpha)\).
\end{claim}
\begin{claimproof}
Just some computation. 
\begin{align*}
            F(\pi)F(\pi-\alpha)&=\begin{pmatrix}
                -1&0\\ 
                0&1
            \end{pmatrix}\begin{pmatrix}
                \cos(\pi-\alpha)&\sin (\pi-\alpha) \\ 
                \sin(\pi-\alpha)&-\cos (\pi-\alpha)
            \end{pmatrix}\\ 
            &=\begin{pmatrix}
                -\cos(\pi-\alpha)& -\sin(\pi-\alpha)\\ 
                \sin(\pi-\alpha)&-\cos (\pi-\alpha)
            \end{pmatrix}\\ 
            &=\begin{pmatrix}
                \cos \alpha&-\sin \alpha\\ 
                \sin \alpha&\cos \alpha
            \end{pmatrix}\\ 
            &=R(\alpha).
\end{align*}
\end{claimproof}

The claim above shows that an orthogonal matrix can be written as a product of reflections, thus \(O(2)\) is generated by reflections.
\end{enumerate}
\end{solution}

\noindent\rule{7in}{2.8pt}
%%%%%%%%%%%%%%%%%%%%%%%%%%%%%%%%%%%%%%%%%%%%%%%%%%%%%%%%%%%%%%%%%%%%%%%%%%%%%%%%%%%%%%%%%%%%%%%%%%%%%%%%%%%%%%%%%%%%%%%%%%%%%%%%%%%%%%%%
% Exercise 6.5.5
%%%%%%%%%%%%%%%%%%%%%%%%%%%%%%%%%%%%%%%%%%%%%%%%%%%%%%%%%%%%%%%%%%%%%%%%%%%%%%%%%%%%%%%%%%%%%%%%%%%%%%%%%%%%%%%%%%%%%%%%%%%%%%%%%%%%%%%%
\begin{problem}{6.5.5}
Let \(c_1,\ldots,c_l\in \mathbb{F}\) satisfy \(c_1+\cdots+c_l=1\), and \(v_1,\ldots,v_l\in V\). If \(f\) is an affine transformation of \(V\), then 
\[f(c_1v_1+\cdots+c_lv_l)=c_1f(v_1)+\cdots+c_lf(v_l).\]
Deduce that \(f\) fixes points \(v\neq w\) in \(V\) only if it fixes every point of the line through \(v\) and \(w\).
\end{problem}
\begin{solution}
We know that \(AGL(V)=GL(V)T(V)\), so an affine transformation \(f\) can be written as \(f=g t_w\) where \(g\in GL(V)\) is a linear transformation and \(t_w\) is a translation. Now we have 
\begin{align*}
    f(c_1v_1+\cdots+c_lv_l)&=gt_w(c_1v_1+\cdots+c_lv_l)\\ 
                           &=g(c_1v_1+\cdots+c_lv_l+w)\\ 
                           &=c_1g(v_1)+\cdots+c_lg(v_l)+g(w)\\ 
                           &=c_1g(v_1)+\cdots+c_lg(v_l)+(c_1+\cdots+c_l)g(w)\\ 
                           &=c_1(g(v_1)+g(w))+\cdots+c_l(g(v_l)+g(w))\\ 
                           &=c_1(gt_w)(v_1)+\cdots+c_l(gt_w)(v_l)\\ 
                           &=c_1f(v_1)+\cdots+c_lf(v_l).
\end{align*}
Now suppose \(f\) fixes points \(v,w\in V\) with \(v\neq w\). Any point on the line through \(v\) and \(w\) can be written as \(c_1v+c_2w\) for some \(c_1,c_2\in \mathbb{F}\) satisfying \(c_1+c_2=1\). So by the 
previous discussion, we have 
\begin{align*}
    f(c_1v+c_2w)&=c_1f(v)+c_2f(w)\\ 
                &=c_1v+c_2w.
\end{align*}
We can see that \(c_1v+c_2w\) is also a fixed point of \(f\).
\end{solution}

\noindent\rule{7in}{2.8pt}
%%%%%%%%%%%%%%%%%%%%%%%%%%%%%%%%%%%%%%%%%%%%%%%%%%%%%%%%%%%%%%%%%%%%%%%%%%%%%%%%%%%%%%%%%%%%%%%%%%%%%%%%%%%%%%%%%%%%%%%%%%%%%%%%%%%%%%%%
% Exercise 6.5.6
%%%%%%%%%%%%%%%%%%%%%%%%%%%%%%%%%%%%%%%%%%%%%%%%%%%%%%%%%%%%%%%%%%%%%%%%%%%%%%%%%%%%%%%%%%%%%%%%%%%%%%%%%%%%%%%%%%%%%%%%%%%%%%%%%%%%%%%%
\begin{problem}{6.5.6}
Suppose that the ground field \(\mathbb{F}\) has characteristic \(0\). If \(G\) is a finite subgroup of \(AGL(V)\) then there is an element \(v\in V\) fixed by every element of \(G\).
\end{problem}
\begin{solution}
    
\end{solution}

\noindent\rule{7in}{2.8pt}
%%%%%%%%%%%%%%%%%%%%%%%%%%%%%%%%%%%%%%%%%%%%%%%%%%%%%%%%%%%%%%%%%%%%%%%%%%%%%%%%%%%%%%%%%%%%%%%%%%%%%%%%%%%%%%%%%%%%%%%%%%%%%%%%%%%%%%%%
% Exercise 6.5.7
%%%%%%%%%%%%%%%%%%%%%%%%%%%%%%%%%%%%%%%%%%%%%%%%%%%%%%%%%%%%%%%%%%%%%%%%%%%%%%%%%%%%%%%%%%%%%%%%%%%%%%%%%%%%%%%%%%%%%%%%%%%%%%%%%%%%%%%%

%%%%%%%%%%%%%%%%%%%%%%%%%%%%%%%%%%%%%%%%%%%%%%%%%%%%%%%%%%%%%%%%%%%%%%%%%%%%%%%%%%%%%%%%%%%%%%%%%%%%%%%%%%%%%%%%%%%%%%%%%%%%%%%%%%%%%%%%
% Exercise 6.5.8
%%%%%%%%%%%%%%%%%%%%%%%%%%%%%%%%%%%%%%%%%%%%%%%%%%%%%%%%%%%%%%%%%%%%%%%%%%%%%%%%%%%%%%%%%%%%%%%%%%%%%%%%%%%%%%%%%%%%%%%%%%%%%%%%%%%%%%%%
\begin{problem}{6.5.8}
\begin{enumerate}[(1)]
\item The group \(AO(2)\) of motions of the Euclidean space \(\mathbb{R}^2\) is generated by reflections relative to arbitrary lines.
\item Each element of the group \(ASO(2)\) of rigid motions of \(\mathbb{R}^2\) is either a tranlation or a rotation about some point.
\end{enumerate}
\end{problem}
\begin{solution}
    
\end{solution}

\noindent\rule{7in}{2.8pt}
%%%%%%%%%%%%%%%%%%%%%%%%%%%%%%%%%%%%%%%%%%%%%%%%%%%%%%%%%%%%%%%%%%%%%%%%%%%%%%%%%%%%%%%%%%%%%%%%%%%%%%%%%%%%%%%%%%%%%%%%%%%%%%%%%%%%%%%%
% Exercise 6.6.4
%%%%%%%%%%%%%%%%%%%%%%%%%%%%%%%%%%%%%%%%%%%%%%%%%%%%%%%%%%%%%%%%%%%%%%%%%%%%%%%%%%%%%%%%%%%%%%%%%%%%%%%%%%%%%%%%%%%%%%%%%%%%%%%%%%%%%%%%
\begin{problem}{6.6.4(Finite subgroups of \(O(2)\)).}
Let \(G\) be a finite subgroup of \(O(2)\). Then \(G\) is one of the following:
\begin{enumerate}[(i)]
\item \(G=C_n\), the cyclic group of order \(n\) generated by the rotation through \(2\pi/n\);
\item \(G=D_{2n}\), the dihedral group of order \(2n\) generated by the rotation through \(2\pi/n\) and a reflection about a line through the origin.
\end{enumerate}
\end{problem}
\begin{solution}
    
\end{solution}

\noindent\rule{7in}{2.8pt}
%%%%%%%%%%%%%%%%%%%%%%%%%%%%%%%%%%%%%%%%%%%%%%%%%%%%%%%%%%%%%%%%%%%%%%%%%%%%%%%%%%%%%%%%%%%%%%%%%%%%%%%%%%%%%%%%%%%%%%%%%%%%%%%%%%%%%%%%
% Exercise 6.6.6
%%%%%%%%%%%%%%%%%%%%%%%%%%%%%%%%%%%%%%%%%%%%%%%%%%%%%%%%%%%%%%%%%%%%%%%%%%%%%%%%%%%%%%%%%%%%%%%%%%%%%%%%%%%%%%%%%%%%%%%%%%%%%%%%%%%%%%%%
\begin{problem}{6.6.6(Symmetries of a cube)}
Let \(C^3\) be a regular cube in \(\mathbb{R}^3\). Use the action of \(\text{Sym}_+(C^3)\) on the four diagonals of the cube to show that \(\text{Sym}_+(C^3)\cong S_4\). Show that 
\(\text{Sym}(C^3)\cong S_4\times C_2\). Any idea on \(\text{Sym}_+(C^n)\) and \(\text{Sym}(C^n)\)?
\end{problem}
\begin{solution}
    
\end{solution}

\noindent\rule{7in}{2.8pt}
%%%%%%%%%%%%%%%%%%%%%%%%%%%%%%%%%%%%%%%%%%%%%%%%%%%%%%%%%%%%%%%%%%%%%%%%%%%%%%%%%%%%%%%%%%%%%%%%%%%%%%%%%%%%%%%%%%%%%%%%%%%%%%%%%%%%%%%%
% Exercise 6.7.5
%%%%%%%%%%%%%%%%%%%%%%%%%%%%%%%%%%%%%%%%%%%%%%%%%%%%%%%%%%%%%%%%%%%%%%%%%%%%%%%%%%%%%%%%%%%%%%%%%%%%%%%%%%%%%%%%%%%%%%%%%%%%%%%%%%%%%%%%
\begin{problem}{6.7.5}
Prove that the group of upper unitriangular \(3\times 3\) matrices over \(\mathbb{F}_3\) is non-abelian and has exponent 3.
\end{problem}
\begin{solution}
Let \(A=\begin{pmatrix}
    1 &a&c\\ 
    & 1&b \\ 
  & & 1
\end{pmatrix}\) and \(B=\begin{pmatrix}
    1&d&f\\ 
    &1&e\\ 
    &&1
\end{pmatrix}\) where \(a,b,c,d,e,f\in \mathbb{F}_3\). We have 
\[AB=\begin{pmatrix}
    1&a&c\\ 
    &1&b\\ 
    &&1
\end{pmatrix}\begin{pmatrix}
    1&d&f\\ 
    &1&e\\ 
    &&1
\end{pmatrix}=\begin{pmatrix}
    1&a+d&c+f+ae\\ 
    &1&b+e\\ 
    &&1
\end{pmatrix}.\]
On the other hand, we have 
\[BA=\begin{pmatrix}
    1&a+d&c+f+bd\\ 
    &1&b+e\\ 
    &&1
\end{pmatrix}.\]
So \(AB\neq BA\) unless \(bd=ae\). For example, 
\[\begin{pmatrix}
    1&1&0\\ 
    &1&2\\ 
    &&1
\end{pmatrix}\begin{pmatrix}
    1&1&0\\ 
    &1&1\\ 
    &&1
\end{pmatrix}\neq \begin{pmatrix}
    1&1&0\\ 
    &1&1\\ 
    &&1
\end{pmatrix}\begin{pmatrix}
    1&1&0\\ 
    &1&2\\ 
    &&1
\end{pmatrix}.\]
For any upper unitriangular matrix \(A=\begin{pmatrix}
    1&a&c\\ 
    &1&b\\ 
    &&1
\end{pmatrix}\), we have 
\[A^3=\begin{pmatrix}
    1& 3a&3ab+3c\\ 
    &1&3b\\ 
    &&1
\end{pmatrix}=I_3\in GL_3(\mathbb{F}_3).\] 
So this group has exponent \(3\).
\end{solution}

\noindent\rule{7in}{2.8pt}
%%%%%%%%%%%%%%%%%%%%%%%%%%%%%%%%%%%%%%%%%%%%%%%%%%%%%%%%%%%%%%%%%%%%%%%%%%%%%%%%%%%%%%%%%%%%%%%%%%%%%%%%%%%%%%%%%%%%%%%%%%%%%%%%%%%%%%%%
% Exercise 6.7.6
%%%%%%%%%%%%%%%%%%%%%%%%%%%%%%%%%%%%%%%%%%%%%%%%%%%%%%%%%%%%%%%%%%%%%%%%%%%%%%%%%%%%%%%%%%%%%%%%%%%%%%%%%%%%%%%%%%%%%%%%%%%%%%%%%%%%%%%%
\begin{problem}{6.7.6}
Let \(G\) be a finitely generated group. Assume that \(g^3=1\) for all \(g\in G\).
\begin{enumerate}[(a)]
\item Show that \(G\) is finite. 
\item Assume further that \(G\) is generated by two elements. Show that \(|G|\leq 27\) and that this estimate cannot be improved.
\end{enumerate}
\end{problem}
\begin{solution}

\end{solution}

\noindent\rule{7in}{2.8pt}
%%%%%%%%%%%%%%%%%%%%%%%%%%%%%%%%%%%%%%%%%%%%%%%%%%%%%%%%%%%%%%%%%%%%%%%%%%%%%%%%%%%%%%%%%%%%%%%%%%%%%%%%%%%%%%%%%%%%%%%%%%%%%%%%%%%%%%%%
% Exercise 6.8.7
%%%%%%%%%%%%%%%%%%%%%%%%%%%%%%%%%%%%%%%%%%%%%%%%%%%%%%%%%%%%%%%%%%%%%%%%%%%%%%%%%%%%%%%%%%%%%%%%%%%%%%%%%%%%%%%%%%%%%%%%%%%%%%%%%%%%%%%%
\begin{problem}{6.8.7}
The group of upper unitriangular \(3\times 3\) matrices over \(\mathbb{F}_3\) from Exercise 6.7.5. is of the form \(C_3\ltimes (C_3\times C_3)\).
\end{problem}
\begin{solution}
Denote by \(G\) the group of \(3\times 3\) upper unitriangular matrices over \(\mathbb{F}_3\). Consider the following group homomorphism 
\begin{align*}
    C_3&\rightarrow G,\\ 
    a&\mapsto \begin{pmatrix}
        1&a&0\\ 
        &1&0\\ 
        &&1
    \end{pmatrix} 
\end{align*}
and 
\begin{align*}
    C_3\times C_3&\rightarrow G,\\ 
    (b,c)&\mapsto \begin{pmatrix}
        1&0&c\\ 
        &1&b\\ 
        &&1
    \end{pmatrix}. 
\end{align*}
These are well-defined group homomorphisms. Consider the group homomorphism 
\begin{align*}
    \phi:C_3&\rightarrow \text{Aut}(C_3\times C_3),\\ 
         a&\mapsto (\phi(a):(b,c)\rightarrow (b,ab+c)).
\end{align*}
The multiplication in the semidirect product \(C_3\ltimes (C_3\times C_3)\) is given by 
\[(a,b,c)\cdot (d,e,f)=(a+d,b+\phi(a)(e),c+\phi(a)(f))=(a+d,b+e,c+f+ae)\]
which is exactly the matrices multiplication 
\[\begin{pmatrix}
    1&a&c\\ 
    &1&b\\ 
    &&1
\end{pmatrix}\begin{pmatrix}
    1&d&f\\ 
    &1&e\\ 
    &&1
\end{pmatrix}=\begin{pmatrix}
    1&a+d&c+f+ae\\ 
    &1&b+e\\ 
    &&1
\end{pmatrix}.\]
Note that \(a,b,c,d,e,f\in \mathbb{F_3}\cong C_3\). We have an isomorphism \(G\cong C_3\ltimes (C_3\times C_3)\).
\end{solution}

\noindent\rule{7in}{2.8pt}
%%%%%%%%%%%%%%%%%%%%%%%%%%%%%%%%%%%%%%%%%%%%%%%%%%%%%%%%%%%%%%%%%%%%%%%%%%%%%%%%%%%%%%%%%%%%%%%%%%%%%%%%%%%%%%%%%%%%%%%%%%%%%%%%%%%%%%%%
% Exercise 7.2.11
%%%%%%%%%%%%%%%%%%%%%%%%%%%%%%%%%%%%%%%%%%%%%%%%%%%%%%%%%%%%%%%%%%%%%%%%%%%%%%%%%%%%%%%%%%%%%%%%%%%%%%%%%%%%%%%%%%%%%%%%%%%%%%%%%%%%%%%%



%%%%%%%%%%%%%%%%%%%%%%%%%%%%%%%%%%%%%%%%%%%%%%%%%%%%%%%%%%%%%%%%%%%%%%%%%%%%%%%%%%%%%%%%%%%%%%%%%%%%%%%%%%%%%%%%%%%%%%%%%%%%%%%%%%%%%%%%
% Exercise 7.2.16
%%%%%%%%%%%%%%%%%%%%%%%%%%%%%%%%%%%%%%%%%%%%%%%%%%%%%%%%%%%%%%%%%%%%%%%%%%%%%%%%%%%%%%%%%%%%%%%%%%%%%%%%%%%%%%%%%%%%%%%%%%%%%%%%%%%%%%%%

\end{document}
 
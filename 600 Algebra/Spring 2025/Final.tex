\documentclass[letterpaper, 12pt]{article}

\usepackage{/Users/zhengz/Desktop/Math/Workspace/Homework1/homework}

\begin{document}
\noindent
\large\textbf{Zhengdong Zhang} \hfill \textbf{Final Exam - Week 11} \\
Email: zhengz@uoregon.edu \hfill ID: 952091294 \\
\normalsize Course: MATH 649 - Abstract Algebra \hfill Term: Spring 2025 \\
Instructor: Professor Sasha Polishchuk \hfill Due Date: $10^{th}$ June, 2025 \\
\noindent\rule{7in}{2.8pt}
\setstretch{1.1}
%%%%%%%%%%%%%%%%%%%%%%%%%%%%%%%%%%%%%%%%%%%%%%%%%%%%%%%%%%%%%%%%%%%%%%%%%%%%%%%%%%%%%%%%%%%%%%%%%%%%%%%%%%%%%%%%%%%%%%%%%
% Problem 1
%%%%%%%%%%%%%%%%%%%%%%%%%%%%%%%%%%%%%%%%%%%%%%%%%%%%%%%%%%%%%%%%%%%%%%%%%%%%%%%%%%%%%%%%%%%%%%%%%%%%%%%%%%%%%%%%%%%%%%%%%%
\begin{problem}{1}
Prove that the extension \(\mathbb{Q}\subset \mathbb{Q}(\sqrt{2+\sqrt{2}})\) is Galois and compute its Galois group.
\end{problem}
\begin{solution}
Let \(a=\sqrt{2+\sqrt{2}}\) and \(K=\mathbb{Q}(\sqrt{2+\sqrt{2}})\). Consider the polynomial 
\[f(x)=x^4-4x^2+2\in \mathbb{Q}[x].\]
Use the prime number \(2\) and by Eisenstein's Criterion, \(f\) is irreducible in \(\mathbb{Q}[x]\). By direct computation, we have \(f(a)=0\). This means \(f\) is the minimal polynomial of \(a\) over \(\mathbb{Q}\). Factoring \(f\) in \(\mathbb{C}[x]\), and we obtain 
\begin{align*}
    f(x)&=x^4-4x^2+2\\ 
        &=(x^2-2)^2-2\\ 
        &=(x^2-a^2)(x^2-\frac{2}{a^2})\\ 
        &=(x-a)(x+a)(x-\frac{\sqrt{2}}{a})(x+\frac{\sqrt{2}}{a})
\end{align*}
Note that \(\sqrt{2}=a^2-2\in K\). All four roots of \(f\) are in \(K\). This implies \(K\) is the splitting field of \(f\) and we know that every finite extension over a characteristic 0 field is sepaprable, so \(\mathbb{Q}\subset K\) is a Galois extension. Let \(G=\Gal(K/\mathbb{Q})\) be the Galois group. We have 
\[|G|=[K:\mathbb{Q}]=\deg f=4.\]
Since \(K/\mathbb{Q}\) is a finite normal extension and \(f\) is irreducible over \(\mathbb{Q}\), there exists \(\sigma\in G\) such that \(\sigma(a)=\frac{\sqrt{2}}{a}\) by transitivity of Galois action. Then 
\[2-\sqrt{2}=\frac{2}{a^2}=(\sigma(a))^2=\sigma(a^2)=\sigma(2+\sqrt{2})=2+\sigma(\sqrt{2}).\]
because \(\sigma\) fix elements in \(\mathbb{Q}\). So 
\[\sigma(\frac{\sqrt{2}}{a})=\frac{\sigma(\sqrt{2})}{\sigma(a)}=\frac{-\sqrt{2}}{\frac{\sqrt{2}}{a}}=-a.\]
This implies \(\sigma^2(a)\neq a\). So \(\sigma\) does not have order 2 in \(G\), then \(\sigma\) must have order 4 because \(|G|=4\). Thus, \(\sigma\) generates \(G\) and we can see that \(G\cong C_4\). 
\end{solution}

\noindent\rule{7in}{2.8pt}
%%%%%%%%%%%%%%%%%%%%%%%%%%%%%%%%%%%%%%%%%%%%%%%%%%%%%%%%%%%%%%%%%%%%%%%%%%%%%%%%%%%%%%%%%%%%%%%%%%%%%%%%%%%%%%%%%%%%%%%%%
% Problem 2
%%%%%%%%%%%%%%%%%%%%%%%%%%%%%%%%%%%%%%%%%%%%%%%%%%%%%%%%%%%%%%%%%%%%%%%%%%%%%%%%%%%%%%%%%%%%%%%%%%%%%%%%%%%%%%%%%%%%%%%%%%
\begin{problem}{2}
Let \(f\in \mathbb{Q}[x]\) be an irreducible polynomial. Assume \(f\) has both real and non-real roots. Prove that the Galois group of \(f\) is non-ableian.
\end{problem}
\begin{solution}
Let \(K\) be the splitting field of \(f\). \(Q\) has characteristic 0, so \(K/\mathbb{Q}\) is a Galois extension. Let \(G=\Gal(K/\mathbb{Q})\) be the Galois group. Let \(z\in \mathbb{C}\) be a complex root of \(f\) and \(a\in \mathbb{R}\) be a real root of \(f\). Note that \(\bar{z}\) is also a root of \(f\) because 
\[0=\overline{f(z)}=\bar{f}(\bar{z})=f(\bar{z})\]
and \(f\in \mathbb{Q}[x]\) implies that \(f=\bar{f}\). Consider a field automorphism \(\sigma\in G\) by sending all roots of \(f\) to its complex conjugate. We know \(f\) has complex roots, so \(\sigma\) is not the identity element. We have  
\[\sigma(z)=\bar{z},\ \ \sigma(a)=a,\ \ \sigma(\bar{z})=z.\]
Since \(f\) is irreducible over \(\mathbb{Q}\), there exists \(g\in G\) such that \(g(a)=z\) by transitivity of Galois action. Then we have 
\begin{align*}
    g\sigma(a)&=g(a)=z,\\ 
    \sigma g(a)&=\sigma(z)=\bar{z}.
\end{align*}
This implies \(\sigma g\neq g\sigma\). So \(G\) is not an abelian group. 
\end{solution}

\noindent\rule{7in}{2.8pt}
%%%%%%%%%%%%%%%%%%%%%%%%%%%%%%%%%%%%%%%%%%%%%%%%%%%%%%%%%%%%%%%%%%%%%%%%%%%%%%%%%%%%%%%%%%%%%%%%%%%%%%%%%%%%%%%%%%%%%%%%%
% Problem 3
%%%%%%%%%%%%%%%%%%%%%%%%%%%%%%%%%%%%%%%%%%%%%%%%%%%%%%%%%%%%%%%%%%%%%%%%%%%%%%%%%%%%%%%%%%%%%%%%%%%%%%%%%%%%%%%%%%%%%%%%%%
\begin{problem}{3}
Let \(R\) be a commutative Noetherian local ring with maximal ideal \(M\) which satisfies \(M^2=M\). Prove that \(R\) is a field. Show that this is false if \(R\) is not required to be Noetherian.
\end{problem}
\begin{solution}
\(R\) is a Noetherian local ring, so the unique maximal ideal \(M\) is a finitely generated \(R\)-module, and \(M=J(R)\) the Jacobson ideal of \(R\). \(M^2=M\) is equivalent to \(J(R)M=M\), by Nakayama's lemma, \(M=0\). This implies \(R\cong R/(0)\) is a field. 

Next, consider the following ring 
\[R=\mathbb{Q}[x,x^\frac{1}{2},x^\frac{1}{3},\ldots,x^\frac{1}{n},\ldots].\]
Here \(R\) is the rational field \(\mathbb{Q}\) adjoining all the \(n\)th root of \(x\) for \(n\geq 1\). Then \(R\) is not noetherian since it has an ascending chain of ideals
\[(x)\subsetneq (x,x^\frac{1}{2})\subsetneq \cdots\]
Let \(m\) be the maximal ideal 
\[(x,x^\frac{1}{2},x^\frac{1}{3},\ldots,x^\frac{1}{n},\ldots).\]
Note that \(m^2=m\) because for any \(k\geq 1\), we have 
\[x^\frac{1}{k}=x^\frac{1}{2k}\cdot x^\frac{1}{2k}.\]
Consider the local ring \(R_m\), it has a unique maximal ideal \(M=mR_m\) satisfying 
\[M^2=(mR_m)^2=m^2R_m=mR_m=M.\]
Note that here \(R_m\) is not a field because \(x\in R_m\) is not invertible. 
\end{solution}

\noindent\rule{7in}{2.8pt}
%%%%%%%%%%%%%%%%%%%%%%%%%%%%%%%%%%%%%%%%%%%%%%%%%%%%%%%%%%%%%%%%%%%%%%%%%%%%%%%%%%%%%%%%%%%%%%%%%%%%%%%%%%%%%%%%%%%%%%%%%
% Problem 4
%%%%%%%%%%%%%%%%%%%%%%%%%%%%%%%%%%%%%%%%%%%%%%%%%%%%%%%%%%%%%%%%%%%%%%%%%%%%%%%%%%%%%%%%%%%%%%%%%%%%%%%%%%%%%%%%%%%%%%%%%%
\begin{problem}{4}
\begin{enumerate}[(a)]
\item Let \(R\) be a unique factorization domain with \(2\) invertible, and let \(K\) be its field of fractions. For a non-unit \(f\in R\) such that there are no repeated primes in the factorization of \(f\), find the integral closure of \(R\) in \(K(\sqrt{f})\).
\item Prove that the ring \(\mathbb{C}[x,y,z]/(x^2+y^2+z^2)\) is integrally closed.
\end{enumerate}
\end{problem}
\begin{solution}
\begin{enumerate}[(a)]
\item Consider the polynomial \(x^2-f\in R[x]\). \(x^2-f\) is irreducible because \(f\) has no repeated primes. It can be easily seen that \(x^2-f\) is the minimal polynomial of \(\sqrt{f}\), so the field \(K(\sqrt{f})\) is isomorphic to \(K[x]/(x^2-f)\)  and every element \(a\in K(\sqrt{f})\) can be written as \(a=m+n\sqrt{f}\) for some \(m,n\in K\). Note that 
\[(a-m)^2=n^2f.\]
So \(a\) is the root of the polynomial 
\[p(x)=x^2-2mx+m^2-n^2f\in K[x].\]
If \(n=0\), then \(a=m\) is integral over \(R\) if and only if \(a=m\in R\). 

Assume \(n\neq 0\). In this case, \(p(x)\) is irreducible over \(K\) because the two roots: \(m+n\sqrt{f}\) and \(m-n\sqrt{f}\) are not in \(K\). Suppose \(a\) is integral over \(R\), then there exists an irreducible monic polynomial \(q(x)\in R[x]\subseteq K[x]\) such that \(q(a)=0\). This means \(p(x)\) divides \(q(x)\) in \(K[x]\). Suppose \(q(x)=h(x)p(x)\) in \(K[x]\), by Gauss's lemma, \(q(x)\) also has a factorization in \(R[x]\) and because \(q(x)\) is irreducible in \(R[x]\), \(h(x)=1\). So the coefficients of \(p(x)\) lies in \(R\) if \(a\) is integral over \(R\). This implies \(m\in R\) because \(2\) is invertible. So \(n^2f\in R\). Since \(n\in K\), \(n\) can be written as \(n=\frac{n_1}{n_2}\) where \(n_1,n_2\in R\) are non-units and have no common primes in the factorization. There exists \(b\in R\) such that 
\[bn_2^2=fn_1^2.\]
Here \(f\) has no repeated primes, so \(n_2^2\) must have some common primes with \(n_1^2\). This contradicts our assumption \(n_1,n_2\) have no common primes, so \(n_2\) is a unit and \(n\in R\). Thus, the integral closure of \(R\) in \(K(\sqrt{f})\) is 
\[R[\sqrt{f}]\cong R[x]/(x^2-f).\]
\item Let \(R=\mathbb{C}[y,z]\) be a unique factorization domain. The field of fractions is \(K=\mathbb{C}(y,z)\). Consider the non-unit \(f=-(y^2+z^2)\in R\). We have proved in (a) that the integral closure of \(R\) in \(K(\sqrt{-(y^2+z^2)})\) is \(R[x]/(x^2+y^2+z^2)\). This implies that 
\[R[x]/(x^2+y^2+z^2)\cong \mathbb{C}[x,y,z]/(x^2+y^2+z^2)\]
is integrally closed. 
\end{enumerate}
\end{solution}

\noindent\rule{7in}{2.8pt}
%%%%%%%%%%%%%%%%%%%%%%%%%%%%%%%%%%%%%%%%%%%%%%%%%%%%%%%%%%%%%%%%%%%%%%%%%%%%%%%%%%%%%%%%%%%%%%%%%%%%%%%%%%%%%%%%%%%%%%%%%
% Problem 5
%%%%%%%%%%%%%%%%%%%%%%%%%%%%%%%%%%%%%%%%%%%%%%%%%%%%%%%%%%%%%%%%%%%%%%%%%%%%%%%%%%%%%%%%%%%%%%%%%%%%%%%%%%%%%%%%%%%%%%%%%%
\begin{problem}{5}
Consider a quadratic extension \(\mathbb{Z}\subset A:=\mathbb{Z}[x]/(x^2+\alpha x+\beta)\), where \(x^2+\alpha x+\beta\) is an irreducible polynomial in \(\mathbb{Z}[x]\). Let \(p\in \mathbb{Z}\) be a prime number. Assume \(a\in A\) is such that \(\Nm(a)=\pm p\), where \(\Nm(a)=a\cdot \sigma(a)\) is the norm in the corresponding quadratic extension of \(\mathbb{Q}\) (here \(\sigma\) is a nontrivial element of the Galois group). Let \((a)\subset A\) be the corresponding principal ideal. 
\begin{enumerate}[(a)]
\item Prove that \((a)\cap \mathbb{Z}=(p)\).
\item Prove that the ideal \((a)\) is prime. 
\end{enumerate}
\end{problem}
\begin{solution}
\begin{enumerate}[(a)]
\item Obviously \((p)\subseteq \mathbb{Z}\) since \(p\) is a prime number. To show that \(p\in (a)\), it is enough to prove that \(\sigma(a)\in A\). Indeed, \(a\) can be written as \(m+nT\) where \(m,n\in \mathbb{Z}\) and \(T\) is a root of the plynomial \(x^2+\alpha x+\beta\). We know that \(T+\sigma(T)=-\alpha\), so 
\[\sigma(m+nT)=m+n\sigma(T)=m+n(-\alpha-T)=m-n\alpha-nT\in A.\]
This implies that 
\[\Nm(a)=a\cdot \sigma(a)=\pm p\in (a).\]
So we have \((p)\subseteq (a)\cap \mathbb{Z}\). Conversely, we know that \((a)\cap \mathbb{Z}\) is a proper ideal in \(\mathbb{Z}\). Since \(\mathbb{Z}\) is a PID, suppose \((a)\cap \mathbb{Z}=(b)\) for some \(b\in \mathbb{Z}\). We have already proved \(p\in (a)\). So \(b|p\) and because \(p\) is prime, \(b=p\). This proves that \((a)\cap \mathbb{Z}=(p)\).
\item Let \(I\) be the principal ideal generated by the prime number \(p\) in \(A\). Then 
\[A/I\cong (\mathbb{Z}/p \mathbb{Z})[x]/(x^2+\alpha x+\beta).\]
Let \(T\) be one root of \(x^2+\alpha x+\beta=0\). Then every element in \(A/I\) can be written as \(m+nT\) where \(m,n\in \mathbb{Z}/p \mathbb{Z}\). This implies that 
\[|A/I|=p^2\]
because \(\mathbb{Z}/p \mathbb{Z}\) is a finite field with \(p\) elements. Note that we have proved in (a) that \(p\in (a)\), thus \(I\subsetneq (a)\). This is strict inclusion because if \(I=(a)\). Then \(\Nm(p)=p\cdot p=p^2\). A contradiction. So we know that 
\[|A/(a)|<|A/I|=p^2.\]
Here the quotient ring \(A/(a)\) has a additive group structure and can be viewed as an abelian subgroup of \(A/I\), so \(|A/(a)|=1\) or \(|A/(a)|=p\). We know that \((a)\) is a proper ideal of \(A\) (otherwise \((a)\cap \mathbb{Z}=\mathbb{Z}\)), so \(|A/(a)|=p\). The only possible ring with \(p\) elements is \(\mathbb{Z}/p \mathbb{Z}\). In this case, \(A/(a)\) is a domian, so \((a)\) is a prime ideal.  
\end{enumerate}
\end{solution}

\noindent\rule{7in}{2.8pt}
%%%%%%%%%%%%%%%%%%%%%%%%%%%%%%%%%%%%%%%%%%%%%%%%%%%%%%%%%%%%%%%%%%%%%%%%%%%%%%%%%%%%%%%%%%%%%%%%%%%%%%%%%%%%%%%%%%%%%%%%%
% Problem 6
%%%%%%%%%%%%%%%%%%%%%%%%%%%%%%%%%%%%%%%%%%%%%%%%%%%%%%%%%%%%%%%%%%%%%%%%%%%%%%%%%%%%%%%%%%%%%%%%%%%%%%%%%%%%%%%%%%%%%%%%%%
\begin{problem}{6}
Let \(k=\mathbb{C}\). Describe the irreducible components of the following algebraic sets in \(\mathbb{A}^3\). 
\begin{enumerate}[(a)]
\item \(V(y^2-xz,x^4-yz,z^2-x^3y)\).
\item \(V(xz-y^2,z^3-x^5)\).
\end{enumerate}
\end{problem}
\begin{solution}
\begin{enumerate}[(a)]
\item Consider the following ring homomorphism 
\begin{align*}
    \phi:k[x,y,z]&\rightarrow k[t],\\ 
         x&\mapsto t^3,\\
         y&\mapsto t^5,\\ 
         z&\mapsto t^7. 
\end{align*}
Let \(I\subseteq k[x,y,z]\) be the ideal 
\[I=(y^2-xz,x^4-yz,z^2-x^3y).\]
It is easy to check that three polynomials satisfy the relationship, so \(I\subseteq \ker \phi\). Conversely, suppose \(f\in \ker \phi\). \(f\) can be written as 
\[f=f_1(y^2-xz)+f_2(x^4-yz)+f_3(z^2-x^3y)+f_4\]
where \(f_1,f_2,f_3,f_4\in k[x,y,z]\). \(f(t^3,t^5,t^7)=0\) implies that \(f_4(t^3,t^5,t^7)=0\). Note that \(f_4\) can be written as 
\[f_4(x,y,z)=a_1(y,z)x+a_2(y,z)x^2+a_3(y,z)x^3.\]
Here \(a_i(y,z)\) has degree at most 2 and the only possible degree 2 term is \(cyz\) for some \(c\in k\). Write 
\[a_1(y,z)x=c_1yx+c_2zx+c_3yzx+c_4x.\]
Then \(f_4(t^3,t^5,t^7)=0\) implies that 
\[c_1 t^8+c_2 t^{12}+c_3t^{15}+c_4t^3=0.\]
So \(c_1=c_2=c_3=c_4=0\). A similar argument can show that \(a_2=a_3=0\). So \(f_4=0\). This implies that \(f\in I\). We can conclude that \(I=\ker\phi\). Therefore, the coordinate ring \(k[x,y,z]/I\) is isomorphic to a subring of \(k[t]\), which is a domain. So \(I\) is prime and \(V(I)\) is an irreducible algebraic set.
\item Consider the ring homomorphism
\begin{align*}
    \phi:k[x,y,z]&\rightarrow k[t],\\ 
         x&\mapsto t^3,\\ 
         y&\mapsto t^4,\\ 
         z&\mapsto t^5.
\end{align*}
Let \(I\) be the ideal 
\[I=(z^3-x^5,xz-y^2).\]
It is easy to check that \(I\subseteq \ker\phi\). Conversely, suppose \(f\in \ker\phi\). \(f\) can be written as 
\[f(x,y,z)=f_1(x^5-z^3)+f_2(y^2-xz)+f_3\]
where \(f_1,f_2,f_3\in k[x,y,z]\). \(f\in \ker\phi\) implies that \(f_3(t^3,t^4,t^5)=0\). Note that \(f_3\) can be written as 
\[f_3(x,y,z)=g_1(x,z)+g_2(x,z)y.\]
This implies that \(g_1(t^3,t^5)=g_2(t^3,t^5)=0\). Note that here \(g_1,g_2\) can only have finite degrees since \(x^5-z^3=0\). A similar argument as (a) implies that \(g_1=g_2=0\). So \(f\in I\) and the coordinate ring \(k[x,y,z]/I\) is isomorphic to a subring of \(k[t]\), which is a domain. Thus, \(I\) is a prime ideal and \(V(I)\) is an irreducible algebraic set. 
\end{enumerate}
\end{solution}

\noindent\rule{7in}{2.8pt}
%%%%%%%%%%%%%%%%%%%%%%%%%%%%%%%%%%%%%%%%%%%%%%%%%%%%%%%%%%%%%%%%%%%%%%%%%%%%%%%%%%%%%%%%%%%%%%%%%%%%%%%%%%%%%%%%%%%%%%%%%
% Problem 7
%%%%%%%%%%%%%%%%%%%%%%%%%%%%%%%%%%%%%%%%%%%%%%%%%%%%%%%%%%%%%%%%%%%%%%%%%%%%%%%%%%%%%%%%%%%%%%%%%%%%%%%%%%%%%%%%%%%%%%%%%%
\begin{problem}{7}
Let \(X\subset \mathbb{A}^n\) be a non-empty algebraic set (we work over an alegbraically closed field \(k\)).
\begin{enumerate}[(a)]
\item Prove that \(X\) is not connected in Zariski topology if and only if there exists two proper ideals \(I\) and \(J\) in \(k[x_1,\ldots,x_n]\) such that \(I+J=(1)\) and \(I\cap J=I(X)\). 
\item Prove that \(X\) is connected if and only if for any \(f\in k[X]\) such that \(f^2=f\), one has either \(f=0\) or \(f=1\).
\end{enumerate}
\end{problem}
\begin{solution}
\begin{enumerate}[(a)]
\item Assume \(X\) is not connected in Zariski topology. Then there exists two non-empty closed subset \(X_1,X_2\subset X\) such that \(X_1\sqcup X_2=X\). Take \(I=I(X_1)\) and \(J=I(X_2)\). They are proper ideals since both of them are non-empty. Then we have 
\[V(I(X_1)+I(X_2))=V(I(X_1))\cap V(I(X_2))=X_1\cap X_2=\varnothing=V(1)\]
because their disjoint union is equal to \(X\). Similarly, 
\[V(I(X_1)\cap I(X_2))=V(I(X_1))\cup V(I(X_2))=X=V(I(X)).\]
By Nullstellensatz, this implies that \(I+J=(1)\) and \(I\cap J=I(X)\). 

Conversely, assume there exists proper ideals \(I,J\subset k[x_1,\ldots,x_n]\) such that \(I+J=(1)\) and \(I\cap J=I(X)\). Consider two closed subset of \(X\): \(X_1=V(\sqrt{I})\) and \(X_2=V(\sqrt{J})\). Note that \(1\in I+J\subseteq \sqrt{I}+\sqrt{J}\). Then we have 
\begin{align*}
    X_1\cup X_2&=V(\sqrt{I})\cup V(\sqrt{J})=V(\sqrt{I\cap J})=V(\sqrt{I(X)})=X,\\ 
    X_1\cap X_2&=V(\sqrt{I})\cap V(\sqrt{J})=V(\sqrt{I}+\sqrt{J})=V(1)=\varnothing.
\end{align*}
This tells us that \(X=X_1\sqcup X_2\), so \(X\) is not connected.
\item Assume \(X\) is connected and suppose there exists non-constant polynomial \(f\in k[X]\) such that \(f^2=f\). Without loss of generality, we can assume \(f\) is irreducible. Consider the ring homomorphism
\[q:k[x_1,\ldots,x_n]\rightarrow k[X].\]
Consider two ideals \(I=(f)\) and \(J=(1-f)\) in \(k[X]\). The preimage \(q^{-1}(I)\) and \(q^{-1}(J)\) are proper ideals of \(k[x_1,\ldots,x_n]\) because \(f\neq 0\) and \(f\neq 1\). Since \(q\) is surjective, choose \(g\in k[x_1,\ldots,x_n]\) such that \(q(g)=f\). Then \(g\in q^{-1}(I)\) and \(1-g\in q^{-1}(J)\). We have 
\[q^{-1}(I)+q^{-1}(J)=(1)=k[x_1,\ldots,x_n].\]
On the other hand, 
\[q^{-1}(I)\cap q^{-1}(J)=q^{-1}(I\cap J)=q^{-1}((f(1-f)))=q^{-1}((0))=I(X)\]
because \(I(X)\) is the kernel of the map \(q\). From what we proved in (a), we know that \(X\) is not connected. 

Conversely, suppose \(X\) is not connected. Then there exists two proper ideals 
\[I,J\subseteq k[x_1,\ldots,x_n]\] 
such that \(I+J=(1)\) and \(I\cap J=I(X)\). Choose \(g\in I\) satisfying \(g\notin I\cap J\) (This can be done because they are proper ideals). Then \(1-g\in J\). So \(g(1-g)\in I\cap J=I(X)\). Consider the image \(q(g)=f\) in \(k[X]\). \(f\neq 0\) in \(k[X]\) because \(g\notin I(X)\). \(f\neq 1\) because \(I\) is a proper ideal of \(k[x_1,\ldots,x_n]\). And we have \(f^2=f\) since \(g(1-g)\in I(X)\). 
\end{enumerate}
\end{solution}


\end{document}
\documentclass[a4paper, 12pt]{article}

\usepackage{/Users/zhengz/Desktop/Math/Workspace/Homework1/homework}

\begin{document}

\noindent

\large\textbf{Zhengdong Zhang} \hfill \textbf{Homework - Week 4} \\ 
Email: zhengz@uoregon.edu \hfill ID: 952091294 \\ 
\normalsize Course: MATH 649 - Abstract Algebra \hfill Term: Spring 2025 \\
Instructor: Professor Sasha Polishchuk \hfill Due Date: $30^{th}$ April , 2025 \\ 
\noindent\rule{7in}{2.8pt}
\setstretch{1.1}

%%%%%%%%%%%%%%%%%%%%%%%%%%%%%%%%%%%%%%%%%%%%%%%%%%%%%%%%%%%%%%%%%%%%%%%%%%%%%%%%%%%%%%%%%%%%%%%%%%%%%%%%%%%%%%%%%%%%%%%%%
% Problem 12.1.6
%%%%%%%%%%%%%%%%%%%%%%%%%%%%%%%%%%%%%%%%%%%%%%%%%%%%%%%%%%%%%%%%%%%%%%%%%%%%%%%%%%%%%%%%%%%%%%%%%%%%%%%%%%%%%%%%%%%%%%%%%%
\begin{problem}{12.16}
Let \(\text{char}\  k\neq 2\) and \(f\in \Bbbk[x]\) be a cubic whose discriminant has a square root in \(\Bbbk\), then \(f\) is either irreducible or splits in \(\Bbbk\).
\end{problem}
\begin{solution}
Let \(\mathbb{K}\) be the splitting field of \(f\) over \(\Bbbk\). First we suppose \(f\) has multiple roots. Then the discriminant \(\Delta(f)=0\) has a square root in \(\Bbbk\). If \(f\) has only one root \(\alpha\), 
then \(f\) is irreducible when \(\alpha\notin \Bbbk\) and \(f\) splits in \(\Bbbk\) when \(\alpha\in \Bbbk\). If \(\alpha\) as a root of \(f\) has multiplicity \(2\), let \(\beta\) be another root of \(f\), \(f(x)\) can be written as 
\[f(x)=(x-\alpha)^2(x-\beta)\] 
when \(\beta\in \Bbbk\), we know that \((x-\alpha)^2=x^2-2\alpha x+\alpha^2\in \Bbbk[x]\). Note that \(2\) is invertible in \(\Bbbk\), so this implies \(\alpha\in \Bbbk\). Thus, \(f\) splits in \(\Bbbk\). When \(\alpha\in \Bbbk\), it is easy to see that \(\beta\in \Bbbk\) and \(f\) again splits in \(\Bbbk\). 
If neither \(\alpha\) nor \(\beta\) is in \(\Bbbk\), then \(f\) is irreducible over \(\Bbbk\).

Now suppose \(f\) does not have multiple roots. By Theorem 12.1.2, the Galois group \(G=\Gal(\mathbb{K}/\Bbbk)\leq A_3\cong C_3\). We know that \(C_3\) is simple and only have two subgroups: \(\left\{ e \right\}\) or \(C_3\). When \(G=\left\{ e \right\}\), this means \(\mathbb{K}=\Bbbk\), so \(f\) splits in \(\Bbbk\). 
When \(G=C_3\), this means the action of \(G\) on the roots of \(f\) is transitive, thus \(f\) is irreducible. 
\end{solution}

\noindent\rule{7in}{2.8pt}
%%%%%%%%%%%%%%%%%%%%%%%%%%%%%%%%%%%%%%%%%%%%%%%%%%%%%%%%%%%%%%%%%%%%%%%%%%%%%%%%%%%%%%%%%%%%%%%%%%%%%%%%%%%%%%%%%%%%%%%%%
% Problem 12.4.9
%%%%%%%%%%%%%%%%%%%%%%%%%%%%%%%%%%%%%%%%%%%%%%%%%%%%%%%%%%%%%%%%%%%%%%%%%%%%%%%%%%%%%%%%%%%%%%%%%%%%%%%%%%%%%%%%%%%%%%%%%%
\begin{problem}{12.4.9}
Let \(\mathbb{K}/\Bbbk\) be a finite Galois extension and \(\alpha\in \mathbb{K}\). Consider the \(\Bbbk\)-linear operator \(A_\alpha:x\mapsto \alpha x\) on the \(\Bbbk\)-vector space \(\mathbb{K}\). Then \(\det A_\alpha=N_{\mathbb{K}/\Bbbk}(\alpha)\) and 
\(\tr A_\alpha=T_{\mathbb{K}/\Bbbk}(\alpha)\).
\end{problem}
\begin{solution}
Let \(p(x)\in \Bbbk[x]\) be the minimal polynomial of \(\alpha\) over \(\Bbbk\) and \(\deg p=d\). \(p(x)\) has \(d\) roots \(\alpha_1,\alpha_2,\ldots,\alpha_d\) where \(\alpha=\alpha_1\). Suppose \([\mathbb{K}:\Bbbk]=n\) and \(r:=\frac{n}{d}\). We prove 
\(\det A_\alpha\) and \(N_{\mathbb{K}/\Bbbk}(\alpha)\) are both equal to \((\alpha_1\alpha_2\cdots \alpha_d)^r\), and both \(\tr A_\alpha\) and \(T_{\mathbb{K}/\Bbbk}(\alpha) \) are equal to \(r(\alpha_1+\cdots +\alpha_d)\). 
\begin{enumerate}[(1)]
\item In this part we prove that 
\begin{align*}
    \det A_\alpha&=(\alpha_1\cdots \alpha_d)^r,\\
    \tr A_\alpha&=r(\alpha_1+\cdots+\alpha_d).
\end{align*}
Let \(\Bbbk(\alpha)\) be the splitting field of \(p\). Suppose 
\[p(x)=(x-\alpha_1)\cdots (x-\alpha_d)=x^d+a_{d-1}x^{d-1}+\cdots+a_0\] 
\(\Bbbk(\alpha)\) as a \(\Bbbk\)-vector space has a basis \(\left\{ 1,\alpha,\alpha^2,\ldots, \alpha^{d-1} \right\}\). The multiplication of \(\alpha\) in \(\Bbbk(\alpha)\) can be written as a matrix 
\[B=\begin{pmatrix}
   0&0&\cdots&0&-a_0\\ 
   1&0&\cdots&0&-a_1\\ 
   \vdots&\vdots&&\vdots&\vdots\\ 
   0&0&\cdots&1&-a_{d-1}
\end{pmatrix}\]
The determinant of this matrix \(B\) is \((-1)^{d-1}\cdot (-a_0)=(-1)^d a_0\) and the trace of this matrix \(B\) is \(-a_{d-1}\). Note that in \(p(x)\), we have 
\[(x-\alpha_1)\cdots (x-\alpha_d)=x^d+a_{d-1}x^{d-1}+\cdots+a_0\]
By comparing coefficients, we notice that 
\begin{align*}
    (-1)^d (\alpha_1\cdots \alpha_d)&=a_0,\\ 
    -(\alpha_1+\cdots+\alpha_d)&=a_{d-1}
\end{align*} 
This proves that \(\det B=\alpha_1\cdots \alpha_d\) and \(\tr B=\alpha_1+\cdots +\alpha_d\). Now consider the extension \(\mathbb{K}/\Bbbk(\alpha)/\Bbbk\), we have 
\[[\mathbb{K}:\Bbbk(\alpha)]=\frac{[\mathbb{K}:\Bbbk]}{[\Bbbk(\alpha):\Bbbk]}=\frac{n}{d}=r.\]
Choose \(\left\{ \beta_1,\ldots,\beta_r \right\}\) as a \(\Bbbk(\alpha)\)-basis of \(\mathbb{K}\). Then 
\begin{align*}
    \beta_1,\alpha\beta_1,&\ldots,\alpha^{d-1}\beta_1,\\ 
    \beta_2,\alpha\beta_2,&\ldots,\alpha^{d-1}\beta_2,\\ 
    &\cdots\\ 
    \beta_r,\alpha\beta_r,&\ldots,\alpha^{d-1}\beta_r.
\end{align*}
is a \(\Bbbk\)-basis for \(\mathbb{K}\). Note that multiplicating by \(\alpha\) only sends a base vector to linear combinations of the basis in the same row. So the matrix \(A_\alpha\) is a block matrix with \(r\) block each equal to \(B\). Thus, 
\begin{align*}
    \det A_\alpha&=(\det B)^r=(\alpha_1\cdots \alpha_d)^r,\\ 
    \tr A_\alpha&=r(\tr B)=r(\alpha_1+\cdots+\alpha_d).
\end{align*}
\item In this part we prove that 
\begin{align*}
    N_{\mathbb{K}/\Bbbk}(\alpha)&=(\alpha_1\cdots \alpha_d)^r,\\ 
    T_{\mathbb{K}/\Bbbk}(\alpha)&=r(\alpha_1+\cdots +\alpha_d).
\end{align*}
We know that \(\alpha\) is a root of the polynomial \(p(x)\in \Bbbk[x]\), for any \(\sigma\in G=\Gal(\mathbb{K}/\Bbbk)\), \(\sigma\) fixes \(p\), so \(\sigma(\alpha)=\alpha_i\) for some \(1\leq i\leq d\). \(\mathbb{K}/\Bbbk\) is a Galois extension, so 
there exists \(\sigma_i\in G\) such that \(\sigma_i(\alpha)=\alpha_i\) for all \(1\leq i\leq d\). Let \(G_\alpha=\Gal(\mathbb{K}/\Bbbk(\alpha))\). We have proved in Exercise 11.6.2 that \(\bigcup_{i=1}^d (\sigma_iG_\alpha)\) is a coset partition of \(G\) with respect to the subgroup \(G_\alpha\). Each coset has 
\(r\) elements and for any \(\tau\in \sigma_iG_\alpha\), \(\tau(\alpha)=\alpha_i\). Therefore, we have 
\begin{align*}
N_{\mathbb{K}/\Bbbk}(\alpha)&=\prod_{\sigma\in G}\sigma(\alpha)=(\prod_{i=1}^d \sigma_i(\alpha))^r=(\alpha_1\cdots \alpha_d)^r,\\ 
T_{\mathbb{K}/\Bbbk}(\alpha)&=\sum_{\sigma\in G}\sigma(\alpha)=r(\sum_{i=1}^d \sigma_i(\alpha))=r(\alpha+\cdots+\alpha_d).
\end{align*}

\end{enumerate} 
\end{solution}

\noindent\rule{7in}{2.8pt}
%%%%%%%%%%%%%%%%%%%%%%%%%%%%%%%%%%%%%%%%%%%%%%%%%%%%%%%%%%%%%%%%%%%%%%%%%%%%%%%%%%%%%%%%%%%%%%%%%%%%%%%%%%%%%%%%%%%%%%%%%
% Problem 12.4.11
%%%%%%%%%%%%%%%%%%%%%%%%%%%%%%%%%%%%%%%%%%%%%%%%%%%%%%%%%%%%%%%%%%%%%%%%%%%%%%%%%%%%%%%%%%%%%%%%%%%%%%%%%%%%%%%%%%%%%%%%%%
\begin{problem}{12.4.11}
Let \(a,b\in \mathbb{Q}\). 
\begin{enumerate}[(a)]
\item \(a^2+b^2=1\) is equivalent to \(N_{\mathbb{Q}(i)/\mathbb{Q}}(a+ib)=1\).
\item Use Hilbert's Theorem 90 to prove that the rational solutions of \(a^2+b^2=1\) are of the form \(a=(s^2-t^2)/(s^2+t^2)\) and \(b=2st/(s^2+t^2)\) for \(s,t\in \mathbb{Q}\).
\end{enumerate}
\end{problem}
\begin{solution}
\begin{enumerate}[(a)]
\item We have proved in Exercise 12.4.9 that \(N_{\mathbb{Q}(i)/\mathbb{Q}}(a+ib)=\det A_{a+ib}\) where \(A_{a+ib}\) is the matrix given by the multiplication \(x\mapsto (a+ib)x\) for all \(x\in \mathbb{Q}(i)\). Choose \(\left\{ 1,i \right\}\) as a \(\mathbb{Q}\)-basis for \(\mathbb{Q}(i)\) and the matrix \(A_{a+ib}\) can be written as 
\[A_{a+ib}=\begin{pmatrix}
   a & -b\\ 
   b & a
\end{pmatrix}.\]
By direct calculation, we know that 
\[N_{\mathbb{Q}(i)/\mathbb{Q}}(a+ib)=\det A_{a+ib}=a^2+b^2.\]
Therefore, \(a^2+b^2=1\) is equivalent to \(N_{\mathbb{Q}(i)/\mathbb{Q}}(a+ib)=1\).
\item From (a), we know that \(a^2+b^2=1\) has rational solutions if and only if \(N_{\mathbb{Q}(i)/\mathbb{Q}}(a+ib)=1\). We know that \(\mathbb{Q}(i)/\mathbb{Q}\) is a quadratic extension so the Galois group \(\Gal(\mathbb{Q}(i)/\mathbb{Q})=C^2\) generated by sending \(i\) to \(-i\). Choose \(\left\{ 1,i \right\}\) as a \(\mathbb{Q}\)-basis for \(\mathbb{Q}(i)\). By Hilbert's Theorem 90, there exists \(s+it\in \mathbb{Q}(i)\) for some \(t,s\in \mathbb{Q}\) and \(s^2+t^2\neq 0\) such that 
\[\frac{s+it}{s-it}=a+ib.\]
This is equivalent to 
\[\frac{(s^2-t^2+i(2st))}{s^2+t^2}=a+ib.\]
By comparing coefficients we know that \(a, b\) must have the form 
\begin{align*}
    a&=\frac{s^2-t^2}{s^2+t^2},\\[5pt]
    b&=\frac{2st}{s^2+t^2}.
\end{align*} 
\end{enumerate}
\end{solution}

\noindent\rule{7in}{2.8pt}
%%%%%%%%%%%%%%%%%%%%%%%%%%%%%%%%%%%%%%%%%%%%%%%%%%%%%%%%%%%%%%%%%%%%%%%%%%%%%%%%%%%%%%%%%%%%%%%%%%%%%%%%%%%%%%%%%%%%%%%%%
% Problem 13.2.9
%%%%%%%%%%%%%%%%%%%%%%%%%%%%%%%%%%%%%%%%%%%%%%%%%%%%%%%%%%%%%%%%%%%%%%%%%%%%%%%%%%%%%%%%%%%%%%%%%%%%%%%%%%%%%%%%%%%%%%%%%%
\begin{problem}{13.2.9}
True or false? Let \(\mathbb{K}/\mathbb{F}_q\) be a finite extension, and \(\mathbb{L}\), \(\mathbb{M}\) be two intermediate subfields. Then either \(\mathbb{L}\subseteq \mathbb{M}\) or \(\mathbb{M}\subseteq \mathbb{L}\). 
\end{problem}
\begin{solution}
This is true. Suppose \(q=p^d\) for some prime \(p\). Then the field extension \(\mathbb{K}/\mathbb{F}_q\) must be \(\mathbb{K}\cong \mathbb{F}_{p^n}\) for some \(n\) satisfying \(d|n\) by Corollary 13.2.8. We know the Galois group 
\(\Gal(\mathbb{K}/\mathbb{F}_q)\) is isomorphic to the cyclic group \(C_{n/d}\). By Galois correspondence, \(\mathbb{M}^*\) and \(\mathbb{L}^*\) are subgroups of \(C_{n/d}\). We know in cyclic groups, either \(\mathbb{M}^*\subseteq \mathbb{L}^*\) or \(\mathbb{L}^*\subseteq \mathbb{M}^*\). 
This implies \(\mathbb{L}\subseteq \mathbb{M}\) or \(\mathbb{L}\subseteq \mathbb{M}\).
\end{solution}

\noindent\rule{7in}{2.8pt}
%%%%%%%%%%%%%%%%%%%%%%%%%%%%%%%%%%%%%%%%%%%%%%%%%%%%%%%%%%%%%%%%%%%%%%%%%%%%%%%%%%%%%%%%%%%%%%%%%%%%%%%%%%%%%%%%%%%%%%%%%
% Problem 13.2.12
%%%%%%%%%%%%%%%%%%%%%%%%%%%%%%%%%%%%%%%%%%%%%%%%%%%%%%%%%%%%%%%%%%%%%%%%%%%%%%%%%%%%%%%%%%%%%%%%%%%%%%%%%%%%%%%%%%%%%%%%%%
\begin{problem}{13.2.12}
Let \(p\) be a prime. Then there are exactly \((q^p-q)/p\) monic irreducible polynomials of degree \(p\) in \(\mathbb{F}_q[x]\) (\(q\) is not necessarily a power of \(p\)).
\end{problem}
\begin{solution}
Let \(\mathbb{K}\) be a degree \(p\) extension of \(\mathbb{F}_q\). Then \(\mathbb{K}\) is a \(p\) dimensional \(\mathbb{F}_q\)-vector space, thus having \(q^p\) elements. By Theorem 13.2.3, \(\mathbb{K}\) is the splitting field of the polynomial \(x^{q^p}-x\). The field \(\mathbb{K}\) has exactly 
\(q^p\) elements, so \(x^{q^p}-x\) has \(q^p\) different roots in \(\mathbb{K}\). Let \(f\in \mathbb{F}_q[x]\) be a degree \(p\) irreducible polynomial. If \(\alpha\) is a root of \(f\), then \(\mathbb{F}_q(\alpha)/\mathbb{F}_q\) is a degree \(p\) extension and thus, \(\mathbb{F}_q(\alpha)\cong \mathbb{K}\) 
as a finite field extension. This means \(\alpha\) is also a root of the polynomial \(x^{q^p}-x\). Since every finite field is separable, every irreducible polynomial \(f\) contributes \(p\) different roots for the polynomial \(x^{q^p}-x\). Note that \([\mathbb{K}:\mathbb{F}_q]=p\) is a prime number, only \(1\) and \(p\) divides \([\mathbb{K}:\mathbb{F}_q]\), so 
the roots of \(x^{q^p}-x\) either coming from a degree \(p\) irreducible polynomial or coming from a degree \(1\) irreducible polynomial. We have \(q\) elements in \(\mathbb{F}_q\), which counts as \(q\) irreducible degree \(1\) polynomial. So the number of degree \(p\) polynomial over \(\mathbb{F}_q\) is equal to \(\frac{q^p-q}{p}\).  
\end{solution}

\noindent\rule{7in}{2.8pt}
%%%%%%%%%%%%%%%%%%%%%%%%%%%%%%%%%%%%%%%%%%%%%%%%%%%%%%%%%%%%%%%%%%%%%%%%%%%%%%%%%%%%%%%%%%%%%%%%%%%%%%%%%%%%%%%%%%%%%%%%%
% Problem 13.2.13
%%%%%%%%%%%%%%%%%%%%%%%%%%%%%%%%%%%%%%%%%%%%%%%%%%%%%%%%%%%%%%%%%%%%%%%%%%%%%%%%%%%%%%%%%%%%%%%%%%%%%%%%%%%%%%%%%%%%%%%%%%
\begin{problem}{13.2.13}
What is \(\sum_A A^{100}\), where the sum is over all \(17\times 17\) matrices \(A\) over \(\mathbb{F}_{17}\)?
\end{problem}
\begin{solution}
We know that \(\mathbb{F}_{17}^\times\) is a multiplicative group generated by \(a\) where \(a^{16}=1\).  We first prove a claim. 
\begin{claim}
The sum over all elements \(x\in \mathbb{F}_{17}\) is \(\sum_{x\in \mathbb{F}_{17}}x^{100}=0\).
\end{claim}
\begin{claimproof}
Write \(S=\sum_{x\in \mathbb{F}_{17}}x^{100}\). Consider \(a^{100}S\). Note that \(a\) acting by multiplication on the field 
\[\mathbb{F}_{17}=\left\{ 0,1,a,a^2,\ldots,a^{16} \right\}\]
is just a permutation of these elements. So we have 
\[a^{100}S=a^{100}\sum_{x\in \mathbb{F}_{17}}x^{100}=\sum_{x\in \mathbb{F}_{17}} (ax)^{100}=\sum_{x\in \mathbb{F}_{17}}x^{100}=S.\]
This implies \((a^{100}-1)S=0\) in the field \(\mathbb{F}_{17}\). Since \(16\nmid 100\), \(a^{100}-1\neq 0\). This implies \(S=0\).
\end{claimproof}

Now consider \(a^{100}\) acts on a matrix \(A\in M_{17}(\mathbb{F}_{17})\) by multiplication on each entry. We have 
\[a^{100}\sum_A A^{100}=\sum_A (aA)^{100}.\]
We claim the following map 
\begin{align*}
    m_a:M_{17}(\mathbb{F}_{17})&\rightarrow M_{17}(\mathbb{F}_{17}),\\ 
        A&\mapsto aA
\end{align*}
is a bijection. Indeed, since \(a\) is multiplicatively invertible in \(\mathbb{F}_{17}\), multiplying by \(\frac{1}{a}=a^{15}\) is the inverse map. So \(m_a\) is both injective and surjective. 
By the same argument as in the claim on each entry, we have \(\sum_A A^{100}=0\).
\end{solution}

\end{document}
\documentclass[letterpaper, 12pt]{article}

\usepackage{/Users/zhengz/Desktop/Math/Workspace/Homework1/homework}

\begin{document}
\noindent
\large\textbf{Zhengdong Zhang} \hfill \textbf{Homework - Week 9} \\
Email: zhengz@uoregon.edu \hfill ID: 952091294 \\
\normalsize Course: MATH 649 - Abstract Algebra \hfill Term: Spring 2025 \\
Instructor: Professor Sasha Polishchuk \hfill Due Date: $4^{th}$ June, 2025 \\
\noindent\rule{7in}{2.8pt}
\setstretch{1.1}
%%%%%%%%%%%%%%%%%%%%%%%%%%%%%%%%%%%%%%%%%%%%%%%%%%%%%%%%%%%%%%%%%%%%%%%%%%%%%%%%%%%%%%%%%%%%%%%%%%%%%%%%%%%%%%%%%%%%%%%%%
% Problem 21.2.2
%%%%%%%%%%%%%%%%%%%%%%%%%%%%%%%%%%%%%%%%%%%%%%%%%%%%%%%%%%%%%%%%%%%%%%%%%%%%%%%%%%%%%%%%%%%%%%%%%%%%%%%%%%%%%%%%%%%%%%%%%%
\begin{problem}{21.2.2}
Let \(I_1,\ldots,I_m\) be ideals in \(\mathbb{F}[T_1,\ldots,T_n]\). Then \(\mathcal{V}(I_1\cdots I_m)=\mathcal{V}(I_1\cap\cdots\cap I_m)\).
\end{problem}
\begin{solution}
We only need to prove the case \(m=2\), the rest can be obtained from induction. To prove \(\mathcal{V}(I_1I_2)=\mathcal{V}(I_1\cap I_2)\), by Corollary 21.1.10, it is the same as proving 
\[\sqrt{I_1I_2}=\sqrt{I_1\cap I_2}.\]
Suppose \(a\in \sqrt{I_1I_2}\), then there exists \(n\geq 1\) such that \(a^n\in I_1I_2\subseteq I_1\cap I_2\). This implies that \(a\in \sqrt{I_1\cap I_2}\). On the other hand, suppose \(b\in \sqrt{I_1\cap I_2}=\sqrt{I_1}\cap \sqrt{I_2}\), then there exists \(k,l\geq 1\) such that \(b^k\in I_1\) and \(b^l\in I_2\). This implies \(b^{k+l}=b^k\cdot b^l\in I_1I_2\), so \(b\in \sqrt{I_1I_2}\). This proves \(\sqrt{I_1I_2}=\sqrt{I_1\cap I_2}\). 
\end{solution}

\noindent\rule{7in}{2.8pt}
%%%%%%%%%%%%%%%%%%%%%%%%%%%%%%%%%%%%%%%%%%%%%%%%%%%%%%%%%%%%%%%%%%%%%%%%%%%%%%%%%%%%%%%%%%%%%%%%%%%%%%%%%%%%%%%%%%%%%%%%%
% Problem 21.2.3
%%%%%%%%%%%%%%%%%%%%%%%%%%%%%%%%%%%%%%%%%%%%%%%%%%%%%%%%%%%%%%%%%%%%%%%%%%%%%%%%%%%%%%%%%%%%%%%%%%%%%%%%%%%%%%%%%%%%%%%%%%
\begin{problem}{21.2.3}
Let \(f\in \mathbb{F}[T_1,\ldots,T_n]\). The corresponding \textit{principal open set} is 
\[\mathbb{A}^n\setminus \mathcal{V}(f)=\left\{ x\in \mathbb{A}^n\mid f(x)\neq 0 \right\}.\]
Show that each open set in \(\mathbb{A}^n\) is finite union of principal open sets, so principal open sets form a base of Zariski topology.
\end{problem}
\begin{solution}
We know that the Zariski closed sets of \(\mathbb{A}^n\) have the form \(\mathcal{V}(I)\) for some ideal \(I\subseteq \mathbb{F}[T_1,\ldots,T_n]\). So for any open set \(U\subseteq \mathbb{A}^n\), \(U\) can be written as \(U=\mathbb{A}^n-\mathcal{V}(I)\) for some radical ideal \(I\). Since \(\mathbb{F}[T_1,\ldots,T_n]\) is noetherian, \(I\) is finitely generated by \(f_1,\ldots,f_k\in \mathbb{F}[T_1,\ldots,T_n]\). This implies 
\[\mathcal{V}(I)=\mathcal{V}(f_1,\ldots,f_k)=\mathcal{V}(f_1)\cap\cdots\cap \mathcal{V}(f_k).\]
Thus, we can write \(U\) as 
\begin{align*}
    U&=\mathbb{A}^n-\mathcal{V}(I)\\ 
     &=\mathbb{A}^n-\mathcal{V}(f_1)\cap \cdots\cap \mathcal{V}(f_k)\\ 
     &=(\mathbb{A}^n-\mathcal{V}(f_1))\cup \cdots\cup (\mathbb{A}^n-\mathcal{V}(f_k)).
\end{align*}
This proves that any Zariski open set can be written as a finite union of principal open sets.
\end{solution}

\noindent\rule{7in}{2.8pt}
%%%%%%%%%%%%%%%%%%%%%%%%%%%%%%%%%%%%%%%%%%%%%%%%%%%%%%%%%%%%%%%%%%%%%%%%%%%%%%%%%%%%%%%%%%%%%%%%%%%%%%%%%%%%%%%%%%%%%%%%%
% Problem 21.2.13
%%%%%%%%%%%%%%%%%%%%%%%%%%%%%%%%%%%%%%%%%%%%%%%%%%%%%%%%%%%%%%%%%%%%%%%%%%%%%%%%%%%%%%%%%%%%%%%%%%%%%%%%%%%%%%%%%%%%%%%%%%
\begin{problem}{21.2.13}
Let \(X=\mathcal{V}(x^2+y^2+z^2,xyz)\subseteq \mathbb{A}^3\). Decompose \(X\) into irreducible components. 
\end{problem}
\begin{solution}
We need to find all the points \((x,y,z)\in \mathbb{A}^3\) satisfying \(x^2+y^2+z^2=0\) and \(xyz=0\). Since \(\mathbb{A}^3\) has no nilpotents, \(xyz=0\) implies at least one of the coordinates is \(0\). Suppose \(x=0\). The \(y\) and \(z\) satisfy the equation \(y^2+z^2=0\). Note that \(\mathbb{F}\) is algebraically closed, if \(\text{char}\ \mathbb{F}=2\), then \(y+z=0\). \(X\) has three irreducible components 
\[X=\mathcal{V}(x+y+z,xyz)=\mathcal{V}(x,y+z)\cup \mathcal{V}(y,x+z)\cup \mathcal{V}(z,x+y).\]
Each of them is isomorphic to \(\mathbb{A}^1\) because the coordinate ring 
\[\mathbb{F}[x,y,z]/(x,y+z)\cong \mathbb{F}[y,-y]\cong \mathbb{F}[y].\]
Next, assume \(\text{char}\ \mathbb{F}\neq 2\), then \(y^2+z^2=(y+iz)(y-iz)=0\). This is the union of two algebraic sets \(\mathcal{V}(y+iz)\) and \(\mathcal{V}(y-iz)\). Thus, \(X\) has six irreducible components 
\[X=\mathcal{V}(x,y+iz)\cup \mathcal{V}(x,y-iz)\cup \mathcal{V}(y,x+iz)\cup \mathcal{V}(y,x-iz)\cup \mathcal{V}(z,x+iy)\cup \mathcal{V}(z,x-iy).\]
Each of them is isomorphic to \(\mathbb{A}^1\) because the coordinate ring 
\[\mathbb{F}[x,y,z]/(x,y-iz)\cong \mathbb{F}[z,iz]\cong \mathbb{F}[z].\]
\end{solution}

\noindent\rule{7in}{2.8pt}
%%%%%%%%%%%%%%%%%%%%%%%%%%%%%%%%%%%%%%%%%%%%%%%%%%%%%%%%%%%%%%%%%%%%%%%%%%%%%%%%%%%%%%%%%%%%%%%%%%%%%%%%%%%%%%%%%%%%%%%%%
% Problem 21.2.14
%%%%%%%%%%%%%%%%%%%%%%%%%%%%%%%%%%%%%%%%%%%%%%%%%%%%%%%%%%%%%%%%%%%%%%%%%%%%%%%%%%%%%%%%%%%%%%%%%%%%%%%%%%%%%%%%%%%%%%%%%%
\begin{problem}{21.2.14}
Let \(\text{char}\, \mathbb{F}\neq 2\). Decompose \(\mathcal{V}(x^2+y^2+z^2,x^2-y^2-z^2+1)\) into irreducible components.
\end{problem}
\begin{solution}
We need to find all the pointd \((x,y,z)\in \mathbb{A}^3\) satisfying \(x^2+y^2+z^2=0\) and \(x^2-y^2-z^2+1=0\). From these two equations, we obtain 
\[0=2x^2+1.\]
We know \(\text{char}\, \mathbb{F}\neq 2\). So this equation has two different solutions: \(x=\frac{i}{\sqrt{2}}\) and \(x=\frac{-i}{\sqrt{2}}\). When \(x=\frac{i}{\sqrt{2}}\), \(y\) and \(z\) satisfy the equation \(y^2+z^2=\frac{1}{2}\). This is a hyperbola and \((y^2+z^2-\frac{1}{2})\) is a prime ideal in \(\mathbb{F}[x,y]\) since we proved in Exercise 21.4.14 that 
\[\mathbb{F}[y,z]/(y^2+z^2-1)\cong \mathbb{F}[u,v]/(uv-1)\cong \mathbb{F}[u,u^{-1}].\]
Thus, \(X\) has two irreducible components
\[X=\mathcal{V}(x-\frac{i}{\sqrt{2}},y^2+z^2-\frac{1}{2})\cup \mathcal{V}(x+\frac{i}{\sqrt{2}}, y^2+z^2-\frac{1}{2}).\]
\end{solution}

\noindent\rule{7in}{2.8pt}
%%%%%%%%%%%%%%%%%%%%%%%%%%%%%%%%%%%%%%%%%%%%%%%%%%%%%%%%%%%%%%%%%%%%%%%%%%%%%%%%%%%%%%%%%%%%%%%%%%%%%%%%%%%%%%%%%%%%%%%%%
% Problem 21.3.4
%%%%%%%%%%%%%%%%%%%%%%%%%%%%%%%%%%%%%%%%%%%%%%%%%%%%%%%%%%%%%%%%%%%%%%%%%%%%%%%%%%%%%%%%%%%%%%%%%%%%%%%%%%%%%%%%%%%%%%%%%%
\begin{problem}{21.3.4}
If \(f:A\rightarrow B\) is a homomorphism of affine algebras and \(M\) is a maximal ideal of \(B\), then \(f^{-1}(M)\) is a maximal ideal of \(A\).
\end{problem}
\begin{solution}
\(A,B\) are finitely generated \(\mathbb{F}\)-algebras, so \(B/M\) is also a finitely generated \(\mathbb{F}\)-algebra. We have a map 
\begin{align*}
    \phi: A/f^{-1}(M)&\rightarrow B/M,\\ 
          a+f^{-1}(M)&\mapsto f(a)+M.
\end{align*}
This is a well-defined \(\mathbb{F}\)-algebra homomorphism. Indeed, suppose \(a,b\in A\) and \(a-b\in f^{-1}(M)\). This means \(f(a-b)=f(a)-f(b)\in M\), so \(f(a)+M=f(b)+M\) is the same element in \(B/M\). Moreover, \(\phi\) is injective. Let \(a+f^{-1}(M)\in \ker \phi\) and assume \(f(a)+M=M\), namely, \(f(a)\in M\). Then \(a\in f^{-1}(M)\) and \(a+f^{-1}(M)=f^{-1}(M)\) is trivial in \(A/f^{-1}(M)\). 

\(M\) is a maximal ideal, so \(B/M\) is a field and is a finitely generator \(\mathbb{F}\)-algebra. By the first version of Nullstellensatz we proved in class, \(\mathbb{F}\subseteq B/M\) is an algebraic and finite extension. We know that \(A/f^{-1}(M)\) is a domain as \(f^{-1}(M)\) is a prime ideal in \(A\), so we have 
\[\mathbb{F}\subseteq A/f^{-1}(M)\subseteq B/M\]
and \(A/f^{-1}(M)\) is a subring of \(B/M\). By Exercise 10.1.11, \(A/f^{-1}(M)\) is a field, thus \(f^{-1}(M)\) is a maximal ideal in \(A\). 
\end{solution}

\noindent\rule{7in}{2.8pt}
%%%%%%%%%%%%%%%%%%%%%%%%%%%%%%%%%%%%%%%%%%%%%%%%%%%%%%%%%%%%%%%%%%%%%%%%%%%%%%%%%%%%%%%%%%%%%%%%%%%%%%%%%%%%%%%%%%%%%%%%%
% Problem 21.4.6
%%%%%%%%%%%%%%%%%%%%%%%%%%%%%%%%%%%%%%%%%%%%%%%%%%%%%%%%%%%%%%%%%%%%%%%%%%%%%%%%%%%%%%%%%%%%%%%%%%%%%%%%%%%%%%%%%%%%%%%%%%
\begin{problem}{21.4.6}
The hyperbola \(xy=1\) and \(\mathbb{A}^1\) are not isomorphic.
\end{problem}
\begin{solution}
The coordinate ring of the hyperbola \(xy=1\) is 
\[\mathbb{F}[x,y]/(xy-1)\cong \mathbb{F}[x,x^{-1}]\cong \mathbb{F}[x]_x.\]
Here, \(\mathbb{F}[x]_x\) is \(\mathbb{F}[x]\) localized with respect to the multiplicative set \(\left\{ 1,x,x^2,\ldots \right\}\). On the other hand, the coordinate ring of \(\mathbb{A}^1\) is \(\mathbb{F}[x]\). The two rings \(\mathbb{F}[x]_x\) and \(\mathbb{F}[x]\) are not isomorphic as \(\mathbb{F}[x,x^{-1}]=\mathbb{F}[x]_x\). Suppose 
\[\phi:\mathbb{F}[x]_x\rightarrow \mathbb{F}[y]\]
is a map of \(\mathbb{F}\)-algebras. We know that the units must be sent to units in \(\mathbb{F}[y]\), and \(x\) is a unit in \(\mathbb{F}[x,x^{-1}]\), so \(x\) must be sent to some element in \(\mathbb{F}\). Note that \(x\) generateds \(\mathbb{F}[x]\) as an \(\mathbb{F}\)-algebra, so there is no element which can be sent to \(y\). This implies \(\phi\) can never be surjective, so we do not have such isomorphism. This implies that \(xy=1\) and \(\mathbb{A}^1\) are not isomorphic as they have different coordinate rings.
\end{solution}

\noindent\rule{7in}{2.8pt}
%%%%%%%%%%%%%%%%%%%%%%%%%%%%%%%%%%%%%%%%%%%%%%%%%%%%%%%%%%%%%%%%%%%%%%%%%%%%%%%%%%%%%%%%%%%%%%%%%%%%%%%%%%%%%%%%%%%%%%%%%
% Problem 21.4.14
%%%%%%%%%%%%%%%%%%%%%%%%%%%%%%%%%%%%%%%%%%%%%%%%%%%%%%%%%%%%%%%%%%%%%%%%%%%%%%%%%%%%%%%%%%%%%%%%%%%%%%%%%%%%%%%%%%%%%%%%%%
\begin{problem}{21.4.14}
The circle \(x^2+y^2=1\) and \(\mathbb{A}^1\) are isomorphic if and only if \(\text{char}\, \mathbb{F}=2\).
\end{problem}
\begin{solution}
Suppose \(\text{char}\ \mathbb{F}=2\). The radical ideal of \((x^2+y^2-1)\) is \((x+y-1)\). The coordinate ring 
\[\mathbb{F}[x,y]/(x^2+y^2-1)\cong \mathbb{F}[x,y]/(x+y-1)\cong \mathbb{F}[t]\]
if we consider the isomorphism 
\begin{align*}
    \mathbb{F}[x,y]/(x+y-1)&\rightarrow \mathbb{F}[t],\\ 
    x&\mapsto t,\\ 
    y&\mapsto t+1.
\end{align*}
This proves the circle \(x^2+y^2=1\) is isomorphic to \(\mathbb{A}^1\) if \(\text{char}\ \mathbb{F}=2\). 

Suppose \(\text{char}\, \mathbb{F}\neq 2\). Then consider the following map 
\begin{align*}
    \phi:\mathbb{F}[u,v]/(uv-1)&\rightarrow \mathbb{F}[x,y]/(x^2+y^2-1),\\ 
    u&\mapsto x+iy,\\ 
    v&\mapsto x-iy. 
\end{align*}
This map is a regular map since it is given by a polynomial in \(y\) and \(x\). It is an isomorphism because it has an inverse 
\begin{align*}
    \phi^{-1}:\mathbb{F}[x,y]/(x^2+y^2-1)&\rightarrow \mathbb{F}[u,v]/(uv-1),\\ 
    x&\mapsto \frac{1}{2}u+\frac{1}{2}v,\\ 
    y&\mapsto \frac{-i}{2}u+\frac{i}{2}v.
\end{align*}
This implies that the circle \(x^2+y^2=1\) is isomorphic to the hyperbola \(uv=1\), and we have proved in Exercise 21.4.6 that the hyperbola \(uv=1\) is not isomorphic to \(\mathbb{A}^1\). So the circle \(x^2+y^2=1\) is not isomorphic to \(\mathbb{A}^1\) when \(\text{char}\, \mathbb{F}\neq 2\). 
\end{solution}

\end{document}
\documentclass[a4paper, 12pt]{article}

\usepackage{/Users/zhengz/Desktop/Math/Workspace/Homework1/homework}
%%%%%%%%%%%%%%%%%%%%%%%%%%%%%%%%%%%%%%%%%%%%%%%%%%%%%%%%%%%%%%%%%%%%%%%%%%%%%%%%%%%%%%%%%%%%%%%%%%%%%%%%%%%%%%%%%%%%%%%%%%%%%%%%%%%%%%%%
\begin{document}
%Header-Make sure you update this information!!!!
\noindent
%%%%%%%%%%%%%%%%%%%%%%%%%%%%%%%%%%%%%%%%%%%%%%%%%%%%%%%%%%%%%%%%%%%%%%%%%%%%%%%%%%%%%%%%%%%%%%%%%%%%%%%%%%%%%%%%%%%%%%%%%%%%%%%%%%%%%%%%
\large\textbf{Zhengdong Zhang} \hfill \textbf{Homework - Week 3}   \\
Email: zhengz@uoregon.edu \hfill ID: 952091294 \\
\normalsize Course: MATH 648 - Abstract Algebra  \hfill Term: Winter 2025\\
Instructor: Professor Arkady Berenstein \hfill Due Date: $29^{th}$ January, 2025 \\
\noindent\rule{7in}{2.8pt}
\setstretch{1.1}
%%%%%%%%%%%%%%%%%%%%%%%%%%%%%%%%%%%%%%%%%%%%%%%%%%%%%%%%%%%%%%%%%%%%%%%%%%%%%%%%%%%%%%%%%%%%%%%%%%%%%%%%%%%%%%%%%%%%%%%%%%%%%%%%%%%%%%%%
% Exercise 15.1.5
%%%%%%%%%%%%%%%%%%%%%%%%%%%%%%%%%%%%%%%%%%%%%%%%%%%%%%%%%%%%%%%%%%%%%%%%%%%%%%%%%%%%%%%%%%%%%%%%%%%%%%%%%%%%%%%%%%%%%%%%%%%%%%%%%%%%%%%%
\begin{problem}{15.1.5}
Calculate the invariant factors of the following matrices, working over the ring \(\mathbb{Z}[i]\) of Gaussian integers:
\begin{multicols}{2}
\begin{enumerate}[(a)]
\item \(\begin{pmatrix}
    1&0&0\\ 
    0&1+i&0\\ 
    0&0&2+i
\end{pmatrix};\)
\item \(\begin{pmatrix}
    2i&i&2+i\\ 
    i-1&1+i&0\\ 
    0&0&2+i\\ 
    1+i&-1&2+i
\end{pmatrix}\)
\end{enumerate}
\end{multicols}
\end{problem}
\begin{solution}
\begin{enumerate}[(a)]
\item Note that we have 
\[\begin{pmatrix}
	1&0&0\\ 
	0&-1&1\\ 
	0&-2-i&1+i
\end{pmatrix}\begin{pmatrix}
	1&0&0\\ 
	0&1+i&0\\ 
	0&0&2+i&
\end{pmatrix}\begin{pmatrix}
	1&0&0\\ 
	0&1&-2-i\\ 
	0&1&-1-i
\end{pmatrix}=\begin{pmatrix}
	1&0&0\\ 
	0&1&0\\ 
	0&0&1+3i
\end{pmatrix}\]
where both 
\[\begin{vmatrix}
	1&0&0\\ 
	0&-1&1\\ 
	0&-2-i&1+i
\end{vmatrix}=1\,\,\, \text{and}\,\,\, \begin{vmatrix}
	1&0&0\\ 
	0&1&-2-i\\ 
	0&1&-1-i
\end{vmatrix}=1\]
So the invariant factors of this matrix is \((1,1,1+3i)\).
\item Note that we have 
\[\begin{pmatrix}
	0&0&0&1\\ 
	0&-i&-2+i&1-i\\ 
	0&-2-i&6i&-1-3i\\ 
	1&-1&0&-1
\end{pmatrix}\begin{pmatrix}
	2i&i&2+i\\ 
	i-1&1+i&0\\
	0&0&2+i\\ 
	1+i&-1&2+i
\end{pmatrix}\begin{pmatrix}
	0&1&-2-i\\ 
	-1&3+2i&-6-8i\\ 
	0&1&-3-i
\end{pmatrix}=\begin{pmatrix}
	1&0&0\\ 
	0&1&0\\ 
	0&0&5-5i\\ 
	0&0&0
\end{pmatrix}\]
where \(\begin{vmatrix}
	0&0&0&1\\ 
	0&-i&-2+i&1-i\\ 
	0&-2-i&6i&-1-3i\\ 
	1&-1&0&-1
\end{vmatrix}=-1\) and \(\begin{vmatrix}
	0&1&-2-i\\ 
	-1&3+2i&-6-8i\\ 
	0&1&-3-i
\end{vmatrix}=-1\) are invertible in \(\mathbb{Z}[i]\). So the invariant factors are \((1,1,5-5i)\).
\end{enumerate} 
\end{solution}

\noindent\rule{7in}{2.8pt}
%%%%%%%%%%%%%%%%%%%%%%%%%%%%%%%%%%%%%%%%%%%%%%%%%%%%%%%%%%%%%%%%%%%%%%%%%%%%%%%%%%%%%%%%%%%%%%%%%%%%%%%%%%%%%%%%%%%%%%%%%%%%%%%%%%%%%%%%
% Exercise 15.1.6
%%%%%%%%%%%%%%%%%%%%%%%%%%%%%%%%%%%%%%%%%%%%%%%%%%%%%%%%%%%%%%%%%%%%%%%%%%%%%%%%%%%%%%%%%%%%%%%%%%%%%%%%%%%%%%%%%%%%%%%%%%%%%%%%%%%%%%%%
\begin{problem}{15.1.6}
Let \(R=\mathbb{C}[[x]]\), the ring of formal power series over \(\mathbb{C}\). Consider the submodule \(W\) of the free module \(V=Rv_1\oplus Rv_2\) generated by 
\[(1-x)^{-1}v_1+(1-x^2)^{-1}v_2\ \ \text{and}\ \ (1+x)^{-1}v_1+(1+x^2){-1}v_2\]
Find a basis \(\left\{ v_1',v_2' \right\}\) of \(V\) and elements \(\delta_1\mid \delta_2\in R\) such that \(W\) is generated by \(\delta_1v_1'\) and \(\delta_2v_2'\). Describe \(V/W\).
\end{problem}
\begin{solution}
Assume \(W\) is the row spaces spanned by the following matrix 
\[A\begin{pmatrix}
v_1\\ 
v_2
\end{pmatrix}=\begin{pmatrix}
	\frac{1}{1-x}&\frac{1}{1-x^2}\\ 
	\frac{1}{1+x}&\frac{1}{1+x^2}
\end{pmatrix}\begin{pmatrix}
	v_1\\ 
	v_2
\end{pmatrix}\]
Note that 
\[\dfrac{\frac{1}{1+x}}{\frac{1}{1-x}}=\frac{1-x}{1+x}=1-\frac{2x}{1+x}\in \mathbb{C}[[x]]\]
since 
\[\frac{2x}{1+x}=2x-2x^2+2x^3-2x^4+\cdots \in \mathbb{C}[[x]].\]
So we have 
\[\begin{pmatrix}
1&0\\ 
\frac{x-1}{x+1}&1
\end{pmatrix}A=\begin{pmatrix}
	\frac{1}{1-x}&\frac{1}{1-x^2}\\ 
	0&\frac{1}{1+x^2}-\frac{1}{(1+x)^2}
\end{pmatrix}\]
Define 
\begin{align*}
	v_1'&=v_1+\frac{1}{1+x}v_2\\
	v_2'&=v_2 
\end{align*}
Let \(\delta_1=\frac{1}{1-x}\) and \(\delta_2=\frac{1}{1+x^2}-\frac{1}{(1+x)^2}\). Note that \(W\) is generated by \(\delta_1v_1'\) and \(\delta_2v_2'\). Moreover, we have 
\(\delta_1\mid \delta_2\) since 
\[\dfrac{\frac{1}{1+x^2}-\frac{1}{(1+x)^2}}{\frac{1}{1-x}}=\frac{1-x}{1+x^2}-\frac{1-x}{(1+x)^2}\in \mathbb{C}[[x]].\]
So 
\[V/W\cong R/(\frac{1}{1-x})\oplus R/(\frac{1}{1+x^2}-\frac{1}{(1+x)^2})\]
Note that \(\frac{1}{1-x}\) is invertible in \(\mathbb{C}[[x]]\) and 
\[\frac{1}{1+x^2}-\frac{1}{(1+x)^2}=x\cdot \frac{x}{(1+x^2)(1+x)^2}\]
where \(\frac{x}{(1+x^2)(1+x)^2}\) is also invertible in \(R\). So \(V/W\cong (0)\oplus R/(x)\cong \mathbb{C}\).
\end{solution}

\noindent\rule{7in}{2.8pt}
%%%%%%%%%%%%%%%%%%%%%%%%%%%%%%%%%%%%%%%%%%%%%%%%%%%%%%%%%%%%%%%%%%%%%%%%%%%%%%%%%%%%%%%%%%%%%%%%%%%%%%%%%%%%%%%%%%%%%%%%%%%%%%%%%%%%%%%%
% Exercise 15.1.8
%%%%%%%%%%%%%%%%%%%%%%%%%%%%%%%%%%%%%%%%%%%%%%%%%%%%%%%%%%%%%%%%%%%%%%%%%%%%%%%%%%%%%%%%%%%%%%%%%%%%%%%%%%%%%%%%%%%%%%%%%%%%%%%%%%%%%%%%
\newpage 
\begin{problem}{15.1.8}
If \(R\) is a PID, then 
\[\hom_R(R/(a),R/(b))\cong R/(\text{gcd}(a,b)).\]
\end{problem}
\begin{solution}
Suppose \(a=p_1^{n_1}\cdots p_k^{n_k}\) is a prime factorization in \(R\) where \(p_1,p_2,\ldots,p_k\) are distinct irreducible elements in \(R\), by Lemma 15.1.7, we have 
\[R/(a)\cong R/(p_1^{n_1})\oplus \cdots \oplus R/(p_k^{n_k}).\]
We can write 
\[\hom (R/(a),R/(b))\cong \hom (R/(p_1^{n_1})\oplus \cdots\oplus R/(p_k^{n_k}), R/(b))\cong \oplus_{i=1}^k\hom (R/(p_i)^{n_i},R/(b)).\]
Now Suppose \(R/(b)\cong \oplus_{j=1}^l R/(q_i^{m_l})\) where \(q_1,\ldots,q_l\) are distinct primes in \(R\). 
\begin{claim}
Let \(\phi:R/(p^n)\rightarrow R/(q^m)\) be a \(R\)-module homomorphism. If \(p\neq q\) are distinct primes in \(R\), then \(\phi=0\).
\end{claim}
\begin{claimproof}
Suppose \(\phi(1+(p^n))=k+(q^m)\in R/(q^m)\). Since \(p\neq q\) are different primes in \(R\), \(p^n\) and \(q^m\) are coprime to each other. There exists \(s,t\in \mathbb{Z}\) such that 
\(sp^n+tq^m=1\). Now we have 
\[k+(q^m)=\phi(1+(p^n))=\phi(tq^m+sp^n+(p^n))=\phi(tq^m+(p^n))=q^m\phi(t+(p^n))=(q^m).\]
This implies \(\phi=0\). 
\end{claimproof}

Now assume \(n,m\) be positive integers. We need to determine \(\hom (R/(p^n),R/(p^m))\). Let \(\phi\in \hom(R/(p^n),R/(p^m))\) and assume \(\phi(1+(p^n))=k+(p^m)\). Note that \(\phi\) is completely determined by \(k+(p^m)\). 
If \(n\geq m\), consider the map 
\begin{align*}
	T:\hom (R/(p^n),R/(p^m))&\rightarrow R/(p^m),\\ 
	                    k+(p^m)&\mapsto k+(p^m). 
\end{align*}
It is easy to see that this map is both injective and surjective. On the other hand, if \(n\leq m\), note that 
\[p^n\phi(1+(p^n))=\phi((p^n))=(p^m)=p^nk+(p^m).\]
So there exists \(r\in R\) such that \(k=rp^{m-n}\). 
Consider the map 
\begin{align*}
	T:\hom (R/(p^n),R/(p^m))&\rightarrow R/(p^n),\\ 
	                    k+(p^m)&\mapsto r+(p^n).
\end{align*}
This map is a well-defined \(R\)-module homomorphism. Suppose \(k_1,k_2\) are two different representatives for the ideal \(k_1+(p^m)\), then \(k_1-k_2\in (p^m)\), this implies there exists \(r_3\in R\) such that 
\(r_1p^{m-n}-r_2p^{m-n}=r_3p^m\), namely \(r_1-r_2=r_3p^n\in (p^n)\). Moreover, suppose \(k+(p^m)\in \ker T\), then \(r\in (p^n)\), so \(k=rp^{m-n}\in (p^m)\). This implies \(k+(p^m)=(p^m)\). So \(T\) is injective. For 
any \(r+(p^n)\), consider the \(k=rp^{m-n}\), the map \(1+(p^n)\mapsto k+(p^m)\) defines an element in \(\hom (R/(p^n).R/(p^m))\). So \(T\) is also surjective. Thus, 
\[\hom(R/(p^n),R/(p^m))\cong R/(p^{\min(n,m) })\]
we can conclude that 
\[\hom(R/(a),R/(b))\cong R/(\text{gcd(a,b)}).\]
\end{solution}

\noindent\rule{7in}{2.8pt}
%%%%%%%%%%%%%%%%%%%%%%%%%%%%%%%%%%%%%%%%%%%%%%%%%%%%%%%%%%%%%%%%%%%%%%%%%%%%%%%%%%%%%%%%%%%%%%%%%%%%%%%%%%%%%%%%%%%%%%%%%%%%%%%%%%%%%%%%
% Exercise 15.1.12
%%%%%%%%%%%%%%%%%%%%%%%%%%%%%%%%%%%%%%%%%%%%%%%%%%%%%%%%%%%%%%%%%%%%%%%%%%%%%%%%%%%%%%%%%%%%%%%%%%%%%%%%%%%%%%%%%%%%%%%%%%%%%%%%%%%%%%%%
\begin{problem}{15.1.12}
True or false? If \(R\) is a PID and \(V\) is a finitely generated \(R\)-module with invariant factors \(\delta_1\mid \delta_2\mid \cdots\mid \delta_k\), then 
\(V\) cannot be generated by fewer than \(k\) elements.
\end{problem}
\begin{solution}
This is true. Assume \(V\) can be generated by \(l\) elements \(v_1,\ldots,v_l\) such that \(l<k\). Then \(V\) can be written as \(V=Rv_1+\cdots+Rv_l\). There exists a surjective 
\(R\)-module homomorphism \(\theta:R^{\oplus l}=F\twoheadrightarrow V\) where \(F=R^{\oplus l}=Rv_1\oplus \cdots\oplus Rv_l\).If \(\ker \theta=0\), then \(V\) is free and by theorem 14.3.7, the basis for \(V\) has the same cardinality, so \(l=k\). Suppose \(\ker \theta\neq 0\). 
We know \(V\cong F/\ker \theta\). By Corollary 15.1.4, there exists a basis \(\left\{ e_1,\ldots,e_l \right\}\) of \(F\) and non-zero elements \(d_1\mid d_2\mid\cdots\mid d_p\) of \(R\) such that 
\[\ker\theta=Rd_1e_1\oplus \cdots\oplus Rd_pe_p\]
for some \(p\leq l\). Therefore, we have 
\[V\cong F/\ker \theta=R/(d_1)\oplus \cdots\oplus R/(d_p).\]
By Theorem 15.1.10 and Lemma 15.1.9, \(l=k\). So \(V\) cannot be generated by fewer than \(k\) elements. 
\end{solution}

\noindent\rule{7in}{2.8pt}
%%%%%%%%%%%%%%%%%%%%%%%%%%%%%%%%%%%%%%%%%%%%%%%%%%%%%%%%%%%%%%%%%%%%%%%%%%%%%%%%%%%%%%%%%%%%%%%%%%%%%%%%%%%%%%%%%%%%%%%%%%%%%%%%%%%%%%%%
% Exercise 15.1.19
%%%%%%%%%%%%%%%%%%%%%%%%%%%%%%%%%%%%%%%%%%%%%%%%%%%%%%%%%%%%%%%%%%%%%%%%%%%%%%%%%%%%%%%%%%%%%%%%%%%%%%%%%%%%%%%%%%%%%%%%%%%%%%%%%%%%%%%%
\begin{problem}{15.1.19}
Find the isomorphism classes of abelian groups of order 108 having exactly 4 subgroups of order \(6\).
\end{problem}
\begin{solution}
Note that \(108=2^2\cdot 3^3\). Suppose \(G\) is an abelian group of order \(108\) and \(H\) is a subgroup of order \(6\). \(H\) must be a product of cyclic groups so \(H\cong C_2\times C_3\). By Theorem 15.1.17, we know 
\(G\) must be product of abelian groups whose order is a prime power. We know \(C_2\times C_2\) has three distinct subgroup of order \(2\), generated by \((1,0),(0,1),(1,1)\). So \(G\) cannot contain a subgroup isomorphic to 
\(C_2\times C_2\). This means \(G\) must contain a subgroup isomorphic to \(C_4\), which has exactly \(1\) subgroup isomorphic to \(C_2\) generated by \(2\). This means \(G/C_4\) must contain exactly 4 distinct subgroup of order \(3\). 
We know \(C_{27}\) has exactly one subgroups of order \(3\) generated by \(9\). And \(C_3\times C_3\times C_3\) has at least five distinct subgroup of order 3, generated relatively by \((1,0,0),(0,1,0),(0,0,1),(1,1,1),(1,1,0)\). Finally, \(C_3\times C_9\) 
has exactly 4 distinct subgroups of order 3, generated relatively by \((1,0),(0,3),(1,3),(2,3)\). So \(G\) has to be isomorphic to \(C_4\times C_3\times C_9\).
\end{solution}

\noindent\rule{7in}{2.8pt}
%%%%%%%%%%%%%%%%%%%%%%%%%%%%%%%%%%%%%%%%%%%%%%%%%%%%%%%%%%%%%%%%%%%%%%%%%%%%%%%%%%%%%%%%%%%%%%%%%%%%%%%%%%%%%%%%%%%%%%%%%%%%%%%%%%%%%%%%
% Exercise 15.2.1
%%%%%%%%%%%%%%%%%%%%%%%%%%%%%%%%%%%%%%%%%%%%%%%%%%%%%%%%%%%%%%%%%%%%%%%%%%%%%%%%%%%%%%%%%%%%%%%%%%%%%%%%%%%%%%%%%%%%%%%%%%%%%%%%%%%%%%%%
\begin{problem}{15.2.1}
Let \(\delta_1,\ldots,\delta_k\) be the invariant factors of a linear transformation \(\phi\) on a finite dimensional vector space \(V\). Then \(\delta_k\) is the minimal polynomial 
of \(\phi\).
\end{problem}
\begin{solution}
By Theorem 15.1.10, \(V\) can be decomposed into \(V_\phi=V_1\oplus \cdots \oplus V_k\) where each \(V_i\) is isomorphic to \(\mathbb{F}[x]/(\delta_i)\) for \(1\leq i\leq k\). This means that for any \(v\in V\), \(v\) can be written as 
\(v=f_1v_1+\cdots +f_kv_k\) for some \(f_i\in \mathbb{F}[x]\) and \(v_i\in V_i\), \(1\leq i\leq k\). For each \(i\), The annihilator \(\Ann(Vi)\subset \mathbb{F}[x]\) is the ideal \((\delta_i)\). And since \(\delta_1\mid \delta_2\mid \cdots\mid \delta_k\), we have 
\(\delta_k\in (\delta_i)\) for all \(1\leq i\leq k\). This is just saying \(\delta_k\cdot v_i=\delta_k(\phi)(v_i)=0\) for any \(1\leq i\leq k\). Thus, \(\delta_k\in \Ann(V)\). Suppose \(\Ann(V)\) is generated by \(p\in \mathbb{F}[x]\) with \(\deg p<\deg \delta_k\), then 
\(p\) annihilating all \(v\in V\) means that \(p\in (\delta_i)\) for all \(1\leq i\leq k\), so \(R/(\delta_1)\oplus \cdots \oplus R/(\delta_{k-1})\oplus R/(p)\) is another decomposition of \(V\) and by Lemma 15.1.9, \(p\) and \(\delta_k\) only differ by a unit in 
\(\mathbb{F}[x]\). We know \(\mathbb{F}[x]^{\times}=\mathbb{F}^{\times}\), so \(\deg \delta_k=\deg p\). A contradiction. This shows that \(\delta_k\) is the minimal polynomial of \(\phi\).
\end{solution}

\noindent\rule{7in}{2.8pt}
%%%%%%%%%%%%%%%%%%%%%%%%%%%%%%%%%%%%%%%%%%%%%%%%%%%%%%%%%%%%%%%%%%%%%%%%%%%%%%%%%%%%%%%%%%%%%%%%%%%%%%%%%%%%%%%%%%%%%%%%%%%%%%%%%%%%%%%%
% Exercise 15.2.2
%%%%%%%%%%%%%%%%%%%%%%%%%%%%%%%%%%%%%%%%%%%%%%%%%%%%%%%%%%%%%%%%%%%%%%%%%%%%%%%%%%%%%%%%%%%%%%%%%%%%%%%%%%%%%%%%%%%%%%%%%%%%%%%%%%%%%%%%
\begin{problem}{15.2.2}
Let \(f\in \mathbb{F}[x]\) be a monic polynomial of degree \(d>0\). \(I_d\in M_d(\mathbb{F})\) be the identity matrix, and consider \(xI_d-\text{Com}(f)\) as a matrix over \(\mathbb{F}[x]\). Then the 
invariant factors of \(xI_d-\text{Com}(f)\) are \(1,\ldots,1,f(x)\) with \(1\) appearing \(d-1\) times.    
\end{problem}
\begin{solution}
Suppose the monic polynomial 
\[f(x)=x^d+a_{d-1}x^{d-1}+\cdots+a_0\]
and write \(A=xI_d-\Com(f)\). By defintion, we have
\[A=\begin{pmatrix}
	x&0&\cdots&0&a_0\\ 
	-1&x&\cdots&0&a_1\\ 
	0&-1&\cdots&0&a_2\\ 
	\vdots&\vdots&&\vdots&\vdots&\\ 
	0&0&\cdots&x&a_{d-2}\\ 
	0&0&\cdots&-1&x+a_{d-1}
\end{pmatrix}\] 
Let \(p\in \mathbb{F}[x]\) and \(L_{i,j}(p)=I_d+pe_{ij}\) where \(e_{ij}\) is a matrix with all zero entries except its \((i,j)\)-entry is \(1\). Note that \(\det L_{i,j}(p)=1\), so it is always 
invertible. Left multiply \(L_{i,j}(p)\) is equivalent to the elemenary transformation that multiply \(j\)th row with \(p\) then add it to \(i\)th row. Right multiply 
\(L_{i,j}(p)\) is equivalent to the elementary transformation that multiply the \(i\)th column with \(p\) then add it to the \(j\)th column. Consider \(EA\) where 
\[E:=L_{1,2}(x)\cdots L_{d-2,d-1}(x)L_{d-1,d}(x)\]
We get a matrix 
\[EA=\begin{pmatrix}
	0&0&\cdots&0&f(x)\\ 
	-1&0&\cdots&0&x^{d-1}+a_{d-2}x^{d-2}+\cdots+a_1\\ 
	0&-1&\cdots&0&x^{d-2}+a_{d-2}x^{d-3}+\cdots+a_2\\ 
	\vdots&\vdots&&\vdots&\vdots&\\ 
	0&0&\cdots&0&x^2+a_{d-1}x+a_{d-2}\\ 
	0&0&\cdots&-1&x+a_{d-1}
\end{pmatrix}\]
Write
\[f_i(x):=x^{d-i}+a_{d-1}x^{d-i-1}+\cdots+a_i\]
for \(0\leq i\leq d-1\). Note that \(f_0(x)=f(x)\) and \(EA\) can be rewrite as 
\[EA=\begin{pmatrix}
	0&0&\cdots&0&f_0(x)\\ 
	-1&0&\cdots&0&f_1(x)\\ 
	0&-1&\cdots&0&f_2(x)\\ 
	\vdots&\vdots&&\vdots&\vdots&\\ 
	0&0&\cdots&0&f_{d-2}(x)\\ 
	0&0&\cdots&-1&f_{d-1}(x)
\end{pmatrix}\]
Next consider \(EAF\) where 
\[F:=L_{1,d}(f_1(x))L_{2,d}(f_2(x))\cdots L_{d-1,d}(f_{d-1}(x)).\]
We have 
\[EAF=\begin{pmatrix}
	0&0&\cdots&0&f(x)\\ 
	-1&0&\cdots&0&0\\ 
	0&-1&\cdots&0&0\\ 
	\vdots&\vdots&&\vdots&\vdots&\\ 
	0&0&\cdots&0&0\\ 
	0&0&\cdots&-1&0
\end{pmatrix}\]
Note that \(-1\) and \(1\) differ by a unit in \(\mathbb{F}[x]\), and by switching rows and columns, \(EAF\) is a diagonal matrix, so the invariant factors of \(A=xI_d-\Com(f)\) are 
\(1,1,\ldots,1,f(x)\) with \(1\) appearing \(d-1\) times.
\end{solution}

\noindent\rule{7in}{2.8pt}
%%%%%%%%%%%%%%%%%%%%%%%%%%%%%%%%%%%%%%%%%%%%%%%%%%%%%%%%%%%%%%%%%%%%%%%%%%%%%%%%%%%%%%%%%%%%%%%%%%%%%%%%%%%%%%%%%%%%%%%%%%%%%%%%%%%%%%%%
% Exercise 15.2.8
%%%%%%%%%%%%%%%%%%%%%%%%%%%%%%%%%%%%%%%%%%%%%%%%%%%%%%%%%%%%%%%%%%%%%%%%%%%%%%%%%%%%%%%%%%%%%%%%%%%%%%%%%%%%%%%%%%%%%%%%%%%%%%%%%%%%%%%%
\begin{problem}{15.2.8}
The number of similarity classes of \(n\times n\) nilpotent matrices over a field \(\mathbb{F}\) is equal to the number of partition of \(n\).
\end{problem}
\begin{solution}
By Exercise 15.2.7, given two matrices \(A,B\in M_n(\mathbb{F})\), \(A\) is similar to \(B\) if and only if \(xI_n-A\) and \(xI_n-B\) have the same 
invariant factors viewed as \(\mathbb{F}[x]\)-modules. By Exercise 15.2.2, we know that the invariant factors of \(xI-A\) are just \(1\) together with invariant factors of \(A\). So to determine the similarity classes of nilpotent matrices, we only need to determine the invariant factors of a nilpotent matrix \(A\). Since \(A\) is nilpotent, the characteristic polynomial of \(A\) is \(x^n\). By Exercise 15.2.1, suppose \(A\) has invariant factors \(\delta_1\mid \delta_2\mid \cdots \mid \delta_k=x^k\) where \(x^k\) is the minimal polynomial. 
For \(1\leq i\leq k-1\), \(\delta_i=x^{d_i}\) where \(1\leq d_i\leq k\) and \(d_1\leq  d_2\leq \cdots\leq d_{k-1}\). By Exercise 15.2.9, \(x^n\) is the product of invariant factors, so \(d_1+d_2+\cdots+d_{k-1}=n\). This gives a partition of \(n\). Now we can see that 
a partition of \(n\) gives a similar class of nilpotent matrices, and any similar class can be obtained in this way.
\end{solution}

\noindent\rule{7in}{2.8pt}
%%%%%%%%%%%%%%%%%%%%%%%%%%%%%%%%%%%%%%%%%%%%%%%%%%%%%%%%%%%%%%%%%%%%%%%%%%%%%%%%%%%%%%%%%%%%%%%%%%%%%%%%%%%%%%%%%%%%%%%%%%%%%%%%%%%%%%%%
% Exercise 15.2.11
%%%%%%%%%%%%%%%%%%%%%%%%%%%%%%%%%%%%%%%%%%%%%%%%%%%%%%%%%%%%%%%%%%%%%%%%%%%%%%%%%%%%%%%%%%%%%%%%%%%%%%%%%%%%%%%%%%%%%%%%%%%%%%%%%%%%%%%%
\begin{problem}{15.2.11}
Give a list of \(2\times 2\) matrices over \(\mathbb{F}_2\) such that every \(2\times 2\) matrix over \(\mathbb{F}_2\) is similar to exactly one on your list.
\end{problem}
\begin{solution}
We first determine the characteristic polynomial of the matrices. We list all degree 2 polynomials over \(\mathbb{F}_2\). There are four: \(x^2\), \(x^2+1=(x+1)^2\), \(x^2+x\) and \(x^2+x+1\). Note that the characteristic polynomial is a product of invariant factors, and by Theorem 15.2.3, 
we can give a matrix in the first canonical form as diag(\(\Com(\delta_1),\ldots,\Com(\delta_k)\)). All the classes of similar matrices are listed below. 
\begin{center}
\begin{tabular}{||c|c| c |c||} 
    \hline
	  & characteristic polynomial & invariant factor & matrix \\ [0.5ex] 
	\hline\hline
	1 & $x^2$ & $x\mid x$ & $\begin{pmatrix}
		0&0\\ 
		0&0
	\end{pmatrix}$ \\ [0.5ex]
	\hline
	2 & $x^2$ & $x^2$ &$\begin{pmatrix}
		0&0\\ 
		1&0
	\end{pmatrix}$  \\[0.5ex]
	\hline
	3 & $x^2+1$ & $x+1\mid x+1$ & $\begin{pmatrix}
		1&0\\ 
		0&1
	\end{pmatrix}$ \\[0.5ex]
	\hline
	4 & $x^2+1$ & $x^2+1$ & $\begin{pmatrix}
		0&1\\ 
		1&0
	\end{pmatrix}$ \\[0.5ex]
	\hline
	5 & $x^2+x$ & $x^2+x$ & $\begin{pmatrix}
		0&0\\ 
		1&1
	\end{pmatrix}$ \\ [0.5ex]
	\hline 
	6 & $x^2+x+1$ & $x^2+x+1$ & $\begin{pmatrix}
		0&1\\ 
		1&1
	\end{pmatrix}$ \\ [1ex] 
	\hline
   \end{tabular}
\end{center}
\end{solution}

\noindent\rule{7in}{2.8pt}
%%%%%%%%%%%%%%%%%%%%%%%%%%%%%%%%%%%%%%%%%%%%%%%%%%%%%%%%%%%%%%%%%%%%%%%%%%%%%%%%%%%%%%%%%%%%%%%%%%%%%%%%%%%%%%%%%%%%%%%%%%%%%%%%%%%%%%%%
% Exercise 15.2.12
%%%%%%%%%%%%%%%%%%%%%%%%%%%%%%%%%%%%%%%%%%%%%%%%%%%%%%%%%%%%%%%%%%%%%%%%%%%%%%%%%%%%%%%%%%%%%%%%%%%%%%%%%%%%%%%%%%%%%%%%%%%%%%%%%%%%%%%%
\begin{problem}{15.2.12}
Let \(V\) be a 7-dimensional vector space over \(\mathbb{Q}\).
\begin{enumerate}[(1)]
\item How many similarity classes of linear transformations on \(V\) have characteristic polynomial \((x-1)^4(x-2)^3\)?
\item Of the similarity classes in (a), how many have minimal polynomial \((x-1)^2(x-2)^2\)?
\item Let \(\phi\) be a linear transformation of \(V\) having characteristic polynomial \((x-1)^4(x-2)^3\) and minimal polynomial 
\((x-1)^2(x-2)^2\). Find \(\dim \ker (\phi-2id)\). 
\end{enumerate}
\end{problem}
\begin{solution}
\begin{enumerate}[(a)]
\item Write \(p=x-1\) and \(q=x-2\). \(p\) and \(q\) are coprime in \(\mathbb{F}[x]\). The characteristic polynomial \(p^4q^3\) is a product of invariant factors \(\delta_1\mid \delta_2\mid \cdots\mid \delta_k\) where 
\(\delta_1,\delta_2,\ldots,\delta_k\) are powers of \(p\), \(q\) or their product. We list the partition of \(4\) and \(3\) as follows:(1's are omitted in the final invariant factors, they are just there to easily determine the corresponding chain of invariant factors)
\begin{center}
\begin{tabular}{||c|c||}
	\hline
	partition of 4 & partition of 3 \\ [0.5ex] 
	\hline\hline
	$p\ p\ p\ p$ & $1\ q\ q\ q$\\ 
	\hline 
	$1\ p\ p\ p^2$ & $1\ 1\ q\ q^2$\\ 
	\hline 
	$1\  1\  p^2\ p^2$ & $1\ 1\ 1\ q^3$\\ 
	\hline 
	$1\ 1\ p\ p^3$ & \\ 
	\hline 
	$1\ 1\ 1\ p^4$ & \\ [1ex]
	\hline 
\end{tabular}
\end{center}
Any choice of one partition of 4 and one partition of 3 from the table gives us a chain of invariant factors just by multiplying them accordingly and omit \(1\)'s in the result. For example, 
\((1,p,p,p^2)\) and \(1,q,q,q\) will gives us the unqiue invariant factors \(pq\mid pq\mid p^2q\). So there are in total 15 similarity classes. 
\item By Exercise 15.2.1, we need to final term multiplied together to get \(p^2q^2\). There are two of them, namely \(p\mid pq\mid p^2q^2\) and \(p^2q\mid p^2q^2\).
\item By Exercise 15.2.10, we need to find the dimension of \(2\)-eigenspace of the linear transformation \(\phi\). \(\phi\) must have invariant factors \(p\mid pq\mid p^2q^2\) or \(p^2q\mid p^2q^2\). 
\begin{claim}
Suppose \(\lambda\) is an eigenvalue of \(\phi\). Then the dimension of \(\lambda\)-eigenspace is equal to the number of Jordan blocks \(J(x-\lambda,a)\) appears in the Jordan normal form. 
\end{claim}
\begin{claimproof}
We need to show that each Jordan block \(J(x-\lambda,a)\) defines a one-dimensional subspace of \(\lambda\)-eigenspace. By definition, 
\[J(x-\lambda,a)\begin{pmatrix}
	v_1\\ 
	v_2\\ 
	\vdots\\ 
	v_m
\end{pmatrix}=\begin{pmatrix}
	\lambda v_1\\ 
	\lambda v_2\\ 
	\vdots\\ 
	\lambda v_m
\end{pmatrix}=\begin{pmatrix}
	\lambda v_1+v_2\\ 
	\lambda v_2+v_3\\ 
	\vdots\\ 
	\lambda v_m
\end{pmatrix}\]
This implies \(v_2=v_3=\cdots=v_m=0\). So this \(J(x-\lambda,a)\) gives a one dimensional eigenspace generated by \(v_1\). And since Jordan Normal Form is diagonal in each block, they do not intersect. So to count 
the dimension of \(\lambda\)-eigenspace, we only need to count the time \(J(x-\lambda,a)\) appears. 
\end{claimproof}

In \(p\mid pq\mid p^2q^2\) or \(p^2q\mid p^2q^2\), \(q\) or \(q^2\) appear exactly twice. So \(\dim \ker (\phi-2id)=2\).
\end{enumerate} 
\end{solution}

\end{document}
\documentclass[a4paper, 12pt]{article}

\usepackage{/Users/zhengz/Desktop/Math/Workspace/Homework1/homework}
%%%%%%%%%%%%%%%%%%%%%%%%%%%%%%%%%%%%%%%%%%%%%%%%%%%%%%%%%%%%%%%%%%%%%%%%%%%%%%%%%%%%%%%%%%%%%%%%%%%%%%%%%%%%%%%%%%%%%%%%%%%%%%%%%%%%%%%%
\begin{document}
%Header-Make sure you update this information!!!!
\noindent
%%%%%%%%%%%%%%%%%%%%%%%%%%%%%%%%%%%%%%%%%%%%%%%%%%%%%%%%%%%%%%%%%%%%%%%%%%%%%%%%%%%%%%%%%%%%%%%%%%%%%%%%%%%%%%%%%%%%%%%%%%%%%%%%%%%%%%%%
\large\textbf{Zhengdong Zhang} \hfill \textbf{Homework - Week 7}   \\
Email: zhengz@uoregon.edu \hfill ID: 952091294 \\
\normalsize Course: MATH 648 - Abstract Algebra  \hfill Term: Winter 2025\\
Instructor: Professor Arkady Berenstein \hfill Due Date: $26^{th}$ February, 2025 \\
\noindent\rule{7in}{2.8pt}
\setstretch{1.1}
%%%%%%%%%%%%%%%%%%%%%%%%%%%%%%%%%%%%%%%%%%%%%%%%%%%%%%%%%%%%%%%%%%%%%%%%%%%%%%%%%%%%%%%%%%%%%%%%%%%%%%%%%%%%%%%%%%%%%%%%%%%%%%%%%%%%%%%%
% Exercise 17.2.5
%%%%%%%%%%%%%%%%%%%%%%%%%%%%%%%%%%%%%%%%%%%%%%%%%%%%%%%%%%%%%%%%%%%%%%%%%%%%%%%%%%%%%%%%%%%%%%%%%%%%%%%%%%%%%%%%%%%%%%%%%%%%%%%%%%%%%%%%
\begin{problem}{17.2.5}
True or false? \(\mathbb{Z}/m \mathbb{Z}\otimes_{\mathbb{Z}}\mathbb{Q}=0\) for any \(m\in \mathbb{Z}_{>0}\).
\end{problem}
\begin{solution}
This is true. From Example 17.2.4 in the book, we know that 
\[\mathbb{Z}/m \mathbb{Z}\otimes_\mathbb{Z}\mathbb{Q}\cong \mathbb{Q}/m \mathbb{Q}.\]
Note that \(\mathbb{Q}\) is a field and \(m\) is invertible in \(\mathbb{Q}\), so \(m \mathbb{Q}\cong \mathbb{Q}\) and we have 
\[\mathbb{Z}/ m \mathbb{Z}\otimes_\mathbb{Z}\mathbb{Q}=0.\]
\end{solution}

\noindent\rule{7in}{2.8pt}
%%%%%%%%%%%%%%%%%%%%%%%%%%%%%%%%%%%%%%%%%%%%%%%%%%%%%%%%%%%%%%%%%%%%%%%%%%%%%%%%%%%%%%%%%%%%%%%%%%%%%%%%%%%%%%%%%%%%%%%%%%%%%%%%%%%%%%%%
% Exercise 17.2.6
%%%%%%%%%%%%%%%%%%%%%%%%%%%%%%%%%%%%%%%%%%%%%%%%%%%%%%%%%%%%%%%%%%%%%%%%%%%%%%%%%%%%%%%%%%%%%%%%%%%%%%%%%%%%%%%%%%%%%%%%%%%%%%%%%%%%%%%%
\begin{problem}{17.2.6}
\(\mathbb{Q}\otimes_{\mathbb{Z}}\mathbb{Q}\cong \mathbb{Q}\) as abelian groups.
\end{problem}
\begin{solution}
We define a map 
\begin{align*}
	f:\mathbb{Q}\times \mathbb{Q}&\rightarrow \mathbb{Q},\\ 
	  (\frac{p}{q},\frac{r}{s})&\mapsto \frac{pr}{qs}.
\end{align*}
where \(\frac{p}{q},\frac{r}{s}\in \mathbb{Q}\). For any \(m\in \mathbb{Z}\), we have 
\[mf(\frac{p}{q},\frac{r}{s})=\frac{mpr}{qs}=f(\frac{mp}{q},\frac{r}{s})=f(\frac{p}{q},\frac{mr}{s}).\]
And for \(\frac{p_1}{q_1},\frac{p_2}{q_2}\in \mathbb{Q}\), we have 
\begin{align*}
f(\frac{p_1}{q_1}+\frac{p_2}{q_2},\frac{r}{s})&=(\frac{p_1}{q_1}+\frac{p_2}{q_2})\cdot \frac{r}{s}\\ 
                                              &=\frac{p_1r}{q_1s}+\frac{p_2r}{q_2s}\\ 
											  &=f(\frac{p_1}{q_1},\frac{r}{s})+f(\frac{p_2}{q_2},\frac{r}{s})
\end{align*}
By symmmetry, this is also true for the second component. Thus, we can conclude that \(f\) is a \(\mathbb{Z}\)-balanced map between \(\mathbb{Z}\)-modules. By the universal property of tensor product, there exists a map 
\(\tilde{f}:\mathbb{Q}\otimes_\mathbb{Z}\mathbb{Q}\rightarrow \mathbb{Q}\) between abelian groups sending \(\frac{p}{q}\otimes \frac{r}{s}\) to \(\frac{pq}{rs}\). Consider a map 
\begin{align*}
	g:\mathbb{Q}&\rightarrow \mathbb{Q}\otimes_\mathbb{Z}\mathbb{Q},\\ 
	  \frac{p}{q}&\mapsto \frac{p}{q}\otimes 1
\end{align*}
It is easy to check this is also a map between abelian groups. For \(\frac{p}{q}, \frac{r}{s}\in \mathbb{Q}\), we have 
\begin{align*}
	(g\circ \tilde{f})(\frac{p}{q}\otimes \frac{r}{s})&=g(\frac{pr}{qs})\\ 
	                                          &=\frac{pr}{qs}\otimes 1\\ 
											  &=(r\cdot \frac{p}{qs})\otimes 1\\ 
											  &=\frac{p}{qs}\otimes (s\cdot \frac{r}{s})\\ 
											  &=(s\cdot \frac{p}{qs})\otimes \frac{r}{s}\\ 
											  &=\frac{p}{q}\otimes \frac{r}{s}.
\end{align*}
This proves that \(g\circ \tilde{f}=id\). Conversely, we know that 
\[(\tilde{f}\circ g)(\frac{p}{q})=\tilde{f}(\frac{p}{q}\otimes 1)=\frac{p}{q}.\]
So \(\tilde{f}\circ g=id\). This proves that \(\tilde{f}\) is an isomorphism between abelian groups. Therefore we can conclude that 
\[\mathbb{Q}\otimes_\mathbb{Z}\mathbb{Q}\cong \mathbb{Q}\]
as abelian groups.
\end{solution}

\noindent\rule{7in}{2.8pt}
%%%%%%%%%%%%%%%%%%%%%%%%%%%%%%%%%%%%%%%%%%%%%%%%%%%%%%%%%%%%%%%%%%%%%%%%%%%%%%%%%%%%%%%%%%%%%%%%%%%%%%%%%%%%%%%%%%%%%%%%%%%%%%%%%%%%%%%%
% Exercise 17.2.7
%%%%%%%%%%%%%%%%%%%%%%%%%%%%%%%%%%%%%%%%%%%%%%%%%%%%%%%%%%%%%%%%%%%%%%%%%%%%%%%%%%%%%%%%%%%%%%%%%%%%%%%%%%%%%%%%%%%%%%%%%%%%%%%%%%%%%%%%
\begin{problem}{17.2.7}
If \(I\) is a right ideal of a ring \(R\) and \(V\) is a left \(R\)-module, then there is an isomorphism of abelian groups 
\[R/I\otimes_R V\cong V/IV,\]
where \(IV\) is the subgroup of \(V\) generated by all elements \(xv\) with \(x\in I\) and \(v\in V\).
\end{problem}
\begin{solution}
We define a map 
\begin{align*}
	f:R/I\times V&\rightarrow V/IV,\\ 
	  (r+I,v)&\mapsto rv+IV.
\end{align*}
We first check \(f\) is well-defined. Suppose \(r_1+I\) and \(r_2+I\) is the same element in \(R/I\), this means \(r_1-r_2\in I\). Then for any \(v\in V\), we have 
\(r_1v-r_2v=(r_1-r_2)v\in IV\). This means \(r_1v+IV\) and \(r_2v+IV\) is the same element in \(V/IV\). Given \(s\in R\), we have 
\[f(rs+I,v)=rsv+IV=f(r+I,sv).\]
So \(f\) is a \(R\)-balanced map. By the universal property of tensor product, there exists a unique abelian group homomorphism \(\tilde{f}:R/I\otimes_R V\rightarrow V/IV\) sending 
\((r+I)\otimes v\) to \(rv+IV\). Next, we are going to show that \(\tilde{f}\) is an isomorphism. 

Given \((r+I)\otimes v\in R/I\otimes_R V\), if \(\tilde{f}((r+I)\otimes v)=0\in V/IV\), then \(rv\in IV\). By definition this means \(r\in I\), so \(r+I\) is the zero element in \(R/I\) and we have 
\((r+I)\otimes v=(0+I)\otimes v=0\in R/I\otimes_R V\). This proves that \(\tilde{f}\) is injective. Conversely, given \(w+IV\in V/IV\), consider \((1+I)\otimes w\in R/I\otimes_R V\), we have 
\[\tilde{f}((1+I)\otimes w)=w+IV.\]
This proves \(\tilde{f}\) is surjective. Thus, we can conclude that \(\tilde{f}\) is an isomorphism between abelian groups and 
\[R/I\otimes_R V\cong V/IV.\]
\end{solution}

\noindent\rule{7in}{2.8pt}
%%%%%%%%%%%%%%%%%%%%%%%%%%%%%%%%%%%%%%%%%%%%%%%%%%%%%%%%%%%%%%%%%%%%%%%%%%%%%%%%%%%%%%%%%%%%%%%%%%%%%%%%%%%%%%%%%%%%%%%%%%%%%%%%%%%%%%%%
% Exercise 17.2.13
%%%%%%%%%%%%%%%%%%%%%%%%%%%%%%%%%%%%%%%%%%%%%%%%%%%%%%%%%%%%%%%%%%%%%%%%%%%%%%%%%%%%%%%%%%%%%%%%%%%%%%%%%%%%%%%%%%%%%%%%%%%%%%%%%%%%%%%%
\begin{problem}{17.2.13}
Prove:
\begin{enumerate}[(1)]
\item If \(M\) is any \(\mathbb{Z}\)-module, then \(M\otimes_{\mathbb{Z}}\mathbb{Q}\) is an injective \(\mathbb{Z}\)-module. 
\item Deduce that given an injective \(\mathbb{Z}\)-module homomorphism \(f:M\rightarrow N\), there exists a \(\mathbb{Z}\)-module homomorphism \(\alpha:N\rightarrow M\otimes_{\mathbb{Z}}\mathbb{Q}\) such that 
\(\alpha(f(m))=m\otimes 1\).
\item Let \(\mu:\mathbb{Q}\otimes_\mathbb{Z}\mathbb{Q}\rightarrow \mathbb{Q}\) be the product map, and 
\[\beta:=(id\otimes \mu)\circ (\alpha\otimes id):N\otimes_\mathbb{Z}\mathbb{Q}\rightarrow M\otimes_\mathbb{Z}\mathbb{Q}.\]
Then \(\beta\circ (f\otimes id)\) is identity on \(M\otimes_\mathbb{Z} \mathbb{Q}\). 
\item Deduce that \(f\otimes id\) is injective and \(\mathbb{Q}\) is a flat \(\mathbb{Z}\)-module.
\end{enumerate}
\end{problem}
\begin{solution}
\begin{enumerate}[(1)]
\item Let \(n\geq 0\) be an integer. \((n)\) is an ideal in \(\mathbb{Z}\), viewed as a \(\mathbb{Z}\)-module. Suppose we have a \(\mathbb{Z}\)-module homomorphism \(p:(n)\rightarrow M\otimes_\mathbb{Z}\mathbb{Q}\). We know that 
\(p\) is completely determined by the image of \(n\in (n)\). Assume \(p(n)=\sum_{i=1}^{k}m_i\otimes \frac{p_i}{q_i}\). Consider a map \(\tilde{p}:\mathbb{Z}\rightarrow M\otimes_\mathbb{Z} \mathbb{Q}\) by sending \(1\in \mathbb{Z}\) to 
\(\sum_{i=1}^{k}m_i\otimes \frac{p_i}{nq_i}\). This is a \(\mathbb{Z}\)-module homomorphism and 
\begin{align*}
\tilde{p}(n)&=n(\sum_{i=1}^{k}m_i\otimes \frac{p_i}{nq_i})\\ 
            &=\sum_{i=1}^{k}m_i\otimes n\cdot \frac{p_i}{nq_i}\\ 
			&=\sum_{i=1}^k m_i\otimes \frac{p_i}{q_i}\\ 
			&=p(n).
\end{align*}
Namely we have a commutative diagram 
\[\begin{tikzcd}
	{(n)} & {\mathbb{Z}} \\
	& {M\otimes_\mathbb{Z}\mathbb{Q}}
	\arrow[hook, from=1-1, to=1-2]
	\arrow["p"', from=1-1, to=2-2]
	\arrow["{\tilde{p}}", from=1-2, to=2-2]
\end{tikzcd}\]
Note that \(\mathbb{Z}\) is a PID and every ideal in \(\mathbb{Z}\) has the form \((n)\) for some \(n\in \mathbb{Z}\). We have proved every \(\mathbb{Z}\)-module homomorphism \(p\) can be extended to a \(\mathbb{Z}\)-module homomorphism \(\tilde{p}\). By 
Baer's Criterion, \(M\otimes_\mathbb{Z}\mathbb{Q}\) is an injective \(\mathbb{Z}\)-mdoule. 
\item We use the definition of injective modules. Consider the following diagram of solid arrows 
\[\begin{tikzcd}
	& {M\otimes_\mathbb{Z}\mathbb{Q}} \\
	0 & M & N
	\arrow[from=2-1, to=2-2]
	\arrow["i", from=2-2, to=1-2]
	\arrow["f"', from=2-2, to=2-3]
	\arrow["\alpha"', dashed, from=2-3, to=1-2]
\end{tikzcd}\]
where \(i:M\rightarrow M\otimes_\mathbb{Z}\mathbb{Q}\) is the inclusion map sending any \(m\in M\) to \(m\otimes 1\in M\otimes_\mathbb{Z}\mathbb{Q}\). There exists a \(\mathbb{Z}\)-module homomorphism \(\alpha:N\rightarrow M\otimes_\mathbb{Z}\mathbb{Q}\) such that the above 
diagram commutes. For any \(m\in M\), we have 
\[\alpha(f(m))=i(m)=m\otimes 1.\]
\item \(-\otimes_\mathbb{Z}\mathbb{Q}\) is a functor and consider the following diagram 
\[\begin{tikzcd}
	{M\otimes_\mathbb{Z}\mathbb{Q}} \\
	{N\otimes_\mathbb{Z}\mathbb{Q}} \\
	{M\otimes_\mathbb{Z}\mathbb{Q}\otimes_\mathbb{Z}\mathbb{Q}} \\
	{M\otimes_\mathbb{Z}\mathbb{Q}}
	\arrow["{f\otimes id_\mathbb{Q}}", from=1-1, to=2-1]
	\arrow["\gamma"', curve={height=60pt}, from=1-1, to=4-1]
	\arrow["{\alpha\otimes id_\mathbb{Q}}", from=2-1, to=3-1]
	\arrow["\beta", curve={height=-60pt}, from=2-1, to=4-1]
	\arrow["{id_M\otimes\mu}", from=3-1, to=4-1]
\end{tikzcd}\]
where \(\mu:\mathbb{Q}\otimes_\mathbb{Z}\mathbb{Q}\rightarrow \mathbb{Q}\) is the product map. For any \(m\in M\) and \(\frac{p}{q}\in \mathbb{Q}\), we have 
\begin{align*}
	\gamma(m\otimes \frac{p}{q})&=(\beta\circ (f\otimes id))(m\otimes \frac{p}{q})\\ 
	                            &=\beta(f(m)\otimes \frac{p}{q})\\ 
								&=(id\otimes \mu)\circ (\alpha\otimes id)(f(m)\otimes \frac{p}{q})\\ 
								&=(id\otimes \mu)(\alpha(f(m))\otimes \frac{p}{q})\\ 
								&=(id\otimes \mu)(m\otimes 1\otimes \frac{p}{q})\\ 
								&=m\otimes \frac{p}{q}.
\end{align*} 
So \(\gamma=id:M\otimes_\mathbb{Z}\mathbb{Q}\rightarrow M\otimes_\mathbb{Z}\mathbb{Q}\) is the identity. 
\item We first prove the following claim.
\begin{claim}
Suppose \(p:X\rightarrow Y\) and \(q:Y\rightarrow Z\) are two maps of \(\mathbb{Z}\)-modules. If \(q\circ p\) is injective, then \(p\) is injective. 
\end{claim}
\begin{claimproof}
Let \(x\in \ker p\). We have \(p(x)=0\in Y\). This implies that \((q\circ p)(x)=q(0)=0\). So \(x\in \ker (q\circ p)\). Since \(q\circ p\) is injective, so 
\(x=0\). This means \(p\) is also injective.
\end{claimproof}

Use the claim above, because \(id=\gamma=\beta\circ (f\otimes id)\) is injective, we can conclude that \(f\otimes id\) is also injective. And since \(-\otimes_\mathbb{Z} \mathbb{Q}\) sends 
injective maps to injective maps, \(\mathbb{Q}\) is a flat \(\mathbb{Z}\)-module.
\end{enumerate}
\end{solution}

\noindent\rule{7in}{2.8pt}
%%%%%%%%%%%%%%%%%%%%%%%%%%%%%%%%%%%%%%%%%%%%%%%%%%%%%%%%%%%%%%%%%%%%%%%%%%%%%%%%%%%%%%%%%%%%%%%%%%%%%%%%%%%%%%%%%%%%%%%%%%%%%%%%%%%%%%%%
% Exercise 17.2.20
%%%%%%%%%%%%%%%%%%%%%%%%%%%%%%%%%%%%%%%%%%%%%%%%%%%%%%%%%%%%%%%%%%%%%%%%%%%%%%%%%%%%%%%%%%%%%%%%%%%%%%%%%%%%%%%%%%%%%%%%%%%%%%%%%%%%%%%%
\begin{problem}{17.2.20}
Prove that a free module is flat. Then prove that a projective module is flat. 
\end{problem}
\begin{solution}
We will use Exercise 17.2.18. We prove it in the claim.
\begin{claim}
Let \((V_i)_{i\in I}\) be a family of \(R\)-modules. Then \(\oplus_{i\in I}V_i\) is flat if and only if all \(V_i\) are flat. 
\end{claim}
\begin{claimproof}
Let \(f:M\rightarrow N\) be an injective \(R\)-module homomorphism. Given a family of \(R\)-modules \((V_i)_{i\in I}\), by Theorem 17.2.16, we have a commutative diagram 
\[\begin{tikzcd}
	{M\otimes_R(\oplus_{i\in I}V_i)} && {N\otimes_R(\oplus_{i\in I}V_i)} \\
	{\oplus_{i\in I}(M\otimes_RV_i)} && {\oplus_{i\in I}(N\otimes_RV_i)}
	\arrow["{f\otimes id}", from=1-1, to=1-3]
	\arrow["{\alpha_M}"', from=1-1, to=2-1]
	\arrow["{\alpha_N}", from=1-3, to=2-3]
	\arrow["{\oplus(f\otimes id)}", from=2-1, to=2-3]
\end{tikzcd}\]
where \(\alpha_M,\alpha_N\) are isomorphism of abelian groups. Assume \(\oplus_{i\in I}V_i\) is flat, this means \(f\otimes id\) in the top row is injective. Then 
\[\alpha_N\circ (f\otimes id)=\oplus(f\otimes id)\circ \alpha_M\]
is also injective because \(\alpha_N\) is an isomorphism. And since \(\alpha_M\) is also an isomorphism, we know that \(\oplus(f\otimes id)\) is injective. Conversely, if \(\oplus(f\otimes id)\) is injective, by the same 
argument, we can see that \(f\otimes id\) in the top row is injective. This proves that \(\oplus_{i\in I}V_i\) is flat if and only if all \(V_i\) are flat.
\end{claimproof}

Let \(f:M\rightarrow N\) be an injective \(R\)-module homomorphism. We know that \(M\otimes_R R\cong M\) and \(N\otimes_R R\cong R\), this isomorphism is functorial so we have a commutative diagram 
\[\begin{tikzcd}
	{M\otimes_R R} & {N\otimes_R R} \\
	M & N
	\arrow["{f\otimes id }", from=1-1, to=1-2]
	\arrow["\sim"', from=1-1, to=2-1]
	\arrow["\sim", from=1-2, to=2-2]
	\arrow["f"', from=2-1, to=2-2]
\end{tikzcd}\]
We can see that \(f\otimes id\) is also injective. So \(R\) is a flat \(R\)-module. By the claim we know that \(\oplus_{i\in I}R\) is also flat. Suppose \(P\) is a projective \(R\)-module, this is equivalent to that there exists 
an \(R\)-module \(P'\) such that \(P\oplus P'=\oplus_{i\in I}R\). We have already known that the free module \(\oplus_{i\in I}R\) is flat, by the claim we know \(P\) is also flat. 

\end{solution}

\noindent\rule{7in}{2.8pt}
%%%%%%%%%%%%%%%%%%%%%%%%%%%%%%%%%%%%%%%%%%%%%%%%%%%%%%%%%%%%%%%%%%%%%%%%%%%%%%%%%%%%%%%%%%%%%%%%%%%%%%%%%%%%%%%%%%%%%%%%%%%%%%%%%%%%%%%%
% Exercise 17.3.2
%%%%%%%%%%%%%%%%%%%%%%%%%%%%%%%%%%%%%%%%%%%%%%%%%%%%%%%%%%%%%%%%%%%%%%%%%%%%%%%%%%%%%%%%%%%%%%%%%%%%%%%%%%%%%%%%%%%%%%%%%%%%%%%%%%%%%%%%
\begin{problem}{17.3.2}
If \(R\) is commutative and \(I,J\) are ideals in \(R\), then there is an isomorphism of \(R\)-modules 
\[R/I\otimes_R R/J\cong R/(I+J).\]
\end{problem}
\begin{solution}
From what we know in Exercise 17.2.7, we can see that have an isomorphism of abelian groups 
\[R/I\otimes_R R/J\cong (R/J)/I(R/J).\]
Since \(R\) is commutative, both sides can be viewed as an \(R\)-module and we the isomorphism we defined before is a \(R\)-module isomorphism. 
\begin{claim}
If \(R\) is commmutative and \(I,J\subset R\) are ideals, then we have the following \(R\)-module isomorphisms.
\[I(R/J)\cong I/(I\cap J)\cong (I+J)/J.\]
\end{claim}
\begin{claimproof}
We define a map of \(R\)-modules. For any \(a(b+I)\in I(R/J)\) where \(a\in I\) and \(b+J\in R/J\),
\begin{align*}
	f:I(R/J)&\rightarrow I/(I\cap J),\\ 
	  a(b+J)&\mapsto ab+I\cap J.
\end{align*}
We check this is well-defined. Suppose \(b_1+J\) and \(b_2+J\) are two representatives for the same element in \(R/J\). This means \(b_1-b_2\in J\). Then we know 
\[f(a(b_1+J))-f(a(b_2+J))=a(b_1-b_2)+I\cap J.\]
Since \(a\in I\) and \(b_1-b_2\in J\), \(a(b_1-b_2)\in I\cap J\), so \(f(a(b_1+J))=f(a(b_2+J))\). Next, we are going to show this is an isomorphism. For any \(a\in I\), consider the map 
\begin{align*}
	g:I/(I\cap J )&\rightarrow I(R/J),\\ 
	  (a+I\cap J)&\mapsto a(1+J).
\end{align*}
This is well-defined. Indeed, suppose for \(a_1,a_2\in I\), \(a_1+I\cap J\) and \(a_2+I\cap J\) represents the same element in \(I/(I\cap J)\). This means \(a_1-a_2\in I\cap J\). Then the image 
\[g(a_1+I\cap J)-g(a_2+I\cap J)=(a_1-a_2)(1+J).\]
Note that \(a_1-a_2\in I\cap J\subset J\), so \((a_1-a_2)(1+J)=(a_1-a_2)+J=J\) is the zero element in \(R/J\). So \(g\) is a well-define \(R\)-module homomorphism. Moreover, we have 
\(f\circ g=id\) and for any \(a\in I\) and \(b+J\in R/J\), 
\[g(ab+I\cap J)=ab(1+J)=a(b+J).\]
This proves that \(f\) is an isomorphism of \(R\)-modules and 
\[I(R/J)\cong I/(I\cap J)\cong (I+J)/J.\]
The next isomorphism is by the second isomorphism theorem in commutative rings. 
\end{claimproof}

Note that \(J\subset I+J\subset R\), use the third isomorphism theorem and we have 
\[R/I\otimes_R R/J\cong (R/J)/I(R/J)\cong (R/J)/(I+J/J)\cong R/(I+J).\]
\end{solution}

\noindent\rule{7in}{2.8pt}
%%%%%%%%%%%%%%%%%%%%%%%%%%%%%%%%%%%%%%%%%%%%%%%%%%%%%%%%%%%%%%%%%%%%%%%%%%%%%%%%%%%%%%%%%%%%%%%%%%%%%%%%%%%%%%%%%%%%%%%%%%%%%%%%%%%%%%%%
% Exercise 17.3.10
%%%%%%%%%%%%%%%%%%%%%%%%%%%%%%%%%%%%%%%%%%%%%%%%%%%%%%%%%%%%%%%%%%%%%%%%%%%%%%%%%%%%%%%%%%%%%%%%%%%%%%%%%%%%%%%%%%%%%%%%%%%%%%%%%%%%%%%%
\begin{problem}{17.3.10}
\(\mathbb{H}\otimes_\mathbb{R}\mathbb{C}\cong M_2(\mathbb{C})\) as \(\mathbb{C}\)-algebras. 
\end{problem}
\begin{solution}
\(\mathbb{H}\) is a \(4\)-dimensional \(\mathbb{R}\)-vector space with standard \(\mathbb{R}\)-basis \(\left\{ 1,i,j,k \right\}\) and \(\mathbb{C}\) is a \(\mathbb{R}\)-vector 
space with \(\mathbb{R}\)-basis \(\left\{ 1,\sqrt{-1} \right\}\). By Theorem 17.3.4, \(\mathbb{H}\otimes_\mathbb{R}\mathbb{C}\) has a \(\mathbb{R}\)-basis 
\begin{align*}
	&1\otimes 1,i\otimes 1,j\otimes 1,k\otimes 1,\\ 
    &1\otimes i,i\otimes i,j\otimes i,k\otimes i.
\end{align*}
If we view \(\mathbb{H}\otimes_\mathbb{R}\mathbb{C}\) as a \(\mathbb{C}\)-algebra, then for any \(a\in\left\{ 1,i,j,k \right\}\), we have 
\[i(a\otimes 1)=a\otimes i.\]
Thus, 
\[1\otimes 1,i\otimes 1,j\otimes 1,k\otimes 1\]
is a basis for \(\mathbb{H}\otimes_\mathbb{R}\mathbb{C}\) as a \(\mathbb{C}\)-algebra. Consider the following map 
\begin{align*}
	f:\mathbb{H}\otimes_\mathbb{R}\mathbb{C}&\rightarrow M_2(\mathbb{C}),\\ 
	1\otimes 1&\mapsto \begin{pmatrix}
		1&0\\ 
		0&1
	\end{pmatrix},\\[5pt]
	i\otimes 1&\mapsto \begin{pmatrix}
		i&0\\ 
		0&-i
	\end{pmatrix},\\[5pt] 
	j\otimes 1&\mapsto \begin{pmatrix}
		0&1\\ 
		-1&0
	\end{pmatrix},\\[5pt]
    k\otimes 1&\mapsto \begin{pmatrix}
		0&i\\ 
		i&0
	\end{pmatrix}.
\end{align*}
It is easy to check that 
\[f(i\otimes 1)^2=f(j\otimes 1)^2=f(k\otimes 1)^2=\begin{pmatrix}
	-1&0\\
	0&-1
\end{pmatrix}\]
and 
\begin{align*}
	f(i\otimes 1)f(j\otimes 1)&=-f(j\otimes 1)f(i\otimes 1)=\begin{pmatrix}
		0&\sqrt{-1}\\ 
		-\sqrt{-1}&0
	\end{pmatrix}=f(k\otimes 1),\\[5pt] 
	f(j\otimes 1)f(k\otimes 1)&=-f(k\otimes 1)f(j\otimes 1)=\begin{pmatrix}
		i&0\\ 
		0&-i
	\end{pmatrix}=f(i\otimes 1),\\[5pt] 
	f(k\otimes 1)f(i\otimes 1)&=-f(i\otimes 1)f(k\otimes 1)=\begin{pmatrix}
		0&1\\ 
		-1&0
	\end{pmatrix}=f(j\otimes 1). 
\end{align*}
So \(f\) defines a \(\mathbb{C}\)-algebra homomorphism. Next, we are going to show \(f\) is surjective. Let \(M\subset M_2(\mathbb{C})\) be the subspace generated by 
\[\begin{pmatrix}
	1&0\\ 
	0&1
\end{pmatrix},\begin{pmatrix}
	i&0\\ 
	0&-i
\end{pmatrix},\begin{pmatrix}
	0&1\\ 
	-1&0
\end{pmatrix},\begin{pmatrix}
	0&i\\ 
	i&0
\end{pmatrix}.\]
We can see that 
\begin{align*}
	\frac{1}{2}(f(1\otimes 1)-if(i\otimes 1))&=\frac{1}{2}(\begin{pmatrix}
		1&0\\ 
		0&1
	\end{pmatrix}-\begin{pmatrix}
		-1&0\\ 
		0&1
	\end{pmatrix})=\begin{pmatrix}
		1&0\\
		0&0
	\end{pmatrix},\\[5pt] 
	\frac{1}{2}(f(1\otimes 1)+if(i\otimes 1))&=\frac{1}{2}(\begin{pmatrix}
		1&0\\ 
		0&1
	\end{pmatrix}+\begin{pmatrix}
		-1&0\\ 
		0&1
	\end{pmatrix})=\begin{pmatrix}
		0&0\\
		0&1
	\end{pmatrix},\\ 
	\frac{1}{2}(f(j\otimes 1)-if(k\otimes 1))&=\frac{1}{2}(\begin{pmatrix}
		0&1\\ 
		-1&0
	\end{pmatrix}-\begin{pmatrix}
		0&-1\\ 
		-1&0
	\end{pmatrix})=\begin{pmatrix}
		0&1\\
		0&0
	\end{pmatrix},\\[5pt]
	-\frac{1}{2}(f(j\otimes 1)+if(k\otimes 1))&=-\frac{1}{2}(\begin{pmatrix}
		0&1\\ 
		-1&0
	\end{pmatrix}+\begin{pmatrix}
		0&-1\\ 
		-1&0
	\end{pmatrix})=\begin{pmatrix}
		0&0\\
		1&0
	\end{pmatrix}. 
\end{align*}
We know that \(M_2(\mathbb{C})\) can be generated by 
\[\begin{pmatrix}
	1&0\\
	0&0
\end{pmatrix},\begin{pmatrix}
	0&1\\
	0&0
\end{pmatrix},\begin{pmatrix}
	0&0\\
	1&0
\end{pmatrix},\begin{pmatrix}
	0&0\\
	0&1
\end{pmatrix}\]
as a \(\mathbb{C}\)-algebra. This proves that \(M=M_2(\mathbb{C})\) and \(f\) is surjective. Note that 
\[\dim_\mathbb{C}(\mathbb{H}\otimes_\mathbb{R}\mathbb{C})=4=\dim_\mathbb{C}(M_2(\mathbb{C})).\]
So we have 
\[\mathbb{H}\otimes_\mathbb{R}\mathbb{C}\cong M_2(\mathbb{C})\]
as \(\mathbb{C}\)-algebras.
\end{solution}

\noindent\rule{7in}{2.8pt}
%%%%%%%%%%%%%%%%%%%%%%%%%%%%%%%%%%%%%%%%%%%%%%%%%%%%%%%%%%%%%%%%%%%%%%%%%%%%%%%%%%%%%%%%%%%%%%%%%%%%%%%%%%%%%%%%%%%%%%%%%%%%%%%%%%%%%%%%
% Exercise 17.3.11
%%%%%%%%%%%%%%%%%%%%%%%%%%%%%%%%%%%%%%%%%%%%%%%%%%%%%%%%%%%%%%%%%%%%%%%%%%%%%%%%%%%%%%%%%%%%%%%%%%%%%%%%%%%%%%%%%%%%%%%%%%%%%%%%%%%%%%%%
\begin{problem}{17.3.11}
\(M_n(R)\otimes_R M_m(R)\cong M_{mn}(R)\) for a commutative ring \(R\).
\end{problem}
\begin{solution}
Let \(A=(a_{ij})_{1\leq i,j\leq n}\in M_n(R)\) and \(B=(b_{kl})_{1\leq k,l\leq m}\in M_m(R)\). We define a map 
\begin{align*}
	f:M_n(R)\times M_m(R)&\rightarrow M_{mn}(R),\\ 
	  (A,B)&\mapsto \begin{pmatrix}
		a_{11}B&a_{12}B&\cdots&a_{1n}B\\ 
		a_{21}B&a_{22}B&\cdots&a_{2n}B\\ 
		\vdots&\vdots&\ddots&\vdots\\ 
		a_{n1}B&a_{n2}B&\cdots&a_{nn}B
	  \end{pmatrix}.
\end{align*}
where the image is a block matrix and for \(1\leq i,j\leq n\), each block \(a_{ij}B\) is a \(m\times m\) matrix with \((k,l)\)-entry \(a_{ij}b_{kl}\) for \(1\leq k,l\leq m\). For any \(r\in R\), we have 
\[f(Ar,B)=\begin{pmatrix}
	a_{11}rB&a_{12}rB&\cdots&a_{1n}rB\\ 
	a_{21}rB&a_{22}rB&\cdots&a_{2n}rB\\ 
	\vdots&\vdots&\ddots&\vdots\\ 
	a_{n1}rB&a_{n2}rB&\cdots&a_{nn}rB
\end{pmatrix}=f(A,rB).\]
So \(f\) is a \(R\)-balanced map and by universal property of tensor product, we have an \(R\)-module homomorphism \(\tilde{f}:M_n(R)\otimes_R M_m(R)\rightarrow M_{mn}(R)\). Suppose \(A\otimes B\in M_n(R)\otimes_R M_m(R)\) satisfying \(\tilde{f}(A\otimes B)=0\), namely 
\(a_{ij}B\) is the zero matrix for any \(1\leq i,j\leq n\). Let \(E_{ij}\) be the \(n\times n\) matrix with \(1\) at the \((i,j)\)-entry and all other entries are zero. Then we can write 
\begin{align*}
A\otimes B&=(\sum_{1\leq i,j\leq n}a_{ij}E_{ij})\otimes B\\ 
          &=\sum_{1\leq i,j\leq n}(a_{ij}E_{ij}\otimes B)\\ 
		  &=\sum_{1\leq i,j\leq n}E_{ij}\otimes a_{ij}B\\ 
		  &=\sum_(1\leq i,j\leq n)E_{ij}\otimes 0\\ 
		  &=0
\end{align*}
This proves \(\ker \tilde{f}=0\), so \(\tilde{f}\) is injective. Note that both the matrix algebra \(M_n(R)\) is a free \(R\)-module of rank \(n^2\), so we have 
\[\rank(M_n(R)\otimes_R M_m(R))=\rank M_n(R)\cdot \rank M_m(R)=n^2\cdot m^2=(mn)^2=\rank M_{mn}(R).\]
This proves that \(\tilde{f}\) is an \(R\)-module isomorphism. The last thing we need to show is that \(\tilde{f}\) is compatible with matrix multiplication. Let \(A\otimes B, C\otimes D\in M_n(R)\otimes_R M_m(R)\), suppose 
\(A=(a_{ij})_{1\leq i,j\leq n}\) and \(C=(c_{ij})_{1\leq i,j\leq n}\). Then we have 
\[\tilde{f}(A\otimes B)\cdot \tilde{f}(C\otimes D)=\begin{pmatrix}
	a_{11}B&a_{12}B&\cdots&a_{1n}B\\ 
		a_{21}B&a_{22}B&\cdots&a_{2n}B\\ 
		\vdots&\vdots&\ddots&\vdots\\ 
		a_{n1}B&a_{n2}B&\cdots&a_{nn}B
\end{pmatrix}\begin{pmatrix}
	c_{11}D&c_{12}D&\cdots&c_{1n}D\\ 
		c_{21}D&c_{22}D&\cdots&c_{2n}D\\ 
		\vdots&\vdots&\ddots&\vdots\\ 
		c_{n1}D&c_{n2}D&\cdots&c_{nn}D
\end{pmatrix}\]
Viewed as a block matrix, for \(1\leq i,j\leq n\), the \((i,j)\)-block entry is 
\[\sum_{k=1}^n (a_{ik}B)(c_{kj}D)=\sum_{k=1}^{n} (a_{ik}c_{kj})BD=(AC)_{ij}BD\]
where \((AC)_{ij}\) means the \((i,j)\)-entry for the matrix \(AC\). On the other hand, we know that 
\[\tilde{f}(AC\otimes BD)=\begin{pmatrix}
	(AC)_{11}BD&(AC)_{12}BD& \cdots&(AC)_{1n}BD\\ 
	(AC)_{21}BD&(AC)_{22}BD&\cdots&(AC)_{2n}BD\\ 
	\vdots&\vdots&\ddots&\vdots\\ 
	(AC)_{n1}BD&(AC)_{n2}BD&\cdots&(AC)_{nn}BD
\end{pmatrix}\]
This proves that 
\[\tilde{f}(A\otimes B)\cdot \tilde{f}(C\otimes D)=\tilde{f}(AC\otimes BD).\]
So \(\tilde{f}\) is indeed an matrix algebra homomorphism. 
\end{solution}

\noindent\rule{7in}{2.8pt}
%%%%%%%%%%%%%%%%%%%%%%%%%%%%%%%%%%%%%%%%%%%%%%%%%%%%%%%%%%%%%%%%%%%%%%%%%%%%%%%%%%%%%%%%%%%%%%%%%%%%%%%%%%%%%%%%%%%%%%%%%%%%%%%%%%%%%%%%
% Exercise 17.3.13
%%%%%%%%%%%%%%%%%%%%%%%%%%%%%%%%%%%%%%%%%%%%%%%%%%%%%%%%%%%%%%%%%%%%%%%%%%%%%%%%%%%%%%%%%%%%%%%%%%%%%%%%%%%%%%%%%%%%%%%%%%%%%%%%%%%%%%%%
\begin{problem}{17.3.13}
True or false? \(\mathbb{Q}\otimes_\mathbb{Z}\mathbb{Q}\cong \mathbb{Q}\) as \(\mathbb{Z}\)-algebras.
\end{problem}
\begin{solution}
This is true. The isomorphism we defined in Exercise 17.2.6 is the map 
\begin{align*}
	f:\mathbb{Q}\otimes_\mathbb{Z}\mathbb{Q}&\rightarrow \mathbb{Q},\\ 
	  \frac{p}{q}\otimes \frac{r}{s}&\mapsto \frac{pq}{rs}
\end{align*}
We need to show \(f\) is compatible with multiplication. Suppose \(\frac{p_1}{q_1}\otimes \frac{r_1}{s_1},\frac{p_2}{q_2}\otimes \frac{r_2}{s_2}\in \mathbb{Q}\otimes_\mathbb{Z}\mathbb{Q}\), we have 
\begin{align*}
f(\frac{p_1}{q_1}\otimes \frac{r_1}{s_1})\cdot f(\frac{p_2}{q_2}\otimes \frac{r_2}{s_2})&=\frac{p_1r_1}{q_1s_1}\cdot \frac{p_2r_2}{q_2s_2}\\ 
                                                                                        &=\frac{p_1p_2r_1r_2}{q_1q_2s_1s_2}\\ 
																						&=f(\frac{p_1p_2}{q_1q_2}\otimes \frac{r_1r_2}{s_1s_2})\\ 
																						&=f((\frac{p_1}{q_1}\otimes \frac{r_1}{s_1})\cdot (\frac{p_2}{q_2}\otimes \frac{r_2}{s_2}))
\end{align*}
Now we know that \(\mathbb{Q}\otimes_\mathbb{Z}\mathbb{Q}\cong \mathbb{Q}\) as \(\mathbb{Z}\)-algebras.
\end{solution}

\noindent\rule{7in}{2.8pt}
%%%%%%%%%%%%%%%%%%%%%%%%%%%%%%%%%%%%%%%%%%%%%%%%%%%%%%%%%%%%%%%%%%%%%%%%%%%%%%%%%%%%%%%%%%%%%%%%%%%%%%%%%%%%%%%%%%%%%%%%%%%%%%%%%%%%%%%%
% Exercise 17.3.21
%%%%%%%%%%%%%%%%%%%%%%%%%%%%%%%%%%%%%%%%%%%%%%%%%%%%%%%%%%%%%%%%%%%%%%%%%%%%%%%%%%%%%%%%%%%%%%%%%%%%%%%%%%%%%%%%%%%%%%%%%%%%%%%%%%%%%%%%
\begin{problem}{17.3.21}
True or false? Let \(A\) and \(B\) be a finite dimensional semisimple algerbras over an alegbraically closed field \(\mathbb{F}\). Then every finite dimenisonal \(A\otimes B\)-module 
is of the form \(V\boxtimes W\) for some \(A\)-module \(V\) and some \(B\)-module \(W\).
\end{problem}
\begin{solution}
This is false. Consider \(A=M_2(\mathbb{C})\times M_3(\mathbb{C})\) and \(B=M_3(\mathbb{C})\times M_4(\mathbb{C})\). By Wedderburn-Artin Theorem for algebras, \(A\) and \(B\) are semisimple. By 
Proposition 16.2.8, simple \(A\)-modules up to isomorphism have the form \(\mathbb{C}^2, \mathbb{C}^3\) and simple \(B\)-modules up to isomorphism have the form \(\mathbb{C}^3, \mathbb{C}^4\). Consider the 
\(A\otimes_\mathbb{C} B\)-module \(M=(\mathbb{C}^2\boxtimes \mathbb{C}^4)\oplus (\mathbb{C}^3\boxtimes \mathbb{C}^3)\). By Theorem 17.3.20, we know that 
both \(\mathbb{C}^2\boxtimes \mathbb{C}^4\) and \(\mathbb{C}^3\boxtimes \mathbb{C}^3\) are simple \(A\otimes_\mathbb{C}B\)-modules, so \(M\) is a finite dimensional \(A\otimes_\mathbb{F}B\)-module. Now we show that 
\(M\) cannot be written as \(V\boxtimes W\) where \(V\) is a \(A\)-module and \(W\) is a \(B\)-module. Note that \(A\) and \(B\) are semisimple,  every \(A\)-module and \(B\)-module are semisimple, namely they can be written as 
direct sum of simple \(A\)-modules and simple \(B\)-modules. For dimension reasons, we have 
\[\dim_\mathbb{C}M=17=\dim_\mathbb{C}V\cdot \dim_\mathbb{C}W.\]
So one of \(\dim_\mathbb{C}V\) or \(\dim_\mathbb{C}W\) has to be \(1\), but \(\mathbb{C}\) is not a simple \(A\)-module or simple \(B\)-module. Thus, \(M\) cannot be written in this form. 
\end{solution}

\end{document}
\documentclass[letterpaper, 12pt]{article}

\usepackage{/Users/zhengz/Desktop/Math/Workspace/Homework1/homework}

%%%%%%%%%%%%%%%%%%%%%%%%%%%%%%%%%%%%%%%%%%%%%%%%%%%%%%%%%%%%%%%%%%%%%%%%%%%%%%%%%%%%%%%%%%%%%%%%%%%%%%%%%%%%%%%%%%%%%%%%%%%%%%%%%%%%%%%%
\begin{document}
%Header-Make sure you update this information!!!!
\noindent
%%%%%%%%%%%%%%%%%%%%%%%%%%%%%%%%%%%%%%%%%%%%%%%%%%%%%%%%%%%%%%%%%%%%%%%%%%%%%%%%%%%%%%%%%%%%%%%%%%%%%%%%%%%%%%%%%%%%%%%%%%%%%%%%%%%%%%%%
\large\textbf{Zhengdong Zhang} \hfill \textbf{Homework - Week 1 Exercises}   \\
Email: zhengz@uoregon.edu \hfill ID: 952091294 \\
\normalsize Course: MATH 616 - Real Analysis \hfill Term: Fall 2025 \\
Instructor: Professor Weiyong He \hfill Due Date: Oct 8th, 2025 \\
\noindent\rule{7in}{2.8pt}
\setstretch{1.1}
%%%%%%%%%%%%%%%%%%%%%%%%%%%%%%%%%%%%%%%%%%%%%%%%%%%%%%%%%%%%%%%%%%%%%%%%%%%%%%%%%%%%%%%%%%%%%%%%%%%%%%%%%%%%%%%%%%%%%%%%%%%%%%%%%%%%%%%%
% Exercise 1.2
%%%%%%%%%%%%%%%%%%%%%%%%%%%%%%%%%%%%%%%%%%%%%%%%%%%%%%%%%%%%%%%%%%%%%%%%%%%%%%%%%%%%%%%%%%%%%%%%%%%%%%%%%%%%%%%%%%%%%%%%%%%%%%%%%%%%%%%%
\begin{problem}{1.2}
The Cantor set \(\mathcal{C}\) can also be described in terms of ternary expansions.
\begin{enumerate}[(a)]
    \item Every number in \([0,1]\) has a ternary expansion 
          \[x=\sum_{k=1}^\infty a_k 3^{-k},\ \ \ \ \text{where}\ a_k=0,1,2\]
          Note that this decomposition is not unique since, for example, 
          \[\frac{1}{3}=\sum_{k=2}^\infty \frac{2}{3^k}.\]
          Prove that \(x\in \mathcal{C}\) if and only if \(x\) has a representation as above where every \(a_k\) is either \(0\) or \(2\). 
    \item The \textbf{Cantor-Lebesgue function} is defined on \(\mathcal{C}\) by 
          \[F(x)=\sum_{k=1}^\infty \frac{b_k}{2^k},\ \ \text{if}\ x=\sum_{k=1}^\infty \frac{a_k}{3^k},\ \ \text{where}\ b_k=\frac{a_k}{2}.\]
          In this definition, we choose the expansion of \(x\) in which \(a_k=0\) or \(2\). Show that \(F\) is well-defined and continuous on \(\mathcal{C}\), and moreover, \(F(0)=0\) as well as \(F(1)=1\).
    \item Prove that \(F:\mathcal{C}\rightarrow [0,1]\) is surjective, that is, for every \(y\in [0,1]\), there exists \(x\in \mathcal{C}\) such that \(F(x)=y\).
    \item One can also extend \(F\) to be a continuous function on \([0,1]\) as follows. Note that if \((a,b)\) is an open interval of the complement of \(\mathcal{C}\), then \(F(a)=F(b)\). Hence, we may define \(F\) to have the constant value \(F(a)\) in that interval. 
\end{enumerate}
\end{problem}
\begin{solution}

\end{solution}

\noindent\rule{7in}{2.8pt}
%%%%%%%%%%%%%%%%%%%%%%%%%%%%%%%%%%%%%%%%%%%%%%%%%%%%%%%%%%%%%%%%%%%%%%%%%%%%%%%%%%%%%%%%%%%%%%%%%%%%%%%%%%%%%%%%%%%%%%%%%%%%%%%%%%%%%%%%
% Exercise 1.5
%%%%%%%%%%%%%%%%%%%%%%%%%%%%%%%%%%%%%%%%%%%%%%%%%%%%%%%%%%%%%%%%%%%%%%%%%%%%%%%%%%%%%%%%%%%%%%%%%%%%%%%%%%%%%%%%%%%%%%%%%%%%%%%%%%%%%%%%
\begin{problem}{1.5}
Suppose \(E\) is a given set, and \(\mathcal{O}_n\) is the open set:
\[\mathcal{O}_n=\left\{ x:\ d(x,E)<\frac{1}{n} \right\}.\]
Show:
\begin{enumerate}[(a)]
    \item If \(E\) is compact, then \(m(E)=\lim_{n\to \infty}<\frac{1}{n}\).
    \item However, the conclusion in (a) may be false for \(E\) closed and unbounded; or \(E\) open and bounded.
\end{enumerate}
\end{problem}
\begin{solution}

\end{solution}

\noindent\rule{7in}{2.8pt}
%%%%%%%%%%%%%%%%%%%%%%%%%%%%%%%%%%%%%%%%%%%%%%%%%%%%%%%%%%%%%%%%%%%%%%%%%%%%%%%%%%%%%%%%%%%%%%%%%%%%%%%%%%%%%%%%%%%%%%%%%%%%%%%%%%%%%%%%
% Exercise 1.6
%%%%%%%%%%%%%%%%%%%%%%%%%%%%%%%%%%%%%%%%%%%%%%%%%%%%%%%%%%%%%%%%%%%%%%%%%%%%%%%%%%%%%%%%%%%%%%%%%%%%%%%%%%%%%%%%%%%%%%%%%%%%%%%%%%%%%%%%
\begin{problem}{1.6}
Using translations and dilations, prove the following: Let \(B\) be a ball in \(\mathbb{R}^d\) of radius \(r\). Then \(m(B)=v_d r^d\), where \(v_d=m(B_1)\), and \(B_1\) is the unit ball. 
\[B_1=\left\{ x\in \mathbb{R}^d: |x|<1 \right\}.\]
\end{problem}
\begin{solution}

\end{solution}

\noindent\rule{7in}{2.8pt}
%%%%%%%%%%%%%%%%%%%%%%%%%%%%%%%%%%%%%%%%%%%%%%%%%%%%%%%%%%%%%%%%%%%%%%%%%%%%%%%%%%%%%%%%%%%%%%%%%%%%%%%%%%%%%%%%%%%%%%%%%%%%%%%%%%%%%%%%
% Exercise 1.9
%%%%%%%%%%%%%%%%%%%%%%%%%%%%%%%%%%%%%%%%%%%%%%%%%%%%%%%%%%%%%%%%%%%%%%%%%%%%%%%%%%%%%%%%%%%%%%%%%%%%%%%%%%%%%%%%%%%%%%%%%%%%%%%%%%%%%%%%
\begin{problem}{1.9}
Give an example of an open set \(\mathcal{O}\) with the following property: the boundary of the closure of \(\mathcal{O}\) has positive Lebesgue measure.
\end{problem}
\begin{solution}

\end{solution}

\noindent\rule{7in}{2.8pt}
%%%%%%%%%%%%%%%%%%%%%%%%%%%%%%%%%%%%%%%%%%%%%%%%%%%%%%%%%%%%%%%%%%%%%%%%%%%%%%%%%%%%%%%%%%%%%%%%%%%%%%%%%%%%%%%%%%%%%%%%%%%%%%%%%%%%%%%%
% Exercise 1.10
%%%%%%%%%%%%%%%%%%%%%%%%%%%%%%%%%%%%%%%%%%%%%%%%%%%%%%%%%%%%%%%%%%%%%%%%%%%%%%%%%%%%%%%%%%%%%%%%%%%%%%%%%%%%%%%%%%%%%%%%%%%%%%%%%%%%%%%%
\begin{problem}{1.10}
Let \(\hat{\mathcal{C}}\) denote a Canton-like set, in particular \(m(\hat{\mathcal{C}})>0\). Let \(F_1\) denote a piecewise linear and continuous function on \([0,1]\), with \(F_1=1\) in the complement of the first interval removed in the construction of \(\hat{\mathcal{C}}\), \(F_1=0\) at the center of this interval, and \(0\leq F_1(x)\leq 1\) for all \(x\). Similarly, construct \(F_2=1\) in the complement of the intervals in stage two of the construction of \(\hat{\mathcal{C}}\), with \(F_2=0\) at the center of these intervals, and \(0\leq F_2\leq 1\). Continuing this way, let \(f_n=F_1\cdot F_2\cdots F_n\). Prove the following:
\begin{enumerate}[(a)]
    \item For all \(n\geq 1\) and all \(x\in [0,1]\), one has\(0\leq f_n(x)\leq 1\) and \(f_n(x)\geq f_{n+1}(x)\). Therefore, \(f_n(x)\) converges to a limit as \(n\to \infty\) which we denote by \(f(x)\). 
    \item The function is discontinuous at every point of \(\hat{\mathcal{C}}\). 
\end{enumerate}
\end{problem}
\begin{solution}

\end{solution}

\noindent\rule{7in}{2.8pt}
%%%%%%%%%%%%%%%%%%%%%%%%%%%%%%%%%%%%%%%%%%%%%%%%%%%%%%%%%%%%%%%%%%%%%%%%%%%%%%%%%%%%%%%%%%%%%%%%%%%%%%%%%%%%%%%%%%%%%%%%%%%%%%%%%%%%%%%%
% Exercise 1.13
%%%%%%%%%%%%%%%%%%%%%%%%%%%%%%%%%%%%%%%%%%%%%%%%%%%%%%%%%%%%%%%%%%%%%%%%%%%%%%%%%%%%%%%%%%%%%%%%%%%%%%%%%%%%%%%%%%%%%%%%%%%%%%%%%%%%%%%%
\begin{problem}{1.13}
The following deals with \(G_\delta\) and \(F_\sigma\) sets. 
\begin{enumerate}[(a)]
  \item Show that a closed set is a \(G_\delta\) and an open set an \(F_\sigma\). 
  \item Give an example of an \(F_\sigma\) which is not a \(G_\delta\).
  \item Give an example of a Borel set which is not a \(G_\delta\) nor an \(F_\sigma\). 
\end{enumerate}
\end{problem}
\begin{solution}

\end{solution}

\end{document}
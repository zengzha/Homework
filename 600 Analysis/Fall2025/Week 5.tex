\documentclass[letterpaper, 12pt]{article}

\usepackage{/Users/zhengz/Desktop/Math/Workspace/Homework1/homework}

%%%%%%%%%%%%%%%%%%%%%%%%%%%%%%%%%%%%%%%%%%%%%%%%%%%%%%%%%%%%%%%%%%%%%%%%%%%%%%%%%%%%%%%%%%%%%%%%%%%%%%%%%%%%%%%%%%%%%%%%%%%%%%%%%%%%%%%%
\begin{document}
%Header-Make sure you update this information!!!!
\noindent
%%%%%%%%%%%%%%%%%%%%%%%%%%%%%%%%%%%%%%%%%%%%%%%%%%%%%%%%%%%%%%%%%%%%%%%%%%%%%%%%%%%%%%%%%%%%%%%%%%%%%%%%%%%%%%%%%%%%%%%%%%%%%%%%%%%%%%%%
\large\textbf{Zhengdong Zhang} \hfill \textbf{Homework - Week 5 Exercises}   \\
Email: zhengz@uoregon.edu \hfill ID: 952091294 \\
\normalsize Course: MATH 616 - Real Analysis \hfill Term: Fall 2025 \\
Instructor: Professor Weiyong He \hfill Due Date: Nov 5th, 2025 \\
\noindent\rule{7in}{2.8pt}
\setstretch{1.1}
%%%%%%%%%%%%%%%%%%%%%%%%%%%%%%%%%%%%%%%%%%%%%%%%%%%%%%%%%%%%%%%%%%%%%%%%%%%%%%%%%%%%%%%%%%%%%%%%%%%%%%%%%%%%%%%%%%%%%%%%%%%%%%%%%%%%%%%%
% Exercise 5.1
%%%%%%%%%%%%%%%%%%%%%%%%%%%%%%%%%%%%%%%%%%%%%%%%%%%%%%%%%%%%%%%%%%%%%%%%%%%%%%%%%%%%%%%%%%%%%%%%%%%%%%%%%%%%%%%%%%%%%%%%%%%%%%%%%%%%%%%%
\begin{problem}{5.1}
Find the limit of the integral when \(n\to \infty\) with justification.
\[\lim_{n\to \infty}\int_0^n \frac{\cos^2(\frac{x}{n})}{x^2+\cos^2(\frac{x}{n})}dx\]
\end{problem}
\begin{solution}
Let 
\[f_n(x)=\mathbbm{1}_{[0,n]}\frac{\cos^2(\frac{x}{n})}{x^2+\cos^2(\frac{x}{n})}\]
be a sequence of functions defined on \((0,+\infty)\). It is easy to see that \(\lim_{n\to \infty}f_n(x)=\frac{1}{x^2+1}\) for any \(x>0\). Consider the function 
\[g(x)=\mathbbm{1}_{(0,1)}\cdot 1+\mathbbm{1}_{[1,+\infty)}\cdot \frac{1}{x^2}.\]
For any \(x\in [0,1]\), we have 
\[0\leq f_n(x)=\frac{\cos^2(\frac{x}{n})}{x^2+\cos^2(\frac{x}{n})}\leq 1=g(x).\] 
and for any \(x\geq 1\), we have 
\[0\leq f_n(x)\leq \frac{\cos^2(\frac{x}{n})}{x^2+\cos^2(\frac{x}{n})}\leq \frac{1}{x^2}=g(x).\]
So \(|f_n(x)|\leq g(x)\) for any \(x>0\) and note that 
\[\int_0^\infty g(x)dx=\int_0^1 1dx+\int_1^\infty \frac{1}{x^2}dx=1+1=2<\infty.\]
So \(g\in L^1\). By Lebesgue dominated convergence theorem
\begin{align*}
     \lim_{n\to \infty}\int_0^n \frac{\cos^2(\frac{x}{n})}{x^2+\cos^2(\frac{x}{n})}dx
     &=\lim_{n\to \infty}\int_0^\infty f_n(x)dx\\[0.2em]
     &=\lim_{n\to \infty}\int_0^\infty \mathbbm{1}_{[0,n]}\frac{\cos^2(\frac{x}{n})}{x^2+\cos^2(\frac{x}{n})}dx\\[0.2em] 
     &=\int_0^\infty \lim_{n\to \infty}\mathbbm{1}_{[0,n]}\frac{\cos^2(\frac{x}{n})}{x^2+\cos^2(\frac{x}{n})}dx\\[0.2em]
     &=\int_0^\infty \frac{1}{x^2+1}dx\\[0.2em]
     &=\lim_{x\to \infty}\arctan x-\arctan 0\\[0.2em]
     &=\frac{\pi}{2}.
\end{align*}
\end{solution}

\noindent\rule{7in}{2.8pt}
%%%%%%%%%%%%%%%%%%%%%%%%%%%%%%%%%%%%%%%%%%%%%%%%%%%%%%%%%%%%%%%%%%%%%%%%%%%%%%%%%%%%%%%%%%%%%%%%%%%%%%%%%%%%%%%%%%%%%%%%%%%%%%%%%%%%%%%%
% Exercise 5.2
%%%%%%%%%%%%%%%%%%%%%%%%%%%%%%%%%%%%%%%%%%%%%%%%%%%%%%%%%%%%%%%%%%%%%%%%%%%%%%%%%%%%%%%%%%%%%%%%%%%%%%%%%%%%%%%%%%%%%%%%%%%%%%%%%%%%%%%%
\begin{problem}{5.2}
Let \(f_n\) be a sequence of real-valued measurable functions in \(L^1(\mu)\) on a measure space \((X,\mu)\). Suppose \(\lim_{n\to \infty}f_n=f\) a.e. and 
\[\lim_{n\to \infty}\int_X|f_n|d\mu=\int_X|f|d\mu.\]
Prove that for any measurable set \(A\), we have 
\[\lim_{n\to \infty}\int_A |f_n|d\mu=\int_A|f|d\mu.\]
Is that true
\[\lim_{n\to \infty}\int_A f_n d\mu=\int_A fd\mu?\]
\end{problem}
\begin{solution}
For any measurable set \(A\subset X\), we know that \(|f_n|\) and \(|f|\) are positive measurable functions on \(A\). By Fatou's lemma,
\[\int_X|f|d\mu=\int_X \liminf_{n\to \infty}|f_n|d\mu\leq \liminf_{n\to \infty}\int_X |f_n|d\mu.\]
On the other hand, by Fatou's lemma and use that fact that
\[\lim_{n\to \infty}\int_X|f_n|d\mu=\int_X|f|d\mu,\]
we have 
\begin{align*}
     0\leq \int_{X\setminus A}|f|d\mu 
      &=\int_{X\setminus A}\liminf_{n\to \infty}|f_n|d\mu\\
      &\leq \liminf_{n\to \infty} \int_{X\setminus A}|f_n|d\mu\\
      &=\liminf_{n\to \infty}(\int_X|f_n|d\mu-\int_A|f_n|d\mu)\\
      &=\int_X|f|d\mu-\limsup_{n\to \infty}\int_A|f_n|d\mu.
\end{align*}
This implies that 
\[\limsup_{n\to \infty}\int_A|f_n|d\mu\leq \int_X|f|d\mu-\int_{X\setminus A}|f|d\mu=\int_A|f|d\mu.\]
Combine this with 
\[\int_A|f|d\mu\leq \liminf_{n\to \infty}\int_A|f_n|d\mu,\]
We obtain that for any measurable set \(A\subset X\), 
\[\lim_{n\to \infty}\int_A|f_n|d\mu=\int_A|f|d\mu.\]

Consider \(X=A=\mathbb{R}^1\) and the sequence
\[f_n=\mathbbm{1}_{[-n,0)}\cdot (-1)+\mathbbm{1}_{(0,n]}.\]
Note that for any \(n\),
\[\int_{\mathbb{R}^1}|f_n|d\mu=2n<+\infty,\]
So \(\left\{ f_n \right\}\) is a sequence of measurable functions in \(L^1(\mu)\) and \(f_n\) converges to 
\[f=\mathbbm{1}_{(-\infty,0)}\cdot (-1)+\mathbbm{1}_{(0,+\infty)}\]
almost everywhere and 
\[\lim_{n\to \infty}\int_{\mathbb{R}^1}|f_n|d\mu=\lim_{n\to \infty}2n=+\infty=\int_{\mathbb{R}^1}|f|d\mu.\]
But on the other hand,
\[\int_{\mathbb{R}^`1}f_nd\mu=-m([-n,0))+m((0,n])=-n+n=0.\]
So 
\[\lim_{n\to \infty}\int_{\mathbb{R}^1}f_nd\mu=0\] 
while
\[\int_{\mathbb{R}^1}fd\mu\]
does not exist because both \(\int_{\mathbb{R}^1}f^+d\mu\) and \(\int_{\mathbb{R}^1}f^-d\mu\) are infinite. 
\end{solution}

\noindent\rule{7in}{2.8pt}
%%%%%%%%%%%%%%%%%%%%%%%%%%%%%%%%%%%%%%%%%%%%%%%%%%%%%%%%%%%%%%%%%%%%%%%%%%%%%%%%%%%%%%%%%%%%%%%%%%%%%%%%%%%%%%%%%%%%%%%%%%%%%%%%%%%%%%%%
% Exercise 5,3
%%%%%%%%%%%%%%%%%%%%%%%%%%%%%%%%%%%%%%%%%%%%%%%%%%%%%%%%%%%%%%%%%%%%%%%%%%%%%%%%%%%%%%%%%%%%%%%%%%%%%%%%%%%%%%%%%%%%%%%%%%%%%%%%%%%%%%%%
\begin{problem}{5,3}
Let \(E\subset \mathbb{R}^1\) be a measurable set with respect to Lebesgue measure and \(m(E)>0\). Prove that for any \(0<\alpha<1\), there exists an open interval \(I\) such that 
\[m(E\cap I)>\alpha m(I).\]
Given \(\frac{3}{4}<\alpha<1\), let \(I\) be an open interval with length \(b=m(I)\) such that 
\[m(E\cap I)>\alpha m(I).\]
Prove that the set 
\[E-E:=\left\{ x-y:x,y\in E \right\}\]
contains an open interval \((-\frac{b}{2},\frac{b}{2})\).
\end{problem}
\begin{solution}
We first assume \(m(E)<\infty\). Assume the opposite that for any open interval \(I\subset \mathbb{R}\), we have 
\[m(E\cap I)\leq \alpha m(I).\]
By the outer regularity of Lebesgue measure, for any \(\varepsilon>0\), there exists an open set \(U\supset E\) such that 
\[m(E)<m(U)<m(E)+\varepsilon.\]
We know that the open set \(U\subset \mathbb{R}\) can be written as 
\[U=\bigcup_{n=1}^\infty I_n\]
where \(I_n\) are open intervals and \(I_i\cap I_j=\varnothing\) if \(i\neq j\). Then 
\begin{align*}
  m(E)&=m(E\cap V)\\
      &=\sum_{n=1}^{\infty}m(E\cap I_n)\\
      &\leq \alpha \sum_{n=1}^{\infty} m(I_n)\\
      &\leq \alpha m(V)\\
      &<\alpha m(E)+\alpha \varepsilon.
\end{align*}
Let \(\varepsilon\to 0\), and we obtain that \(m(E)\leq \alpha m(E)\), but \(0<\alpha<1\). A contradiction. Thus, there exists an open interval \(I\) such that 
\[m(E\cap I)>\alpha m(I).\]
Now assume \(m(E)=+\infty\). Then for a large number \(N\), consider the measurable set 
\[E':=E\cap (-N,N)\]
with finite measure. We have proved that there exists an open interval \(I\) such that 
\[m(E'\cap I)>\alpha m(I).\]
If we choose \(N\) large enough, then 
\[m(E\cap I)\geq m(E'\cap I)>\alpha m(I).\]

Now suppose \(\frac{3}{4}<\alpha<1\). For any point \(z\in (-\frac{b}{2},\frac{b}{2})\), to prove that \(z\in E-E\), it is the same as proving there exists \(x,y\in E\) such that \(x=y+z\), which is the same as 
\[E\cap (E+z)\neq \varnothing\]
for any \(z\in (-\frac{b}{2},\frac{b}{2})\). Assume the opposite that there exists \(|z|<\frac{b}{2}\) such that 
\[E\cap (E+z)=\varnothing.\]
Note that \((E+z)\cap (I+z)=(E\cap I)+z\subset I+z\). Since \(E\) does not intersect \(E+z\), by the translation invariance of Lebesgue measure,
\begin{align*}
     m(I\cup (I+z))&=2m((E\cap I)\cup((E\cap I)+z))\\
                  &\geq m(E\cap I)+m((E\cap I)+z)\\
                  &> \alpha m(I)+\alpha m(I)\\
                  &=2\alpha m(I)\\
                  &>\frac{3}{2}m(I)
\end{align*}
Here \(|z|<\frac{b}{2}=\frac{1}{2}m(I)\), so 
\[m(I\cup (I+z))<m(I)+m(I)-\frac{1}{2}m(I)=\frac{3}{2}m(I).\]
This is a contradiction, so \(E\cap E+z\neq \varnothing\).
\end{solution}

\end{document}
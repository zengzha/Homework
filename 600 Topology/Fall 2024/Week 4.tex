\documentclass[a4paper, 11pt]{article}

\usepackage{/Users/zhengz/Desktop/Math/Workspace/Homework1/homework}
%%%%%%%%%%%%%%%%%%%%%%%%%%%%%%%%%%%%%%%%%%%%%%%%%%%%%%%%%%%%%%%%%%%%%%%%%%%%%%%%%%%%%%%%%%%%%%%%%%%%%%%%%%%%%%%%%%%%%%%%%%%%%%%%%%%%%%%%
\begin{document}
%Header-Make sure you update this information!!!!
\noindent
%%%%%%%%%%%%%%%%%%%%%%%%%%%%%%%%%%%%%%%%%%%%%%%%%%%%%%%%%%%%%%%%%%%%%%%%%%%%%%%%%%%%%%%%%%%%%%%%%%%%%%%%%%%%%%%%%%%%%%%%%%%%%%%%%%%%%%%%
\large\textbf{Zhengdong Zhang} \hfill \textbf{Homework - Week 4}   \\
Email: zhengz@uoregon.edu \hfill ID: 952091294 \\
\normalsize Course: MATH 634 - Algebraic Topology  \hfill Term: Fall 2024\\
Instructor: Dr.Patricia Hersh \hfill Due Date: $29^{nd}$ October, 2024 \\
\noindent\rule{7in}{2.8pt}
%%%%%%%%%%%%%%%%%%%%%%%%%%%%%%%%%%%%%%%%%%%%%%%%%%%%%%%%%%%%%%%%%%%%%%%%%%%%%%%%%%%%%%%%%%%%%%%%%%%%%%%%%%%%%%%%%%%%%%%%%%%%%%%%%%%%%%%%
% Exercise 2.1.15
%%%%%%%%%%%%%%%%%%%%%%%%%%%%%%%%%%%%%%%%%%%%%%%%%%%%%%%%%%%%%%%%%%%%%%%%%%%%%%%%%%%%%%%%%%%%%%%%%%%%%%%%%%%%%%%%%%%%%%%%%%%%%%%%%%%%%%%%
\begin{problem}{2.1.15}
For an exact sequence \(A\rightarrow B\rightarrow C\rightarrow D\rightarrow E\) show that \(C=0\) iff the map \(A\rightarrow B\) is surjective and 
\(D\rightarrow E\) is injective. Hence for a pair of spaces \((X,A)\), the inclusion \(A\hookrightarrow X\) induces isomorphisms on all homology groups iff 
\(H_n(X,A)=0\) for all \(n\).
\end{problem}   
\begin{solution}
Assume \(C=0\). Then \(0\rightarrow D\rightarrow E\) implies that \(\ker (D\rightarrow E)=\text{im}(0\rightarrow D)=0\). So\(D\rightarrow E\) is injective. Similarly, 
\(\text{im}(A\rightarrow B)=\ker (B\rightarrow 0)=B\) implies  that \(A\rightarrow B\) is surjective. Conversely, \(A\rightarrow B\) is surjective implies that \(B=\text{im}(A\rightarrow B)=\ker (B\rightarrow C)\). 
So \(B\rightarrow C\) is the zero map. And \(\text{im}(B\rightarrow C)=\ker (C\rightarrow D)\) while \(C\rightarrow D\) is also the zero map because \(\text{im}(C\rightarrow D)=\ker (D\rightarrow E)=0\) as \(D\rightarrow E\) 
is injective. So \(\text{im}(B\\rightarrow C)=0\). Thus, we have \(C=0\). 
\par 
Let \((A,X)\) be a pair of spaces. Consider the short exact sequence of chain complexes:
$$0\rightarrow C_\bullet(A)\rightarrow C_\bullet(X)\rightarrow C_\bullet(X,A)\rightarrow 0.$$
We have a long exact sequence:
$$\begin{tikzcd}
	\cdots & {H_n(A)} & {H_n(X)} & {H_n(X,A)} \\
	& {H_{n-1}(A)} & {H_{n-1}(X)} & {H_{n-1}(X,A)} & \cdots
	\arrow[from=1-1, to=1-2]
	\arrow[from=1-2, to=1-3]
	\arrow[from=1-3, to=1-4]
	\arrow[from=1-4, to=2-2]
	\arrow[from=2-2, to=2-3]
	\arrow[from=2-3, to=2-4]
	\arrow[from=2-4, to=2-5]
\end{tikzcd}$$
where \(H_n(A,X)=0\) for all \(n\) if and only if \(H_n(A)\xrightarrow{\sim} H_n(X)\) for all \(n\).
\end{solution}
\\ 
\noindent\rule{7in}{2.8pt}
%%%%%%%%%%%%%%%%%%%%%%%%%%%%%%%%%%%%%%%%%%%%%%%%%%%%%%%%%%%%%%%%%%%%%%%%%%%%%%%%%%%%%%%%%%%%%%%%%%%%%%%%%%%%%%%%%%%%%%%%%%%%%%%%%%%%%%%%
% Exercise 2.1.16 (a)
%%%%%%%%%%%%%%%%%%%%%%%%%%%%%%%%%%%%%%%%%%%%%%%%%%%%%%%%%%%%%%%%%%%%%%%%%%%%%%%%%%%%%%%%%%%%%%%%%%%%%%%%%%%%%%%%%%%%%%%%%%%%%%%%%%%%%%%%
\begin{problem}{2.1.16}
Show that \(H_0(X,A)=0\) iff \(A\) meets each path-component of \(X\).
\end{problem}
\begin{solution}
As the Exercise above shows that we have a long exact sequence:
$$\begin{tikzcd}
	\cdots & {H_0(A)} & {H_0(X)} & {H_0(X,A)} & 0
	\arrow[from=1-1, to=1-2]
	\arrow[from=1-2, to=1-3]
	\arrow[from=1-3, to=1-4]
	\arrow[from=1-4, to=1-5]
\end{tikzcd}$$
So \(H_0(X,A)=0\) if and only if \(H_0(A)\rightarrow H_0(X)\) is surjective. This map is induced by the inclusion \(A\hookrightarrow X\). 
For a topological space \(X\), we have proved in class that \(H_0(X)\cong \bigoplus \mathbb{Z}^d\) where each copy of \(\mathbb{Z}\) represents a path-component 
of \(X\). So \(H_0(A)\rightarrow H_0(X)\cong \bigoplus \mathbb{Z}^d\) is surjective if and only if for each path-component of \(X\), there exist a 0-chain \(a\in C_0(A)\) such that 
\(a\) is generated by points in that path-component. This is the same as saying \(A\) meets each path-component of \(X\).
\end{solution}
\\ 






\end{document}
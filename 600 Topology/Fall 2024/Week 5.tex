\documentclass[a4paper, 12pt]{article}

\usepackage{/Users/zhengz/Desktop/Math/Workspace/Homework1/homework}
%%%%%%%%%%%%%%%%%%%%%%%%%%%%%%%%%%%%%%%%%%%%%%%%%%%%%%%%%%%%%%%%%%%%%%%%%%%%%%%%%%%%%%%%%%%%%%%%%%%%%%%%%%%%%%%%%%%%%%%%%%%%%%%%%%%%%%%%
\begin{document}
%Header-Make sure you update this information!!!!
\noindent
%%%%%%%%%%%%%%%%%%%%%%%%%%%%%%%%%%%%%%%%%%%%%%%%%%%%%%%%%%%%%%%%%%%%%%%%%%%%%%%%%%%%%%%%%%%%%%%%%%%%%%%%%%%%%%%%%%%%%%%%%%%%%%%%%%%%%%%%
\large\textbf{Zhengdong Zhang} \hfill \textbf{Homework - Week 5}   \\
Email: zhengz@uoregon.edu \hfill ID: 952091294 \\
\normalsize Course: MATH 634 - Algebraic Topology  \hfill Term: Fall 2024\\
Instructor: Dr.Patricia Hersh \hfill Due Date: $7^{th}$ November, 2024 \\
\noindent\rule{7in}{2.8pt}
%%%%%%%%%%%%%%%%%%%%%%%%%%%%%%%%%%%%%%%%%%%%%%%%%%%%%%%%%%%%%%%%%%%%%%%%%%%%%%%%%%%%%%%%%%%%%%%%%%%%%%%%%%%%%%%%%%%%%%%%%%%%%%%%%%%%%%%%
% Exercise 2.1.20
%%%%%%%%%%%%%%%%%%%%%%%%%%%%%%%%%%%%%%%%%%%%%%%%%%%%%%%%%%%%%%%%%%%%%%%%%%%%%%%%%%%%%%%%%%%%%%%%%%%%%%%%%%%%%%%%%%%%%%%%%%%%%%%%%%%%%%%%
\begin{problem}{2.1.20}
Show that \(\tilde{H}_n(X)\approx \tilde{H}_{n+1}(SX)\) for all \(n\), where \(SX\) is the suspension of \(X\). More generally, thinking of \(SX\) as the union 
of two cones \(CX\) with their bases identified, compute the reduced homology groups of the union of any finite number of cones \(CX\) with their bases identified. 
\end{problem}
\begin{solution}
Let \(X\) be a topological space. The suspension \(SX\) can be viewed as two cones \(CX:=(X\times I)/(X\times \left\{ 1 \right\})\) glued together along the base \(X\times \left\{ 0 \right\}\). Assume 
\(SX=(X\times [-1,1])/(X\times \left\{ 1 \right\},X\times \left\{ -1 \right\})\). The two cones are given by \(A=(X\times [-0.1,1])/(X\times \left\{ 1 \right\})\) and \(B=(X\times [-1,0.1])/(X\times \left\{ -1 \right\})\). 
We know that \(\text{int}(A)\cup \text{int}(B)=SX\) and \(A\cap B=X\times [-0.1,0.1]\simeq  X\). The Mayer-Vietoris sequences give us a long exact sequence:
\[\cdots\rightarrow H_{n+1}(SX)\rightarrow H_n(X)\rightarrow H_n(A)\oplus H_n(B)\rightarrow H_n(SX)\rightarrow H_{n-1}(X)\rightarrow \cdots\]
\begin{claim}
For a topological space \(X\), the cone \(CX\) is contractible.
\end{claim}
\begin{claimproof}
Note that \(CX=(X\times [0,1])/(X\times \left\{ 1 \right\})\).  Consider the continous map 
\(F:(X\times [0,1])\times [0,1]\rightarrow X\times [0,1],\, (x,s,t)\mapsto (x,t+(1-t)s)\). We have \(F((x,s),0)=(x,s)\), \(F((x,s),1)=(x,1)\in X\times [0,1]\) and \(F((x,1),t)=(x,1)\). So \(f\) descends to a 
contionous map between the quotient space \(\tilde{F}:CX\times [0,1]\rightarrow CX\), which gives us a homotopy equivalence between \(CX\) and the one point subspace \(\left\{ p \right\}=X\times \left\{ 1 \right\}\in CX\).
\end{claimproof}\\ 
Since both \(A\) and \(B\) are cones, \(A\) and \(B\) are contractible and we have \(H_n(A)=H_n(B)=0\) for all \(n\). By the Mayer-Vietoris sequence, we have \(H_{n+1}(SX)\cong H_n(X)\) for all \(n\). 
\end{solution}

\noindent\rule{7in}{2.8pt}
%%%%%%%%%%%%%%%%%%%%%%%%%%%%%%%%%%%%%%%%%%%%%%%%%%%%%%%%%%%%%%%%%%%%%%%%%%%%%%%%%%%%%%%%%%%%%%%%%%%%%%%%%%%%%%%%%%%%%%%%%%%%%%%%%%%%%%%%
% Exercise 2.1.22
%%%%%%%%%%%%%%%%%%%%%%%%%%%%%%%%%%%%%%%%%%%%%%%%%%%%%%%%%%%%%%%%%%%%%%%%%%%%%%%%%%%%%%%%%%%%%%%%%%%%%%%%%%%%%%%%%%%%%%%%%%%%%%%%%%%%%%%%
\begin{problem}{2.1.22}
Prove by induction on dimension the following facts about the homology of a finite-dimensional CW complex \(X\), using the observation that \(X^n/X^{n-1}\) is a wedge sum of \(n\)-spheres:
\begin{enumerate}[(a)]
	\item If \(X\) has dimension \(n\), then \(H_i(X)=0\) for \(i>n\) and \(H_n(X)\) is free. 
	\item \(H_n(X)\) is free with basis in bijective correspondence with the \(n\)-cells if there are no cells of dimension \(n-1\) or \(n+1\).
	\item If \(X\) has \(k\) \(n\)-cells, then \(H_n(X)\) is generated by at most \(k\) elements.
\end{enumerate}
\end{problem}
\begin{solution}
\begin{enumerate}[(a)]
	\item \(X\) has dimension \(n\) implies \(X=X^n\), the \((n-1)\)-skeleton \(X^{n-1}\) is a subcomplex of \(X\) and the pair \((X,X^{n-1})\) is a good pair. By proposition 2.22, the quotient map 
	      \(q:(X,X^{n-1})\rightarrow (X/X^{n-1},X^{n-1}/X^{n-1})\) induces isomorphisms 
		  \[q_*:H_i(X,X^{n-1})\xrightarrow{\sim} H_i(X/X^{n-1},X^{n-1}/X^{n-1})\cong \tilde{H}_i(X/X^{n-1})\] 
		  for all \(i\). Note that \(X/X^{n-1}\cong X^n/X^{n-1}\) is a wedge sum of \(n\)-spheres, namely \(\bigvee_k S^n\), where \(k\) is the number of \(n\)-cells. We know that  by Corollary 2.25 for \(i=n\), 
		  \begin{align*}
			\tilde{H}_i(X/X^{n-1}) & \cong \tilde{H}_i(\bigvee_k S^n)\\ 
			                       & \cong \bigoplus_k \tilde{H}_i(S^n)\\ 
								   & \cong \bigoplus_k \mathbb{Z}\\ 
								   & \cong \mathbb{Z}^k
		  \end{align*}
		  otherwise, \(\tilde{H}_i(X/X^{n-1})=0\). 
		  \par 
		  To prove that \(H_i(X)=0\) for \(i>n\), we do an induction on the dimension \(n\). If \(X\) has dimension \(0\), then \(X\) is just a collection of disjoint union of points, we know only \(H_0(X)\neq 0\). Now Suppose 
		  the assumption is true for CW complex with dimension \(n-1\) and assume \(X\) has dimension \(n\). The inclusion of \((n-1)\)-skeleton \(X^{n-1}\) in \(X\) gives us a long exact sequence:
\[\begin{tikzcd}
	\cdots & {H_i(X^{n-1})} & {H_i(X)} & {H_i(X,X^{n-1})} \\
	& {H_{i-1}(X^{n-1})} & \cdots & {H_{n+1}(X,X^{n-1})} \\
	& {H_n(X^{n-1})} & {H_n(X)} & {H_n(X,X^{n-1})} & \cdots
	\arrow[from=1-1, to=1-2]
	\arrow[from=1-2, to=1-3]
	\arrow[from=1-3, to=1-4]
	\arrow[from=1-4, to=2-2]
	\arrow[from=2-2, to=2-3]
	\arrow[from=2-3, to=2-4]
	\arrow[from=2-4, to=3-2]
	\arrow[from=3-2, to=3-3]
	\arrow[from=3-3, to=3-4]
	\arrow[from=3-4, to=3-5]
\end{tikzcd}\]
Note that 
\[H_i(X^{n-1})=H_{i-1}(X^{n-1})=\cdots=H_n(X^{n-1})=0\]
by our assumption and 
\[H_i(X,X^{n-1})=H_{i-1}(X,X^{n-1})=\cdots=H_{n+1}(X,X^{n-1})=0\]
by our previous calculations. So \(H_i(X)=0\) for \(i>n\). Moreover, \(H_n(X^{n-1})=0\) implies that \(H_n(X)\rightarrow H_n(X,X^{n-1})\cong \mathbb{Z}^k\) is injective. Thus, \(H_n(X)\) has no torsion and is a free \(\mathbb{Z}\)-module.
\item First we prove that \(H_n(X)\cong H_n(X^n)\) where \(X^n\) is the \(n\)-skeleton of \(X\) by induction. Since \(X\) does not have \((n+1)\)-cells, we have \(X^{n+1}=X^n\), so \(H_n(X^n)=H_n(X^{n+1})\). Let \(m\geq 1\) be an integer and 
      assume we have proved \(H_n(X^{n+m})\cong H_n(X^{n+m-1})\), we are going to show that \(H_n(X^{n+m+1})\cong H_n(X^{n+m})\). The inclusion \(X^{n+m}\hookrightarrow X^{n+m+1}\) gives us a long exact sequence:
	  \[\cdots\rightarrow H_{n+1}(X^{n+m+1},X^{n+m})\rightarrow H_n(X^{n+m})\rightarrow H_n(X^{n+m+1})\rightarrow H_n(X^{n+m+1},X^{n+m})\rightarrow \cdots\]
	  From the discussion in (a), we know that 
	  \[H_{n+1}(X^{n+m+1},X^{n+m})\cong H_{n+1}(\bigvee_p S^{n+m+1})\cong \bigoplus_p H_{n+1}(S^{n+m+1})=0\]
	  because \(m\geq 1\) and similarly,
	  \[H_n(X^{n+m+1},X^{n+m})\cong H_n(\bigvee_q S^{n+m+1})\cong \bigoplus_q H_n(S^{n+m+1})=0\] 
      So we know that \(H_n(X^{n+m})\cong H_n(X^{n+m+1})\) and by our assumption, \(H_n(X^n)\cong H_n(X^{n+m})\) for any \(m\geq 1\). Note that \(X\) is a finite dimensional CW complex, so there exists \(m\geq 0\) such that \(X=X^m\), therefore we 
	  have \(H_n(X)=H_n(X^n)\).
	\par 
	The inclusion \(X^{n-1}\hookrightarrow X^n\) gives us a long exact sequence:
	\[\cdots\rightarrow H_n(X^{n-1})\rightarrow H_n(X^n)\rightarrow H_n(X^n,X^{n-1})\rightarrow H_{n-1}(X^{n-1})\rightarrow \cdots\]
	Because \(X\) does not have \((n-1)\)-cells, so \(X^{n-1}=X^{n-2}\). By the discussion in (a), we know that \(H_n(X^{n-2})=H_{n-1}(X^{n-2})=0\) and \(H_n(X^n,X^{n-1})=\bigoplus_k \mathbb{Z}^k\) where \(k\) is the number of \(n\)-cells. So we have 
	\(H_n(X)\cong H_n(X^n)\cong \bigoplus_k \mathbb{Z}^k\), which means \(H_n(X)\) is free with basis in bijective with the \(n\)-cells.
\item The discussion in (b) gives us \(H_n(X^{n+m})\cong H_n(X^{n+m+1})\) for \(m\geq 1\) and since \(X\) is finite dimensional, we have \(H_n(X)\cong H_n(X^{n+1})\). Now consider the inclusion \(X^n\hookrightarrow X^{n+1}\) and we have a long exact sequence:
      \[\begin{tikzcd}
	& \cdots & {H_{n+1}(X^{n+1},X^n)} \\
	{H_n(X^n)} & {H_n(X^{n+1})} & {H_n(X^{n+1},X^n)} & \cdots
	\arrow[from=1-2, to=1-3]
	\arrow["\partial"', from=1-3, to=2-1]
	\arrow[from=2-1, to=2-2]
	\arrow[from=2-2, to=2-3]
	\arrow[from=2-3, to=2-4]
\end{tikzcd}\]
Note that \(H_n(X^{n+1},X^n)\cong \tilde{H}_n(\bigvee S^{n+1})=0\), so we have a short exact sequence:
\[0\rightarrow \text{im}\partial \rightarrow H_n(X^n)\rightarrow H_n(X^{n+1})\rightarrow 0\]
where \(\partial:H_{n+1}(X^{n+1},X^n)\rightarrow H_n(X^n)\) is the connecting homomorphism. Since \(X^n\) has dimension \(n\) as a CW complex, from the discussion in (a), we know that \(H_n(X^n)=\bigoplus_k \mathbb{Z}\) is generated by \(k\) elements and from the 
short exact sequence we know that \(H_n(X)\cong H_n(X^{n+1})\cong H_n(X^n)/\text{im}\partial\), so it is generated by at most \(k\) elements.
\end{enumerate}
\end{solution}

\noindent\rule{7in}{2.8pt}

%%%%%%%%%%%%%%%%%%%%%%%%%%%%%%%%%%%%%%%%%%%%%%%%%%%%%%%%%%%%%%%%%%%%%%%%%%%%%%%%%%%%%%%%%%%%%%%%%%%%%%%%%%%%%%%%%%%%%%%%%%%%%%%%%%%%%%%%
% Exercise 2.1.23
%%%%%%%%%%%%%%%%%%%%%%%%%%%%%%%%%%%%%%%%%%%%%%%%%%%%%%%%%%%%%%%%%%%%%%%%%%%%%%%%%%%%%%%%%%%%%%%%%%%%%%%%%%%%%%%%%%%%%%%%%%%%%%%%%%%%%%%%
\begin{problem}{2.1.23}
Show that the second barycentric subdivision of a \(\Delta\)-complex is a simplicial complex. Namely, show that the first barycentric subdivision produces a \(\Delta\)-complex 
with the property that each simplex has all its vertices distinct, then show that for a \(\Delta\)-complex with this property, barycentric subdivision produces a simplicial complex.
\end{problem}
\begin{solution}
Let \(X\) be a \(\Delta\)-complex, and denote the \(\Delta\)-complex obtained after first (respectively second)	barycentric subdivision by \(X_1\) (respectively \(X_2\)). 
We first prove that for each \(n\)-simplex \([v_0,\ldots,v_n]\in X_1\), none of its vertices \(v_i\,(i=0,1,\ldots,n)\) is identified with another. Note that in \(v_0,\ldots,v_n\), only one vertex is from \(X\), the other are all barycenters of 
simplices in \(X\). This vertex cannot be identified with a barycenter since we choose every barycentre using different coordiantes from the vertices in \(X\). Now suppose two barycenters are identified in \(X_1\), it is only possible when they are barycenters 
for different \(k\)-simplices (\(k\leq n\)) because the gluing of faces in \(\Delta\)-complex requires homeomorphisms. But each of the barycenters in \([v_0,\ldots,v_n]\) is the barycenter of faces from last step subdivision, so any two of them cannot be identified in \(X_1\). 
\par 
Now we are going to show that \(X_2\) is a simplicial complex by showing that if two \(n\)-simplices in \(X_2\) have the same set of vertices, then they are the same \(n\)-simplex in \(X_2\). Given a set of distinct vertices \(\left\{ v_0,v_1,\ldots,v_n \right\}\), 
if this is the vertices set of two distinct \(n\)-simplices, then suppose there are two vertices which are connected by two different \(1\)-simplices, meaning two barycenters get identified in \(X_1\). This only happens when they are barycenters of different faces in one simplex in \(X_1\), but 
we have shown that in \(X_1\), the vertices for any simplex are distinct, so we cannot identify faces of one simplex in \(X_1\). Thus,this cannot happen and we proved that \(X_2\) is a simplicial complex.
\end{solution}

\noindent\rule{7in}{2.8pt}
%%%%%%%%%%%%%%%%%%%%%%%%%%%%%%%%%%%%%%%%%%%%%%%%%%%%%%%%%%%%%%%%%%%%%%%%%%%%%%%%%%%%%%%%%%%%%%%%%%%%%%%%%%%%%%%%%%%%%%%%%%%%%%%%%%%%%%%%
% Exercise 2.1.27(a)
%%%%%%%%%%%%%%%%%%%%%%%%%%%%%%%%%%%%%%%%%%%%%%%%%%%%%%%%%%%%%%%%%%%%%%%%%%%%%%%%%%%%%%%%%%%%%%%%%%%%%%%%%%%%%%%%%%%%%%%%%%%%%%%%%%%%%%%%
\begin{problem}{2.1.27(a)}
Let \(f:(X,A)\rightarrow (Y.B)\) be a map such that both \(f:X\rightarrow Y\) and the restriction \(f:A\rightarrow B\) are homotopy equivalences. Show that \(f_*:H_n(X,A)\rightarrow H_n(Y,B)\) is 
an isomorphism for all \(n\).
\end{problem}
\begin{solution}
By naturality of the long exact sequence, the map \(f:(X,A)\rightarrow (Y,B)\) induces the following commutative diagram:
\[\begin{tikzcd}
	\cdots & {H_n(A)} & {H_n(X)} & {H_n(X,A)} & {H_{n-1}(A)} & \cdots \\
	\cdots & {H_n(B)} & {H_n(Y)} & {H_n(Y,B)} & {H_{n-1}(B)} & \cdots
	\arrow[from=1-1, to=1-2]
	\arrow[from=1-2, to=1-3]
	\arrow["{f_*}", from=1-2, to=2-2]
	\arrow[from=1-3, to=1-4]
	\arrow["{f_*}", from=1-3, to=2-3]
	\arrow[from=1-4, to=1-5]
	\arrow["{f_*}", from=1-4, to=2-4]
	\arrow[from=1-5, to=1-6]
	\arrow["{f_*}", from=1-5, to=2-5]
	\arrow[from=2-1, to=2-2]
	\arrow[from=2-2, to=2-3]
	\arrow[from=2-3, to=2-4]
	\arrow[from=2-4, to=2-5]
	\arrow[from=2-5, to=2-6]
\end{tikzcd}\]
Note that \(f:X\rightarrow Y\) is a homotopy equivalence, so \(f_*:H_n(X)\rightarrow H_n(Y)\) is an isomorphism for all \(n\). The map \(f_*:H_n(A)\rightarrow H_n(B)\)	is induced bu the restriction \(f:A\rightarrow B\), which 
is also a homotopy equivalence, so \(f_*:H_n(A)\rightarrow H_n(B)\) is also an isomorphism for all \(n\). By the five lemma, the map \(f_*:H_n(X,A)\rightarrow H_n(Y,B)\) is an isomorphism for all \(n\).
\end{solution}





\end{document}
\documentclass[a4paper, 12pt]{article}

\usepackage{/Users/zhengz/Desktop/Math/Workspace/Homework1/homework}
%%%%%%%%%%%%%%%%%%%%%%%%%%%%%%%%%%%%%%%%%%%%%%%%%%%%%%%%%%%%%%%%%%%%%%%%%%%%%%%%%%%%%%%%%%%%%%%%%%%%%%%%%%%%%%%%%%%%%%%%%%%%%%%%%%%%%%%%
\begin{document}
%Header-Make sure you update this information!!!!
\noindent
%%%%%%%%%%%%%%%%%%%%%%%%%%%%%%%%%%%%%%%%%%%%%%%%%%%%%%%%%%%%%%%%%%%%%%%%%%%%%%%%%%%%%%%%%%%%%%%%%%%%%%%%%%%%%%%%%%%%%%%%%%%%%%%%%%%%%%%%
\large\textbf{Zhengdong Zhang} \hfill \textbf{Homework - Week 6}   \\
Email: zhengz@uoregon.edu \hfill ID: 952091294 \\
\normalsize Course: MATH 634 - Algebraic Topology  \hfill Term: Fall 2024\\
Instructor: Dr.Patricia Hersh \hfill Due Date: $16^{th}$ November, 2024 \\
\noindent\rule{7in}{2.8pt}
%%%%%%%%%%%%%%%%%%%%%%%%%%%%%%%%%%%%%%%%%%%%%%%%%%%%%%%%%%%%%%%%%%%%%%%%%%%%%%%%%%%%%%%%%%%%%%%%%%%%%%%%%%%%%%%%%%%%%%%%%%%%%%%%%%%%%%%%
% Exercise 2.1.24
%%%%%%%%%%%%%%%%%%%%%%%%%%%%%%%%%%%%%%%%%%%%%%%%%%%%%%%%%%%%%%%%%%%%%%%%%%%%%%%%%%%%%%%%%%%%%%%%%%%%%%%%%%%%%%%%%%%%%%%%%%%%%%%%%%%%%%%%
\begin{problem}{2.1.24}
Show that each \(n\)-simplex in the barycentric subdivision of \(\Delta^n\) is defined by \(n\) inequalities \(t_{i_0}\leq t_{i_1}\leq \cdots \leq t_{i_n}\) in its 
barycentric coordinates, where \((i_0,\ldots,i_n)\) is a permutation of \((0,\ldots,n)\). 
\end{problem}
\begin{solution}
Let \([v_0\ldots,v_n]\) be a standard \(n\)-simplex. We prove this using the induction on \(n\). 
\par 
When \(n=1\), under barycentric coordinates, the \(1\)-simplex is an interval \([v_0,v_1]\) with two vertices \(v_0=(1,0)\) and \(v_1=(0,1)\). The barycenter is \((\frac{1}{2},\frac{1}{2})\). After barycentric 
subdivision, the 2 \(1\)-simplices are just \((t,1-t)\) given by \(0\leq t\leq \frac{1}{2}\) and \(\frac{1}{2}\leq t\leq 1\) respectively. So it satifies the assumption. 
\par 
Now assume \(n\geq 2\) and we have prove the case for \(n-1\). The barycenter for \([v_0,\ldots,v_n]\) has coordinates \(b=(\frac{1}{n+1},\ldots,\frac{1}{n+1})\). Consider one of its faces \([v_0,\ldots,\hat{v_k},\ldots,v_n]\), by our assumption we know that 
each \((n-1)\)-simplex after the barycentric subdivision in this face is given by an equality \(0\leq t_{i_0}\leq \cdots\leq t_{i_{k-1}}\leq t_{i_{k+1}}\leq \cdots\leq t_{i_n}\) where \((i_0,i_1,\ldots,i_{k-1},i_{k+1},\ldots,i_n )\) is a permutation of \((0,1,\ldots,k-1,k+1,\ldots,n)\). Fix such a 
\((n-1)\)-simplex \(\Delta^{n-1}\) (namely, an inequality as above), we will try to describe any point \(x=(t_0,t_1,\ldots,t_n)\) in the \(n\)-simplex formed using vertices from \(\Delta^{n-1}\) and \(b\). Consider the line passing through \(x\) and \(b\) and it intersects with \(\Delta^{n-1}\) at the point \(y\in \Delta^{n-1}\). 
By colinearity we can write the coordinate 
\[y=(\frac{t_0-t_k}{n+1},\frac{t_1-t_k}{n+1},\ldots,\frac{t_{k-1}-t_k}{n+1},0,\frac{t_{k+1}-t_k}{n+1},\ldots,\frac{t_n-t_k}{n+1}).\]
The inequality implies that 
\[0\leq \frac{t_{i_0}-t_k}{n+1}\leq \frac{t_{i_1}-t_k}{n+1}\leq\cdots \leq\frac{t_{i_{k-1}}-t_k}{n+1}\leq \frac{t_{i_{k+1}}-t_k}{n+1}\leq \cdots\leq \frac{t_{i_n}-t_k}{n+1}\leq 1.\]
Combine this with the requirements that the coordinate \(\frac{t_{i_j}-t_k}{n+1}\geq 0\) for all \(j=0,1,\ldots,k-1,k+1,\ldots,n\) gives us a total order
\[0\leq t_k\leq t_{i_0}\leq t_{i_1}\leq \cdots\leq t_{i_{k-1}}\leq t_{i_{k+1}}\leq \cdots\leq t_{i_n}\leq 1.\]
Varying \(k=0,1,\ldots,n\) and repeat the same process for each face we run through all the \(n\)-simplex in the barycentric subdivision, giving us an inequality as above each time. We are done. 
\end{solution}

\noindent\rule{7in}{2.8pt}
%%%%%%%%%%%%%%%%%%%%%%%%%%%%%%%%%%%%%%%%%%%%%%%%%%%%%%%%%%%%%%%%%%%%%%%%%%%%%%%%%%%%%%%%%%%%%%%%%%%%%%%%%%%%%%%%%%%%%%%%%%%%%%%%%%%%%%%%
% Exercise 2.1.26
%%%%%%%%%%%%%%%%%%%%%%%%%%%%%%%%%%%%%%%%%%%%%%%%%%%%%%%%%%%%%%%%%%%%%%%%%%%%%%%%%%%%%%%%%%%%%%%%%%%%%%%%%%%%%%%%%%%%%%%%%%%%%%%%%%%%%%%%
\begin{problem}{2.1.26}
Show that \(H_1(X,A)\) is not isomorphic to \(\tilde{H}_1(X/A)\) if \(X=[0,1]\) and \(A\) is the sequence \(1,\frac{1}{2},\frac{1}{3},\ldots\) together with its limit \(0\).
\end{problem}
\begin{solution}
We first use the long exact sequence for relative homology to calculate \(H_1(X,A)\)
\[\begin{tikzcd}
	\cdots & {H_1(X)} & {H_1(X,A)} & {H_0(A)} & {H_0(X)} & \cdots
	\arrow[from=1-1, to=1-2]
	\arrow[from=1-2, to=1-3]
	\arrow[from=1-3, to=1-4]
	\arrow[from=1-4, to=1-5]
	\arrow[from=1-5, to=1-6]
\end{tikzcd}\]
We know that \(X=[0,1]\) is contractible, so \(H_1(X)=0\). Note that \(A\) is countable many points \(\left\{ 0,1,\frac{1}{2},\frac{1}{3},\ldots \right\}\), so by Proposition 2.7, \(H_0(A)\cong \bigoplus_{i=1}^\infty \mathbb{Z}\). We have an injective map \(\partial:H_1(X,A)\rightarrow H_0(A)\cong \bigoplus_{i=1}^\infty \mathbb{Z}\). 
On the other hand, the quotient space \(X/A\) is homeomorphic to the shrinking wedge of circles, each interval \([\frac{1}{n+1},\frac{1}{n}]\) in \(X/A\) is a small circle, the radius of which is shrinking as \(n\) gets larger. Denote this circle as \(C_n\). Consider the 
retraction \(r_n:X/A\rightarrow C_n\) collapsing all circles except \(C_n\) to the point represented by \(A\). The induced map in homology \(r_{n,*}:H_1(X/A)\rightarrow H_1(C_n)\cong \mathbb{Z}\) is surjective. Moreover, by the universal property of products, we have a surjective map \(r_*:H_1(X/A)\rightarrow \prod_{i=1}^\infty \mathbb{Z}\). Note that for 
any topological space, \(H_1(X/A)\cong \tilde{H}_1(X/A)\). And \(\prod_{i=1}^\infty \mathbb{Z}\) is not isomorphic to \(\bigoplus_{i=1}^\infty \mathbb{Z}\), so \(H_1(X,A)\) cannot be isomorphic to \(\tilde{H}_1(X/A)\).
\end{solution}

\noindent\rule{7in}{2.8pt}
%%%%%%%%%%%%%%%%%%%%%%%%%%%%%%%%%%%%%%%%%%%%%%%%%%%%%%%%%%%%%%%%%%%%%%%%%%%%%%%%%%%%%%%%%%%%%%%%%%%%%%%%%%%%%%%%%%%%%%%%%%%%%%%%%%%%%%%%
% Exercise 2.1.30
%%%%%%%%%%%%%%%%%%%%%%%%%%%%%%%%%%%%%%%%%%%%%%%%%%%%%%%%%%%%%%%%%%%%%%%%%%%%%%%%%%%%%%%%%%%%%%%%%%%%%%%%%%%%%%%%%%%%%%%%%%%%%%%%%%%%%%%%
\begin{problem}{2.1.30}
In each of the following commutative diagrams assume that all maps but one are isomorphisms. Show that the remaining map must be isomorphism as well. 
\[\begin{tikzcd}
	A && B && A & B && A & B \\
	& C &&& C & D && C & D
	\arrow[from=1-1, to=1-3]
	\arrow[from=1-1, to=2-2]
	\arrow[from=1-5, to=1-6]
	\arrow[from=1-5, to=2-5]
	\arrow[from=1-6, to=2-6]
	\arrow[from=1-8, to=1-9]
	\arrow[from=1-8, to=2-8]
	\arrow[from=2-2, to=1-3]
	\arrow[from=2-5, to=2-6]
	\arrow[from=2-8, to=2-9]
	\arrow[from=2-9, to=1-9]
\end{tikzcd}\]
\end{problem}
\begin{solution}
\begin{enumerate}[(1)]
\item \[\begin{tikzcd}
	A && B \\
	& C
	\arrow["f", from=1-1, to=1-3]
	\arrow["g"', from=1-1, to=2-2]
	\arrow["h"', from=2-2, to=1-3]
\end{tikzcd}\]
(a) Assume \(g,h\) are isomorphisms, then \(f=h\circ g\) is also an isomorphism since it is the composition of two isomorphisms. 
\par 
(b) Assume \(f,g\) are isomorphisms. \(g\) is an isomorphism implies that there exist a map \(g^{-1}:C\rightarrow A\) such that \(g\circ g^{-1}=id_C\), then 
\[h=h\circ id_C=h\circ (g\circ g^{-1})=(h\circ g)\circ g^{-1}=f\circ g^{-1}\]
where both \(f\) and \(g^{-1}\) are isomorphisms, so is \(h\).
\par 
(c) Assume \(f,h\) are isomorphisms. \(h\) is an isomorphism implies that there exists a map \(h^{-1}:B\rightarrow C\) such that \(h^{-1}\circ h=id_C\)., then 
\[g=id_C\circ g=(h^{-1}\circ h)\circ g=h^{-1}\circ (h\circ g)=h^{-1}\circ f\]
where both \(f\) and \(h^{-1}\) are isomorphisms, so is \(g\).
\item \[\begin{tikzcd}
	A & B \\
	C & D
	\arrow["f", from=1-1, to=1-2]
	\arrow["i"', from=1-1, to=2-1]
	\arrow["g", from=1-2, to=2-2]
	\arrow["h"', from=2-1, to=2-2]
\end{tikzcd}\]
Assume \(i,g,h\) are isomorphisms. Then view the composition \(h\circ i\) as one isomorphism, and we are back to the situation (c) in (1).\\ 
Assume \(i,f,g\) are isomorphisms. Then view the composition \(g\circ f\) as one isomorphism, and we are back to the situation (b) in (1).\\
Assume \(f,i,h\) are isomorphisms. Then view the composition \(h\circ i\) as one isomorphism, and we are back to the situation (b) in (1).\\ 
Assume \(f,g,h\) are isomorphisms. Then view the composition \(g\circ f\) as one isomorphism, and we are back to the situation (c) in (1).
\item \[\begin{tikzcd}
	A & B \\
	C & D
	\arrow["f", from=1-1, to=1-2]
	\arrow["i"', from=1-1, to=2-1]
	\arrow["h"', from=2-1, to=2-2]
	\arrow["g"', from=2-2, to=1-2]
\end{tikzcd}\]
Assume \(i,g,h\) are isomorphisms, then \(f=g\circ h\circ i\) is also an isomorphism since it is the composition of isomorphisms.\\ 
Assume \(i,f,h\) are isomorphisms, then view the composition \(h\circ i\) as one isomorphism, and we are back to the situation (b) in (1).\\ 
Assume \(i,f,g\) are isomorphisms. \(i,g\) are isomorphisms implies that there exist \(g^{-1}:B\rightarrow D\) and \(i^{-1}:C\rightarrow A\) such that \(g^{-1}\circ g=id_D\) and 
\(i\circ i^{-1}=id_C\). Now we have 
\[h=id_D\circ h\circ id_C=g^{-1}\circ g\circ h\circ i\circ i^{-1}=g^{-1}\circ (g\circ h\circ i)\circ i^{-1}=g^{-1}\circ f\circ i^{-1} \]
where \(i^{-1},f,g^{-1}\) are isomorphisms, so is \(h\).\\ 
Assume \(f,g,h\) are isomorphisms, then view the composition \(g\circ h\) as one isomorphism, and we are back to the situation (c) in (1).
\end{enumerate}
\end{solution}

\noindent\rule{7in}{2.8pt}
%%%%%%%%%%%%%%%%%%%%%%%%%%%%%%%%%%%%%%%%%%%%%%%%%%%%%%%%%%%%%%%%%%%%%%%%%%%%%%%%%%%%%%%%%%%%%%%%%%%%%%%%%%%%%%%%%%%%%%%%%%%%%%%%%%%%%%%%
% Exercise 2.1.31
%%%%%%%%%%%%%%%%%%%%%%%%%%%%%%%%%%%%%%%%%%%%%%%%%%%%%%%%%%%%%%%%%%%%%%%%%%%%%%%%%%%%%%%%%%%%%%%%%%%%%%%%%%%%%%%%%%%%%%%%%%%%%%%%%%%%%%%%
\begin{problem}{2.1.31}
Using the notation of the five lemma, give an example where the maps \(\alpha,\beta,\delta\) and \(\varepsilon\) are zero but \(\gamma\) is nonzero. 
This can be done with short exact sequencs in which all the groups are either \(\mathbb{Z}\) or \(0\).
\end{problem}
\begin{solution}
Consider the following diagrams:
\[\begin{tikzcd}
	{\mathbb{Z}} & {\mathbb{Z}} & {\mathbb{Z}} & {\mathbb{Z}} & 0 \\
	0 & {\mathbb{Z}} & {\mathbb{Z}} & {\mathbb{Z}} & {\mathbb{Z}}
	\arrow["\sim", from=1-1, to=1-2]
	\arrow["0"', from=1-1, to=2-1]
	\arrow["0", from=1-2, to=1-3]
	\arrow["0"', from=1-2, to=2-2]
	\arrow["\sim", from=1-3, to=1-4]
	\arrow["\sim", from=1-3, to=2-3]
	\arrow[from=1-4, to=1-5]
	\arrow["0", from=1-4, to=2-4]
	\arrow["0", from=1-5, to=2-5]
	\arrow[from=2-1, to=2-2]
	\arrow["\sim"', from=2-2, to=2-3]
	\arrow["0"', from=2-3, to=2-4]
	\arrow["\sim"', from=2-4, to=2-5]
\end{tikzcd}\]
The top row and the bottom row are exact. And we have \(\alpha=\beta=\delta=\varepsilon=0\), and \(\gamma:\mathbb{Z}\xrightarrow{\sim} \mathbb{Z}\) is an isomorphism and nonzero.
\end{solution}

\noindent\rule{7in}{2.8pt}
%%%%%%%%%%%%%%%%%%%%%%%%%%%%%%%%%%%%%%%%%%%%%%%%%%%%%%%%%%%%%%%%%%%%%%%%%%%%%%%%%%%%%%%%%%%%%%%%%%%%%%%%%%%%%%%%%%%%%%%%%%%%%%%%%%%%%%%%
% Exercise 2.2.1
%%%%%%%%%%%%%%%%%%%%%%%%%%%%%%%%%%%%%%%%%%%%%%%%%%%%%%%%%%%%%%%%%%%%%%%%%%%%%%%%%%%%%%%%%%%%%%%%%%%%%%%%%%%%%%%%%%%%%%%%%%%%%%%%%%%%%%%%
\begin{problem}{2.2.1}
Prove the Brouwer fixed point theorem for maps \(f:D^n\rightarrow D^n\) by applying degree theory to the map \(S^n\rightarrow S^n\) that sends both the northern and southern hemispheres 
of \(S^n\) to the southern hemisphere via \(f\). 	
\end{problem}
\begin{solution}
Denote the descirbed map by \(\bar{f}:S^n\rightarrow S^n\). Since \(\bar{f}\) is not surjective, so we know \(\deg \bar{f}=0\). Moreover, \(\bar{f}\) has no fix point unless \(\deg \bar{f}=(-1)^{n+1}\). There 
exist \(x\in S^n\) such that \(\bar{f}(x)=x\). Note that \(x\) cannot be in the northern hemisphere because the northern hemisphere is not in the image of \(\bar{f}\). And we know that when \(\bar{f}\) restricts to the southern hemisphere, 
it is just the map \(f:D^n\rightarrow D^n\), so we can conclude that \(f\) has a fixed point. 
\end{solution}


\end{document}
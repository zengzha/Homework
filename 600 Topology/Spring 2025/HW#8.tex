\documentclass[letterpaper, 12pt]{article}

\usepackage{/Users/zhengz/Desktop/Math/Workspace/Homework1/homework}

\begin{document}
\noindent
\large\textbf{Zhengdong Zhang} \hfill \textbf{Homework 8}  \\
Email: zhengz@uoregon.edu \hfill ID: 952091294  \\
\normalsize Course: MATH 636 - Algebraic Topology III \hfill Term: Spring 2025 \\
Instructor: Dr.Daniel Dugger \hfill Due Date: $6^{th}$ June, 2025  \\
\noindent\rule{7in}{2.8pt}
\setstretch{1.1}
%%%%%%%%%%%%%%%%%%%%%%%%%%%%%%%%%%%%%%%%%%%%%%%%%%%%%%%%%%%%%%%%%%%%%%%%%%%%%%%%%%%%%%%%%%%%%%%%%%%%%%%%%%%%%%%%%%%%%%%%%
% Problem 1
%%%%%%%%%%%%%%%%%%%%%%%%%%%%%%%%%%%%%%%%%%%%%%%%%%%%%%%%%%%%%%%%%%%%%%%%%%%%%%%%%%%%%%%%%%%%%%%%%%%%%%%%%%%%%%%%%%%%%%%%%%
\begin{problem}{1}
Suppose that \(M\) is a compact 3-manifold with \(\pi_1(M)\cong \mathbb{Z}/5\). 
\begin{enumerate}[(a)]
\item Prove that \(M\) is orientable, and then calculate all of the homology and cohomology groups of \(M\). 
\item Prove that every map \(M\rightarrow \mathbb{R}P^3\) has even degree.
\end{enumerate}
\end{problem}
\begin{solution}
\begin{enumerate}[(a)]
\item \(\pi_1(M)\cong \mathbb{Z}/5\), so there does not exist \(\pi_1(M)\)-set of index 2 because \(2\) does not divide \(5\). This implies every degree 2 covering space of \(M\) must be disconnected, so \(\tilde{M}\rightarrow M\) is the trivial covering map with 2 connected components. This implies \(M\) is orientable. 

By Hurewicz theorem, we know that \(H_1(M)\) is the abelianization of \(\pi_1(M)\). So 
\[H_1(M)\cong \pi_1(M)\cong \mathbb{Z}/5.\]
\(M\) being orientable implies that \(H_3(M)\cong \mathbb{Z}\). By UCT, we have 
\[H^1(M)\cong \hom(H_1(M),\mathbb{Z})\oplus \Ext^1(H_0(M),\mathbb{Z})\cong 0.\]
By Poincaré duality, \(H_2(M)\cong H^1(M)\cong 0\). We have all the homology groups of \(M\) and use Poincaré duality again, we can obtain all the cohomology groups.
\begin{table}[ht]
    \centering
    \resizebox{0.3\columnwidth}{!}{%
    \(\begin{array}{|c|c|c|}
    \hline
      & H_*(M)       & H^*(M)       \\ \hline
    0 & \mathbb{Z}   & \mathbb{Z}   \\ \hline
    1 & \mathbb{Z}/5 & 0            \\ \hline
    2 & 0            & \mathbb{Z}/5 \\ \hline
    3 & \mathbb{Z}   & \mathbb{Z}   \\ \hline
    \end{array}\)%
    }
    \end{table}
\item By UCT, we have 
\[H^1(M;\mathbb{Z}/2)\cong H^1(M)\otimes_\mathbb{Z} \mathbb{Z}/2 \oplus \Tor_1(H^2(M), \mathbb{Z}/2)\cong 0.\]
Given a map \(f:M\rightarrow \mathbb{R}P^3\), the induced map 
\[f^*:H^*(\mathbb{R}P^3;\mathbb{Z}/2)\rightarrow H^*(M;\mathbb{Z}/2)\]
must be the zero map because we know \(H^*(\mathbb{R}P^3;\mathbb{Z}/2)\cong (\mathbb{Z}/2)[x]/(x^4)\), and it is generated by \(x\) in degree \(1\), but \(H^1(M)=0\). By naturality of UCT, we have a commutative diagram 
% https://q.uiver.app/#q=WzAsNCxbMCwwLCJIXjMoXFxtYXRoYmJ7Un1QXjMpXFxvdGltZXMgXFxtYXRoYmJ7Wn0vMiJdLFswLDEsIkheMyhNKVxcb3RpbWVzXFxtYXRoYmJ7Wn0vMiJdLFsxLDAsIkheMyhcXG1hdGhiYntSfVBeMztcXG1hdGhiYntafS8yKSJdLFsxLDEsIkheMyhNO1xcbWF0aGJie1p9LzIpIl0sWzAsMiwiIiwwLHsic3R5bGUiOnsidGFpbCI6eyJuYW1lIjoiaG9vayIsInNpZGUiOiJ0b3AifX19XSxbMSwzLCIiLDAseyJzdHlsZSI6eyJ0YWlsIjp7Im5hbWUiOiJob29rIiwic2lkZSI6InRvcCJ9fX1dLFsyLDMsIjAiXSxbMCwxXV0=
\[\begin{tikzcd}
	{H^3(\mathbb{R}P^3)\otimes \mathbb{Z}/2} & {H^3(\mathbb{R}P^3;\mathbb{Z}/2)} \\
	{H^3(M)\otimes\mathbb{Z}/2} & {H^3(M;\mathbb{Z}/2)}
	\arrow[hook, from=1-1, to=1-2]
	\arrow[from=1-1, to=2-1]
	\arrow["0", from=1-2, to=2-2]
	\arrow[hook, from=2-1, to=2-2]
\end{tikzcd}\]
We know that \(H^3(M)\otimes \mathbb{Z}/2\cong \mathbb{Z}/2\) and the map \(H^3(M)\otimes \mathbb{Z}/2\rightarrow H^3(M;\mathbb{Z}/2)\) is injective, so it cannot be the zero map. Thus, 
\[H^3(\mathbb{R}P^3)\otimes \mathbb{Z}/2\rightarrow H^3(M)\otimes \mathbb{Z}/2\]
must be the zero map. This implies \(H^3(\mathbb{R}P^3)\rightarrow H^3(M)\) is given by multiplication of an even number, namely the map \(f:M\rightarrow \mathbb{R}P^3\) has even degree.
\end{enumerate}
\end{solution}

\noindent\rule{7in}{2.8pt}
%%%%%%%%%%%%%%%%%%%%%%%%%%%%%%%%%%%%%%%%%%%%%%%%%%%%%%%%%%%%%%%%%%%%%%%%%%%%%%%%%%%%%%%%%%%%%%%%%%%%%%%%%%%%%%%%%%%%%%%%%
% Problem 2
%%%%%%%%%%%%%%%%%%%%%%%%%%%%%%%%%%%%%%%%%%%%%%%%%%%%%%%%%%%%%%%%%%%%%%%%%%%%%%%%%%%%%%%%%%%%%%%%%%%%%%%%%%%%%%%%%%%%%%%%%%
\begin{problem}{2}
\begin{enumerate}[(a)]
\item Explain why the Euler characteristic of an odd-dimensional compact manifold must be zero. 
\item Suppose that \(M\) is a \((2d+1)\)-dimensional compact manifold, and let \(W=\partial M\). Let \(X\) be the manifold obtained by gluing two copies of \(M\) together along their boundary. Using Mayer-Vietoris (or otherwise) prove that \(\chi(W)\equiv \chi(X)\) mod \(2\), and so deduce that \(\chi(W)\) must be even.
\end{enumerate}
\end{problem}
\begin{solution}
\begin{enumerate}[(a)]
\item Suppose \(M\) is a compact mainifold of dimension \(2n-1\) where \(n\geq 1\). We have proved in class that 
\begin{align*}
    \chi(M)&=\sum_{i=0}^{2n-1} (-1)^{i}\rank H_i(M)\\ 
           &=\sum_{i=0}^{2n-1} (-1)^i\dim_{\mathbb{Z}_2} H_i(M;\mathbb{Z}/2).
\end{align*}
\(M\) is \(\mathbb{Z}_2\)-orientable, by Poincaré duality, we know that 
\[H_{2n-1-i}(M;\mathbb{Z}/2)\cong H^{i}(M;\mathbb{Z}/2).\]
Moreover, by UCT and note that \(\mathbb{Z}/2\) is a field, we have an isomorphism
\[H^i(M;\mathbb{Z}/2)\xrightarrow{\cong} \hom_{\mathbb{Z}/2}(H_i(M;\mathbb{Z}/2),\mathbb{Z}/2).\]
So we have 
\[\dim_{\mathbb{Z}/2} H_i(M;\mathbb{Z}/2)=\dim_{\mathbb{Z}/2} H^i(M;\mathbb{Z}/2).\]
Combine these two together, and we have 
\[\dim_{\mathbb{Z}/2} H_{2n-1-i}(M;\mathbb{Z}/2)=\dim_{\mathbb{Z}/2} H_i (M;\mathbb{Z}/2).\]
Note that \((-1)^i+(-1)^{2n-1-i}=0\), therefore, we have 
\begin{align*}
    \chi(M)&=\sum_{i=0}^{2n-1}(-1)^i \dim_{\mathbb{Z}/2} H_i(M;\mathbb{Z}/2)\\ 
           &=\sum_{i=0}^{n-1} [(-1)^i \dim_\mathbb{Z}/2 H_i(M;\mathbb{Z}/2)+(-1)^{2n-1-i}\dim_{\mathbb{Z}/2} H_i(M;\mathbb{Z}/2)]\\ 
           &=0
\end{align*}
\item \(X\) has an open cover \(U\cup V\) where \(U\cong V\) is homotopy equivalent to \(M\) and \(U\cap V\) is homotopy equivalent to \(\partial M\). We have the Mayer-Vietoris sequence 
\[\cdots\rightarrow H_k(\partial M)\rightarrow H_k(M)\oplus H_k(M)\rightarrow H_k(X)\rightarrow H_{k-1}(\partial M)\rightarrow \cdots\]
\begin{claim}
Suppose \(n>0\) and we have a long exact sequence
% https://q.uiver.app/#q=WzAsMTIsWzAsMSwiQV9uIl0sWzIsMCwiMCJdLFsxLDEsIkJfbiJdLFsyLDEsIkNfbiJdLFswLDIsIkFfe24tMX0iXSxbMSwwLCJcXGNkb3RzIl0sWzEsMiwiXFxjZG90cyJdLFsyLDIsIkNfMSJdLFswLDMsIkFfMCJdLFsxLDMsIkJfMCJdLFsyLDMsIkNfMCJdLFswLDQsIjAiXSxbMSwwXSxbNSwxXSxbMCwyXSxbMiwzXSxbMyw0XSxbNCw2XSxbNyw4XSxbOCw5XSxbOSwxMF0sWzYsN10sWzEwLDExXV0=
\[\begin{tikzcd}
	& \cdots & 0 \\
	{A_n} & {B_n} & {C_n} \\
	{A_{n-1}} & \cdots & {C_1} \\
	{A_0} & {B_0} & {C_0} \\
	0
	\arrow[from=1-2, to=1-3]
	\arrow[from=1-3, to=2-1]
	\arrow[from=2-1, to=2-2]
	\arrow[from=2-2, to=2-3]
	\arrow[from=2-3, to=3-1]
	\arrow[from=3-1, to=3-2]
	\arrow[from=3-2, to=3-3]
	\arrow[from=3-3, to=4-1]
	\arrow[from=4-1, to=4-2]
	\arrow[from=4-2, to=4-3]
	\arrow[from=4-3, to=5-1]
\end{tikzcd}\]
Here \(A_i,B_i\) and \(C_i\) are finitely generated abelian groups. Then 
\[\chi(B)=\chi(A)+\chi(C).\]
\end{claim}
\begin{claimproof}
We need to prove that 
\[\sum_{i=0}^{n}(-1)^i \rank A_i-\sum_{j=0}^{n}(-1)^j\rank B_j+\sum_{k=0}^{n}(-1)^k\rank C_k=\sum_{i=0}^{n}(-1)^i(\rank A_i-\rank B_i+\rank C_i)=0.\]
This is equivalent as proving the following fact: Given an exact sequence of finitely generated abelian groups
\[0\rightarrow X_n\xrightarrow{f_n}X_{n-1}\xrightarrow{f_{n-1}}\cdots\xrightarrow{f_1}X_0\xrightarrow{f_0} 0\]
We have 
\[\chi(X)=\sum_{i=0}^{n}(-1)^i\rank X_i=0.\]
By the first isomorphism theorem, for any \(0\leq i\leq n\), we have 
\[X_i/\ker f_i\cong \im f_i.\]
So by exactness,
\begin{align*}
    \rank X_i&=\rank \ker f_i+\rank \im f_i\\ 
             &=\rank \ker f_i+\rank \im f_{i-1}.
\end{align*}
Thus, 
\begin{align*}
    \chi(X)=&\sum_{i=0}^{n}(-1)^i\rank X_i\\ 
           =&\quad \rank \ker f_0\\ 
            &-\rank \ker f_0-\rank \ker f_1\\ 
            &+\rank \ker f_1+\rank \ker f_2\\ 
            &\cdots \\ 
            &+(-1)^{n}\rank \ker f_{n-1}+(-1)^n \rank \ker f_n\\ 
           =&(-1)^n \rank \ker f_n. 
\end{align*}
Note that \(f_n:X_n\rightarrow X_{n-1}\) is injective because \(X_{n+1}=0\). This proves that \(\chi(X)=0\).
\end{claimproof}

By the claim, we know that 
\[\chi(X)+\chi(\partial M)=2\chi(M)\]
This implies \(\chi(X)\equiv \chi(\partial M)\) (mod 2). And note that here \(X\) is a closed \((2d+1)\)-dimensional manifold, so \(\chi(X)=0\) from what we have proved in (a). So \(\chi(\partial M)\) must be even. 
\end{enumerate}
\end{solution}

\noindent\rule{7in}{2.8pt}
%%%%%%%%%%%%%%%%%%%%%%%%%%%%%%%%%%%%%%%%%%%%%%%%%%%%%%%%%%%%%%%%%%%%%%%%%%%%%%%%%%%%%%%%%%%%%%%%%%%%%%%%%%%%%%%%%%%%%%%%%
% Problem 3
%%%%%%%%%%%%%%%%%%%%%%%%%%%%%%%%%%%%%%%%%%%%%%%%%%%%%%%%%%%%%%%%%%%%%%%%%%%%%%%%%%%%%%%%%%%%%%%%%%%%%%%%%%%%%%%%%%%%%%%%%%
\begin{problem}{3}
Suppose that there is a fiber bundle \(p:X\xrightarrow{p} S^8\) with fiber \(S^3\). 
\begin{enumerate}[(a)]
\item Prove that \(X\) is an orientable manifold.
\item Prove that \(H_*(X)\) is isomorphic to \(H_*(S^3\times S^8)\). 
\end{enumerate}
\end{problem}
\begin{solution}
\begin{enumerate}[(a)]
\item The fiber bundle \(S^3\rightarrow X\xrightarrow{p} S^8\) implies that \(X\) is a \(11\)-dimensional manifold and induces a long exact sequence in homotopy groups 
\[\cdots\rightarrow \pi_1(S^3)\rightarrow \pi_1(X)\rightarrow \pi_1(S^8)\rightarrow \cdots\]
We know that \(\pi_1(S^3)\cong \pi_1(S^*)\cong \left\{ * \right\}\) is trivial. This implies \(\pi_1(X)=\left\{ * \right\}\) is trivial, so \(X\) does not have degree 2 connected coverings, namely the orientation covering \(\tilde{X}\rightarrow X\) is a trivial covering, so \(X\) is orientable. 
\item Let \(D_+\) and \(D_-\) be the open upper and lower half hemispheres of \(S^8\). Let \(U:=p^{-1}(D_+)\) and \(V:=p^{-1}(D_-)\). \(U\cup V\) is an open cover of \(X\). Note that \(p^{-1}(D_+)\xrightarrow{p} D_+\) is a subbundle of \(X\xrightarrow{p} S^8\), since \(D_+\cong \mathbb{R}^8\) is contracitble, \(U\xrightarrow{p} D_+\) is isomorphic to the trivial bundle \(S^3\times D_+\rightarrow D_+\). The space \(U\) is homotopy equivalent to \(S^3\). Same for \(V\). Moreover, \(p^{-1}(D_+\cap D_-)\rightarrow D_+\cap D_-\) is a subbundle of \(p^{-1}(D_+)\rightarrow D_+\), which is isomorphic to a trivial bundle, so it is also isomorphic to a trivial bundle. Thus, we have \(p^{-1}(D_+\cap D_-)\) is homotopy equivalent to \(S^7\times S^3\) as \(D_+\cap D_-\) is homotopy equivalent to \(S^7\subseteq S^8\). Consider the Mayer-Vietoris sequence given by the cover \(p^{-1}(D_+)\cup p^{-1}(D_-)\) on \(X\):
\[\cdots\rightarrow H_k(S^7\times S^3)\rightarrow H_k(S^3)\oplus H_k(S^3)\rightarrow H_k(X)\rightarrow H_{k-1}(S^7\times S^3)\rightarrow \cdots\]
For \(k\geq 4\), \(H_k(S^3)=0\). So \(H_k(X)\cong H_{k-1}(S^3\times S^7)\) for \(k\geq 5\). Namely, \(H_8(X)\cong H_7(S^7\times S^3)\cong \mathbb{Z}\) and \(H_{11}(X)\cong H_{10}(S^7\times S^3)\cong \mathbb{Z}\), else \(H_k(X)=0\) for \(k\geq 5\). In addition, by the same argument, \(H_1(X)=H_2(X)=0\) since \(H_1(S^3)=H_2(S^3)=0\), and \(H_0(X)\cong \mathbb{Z}\) because \(X\) is connected. We need to determine \(H_3(X)\) and \(H_4(X)\) from the following exact sequence:
\[0\rightarrow H_4(X)\rightarrow H_3(S^3\times S^7)\rightarrow H_3(S^3)\oplus H_3(S^3)\rightarrow H_3(X)\rightarrow 0.\]
That is 
\[0\rightarrow H_4(X)\rightarrow \mathbb{Z}\rightarrow \mathbb{Z}^2\rightarrow H_3(X)\rightarrow 0.\]
By exactness, \(H_4(X)\rightarrow \mathbb{Z}\) is injective, so \(H_4(X)=0\) or \(H_4(X)\cong \mathbb{Z}\). By Poincaré duality, \(H_4(X)\) and \(H_7(X)\) has the same rank, and since \(H_7(X)=0\), \(H_4(X)=0\). \(H_3(X)\) is the cokernel of an injective map \(\mathbb{Z}\rightarrow \mathbb{Z}^2\). By Poincaré duality, \(\rank H_3(X)=\rank H_8(X)=1\). By UCT, 
\[H^8(X)\cong \hom(H_8(X),\mathbb{Z})\oplus \Ext^1(H_7(X),\mathbb{Z})\cong \mathbb{Z}\]
does not have any torsion. So by Poincaré duality, \(H_3(X)\cong H^8(X)\) also does not have torsion. This implies \(H_3(X)\cong \mathbb{Z}\). We can summarize that \(H_*(X)=0\) except 
\[H_0(X)\cong H_3(X)\cong H_8(X)\cong H_{11}(X)\cong \mathbb{Z}.\]
This means \(H_*(X)\) is isomorphic to \(H_*(S^3\times S^8)\) for all \(*\). 
\end{enumerate}
\end{solution}

\noindent\rule{7in}{2.8pt}
%%%%%%%%%%%%%%%%%%%%%%%%%%%%%%%%%%%%%%%%%%%%%%%%%%%%%%%%%%%%%%%%%%%%%%%%%%%%%%%%%%%%%%%%%%%%%%%%%%%%%%%%%%%%%%%%%%%%%%%%%
% Problem 4
%%%%%%%%%%%%%%%%%%%%%%%%%%%%%%%%%%%%%%%%%%%%%%%%%%%%%%%%%%%%%%%%%%%%%%%%%%%%%%%%%%%%%%%%%%%%%%%%%%%%%%%%%%%%%%%%%%%%%%%%%%
\begin{problem}{4}
Compute the cohomology ring of \(\mathbb{R}P^4\vee S^5\) with \(\mathbb{Z}/2\)-coefficients. Then use this to prove that \(\mathbb{R}P^4\vee S^5\) is not homotopy equivalent to a compact manifold. 
\end{problem}
\begin{solution}
We know that \(H^*(\mathbb{R}P^4;\mathbb{Z}/2)=(\mathbb{Z}/2)[x]/(x^5)\) and \(H^*(S^5;\mathbb{Z}/2)=(\mathbb{Z}/2)[y]/(y^2)\) (\(y\) is in degree 5). Since 
\[\tilde{H}^*(\mathbb{R}P^4\vee S^5;\mathbb{Z}/2)\cong \tilde{H}^*(\mathbb{R}P^4;\mathbb{Z}/2)\oplus \tilde{H}^*(S^5;\mathbb{Z}/2)\]
and \(\mathbb{R}P^4\vee S^5\) is still connected, we know that 
\[H^*(\mathbb{R}P^4\vee S^5;\mathbb{Z}/2)\cong (\mathbb{Z}/2)[x,y]/(x^5,y^2,xy)\]
where \(x\) is in degree 1 and \(y\) is in degree 5. Suppose \(\mathbb{R}P^4\vee S^5\) is homotopy equivalent to a compact manifold, then it is \(\mathbb{Z}/2\)-orientable. By Poincaré duality, the pairing 
\[H^1(\mathbb{R}P^4\vee S^5;\mathbb{Z}/2)\otimes H^4(\mathbb{R}P^4\vee S^5;\mathbb{Z}/2)\xrightarrow{\cup} H^5(\mathbb{R}P^4\vee S^5;\mathbb{Z}/2)\cong \mathbb{Z}/2\]
is a perfect pairing. Here \(H^1(\mathbb{R}P^4\vee S^5;\mathbb{Z}/2)\) is generated by \(x\) and \(H^4(\mathbb{R}P^4\vee S^5;\mathbb{Z}/2)\) is generated by \(x^4\), and \(x\cup x^4=x^5=0\). A contradiction.
\end{solution}

\noindent\rule{7in}{2.8pt}
%%%%%%%%%%%%%%%%%%%%%%%%%%%%%%%%%%%%%%%%%%%%%%%%%%%%%%%%%%%%%%%%%%%%%%%%%%%%%%%%%%%%%%%%%%%%%%%%%%%%%%%%%%%%%%%%%%%%%%%%%
% Problem 5
%%%%%%%%%%%%%%%%%%%%%%%%%%%%%%%%%%%%%%%%%%%%%%%%%%%%%%%%%%%%%%%%%%%%%%%%%%%%%%%%%%%%%%%%%%%%%%%%%%%%%%%%%%%%%%%%%%%%%%%%%%
\begin{problem}{5}
Suppose that \(X\) is a compact, orientable \(n\)-manifold and that \(S^n\rightarrow X\) is a map of positive degree. Prove that \(H_*(X;\mathbb{Q})\cong H_*(S^n;\mathbb{Q})\).
\end{problem}
\begin{solution}
Given a map \(f:S^n\rightarrow X\) of positive degree, we prove that the induced map 
\[f^*:H^k(X;\mathbb{Q})\rightarrow H^k(S^n;\mathbb{Q})\]
is an injective map between \(\mathbb{Q}\)-vector space for \(0\leq k\leq n\). Suppose \(a\in \ker f^*\). If \(a\neq 0\) in \(H^k(X;\mathbb{Q})\), then by Poincaré duality, there exists \(a'\in H^{n-k}(X;\mathbb{Q})\) such that \(a\cup a'=\widehat{[X]}\) where \(\widehat{[X]}\) is the cohomological fundamantal class of \(X\). Then we have 
\[(\deg f)\widehat{[S^n]}=f^*(\widehat{[X]})=f^*(a\cup a')=f^*(a)\cup f^*(a')=0.\]
Here \(\deg f>0\). We have a contradiction. So \(a=0\). This proves that \(f^*\) is injective for all \(k\). Since \(H^k(S^n;\mathbb{Q})=0\) for \(1\leq k\leq n-1\), we know that \(H^k(X;\mathbb{Q})=0\) for \(1\leq k\leq n-1\). Moreover, \(X\) being connected and orientable implies that \(H^0(X;\mathbb{Q})\cong H^n(X;\mathbb{Q})\cong \mathbb{Q}\). By Poincaré duality, we have 
\[H_*(S^n;\mathbb{Q})\cong H_*(X;\mathbb{Q}).\]
\end{solution}

\noindent\rule{7in}{2.8pt}
%%%%%%%%%%%%%%%%%%%%%%%%%%%%%%%%%%%%%%%%%%%%%%%%%%%%%%%%%%%%%%%%%%%%%%%%%%%%%%%%%%%%%%%%%%%%%%%%%%%%%%%%%%%%%%%%%%%%%%%%%
% Problem 6
%%%%%%%%%%%%%%%%%%%%%%%%%%%%%%%%%%%%%%%%%%%%%%%%%%%%%%%%%%%%%%%%%%%%%%%%%%%%%%%%%%%%%%%%%%%%%%%%%%%%%%%%%%%%%%%%%%%%%%%%%%
\begin{problem}{6}
Find the mistake in the following "proof" that \(0=1\):

Let \(A:S^2\rightarrow S^2\) be the antipodal map, and \(p:S^2\rightarrow \mathbb{R}P^2\) the projection. Consider the diagram 
% https://q.uiver.app/#q=WzAsNCxbMCwwLCJcXHBpXzIoU14yKSJdLFsxLDAsIlxccGlfMihTXjIpIl0sWzAsMSwiSF8yKFNeMikiXSxbMSwxLCJIXzIoU14yKSJdLFswLDEsIkFfKiJdLFsyLDMsIkFfKiJdLFswLDIsImhfMiIsMl0sWzEsMywiaF8yIiwyXV0=
\[\begin{tikzcd}
	{\pi_2(S^2)} & {\pi_2(S^2)} \\
	{H_2(S^2)} & {H_2(S^2)}
	\arrow["{A_*}", from=1-1, to=1-2]
	\arrow["{h_2}"', from=1-1, to=2-1]
	\arrow["{h_2}"', from=1-2, to=2-2]
	\arrow["{A_*}", from=2-1, to=2-2]
\end{tikzcd}\]
where \(h_2\) is the Hurewicz map. We know that \(h_2\) is an isomorphism, and we know that the lower map \(A_*\) is multiplication by \((-1)^3\). So it follows that the upper \(A_*\) is also multiplication by \((-1)\). \\ 
Next consider the diagram 
% https://q.uiver.app/#q=WzAsMyxbMCwwLCJcXHBpXzIoU14yKSJdLFsyLDAsIlxccGlfMihTXjIpIl0sWzEsMSwiXFxwaV8yKFxcbWF0aGJie1J9UF4yKSJdLFswLDEsIkFfKiJdLFswLDIsInBfKiIsMl0sWzEsMiwicF8qIl1d
\[\begin{tikzcd}
	{\pi_2(S^2)} && {\pi_2(S^2)} \\
	& {\pi_2(\mathbb{R}P^2)}
	\arrow["{A_*}", from=1-1, to=1-3]
	\arrow["{p_*}"', from=1-1, to=2-2]
	\arrow["{p_*}", from=1-3, to=2-2]
\end{tikzcd}\]
This commutes because of functoriality, since \(p\circ A=p\). We know from the long exact sequence for the fibration \(p:S^2\rightarrow \mathbb{R}P^2\) that \(p_*\) is an isomorphism. Let \(g\in \pi_2(S^2)\) be a generator. Then we have 
\[p_*(g)=p_*(A_*(g))=p_*(-g)=-p_*(g).\]
But \(\pi_2(\mathbb{R}P^2)\cong \pi_2(S^2)\cong \mathbb{Z}\), and so the above equation implies \(p_*(g)=0\). Therefore \(p_*\) is the zero map. But we have already said that \(p_*\) is an isomorphism, therefore \(\pi_2(\mathbb{R}P^2)=0\). Since we have also said that \(\pi_2(\mathbb{R}P^2)\cong \mathbb{Z}\), it must be that \(\mathbb{Z}\cong 0\). So \(\mathbb{Z}\) has only one element and, in particular, \(0=1\). 
\end{problem}
\begin{solution}
Choose a point \(x_0\in S^2\) as the base point. The commutative triangle between pointed space is 
% https://q.uiver.app/#q=WzAsMyxbMCwwLCIoU14yLHhfMCkiXSxbMiwwLCIoU14yLC14XzApIl0sWzEsMSwiKFxcbWF0aGJie1J9UF4yLFxcdGlsZGV7eF8wfSkiXSxbMCwyLCJwXzEiLDJdLFswLDEsIkEiXSxbMSwyLCJwXzIiXV0=
\[\begin{tikzcd}
	{(S^2,x_0)} && {(S^2,-x_0)} \\
	& {(\mathbb{R}P^2,\tilde{x_0})}
	\arrow["A", from=1-1, to=1-3]
	\arrow["{p_1}"', from=1-1, to=2-2]
	\arrow["{p_2}", from=1-3, to=2-2]
\end{tikzcd}\]
Note here \(p_1\) and \(p_2\) are different as pointed maps. This induces a commutative triangle in homotopy groups 
% https://q.uiver.app/#q=WzAsMyxbMCwwLCJcXHBpXzIoU14yLHhfMCkiXSxbMiwwLCJcXHBpXzIoU14yLC14XzApIl0sWzEsMSwiXFxwaV8yKFxcbWF0aGJie1J9UF4yLFxcdGlsZGV7eF8wfSkiXSxbMCwyLCJwX3sxLCp9IiwyXSxbMCwxLCJBXyoiXSxbMSwyLCJwX3syLCp9Il1d
\[\begin{tikzcd}
	{\pi_2(S^2,x_0)} && {\pi_2(S^2,-x_0)} \\
	& {\pi_2(\mathbb{R}P^2,\tilde{x_0})}
	\arrow["{A_*}", from=1-1, to=1-3]
	\arrow["{p_{1,*}}"', from=1-1, to=2-2]
	\arrow["{p_{2,*}}", from=1-3, to=2-2]
\end{tikzcd}\]
Let \(g\in \pi_2(S^2,x_0)\) be the generator. We have 
\[p_{1,*}(g)=p_{2,*}(-g).\]
Here \(p_{1,*}\) and \(p_{2,*}\) are different isomorphisms as we choose different base point for \(S^2\). And here \(g\in \pi_2(S^2,x_0)\) and \(-g\in \pi_2(S^2,-x_0)\). They are not in the same group, so the next line of reasoning does not make sense. 
\end{solution}

\end{document}
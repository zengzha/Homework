\documentclass[a4paper, 11pt]{article}
\usepackage{comment} % enables the use of multi-line comments (\ifx \fi) 
\usepackage{lipsum} %This package just generates Lorem Ipsum filler text. 
\usepackage{fullpage} % changes the margin
\usepackage[a4paper, total={7in, 10in}]{geometry}
\usepackage[fleqn]{amsmath}
\usepackage{amssymb,amsthm}  % assumes amsmath package installed
\newtheorem{theorem}{Theorem}
\newtheorem{corollary}{Corollary}
\usepackage{graphicx}
\usepackage{tikz}
\usepackage{quiver}
\usetikzlibrary{arrows}
\usepackage{verbatim}
\usepackage[numbered]{mcode}
\usepackage{float}
\usepackage{tikz-cd}


    
\usepackage{xcolor}
\usepackage{mdframed}
\usepackage[shortlabels]{enumitem}
\usepackage{indentfirst}
\usepackage{hyperref}
    
\renewcommand{\thesubsection}{\thesection.\alph{subsection}}

\newenvironment{problem}[2][Exercise]
    { \begin{mdframed}[backgroundcolor=gray!20] \textbf{#1 #2} \\}
    {  \end{mdframed}}

% Define solution environment
\newenvironment{solution}
    {\textit{Solution:}}
    {}

%Define the claim environment
\newenvironment{claim}[1]{\par\noindent\underline{Claim:}\space#1}{}
\newenvironment{claimproof}[1]{\par\noindent\underline{Proof:}\space#1}{\hfill $\blacksquare$}

\renewcommand{\qed}{\quad\qedsymbol}
%%%%%%%%%%%%%%%%%%%%%%%%%%%%%%%%%%%%%%%%%%%%%%%%%%%%%%%%%%%%%%%%%%%%%%%%%%%%%%%%%%%%%%%%%%%%%%%%%%%%%%%%%%%%%%%%%%%%%%%%%%%%%%%%%%%%%%%%
\begin{document}
%Header-Make sure you update this information!!!!
\noindent
%%%%%%%%%%%%%%%%%%%%%%%%%%%%%%%%%%%%%%%%%%%%%%%%%%%%%%%%%%%%%%%%%%%%%%%%%%%%%%%%%%%%%%%%%%%%%%%%%%%%%%%%%%%%%%%%%%%%%%%%%%%%%%%%%%%%%%%%
\large\textbf{Zhengdong Zhang} \hfill \textbf{Homework - Week 3}   \\
Email: zhengz@uoregon.edu \hfill ID: 952091294 \\
\normalsize Course: MATH 647 - Abstract Algebra  \hfill Term: Fall 2024\\
Instructor: Dr.Victor Ostrik \hfill Due Date: $23^{th}$ October, 2024 \\
\noindent\rule{7in}{2.8pt}
%%%%%%%%%%%%%%%%%%%%%%%%%%%%%%%%%%%%%%%%%%%%%%%%%%%%%%%%%%%%%%%%%%%%%%%%%%%%%%%%%%%%%%%%%%%%%%%%%%%%%%%%%%%%%%%%%%%%%%%%%%%%%%%%%%%%%%%%
% Exercise 2.4.10
%%%%%%%%%%%%%%%%%%%%%%%%%%%%%%%%%%%%%%%%%%%%%%%%%%%%%%%%%%%%%%%%%%%%%%%%%%%%%%%%%%%%%%%%%%%%%%%%%%%%%%%%%%%%%%%%%%%%%%%%%%%%%%%%%%%%%%%%
\begin{problem}{2.4.10}
A \textit{set-through-time} \(X\) is the data of sets \(X_n\) and functors \(X_n\rightarrow X_{n+1}\) for each \(n\in \mathbb{N}\):
$$X=(X_0\rightarrow X_1\rightarrow X_2\rightarrow \cdots).$$
A morphism \(f:X\rightarrow Y\) between two sets-through-time \(X\) and \(Y\) is the data of functions \(f_n:X_n\rightarrow Y_n\) for each 
\(n\in \mathbb{N}\), such that all the squares in the following diagram commute:
$$\begin{tikzcd}
	{X_0} & {X_1} & {X_2} & \cdots \\
	{Y_0} & {Y_1} & {Y_2} & \cdots
	\arrow[from=1-1, to=1-2]
	\arrow["{f_0}"', from=1-1, to=2-1]
	\arrow[from=1-2, to=1-3]
	\arrow["{f_1}"', from=1-2, to=2-2]
	\arrow[from=1-3, to=1-4]
	\arrow["{f_2}"', from=1-3, to=2-3]
	\arrow[from=2-1, to=2-2]
	\arrow[from=2-2, to=2-3]
	\arrow[from=2-3, to=2-4]
\end{tikzcd}$$
Let \textbf{SetThruTime} be the category of all sets-through-time. Also let \(\underline{\mathbb{N}}\) be the category from Exercise 2.1.4. for the natural ordering 
and let \textbf{Func}(\(\underline{\mathbb{N}}\),\textbf{Sets}) be the functor category. Prove that there is an isomorphism of categories:
$$\textbf{Fun}(\underline{\mathbb{N}},\textbf{Sets})\xrightarrow{\sim} \textbf{SetsThruTime}.$$
mapping a functor \(\mathcal{F}:\underline{\mathbb{N}}\rightarrow \textbf{Sets}\) to the set-through-time \(X\) such that \(X_n:=\mathcal{F}(n)\) and \(X_n\rightarrow X_{n+1}\) 
is the function obtained by applying \(\mathcal{F}\) to the unique morphism \(n\rightarrow(n+1)\).
\end{problem}
\begin{solution}
Let \(\mathcal{E}:\textbf{Fun}(\underline{\mathbb{N}},\textbf{Sets})\rightarrow \textbf{SetsThruTime}\) be the functor described in the problem. We define a functor 
$$\mathcal{G}:\textbf{SetsThruTime}\rightarrow \textbf{Fun}(\underline{\mathbb{N},\textbf{Sets}})$$
as follows. Let \(X=(X_0\rightarrow X_1\rightarrow X_2\rightarrow\cdots)\) be a set-through-time. \(\mathcal{G}X\) is a functor from \(\underline{\mathbb{N}}\) to \textbf{Sets}. We 
define for each \(n\in \mathbb{N}\), \(\mathcal{G}X(n):=X_n\) and \(\mathcal{G}X(n\leq n+1):=(X_n\rightarrow X_{n+1})\), which is the map between the sets \(X_n\rightarrow X_{n+1}\). We need to check that \(\mathcal{G}X\) is indeed a functor. 
Note that \(\mathcal{G}X(n\leq n)=X_n\rightarrow X_n\) is just the identity map of the set \(X_n\). Furthermore, 
$$\mathcal{G}X(n\leq m\leq l)=(X_n\rightarrow X_m\rightarrow X_l)=(\mathcal{G}X(n\leq m))(\mathcal{G}X(m\leq l))$$
since functions between the sets can be composed. This proves that for each \(X\in \textbf{SetsThruTime}\), \(\mathcal{G}X\) is a functor. 
\par 
Next, we are going to show that \(\mathcal{G}\) itself is also a functor. Consider a morphism in \textbf{SetsThruTime}:
$$\begin{tikzcd}
	{X_0} & {X_1} & {X_2} & \cdots \\
	{Y_0} & {Y_1} & {Y_2} & \cdots
	\arrow[from=1-1, to=1-2]
	\arrow["{f_0}"', from=1-1, to=2-1]
	\arrow[from=1-2, to=1-3]
	\arrow["{f_1}"', from=1-2, to=2-2]
	\arrow[from=1-3, to=1-4]
	\arrow["{f_2}"', from=1-3, to=2-3]
	\arrow[from=2-1, to=2-2]
	\arrow[from=2-2, to=2-3]
	\arrow[from=2-3, to=2-4]
\end{tikzcd}$$
We need to show that we have a natural transformation from \(\mathcal{G}X\) to \(\mathcal{G}Y\). Note that 
$$\mathcal{G}f_i:\mathcal{G}X(i)=X_i\rightarrow \mathcal{G}Y(i)=Y_i$$ and morphisms are the same as \(X_i\rightarrow X_{i+1}\) or \(Y_i\rightarrow Y_{i+1}\). So the diagram for 
natural transformations are exactly given by the diagram above. 
\par 
Finally, note that by defintion \(\mathcal{E}\mathcal{G}\) and \(\mathcal{G}\mathcal{E}\) are both identity functors by checking objects and morphisms. This proves that we have an isomorphism of categories. 
\end{solution}
\\ 
\noindent\rule{7in}{2.8pt}
%%%%%%%%%%%%%%%%%%%%%%%%%%%%%%%%%%%%%%%%%%%%%%%%%%%%%%%%%%%%%%%%%%%%%%%%%%%%%%%%%%%%%%%%%%%%%%%%%%%%%%%%%%%%%%%%%%%%%%%%%%%%%%%%%%%%%%%%
% Exercise 2.5.6
%%%%%%%%%%%%%%%%%%%%%%%%%%%%%%%%%%%%%%%%%%%%%%%%%%%%%%%%%%%%%%%%%%%%%%%%%%%%%%%%%%%%%%%%%%%%%%%%%%%%%%%%%%%%%%%%%%%%%%%%%%%%%%%%%%%%%%%%
\begin{problem}{2.5.6}
Given any category \(\mathbf{C}\), a \textit{skeletal subcategory} of \(\mathbf{C}\) is a full subcategory \(\mathbf{S}\) such that Ob \(\mathbf{S}\) consists of 
exactly one representative for each isomorphism class of objects in \(\mathbf{C}\). Use Theorem 2.5.3. to show that \(\mathbf{C}\) and \(\mathbf{S}\) are equivalent categories.
\end{problem}
\begin{solution}
We define a functor \(\mathcal{F}:\mathbf{S}\rightarrow \mathbf{C}\) via the inclusion of the subcategory. Since \(\mathbf{S}\) is a full subcategory of \(\mathbf{C}\), the inclusion functor 
is both full and faithful. Furthermore, every object in \(\mathbf{C}\) has an isomorphic representative in \(\mathbf{S}\) by the definition of the skeletal subcategory. We have shown that 
\(\mathcal{F}\) is full, faithful and dense, by Theorem 2.5.3., \(\mathcal{F}\) is an equivalence of categories.
\end{solution}
\\ 
\noindent\rule{7in}{2.8pt}
%%%%%%%%%%%%%%%%%%%%%%%%%%%%%%%%%%%%%%%%%%%%%%%%%%%%%%%%%%%%%%%%%%%%%%%%%%%%%%%%%%%%%%%%%%%%%%%%%%%%%%%%%%%%%%%%%%%%%%%%%%%%%%%%%%%%%%%%
% Exercise 2.6.1
%%%%%%%%%%%%%%%%%%%%%%%%%%%%%%%%%%%%%%%%%%%%%%%%%%%%%%%%%%%%%%%%%%%%%%%%%%%%%%%%%%%%%%%%%%%%%%%%%%%%%%%%%%%%%%%%%%%%%%%%%%%%%%%%%%%%%%%%
\begin{problem}{2.6.1}
Let \(\mathbf{C}\) be a locally small category, \(X\in \text{Ob}\, \mathbf{C}\), \(\mathcal{F}:\mathbf{C}\rightarrow \textbf{Sets}\) be any functors, and \(g\in \mathcal{F}X\).
\begin{enumerate}
    \item Show that there is a functor \(h^X:\mathbf{C}\rightarrow \textbf{Sets}\) defined on the object \(A\) by setting \(h^X A:=\text{Hom}_{\mathbf{C}}(X,A)\), and on a morphism 
          \(f:A\rightarrow B\) by letting \(h^X f:h^X A\rightarrow h^X B\) be the function with \((h^X f)(\theta)=f\circ \theta\) for each \(\theta\in \text{Hom}_{\mathbf{C}}(X,A)\).
    \item Show that there is a natural transformation \(h^g:h^X\Rightarrow \mathcal{F}\) such that \(h^g_A: h^X A\rightarrow \mathcal{F}A\) maps \(\theta\in \text{Hom}_{\mathbf{C}}(X,A)\) 
          to \((\mathcal{F}\theta)(g)\in \mathcal{F}A\) for each \(A\in \text{Ob}\, \mathbf{C}\).
    \item If \(\mathcal{F}=h^Y\) for some \(Y\in \text{Ob}\, ,\mathbf{C}\), so that \(g\in \mathcal{F}X\) is an arrow \(g:Y\rightarrow X\) in \(\mathbf{C}\), show that the natural transformtion 
          \(h^g:h^X\Rightarrow h^Y\) from (2) is given by
          \(h^g_A:h^X A\rightarrow h^Y A\), \(\theta \mapsto \theta\circ g\) for some \(A\in \text{Ob}\, \mathbf{C}\). 
\end{enumerate}
\end{problem}
\begin{solution}
\begin{enumerate}
 \item We need to check that \(h^X\) is indeed a functor. Let \(\text{id}:A\rightarrow A\) is the identity morphism of object \(A\) in \(\mathbf{C}\). We have \(h^X\text{id}:\hom (X,A)\rightarrow \hom (X,A)\) which sends 
       any morphism \(f:X\rightarrow A\) to \(\text{id}\circ f=f\), so \(h^X\text{id}\) is still the identity morphism. Furthermore, suppose we have two morphisms \(f:A\rightarrow B\) and \(g:B\rightarrow C\), given a morphism \(h:X\rightarrow A\), apply \(h^X\) and we get:
       $$h^X(g\circ f)(h)=(g\circ f)\circ h=g\circ (f\circ h)=g\circ (h^Xf)(h)=(h^Xg)\circ(h^Xf)(h).$$
 \item We need to show that for any morphism \(f:A\rightarrow B\) in \(\mathbf{C}\), we have a commutative diagram:
       $$\begin{tikzcd}
	{h^XA} & {\mathcal{F}A} \\
	{h^XB} & {\mathcal{F}B}
	\arrow["{h^g_A}", from=1-1, to=1-2]
	\arrow["{h^Xf}"', from=1-1, to=2-1]
	\arrow["{\mathcal{F}f}", from=1-2, to=2-2]
	\arrow["{h^g_B}"', from=2-1, to=2-2]
\end{tikzcd}$$
       Given a morphism \(p:X\rightarrow A\), we have 
       $$(h^g_B\circ h^Xf)(p)=h^g_B(f\circ p)=(\mathcal{F}(f\circ p))(g)=(\mathcal{F}f\circ \mathcal{F}p)(g).$$
       On the other hand, 
       $$(\mathcal{F}f\circ h^g_A)(p)=\mathcal{F}f(\mathcal{F}p)(g)=(\mathcal{F}f\circ \mathcal{F}p)(g).$$
       This proves that the diagram commutes. So \(h^g:h^X\Rightarrow \mathcal{F}\) is indeed a natural transformation.
 \item By the definition in (2), we need to show that given a morphism \(\theta:X\rightarrow A\) in \(\mathbf{C}\), \(h^g_A\) maps \(\theta\) to 
       $$(h^Y\theta)(g)=\theta\circ g$$
       is just the composition.  
\end{enumerate}
\end{solution}

\noindent\rule{7in}{2.8pt}
%%%%%%%%%%%%%%%%%%%%%%%%%%%%%%%%%%%%%%%%%%%%%%%%%%%%%%%%%%%%%%%%%%%%%%%%%%%%%%%%%%%%%%%%%%%%%%%%%%%%%%%%%%%%%%%%%%%%%%%%%%%%%%%%%%%%%%%%
% Exercise 2.6.3
%%%%%%%%%%%%%%%%%%%%%%%%%%%%%%%%%%%%%%%%%%%%%%%%%%%%%%%%%%%%%%%%%%%%%%%%%%%%%%%%%%%%%%%%%%%%%%%%%%%%%%%%%%%%%%%%%%%%%%%%%%%%%%%%%%%%%%%%
\begin{problem}{2.6.3}
Let \(\mathcal{P}:\textbf{Sets}\rightarrow \textbf{Sets}\) be the covariant power set functor, i.e. for a set \(X\), \(\mathcal{P}X\) is its power 
set \(P(X)\), and for a function \(f:X\rightarrow Y\) sends a subset of \(X\) to its image under \(f\). Show that \(\mathcal{P}\) is not representable.
\end{problem}
\begin{solution}
Suppose the covariant power set functor is represented by a set \(A\). Let \(X\) be a set with only one element. Then the power set \(\mathcal{P}X\) has two elements, the empty set \(\varnothing\) and \(X\) itself. 
So, the Hom set \(\hom(A,X)\) should also have two maps. However, there only exists one morphism from any set to a set with only one element. A contradiction. So \(\mathcal{P}\) is not representable.  
\end{solution}
\\
\noindent\rule{7in}{2.8pt}
%%%%%%%%%%%%%%%%%%%%%%%%%%%%%%%%%%%%%%%%%%%%%%%%%%%%%%%%%%%%%%%%%%%%%%%%%%%%%%%%%%%%%%%%%%%%%%%%%%%%%%%%%%%%%%%%%%%%%%%%%%%%%%%%%%%%%%%%
% Exercise 3.1.4
%%%%%%%%%%%%%%%%%%%%%%%%%%%%%%%%%%%%%%%%%%%%%%%%%%%%%%%%%%%%%%%%%%%%%%%%%%%%%%%%%%%%%%%%%%%%%%%%%%%%%%%%%%%%%%%%%%%%%%%%%%%%%%%%%%%%%%%%
\begin{problem}{3.1.4}
Show that the ring \(\mathbb{Z}\) is an initial object in the category \textbf{Rings}, and that any trivial ring is a terminal object. What about in the category \(\textbf{Alg}(\mathbb{F})\)?
\end{problem}
\begin{solution}
Let \(R\) be the ring with multiplicative identity \(1_R\). We define a map \(f:\mathbb{Z}\rightarrow R\) by sending \(a\in \mathbb{Z}\) to \(a\cdot 1_R\in R\). This is a ring homomorphism because 
for any \(a,b\in \mathbb{Z}\), we have \(f(ab)=ab\cdot 1_R=a\cdot 1_R\cdot b\cdot 1_R=f(a)f(b)\) and \(f(a+b)=(a+b)\cdot 1_R=a\cdot 1_R+b\cdot 1_R=f(a)+f(b)\). This ring homomorphism \(f\) is also 
unique because the definition of ring homomorphisms requires that it must send \(1\) to \(1_R\), and then by extending it \(\mathbb{Z}\)-linearly we can define it on the whole \(\mathbb{Z}\). This proves that 
\(\mathbb{Z}\) is the initial object in \textbf{Rings}.
\par 
Any ring \(R\) has a unique zero map to the trivial ring. This shows that the trivial ring the terminal object in \textbf{Rings}.
\par 
The initial object in \(\textbf{Alg}(\mathbb{F})\) is the field \(\mathbb{F}\). Indeed, \(\mathbb{F}\) is the trivial \(\mathbb{F}\)-algebra and for any 
\(\mathbb{F}\)-algebra \(S\), we have a homomorphism \(\mathbb{F}\rightarrow S\) by sending \(a\in \mathbb{F}\) to \(a\cdot 1_S\). This homomorphism is also unique as we have to send \(1\in \mathbb{F}\) to 
\(1_S\in S\). The trivial ring \(\left\{ 0 \right\}\) can also be viewed as an \(\mathbb{F}\)-algebra, so it is the terminal object in \(\textbf{Alg}(\mathbb{F})\).
\end{solution}
\\ 
\noindent\rule{7in}{2.8pt}
%%%%%%%%%%%%%%%%%%%%%%%%%%%%%%%%%%%%%%%%%%%%%%%%%%%%%%%%%%%%%%%%%%%%%%%%%%%%%%%%%%%%%%%%%%%%%%%%%%%%%%%%%%%%%%%%%%%%%%%%%%%%%%%%%%%%%%%%
% Exercise 3.1.7
%%%%%%%%%%%%%%%%%%%%%%%%%%%%%%%%%%%%%%%%%%%%%%%%%%%%%%%%%%%%%%%%%%%%%%%%%%%%%%%%%%%%%%%%%%%%%%%%%%%%%%%%%%%%%%%%%%%%%%%%%%%%%%%%%%%%%%%%
\begin{problem}{3.1.7}
Prove that there are no initial or terminal object in \textbf{Fields}.
\end{problem}
\begin{solution}
By definition a field needs to have at least two distinctive elements, the additive identity \(0\) and the multiplicative identity \(1\). By definition, for any 
field homomorphism \(f:E\rightarrow F\), we must have \(f(0_E)=0_F\) and \(f(1_E)=1_F\). Note that \(p\cdot 1_F=p\cdot f(1_E)=f(p\cdot 1_E)\), so if the characteristic of \(E\) is p, 
then the characteristic of \(F\) must also be \(p\), for any prime number \(p\) or \(p=0\). This means that there does not exist field homomorphisms between fields of different 
characteristics. Therefore, we do not have initial or terminal object in the category \textbf{Fields}.   
\end{solution}
\\ 
\noindent\rule{7in}{2.8pt}
%%%%%%%%%%%%%%%%%%%%%%%%%%%%%%%%%%%%%%%%%%%%%%%%%%%%%%%%%%%%%%%%%%%%%%%%%%%%%%%%%%%%%%%%%%%%%%%%%%%%%%%%%%%%%%%%%%%%%%%%%%%%%%%%%%%%%%%%
% Exercise 3.2.4
%%%%%%%%%%%%%%%%%%%%%%%%%%%%%%%%%%%%%%%%%%%%%%%%%%%%%%%%%%%%%%%%%%%%%%%%%%%%%%%%%%%%%%%%%%%%%%%%%%%%%%%%%%%%%%%%%%%%%%%%%%%%%%%%%%%%%%%%
\begin{problem}{3.2.4}
Let \(X\) be a partially ordered set. In the category \underline{X} from Exercise 2.1.4., show that the coproduct of \(x\) and \(y\) is their 
supremum \(x\vee y\), and the product of \(x\) and \(y\) is their infimum \(x\wedge y\). If \(x\vee y\)(resp. \(x\wedge y\)) does not exist then the 
coproduct (resp. product) of \(x\) and \(y\) does not exist.
\end{problem}
\begin{solution}
Let \(x,y,z\in \underline{X}\) satisfying \(z\leq x\) and \(z\leq y\). By definition, it is easy to see that \((x\wedge y)\leq x\) and \((x\wedge y)\leq y\). We have a 
diagram as follows:
$$\begin{tikzcd}
	z \\
	& {x\wedge y} & y \\
	& x
	\arrow[dashed, from=1-1, to=2-2]
	\arrow[curve={height=-12pt}, from=1-1, to=2-3]
	\arrow[curve={height=18pt}, from=1-1, to=3-2]
	\arrow[from=2-2, to=2-3]
	\arrow[from=2-2, to=3-2]
\end{tikzcd}$$
We need to show that there exists a unique map \(f:z\rightarrow x\wedge y\). And this is true because by definition of infimum, \(z\leq x\) and \(z\leq y\) give us \(z\leq x\wedge y\). This is unique since 
each Hom set has only one element. Thus, \(x\wedge y\), if exists, is the product of \(x\) and \(y\) in \(\underline{X}\). Reverse all the arrows and all the inequalities, we get that 
\(x\vee y\), if exists, is the coproduct of \(x\) and \(y\) in \(\underline{X}\). 
\end{solution}
\\ 
\noindent\rule{7in}{2.8pt}
%%%%%%%%%%%%%%%%%%%%%%%%%%%%%%%%%%%%%%%%%%%%%%%%%%%%%%%%%%%%%%%%%%%%%%%%%%%%%%%%%%%%%%%%%%%%%%%%%%%%%%%%%%%%%%%%%%%%%%%%%%%%%%%%%%%%%%%%
% Exercise 3.2.8
%%%%%%%%%%%%%%%%%%%%%%%%%%%%%%%%%%%%%%%%%%%%%%%%%%%%%%%%%%%%%%%%%%%%%%%%%%%%%%%%%%%%%%%%%%%%%%%%%%%%%%%%%%%%%%%%%%%%%%%%%%%%%%%%%%%%%%%%
\begin{problem}{3.2.8}
Let \(V\) be an abelian group (resp. a vector spce over \(\mathbb{F}\)) and \((V_i)_{i\in I}\) be a family of subgroups (resp. subspaces). Define \(\Sigma_{i\in I}V_i\) 
to be the subgroup (resp. subspaces) of \(V\) consisting of the elements of the form \(\Sigma_{i\in I}v_i\) for \(v_i\in V_i\), all but finitely many of which are zero. Show that 
the map 
$$\oplus_{i\in I}V_i\rightarrow V,\ (v_i)_{i\in I}\mapsto \Sigma_{i\in I}v_i$$
is an isomorphism if and only if the following conditions hold:
\begin{enumerate}
    \item \(V=\sum_{i\in I}V_i\).
    \item \(V_i\cap \sum_{j\in I, j\neq i}V_j=\left\{ 0 \right\}\) for each \(i\in I\).
\end{enumerate}
The conditions (1)-(2) are equivalent to saying that every \(v\in V\) can be written as \(v=\sum_{i\in I}v_i\) for unique vectors \(v_i\in V_i\), all but 
finitely many of which are zero.
\end{problem}
\begin{solution}
Suppose the map is an isomorphism. By Theorem 3.2.7, we know that \(V\cong \oplus_{i\in I}V_i\) is the coproduct of \((V_i)_{i\in I}\). Note that for each \(i\in I\), 
\(V_i\subset V\) is a subgroup (resp. subspace), we have \(\Sigma_{i\in I}V_i\subset V\). On the other hand, since \(\oplus_{i\in I}V_i\rightarrow V\) is an isomorphism, every \(v\in V\) can be written as 
\(\Sigma_{i\in I}v_i\). Thus, \(V=\Sigma_{i\in I}V_i\) and we have proved (1). For (2), fix an \(i\in I\), suppose \(w\in (V_i\cap (\Sigma_{j\in I, j\neq i}V_j))\). We know that \(w\) 
can be written as:
$$w=\Sigma_{j\in I,j\neq i}v_j$$
with only finitely many \(v_j\) nonzero. Consider \(w\) as a single entry of \(V_i\) and the element \((v_j)_{j\in I, j\neq i}\in \oplus_{j\in I}V_j\). Since the map \(\oplus_{j\in I}V_j\rightarrow V\) is an isomorphism, \(w\) and \((v_j)_{j\in I, j\neq i}\in \oplus_{j\in I}V_j\) 
is the same element in \(\oplus_{j\in I}V_j\). This shows that for each \(j\in I, j\neq i\), \(v_j=0\) and \(w=0\). So we proved (2). 
\par 
Conversely, assume the conditions (1) and (2) are satisfied. Write the original map:
\begin{align*}
    f:\oplus_{i\in I}V_i & \rightarrow V,\\ 
    (v_i)_{i\in I} & \mapsto \Sigma_{i\in I}v_i.
\end{align*}
We are going to construct an inverse \(g:V\rightarrow \oplus_{i\in I}V_i\). For each \(i\in I\), given \(v\in V\), since \(V=\Sigma_{i\in I}V_i\), we could write \(v=(v_i)_{i\in I}\). We define \(g_i:V\rightarrow V_i\) by sending 
\(v\) to \(v_i\). Note that \(\oplus_{i\in I}V_i\) is the coproduct of \((V_i)_{i\in I}\), so by the universal property, we have a map \(g:V\rightarrow \oplus_{i\in I}V_i\). This map \(g\) is unique since condition (2) is satisfied, which means 
every element \(v\in V\) can be uniquely written as finite sum of elements from \(V_i\). Next, we need to check \(f\circ g=\text{id}\) and \(g\circ f=\text{id}\). Write \(v=(v_i)_{i\in I}\) and use the universal property of \(g\), this is true. 
\end{solution}
\\ 
\noindent\rule{7in}{2.8pt}
%%%%%%%%%%%%%%%%%%%%%%%%%%%%%%%%%%%%%%%%%%%%%%%%%%%%%%%%%%%%%%%%%%%%%%%%%%%%%%%%%%%%%%%%%%%%%%%%%%%%%%%%%%%%%%%%%%%%%%%%%%%%%%%%%%%%%%%%
% Exercise 3.3.3
%%%%%%%%%%%%%%%%%%%%%%%%%%%%%%%%%%%%%%%%%%%%%%%%%%%%%%%%%%%%%%%%%%%%%%%%%%%%%%%%%%%%%%%%%%%%%%%%%%%%%%%%%%%%%%%%%%%%%%%%%%%%%%%%%%%%%%%%
\begin{problem}{3.3.3}
Let \(M\) be a commutative monoid with operation \(+\). Let 
$$A(M):=(M\times M)/\sim$$
where \(\sim\) is the equivalence relation on \(M\times M\) defined from \((p,m)\sim (p',m')\Leftrightarrow p+m'+q=p'+m+q\) for some \(q\in M\). 
Introduce the symbol \(p-m\) to denote the equivalence class of \((p,m)\in M\times M\). Show that \(A(M)\) is a well-defined abelian group with operation 
\((p-m)+(p'+m'):=(p+p')-(m+m')\) and that the map \(\iota:M\rightarrow A(M), p\mapsto p-0_M\) is a monoid homomorphism. Show further that \(\iota\) is injective 
if and only if \(M\) is cancellative in the obvious sense. What well-known abelian group is \(A(\mathbb{N})\)?
\end{problem}
\begin{solution}
For every \(m,n\in M\), we have \((m,m)\sim (n,n)\) since \(m+n=m+n\). And \((m,m)\) is the identity element in \(A(M)\) because for \((p,q)\in A(M)\), we have 
\((m,m)+(p,q)=(m+p,m+q)\sim (p,q)\) as \(m+p+q=m+q+p\). Write \((m,m)=e\). The operation defined is compatible with the equivalence relations. Suppose \((p,q)\sim (p',q')\) and 
\((s,t)\sim (s',t')\), which means there exists \(m,n\in M\) such that \(p+q'+m=p'+q+m\) and \(s+t'+n=s'+t+n\). Add these two together, we have \(p+s+q'+t'+m+n=q+t+p'+s'+m+n\). This implies 
\((p+q,s+t)\sim (p'+q',s'+t')\).
\par 
Next, we show that the operation is associative and commutative. Let \((m_1,n_1),(m_2,n_2),(m_3,n_3)\) be in \(A(M)\). We have 
$$(m_1,n_1)+(m_2,n_2)=(m_1+m_2,n_1+n_2)=(m_2+m_1,n_2+n_1)=(m_2,n_2)+(m_1,n_1).$$
This proves that \(A(M)\) is commutative. Moreover, 
$$\begin{align*}
    ((m_1,n_1)+(m_2,n_2))+(m_3,n_3) & = (m_1+m_2,n_1+n_2)+(m_3,n_3)\\ 
                                    & = (m_1+m_2+m_3,n_1+n_2+n_3)\\ 
                                    & = (m_1,n_1)+(m_2+m_3,n_2+n+3)\\ 
                                    & = (m_1,n_1)+((m_2,n_2)+(m_3,n_3)).
\end{align*}$$ 
This is the operation in \(A(M)\) is associative. Now we need to show that every \((p,q)\in A(M)\) has inverse. Indeed, we have 
$$(p,q)+(q,p)=(p+q,q+p)\sim e.$$
So \((q,p)\) is the inverse of \((p,q)\) in \(A(M)\). This concludes that \(A(M)\) is an abelian group. Consider the map:
$$\begin{align*}
    \iota:M &\rightarrow A(M),\\ 
          p &\mapsto (p,0_M)
\end{align*}$$
This is a monoid homomorphism as \(\iota(p+q)=(p+q,0_M)=(p,0_M)+(q,0_M)=\iota(p)+\iota(q)\). If we further assume \(M\) is cancellative, let \(p\in \ker \iota\), we know that 
\((p,0_M)\sim (p,p)\), which means \(2p+m=p+m\) for some \(m\in M\). If \(M\) is cancellative, then \(p=0_M\) and thus \(\iota\) is injective. Conversely, assume \(\iota\) is injective. Let 
\(a,b\in M\) and \(a+m=b+m\) for some \(m\in M\). This implies \(\iota(a)=(a,0_M)\sim (b,0_M)=\iota(b)\) in \(A(M)\). Because \(\iota\) is injective, so \(a=b\) in \(M\). This shows that \(M\) is a 
cancellative monoid.
\par
The group completion \(A(\mathbb{N})\) is the abelian group \((\mathbb{Z},+)\). To see this, we prove the universal property for group completion. Suppose \(A\) is an abelian group with a monoid homomorphism \(f:\mathbb{N}\rightarrow A\). 
Then we can define a group homomorphism \(\bar{f}:\mathbb{Z}\rightarrow A\) as follows. If \(n\in \mathbb{Z}\) is natural number, then \(\bar{f}(n)=f(n)\). If \(n\in \mathbb{Z}\) is negative, then \(\bar{f}(n)=f(0)-f(-n)\). We have a commutative diagram: 
$$\begin{tikzcd}
	{\mathbb{N}} & {\mathbb{Z}} \\
	A
	\arrow[from=1-1, to=1-2]
	\arrow[from=1-1, to=2-1]
	\arrow[dashed, from=1-2, to=2-1]
\end{tikzcd}$$
This morphism \(\bar{f}\) is unique as it has been restricted by how it maps on \(\mathbb{N}\subset \mathbb{Z}\). This shows that \(A(\mathbb{N})=\mathbb{Z}\). 

\end{solution}

\noindent\rule{7in}{2.8pt}
%%%%%%%%%%%%%%%%%%%%%%%%%%%%%%%%%%%%%%%%%%%%%%%%%%%%%%%%%%%%%%%%%%%%%%%%%%%%%%%%%%%%%%%%%%%%%%%%%%%%%%%%%%%%%%%%%%%%%%%%%%%%%%%%%%%%%%%%
% Exercise 3.3.9
%%%%%%%%%%%%%%%%%%%%%%%%%%%%%%%%%%%%%%%%%%%%%%%%%%%%%%%%%%%%%%%%%%%%%%%%%%%%%%%%%%%%%%%%%%%%%%%%%%%%%%%%%%%%%%%%%%%%%%%%%%%%%%%%%%%%%%%%
\begin{problem}{3.3.9}
If \(f:G\rightarrow H\) is a homomorphism of groups, its image \(\text{im}f\) is a subgroup of \(H\) and its kernel \(\text{ker}f:=f^{-1}(1_H)\) is a normal 
subgroup of \(G\). Show that \(f\) induces a group isomorphism \(\bar{f}:G/\text{ker}f\xrightarrow{\sim} \text{im}f\).
\end{problem}
\begin{solution}
Without loss of generality, we could assume \(f\) is surjective and in this case \(\text{im}f=H\). Write \(\ker f=K\unlhd G\) and by the definition of the kernel, we have 
\(f(K)=\left\{ 1_H \right\}\). Use Theorem 3.3.8 (Universal property of quotient groups), we have a unique group homomorphism \(\bar{f}:G/K\rightarrow H\). By surjectivity of \(f\), for 
every \(h\in H\), there exist \(g\in G\) such that \(f(g)=h\), thus \(gK\) is the preimage of \(h\) under the map \(\bar{f}\). This proves that \(\bar{f}\) is surjective. Moreover, 
Suppose \(gK\in \ker \bar{f}\), we know that \(1_H=\bar{f}(gK)=(\bar{f}\circ \pi)(g)=f(g)\) where \(\pi:G\rightarrow G/K\) is the canonical projection. This implies \(g\in \ker f=K\), therefore \(gK=K\) is the 
identity element in \(\ker \bar{f}\). We have shown \(\ker \bar{f}\) is trivial, thus \(\bar{f}\) is injective. Now we can conclude that \(\bar{f}\) is a group isomorphism.
\end{solution}



\end{document}
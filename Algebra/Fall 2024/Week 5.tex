\documentclass[a4paper, 12pt]{article}
\usepackage{comment} % enables the use of multi-line comments (\ifx \fi) 
\usepackage{lipsum} %This package just generates Lorem Ipsum filler text. 
\usepackage{fullpage} % changes the margin
\usepackage[a4paper, total={7in, 10in}]{geometry}
\usepackage{amsmath}
\usepackage{amssymb,amsthm}  % assumes amsmath package installed
\newtheorem{theorem}{Theorem}
\newtheorem{corollary}{Corollary}
\usepackage{graphicx}
\usepackage{tikz}
\usepackage{quiver}
\usetikzlibrary{arrows}
\usepackage{verbatim}
\usepackage{float}
\usepackage{tikz-cd}
\usepackage[backend=biber,bibencoding=utf8,style=numeric,sorting=ynt]{biblatex}

    
\usepackage{xcolor}
\usepackage{mdframed}
\usepackage[shortlabels]{enumitem}
\usepackage{indentfirst}
\usepackage{hyperref}
    
\renewcommand{\thesubsection}{\thesection.\alph{subsection}}

\newenvironment{problem}[2][Exercise]
    { \begin{mdframed}[backgroundcolor=gray!20] \textbf{#1 #2} \\}
    {  \end{mdframed}}

% Define solution environment
\newenvironment{solution}
    {\textit{Solution:}}
    {}

%Define the claim environment
\newenvironment{claim}[1]{\par\noindent\underline{Claim:}\space#1}{}
\newenvironment{claimproof}[1]{\par\noindent\underline{Proof:}\space#1}{\hfill $\blacksquare$}

\renewcommand{\qed}{\quad\qedsymbol}
%%%%%%%%%%%%%%%%%%%%%%%%%%%%%%%%%%%%%%%%%%%%%%%%%%%%%%%%%%%%%%%%%%%%%%%%%%%%%%%%%%%%%%%%%%%%%%%%%%%%%%%%%%%%%%%%%%%%%%%%%%%%%%%%%%%%%%%%
\begin{document}
%Header-Make sure you update this information!!!!
\noindent
%%%%%%%%%%%%%%%%%%%%%%%%%%%%%%%%%%%%%%%%%%%%%%%%%%%%%%%%%%%%%%%%%%%%%%%%%%%%%%%%%%%%%%%%%%%%%%%%%%%%%%%%%%%%%%%%%%%%%%%%%%%%%%%%%%%%%%%%
\large\textbf{Zhengdong Zhang} \hfill \textbf{Homework - Week 5}   \\
Email: zhengz@uoregon.edu \hfill ID: 952091294 \\
\normalsize Course: MATH 647 - Abstract Algebra  \hfill Term: Fall 2024\\
Instructor: Dr.Victor Ostrik \hfill Due Date: $6^{th}$ November, 2024 \\
\noindent\rule{7in}{2.8pt}
%%%%%%%%%%%%%%%%%%%%%%%%%%%%%%%%%%%%%%%%%%%%%%%%%%%%%%%%%%%%%%%%%%%%%%%%%%%%%%%%%%%%%%%%%%%%%%%%%%%%%%%%%%%%%%%%%%%%%%%%%%%%%%%%%%%%%%%%
% Exercise 3.7.6
%%%%%%%%%%%%%%%%%%%%%%%%%%%%%%%%%%%%%%%%%%%%%%%%%%%%%%%%%%%%%%%%%%%%%%%%%%%%%%%%%%%%%%%%%%%%%%%%%%%%%%%%%%%%%%%%%%%%%%%%%%%%%%%%%%%%%%%%
\begin{problem}{3.7.6}
Let \(X\) and \(Y\) be pointed topological spaces and \(Z\) be the one point space. Let \(p:Z\rightarrow X\) and \(q:Z\rightarrow Y\) be the inclusions of the 
basepoints. Show that a pushout is given by the wedge sum \(X\vee  Y\).
\end{problem}
\begin{solution}
Suppose \(X\) is pointed at \(x_0\in X\) and \(Y\) is pointed at \(y_0\in Y\). Assume \(Z=\left\{ z \right\}\) is the one-point space. We are going to prove the universal property of the 
wedge sum \(X\vee Y:=X\sqcup Y/x_0\sim y_0\). Given a topological space \(W\) along the following commutative diagram:
\[\begin{tikzcd}
	Z & X \\
	Y & W
	\arrow["p", from=1-1, to=1-2]
	\arrow["q"', from=1-1, to=2-1]
	\arrow["f", from=1-2, to=2-2]
	\arrow["g"', from=2-1, to=2-2]
\end{tikzcd}\]
We write \(f(p(z))=f(x_0)=g(q(z))=g(y_0)=w\in W\). We define a unique map \(h:X\sqcup Y\rightarrow W\) in the obvious way as follows: for any \(x\in X\), \(h(x)=f(x)\), and for any \(y\in Y\), \(h(y)=g(y)\). This map is continous since 
\(f\) and \(g\) are continous. Note that \(h(x_0)=f(x_0)=w=g(y_0)=h(y_0)\), so by the universal property of quotient maps, we have a commutative diagram:
\[\begin{tikzcd}
	{X\sqcup Y} & W \\
	{X\vee Y}
	\arrow["h", from=1-1, to=1-2]
	\arrow[from=1-1, to=2-1]
	\arrow["{\bar{h}}"', from=2-1, to=1-2]
\end{tikzcd}\]
where \(X\sqcup Y\rightarrow X\vee Y\) is the quotient map. Here the unique map \(\bar{h}:X\vee Y\rightarrow W\) is also continous. Denote by \(p'\) the inclusion map \(p':X\rightarrow X\vee Y\) and by \(q'\) 
the inclusion map \(q':Y\rightarrow X\vee Y\). We know that for any \(x\in X\), \(\bar{h}(p'(x))=h(x)=f(x)\) and for any \(y\in Y\), \(\bar{h}(q'(y))=h(y)=g(y)\). We have a commutative diagram:
\[\begin{tikzcd}
	Z & X \\
	Y & {X\vee Y} \\
	&& W
	\arrow["p", from=1-1, to=1-2]
	\arrow["q"', from=1-1, to=2-1]
	\arrow["{p'}", from=1-2, to=2-2]
	\arrow["f", curve={height=-12pt}, from=1-2, to=3-3]
	\arrow["{q'}"', from=2-1, to=2-2]
	\arrow["g"', curve={height=12pt}, from=2-1, to=3-3]
	\arrow["{\exists! \bar{h}}", dashed, from=2-2, to=3-3]
\end{tikzcd}\]
This proves the universal property of \(X\vee Y\), thus, \(X\vee Y\) is the pushout of the diagram:
\[\begin{tikzcd}
	Z & X \\
	Y
	\arrow["p", from=1-1, to=1-2]
	\arrow["q"', from=1-1, to=2-1]
\end{tikzcd}\]
\end{solution}

\noindent\rule{7in}{2.8pt}
%%%%%%%%%%%%%%%%%%%%%%%%%%%%%%%%%%%%%%%%%%%%%%%%%%%%%%%%%%%%%%%%%%%%%%%%%%%%%%%%%%%%%%%%%%%%%%%%%%%%%%%%%%%%%%%%%%%%%%%%%%%%%%%%%%%%%%%%
% Exercise 3.7.8
%%%%%%%%%%%%%%%%%%%%%%%%%%%%%%%%%%%%%%%%%%%%%%%%%%%%%%%%%%%%%%%%%%%%%%%%%%%%%%%%%%%%%%%%%%%%%%%%%%%%%%%%%%%%%%%%%%%%%%%%%%%%%%%%%%%%%%%%
\begin{problem}{3.7.8}
Supoose the directed set \(I\) has a greatest element \(n\) (there can be more than one such element since \(I\) is merely preordered not partially ordered). Prove that 
\((X_n,(f_{i,n})_{i\in I})\) is an inverse limit of \((X_i)_{i\in I}\) for any inverse system \((X_i)_{i\in I}\) in any category \(\mathbb{C}\).
\end{problem}
\begin{solution}
That \(n\) is a greatest element in \(I\) implies that for every \(i\in I\), we have \(i\leq n\). We have a collection of morphism \(f_{i,n}:X_n\rightarrow X_i\) since \((X_i)_{i\in I}\) is an 
inverse system. To show that \((X_n,(f_{i,n})_{i\in I})\) is the limit, we prove its universal property. Suppose \((Y,g)\) is a cone in \(\mathbf{C}\) and for the index category \(\mathbf{I}\). For every \(i\in I\), we have a commutative diagram:
\[\begin{tikzcd}
	& Y \\
	{X_n} && {X_i}
	\arrow["{g_n}"', from=1-2, to=2-1]
	\arrow["{g_i}", from=1-2, to=2-3]
	\arrow["{f_{i,n}}"', from=2-1, to=2-3]
\end{tikzcd}\]
Given any \(i\leq j\) in \(\mathbf{I}\), we have \(i\leq j\leq n\) because \(n\) is a greast element. That \((X_i)_{i\in I}\) is an inverse system gives us a commutative diagram:
\[\begin{tikzcd}
	& {X_n} \\
	{X_j} && {X_i}
	\arrow["{f_{j,n}}"', from=1-2, to=2-1]
	\arrow["{f_{i,n}}", from=1-2, to=2-3]
	\arrow["{f_{i,j}}"', from=2-1, to=2-3]
\end{tikzcd}\]
Combine the above two commutative diagrams together, and we have:
\[\begin{tikzcd}
	& Y \\
	& {X_n} \\
	{X_j} && {X_i}
	\arrow["{g_n}", from=1-2, to=2-2]
	\arrow["{g_j}"', curve={height=12pt}, from=1-2, to=3-1]
	\arrow["{g_i}", curve={height=-12pt}, from=1-2, to=3-3]
	\arrow["{f_{j,n}}"', from=2-2, to=3-1]
	\arrow["{f_{i,n}}", from=2-2, to=3-3]
	\arrow["{f_{i,j}}"', from=3-1, to=3-3]
\end{tikzcd}\]
Here \(g_n\) is unique because to make the following diagram commutes:
\[\begin{tikzcd}
	& Y \\
	{X_n} && {X_n}
	\arrow["{g_n}"', from=1-2, to=2-1]
	\arrow["{g_n}", from=1-2, to=2-3]
	\arrow["id"', from=2-1, to=2-3]
\end{tikzcd}\]
We only have one choice for \(g_n\), this proves the universal property of \(X_n\).
\end{solution}

\noindent\rule{7in}{2.8pt}
%%%%%%%%%%%%%%%%%%%%%%%%%%%%%%%%%%%%%%%%%%%%%%%%%%%%%%%%%%%%%%%%%%%%%%%%%%%%%%%%%%%%%%%%%%%%%%%%%%%%%%%%%%%%%%%%%%%%%%%%%%%%%%%%%%%%%%%%
% Exercise 3.7.9
%%%%%%%%%%%%%%%%%%%%%%%%%%%%%%%%%%%%%%%%%%%%%%%%%%%%%%%%%%%%%%%%%%%%%%%%%%%%%%%%%%%%%%%%%%%%%%%%%%%%%%%%%%%%%%%%%%%%%%%%%%%%%%%%%%%%%%%%
\begin{problem}{3.7.9}
Given an inverse system of sets, if all the functions \(f_{i,j}:X_j\rightarrow X_i\) are injective (resp. bijective), show that all the functions \(\pi_i:\varprojlim X_i\rightarrow X_i\) are 
injective (resp. bijective) too.
\end{problem}
\begin{solution}
First note that if \(\varprojlim X_i=\varnothing\), then all the maps are automatically injective. Now assume the inverse limit is not empty and there exist \(j\in I\) such that \(\pi_j:\varprojlim X_i\rightarrow X_j\) is not injective. More specifically, there exists \(a,b\in \varprojlim X_i\) with \(a\neq b\) such that 
\(\pi_j(a)=\pi_j(b)\in X_j\). Consider the following set \(Y:=\varprojlim X_i\setminus \left\{ b \right\}\). For each \(i\in I\), we define \(\pi_i':Y\rightarrow X_i\) by just restricting \(\pi_i\) to \(Y\). For each \(X_i\) with a morphism 
\(X_j\rightarrow X_i\) and \(X_k\) with a morphism \(X_k\rightarrow X_j\), the following diagrams still commutes:
\[\begin{tikzcd}
	& Y &&& Y \\
	{X_j} && {X_i} & {X_k} && {X_j}
	\arrow["{\pi_j'}"', from=1-2, to=2-1]
	\arrow["{\pi_i'}", from=1-2, to=2-3]
	\arrow["{\pi_k'}"', from=1-5, to=2-4]
	\arrow["{\pi_j'}", from=1-5, to=2-6]
	\arrow["{f_{i,j}}"', from=2-1, to=2-3]
	\arrow["{f_{j,k}}"', from=2-4, to=2-6]
\end{tikzcd}\]
because \(f_{i,j}\) and \(f_{j,k}\) are injective. So \(\pi_i':Y\rightarrow X_i\) are well-defined morphisms of sets for all \(i\). We have th following two diagrams:
\[\begin{tikzcd}
	& Y &&& Y \\
	& {\varprojlim X_i} &&& {\varprojlim X_i} \\
	{X_j} && {X_j} & {X_j} && {X_j}
	\arrow["{f_1}"', from=1-2, to=2-2]
	\arrow["{\pi_j'}"', curve={height=12pt}, from=1-2, to=3-1]
	\arrow["{\pi_j'}", curve={height=-12pt}, from=1-2, to=3-3]
	\arrow["{f_2}"', from=1-5, to=2-5]
	\arrow["{\pi_j'}"', curve={height=12pt}, from=1-5, to=3-4]
	\arrow["{\pi_j'}", curve={height=-12pt}, from=1-5, to=3-6]
	\arrow["{\pi_j}"', from=2-2, to=3-1]
	\arrow["{\pi_j}", from=2-2, to=3-3]
	\arrow["{\pi_j}"', from=2-5, to=3-4]
	\arrow["{\pi_j}", from=2-5, to=3-6]
	\arrow["id"', from=3-1, to=3-3]
	\arrow["id"', from=3-4, to=3-6]
\end{tikzcd}\]
where \(f_1:Y\rightarrow \varprojlim X_i\) sending \(a\) to \(a\) and \(f_2:Y\rightarrow \varprojlim X_i\) sending \(a\) to \(b\). Since \(\pi_j(a)=\pi_j(b)\), both two above diagrams commutes. This contradicts the universal 
property of \(\varprojlim X_i\) as we only have one unique morphism from \(Y\) to \(\varprojlim X_i\).
\par 
Now assume in addition each \(X_j\rightarrow X_i\) is a bijective map of sets. Suppose there exist \(j\in I\) such that \(\pi_j:\varprojlim X_i\rightarrow X_j\) is injective but not surjective. Since every morphism in the inverse system is 
bijective, therefore every \(\pi_i:\varprojlim X_i\rightarrow X_i \) is not surjective. Consider the set \(Y\) which is bijective to every \(X_i\) and the inclusion map \(f:\varprojlim X_i\rightarrow Y\). For 
any \(i,j\in I\), we have a commutative triangle:
\[\begin{tikzcd}
	& Y \\
	{X_j} && {X_i}
	\arrow[from=1-2, to=2-1]
	\arrow[from=1-2, to=2-3]
	\arrow[from=2-1, to=2-3]
\end{tikzcd}\]
because every map here is bijective. This shows that \(Y\) is also a cone along with the commutative diagrams:
\[\begin{tikzcd}
	& {\varprojlim X_i} \\
	& Y \\
	{X_j} && {X_i}
	\arrow["f", from=1-2, to=2-2]
	\arrow["{\pi_j}"', curve={height=12pt}, from=1-2, to=3-1]
	\arrow["{\pi_i}", curve={height=-12pt}, from=1-2, to=3-3]
	\arrow["\sim"', from=2-2, to=3-1]
	\arrow["\sim", from=2-2, to=3-3]
	\arrow["\sim"', from=3-1, to=3-3]
\end{tikzcd}\]
By the universal property of inverse limit and uniqueness we know that \(Y\) must be bijective to \(\varprojlim X_i\) but \(f\) is a strict inclusion since \(\pi_i\) is not surjective. A contradiction.
\end{solution}

\noindent\rule{7in}{2.8pt}
%%%%%%%%%%%%%%%%%%%%%%%%%%%%%%%%%%%%%%%%%%%%%%%%%%%%%%%%%%%%%%%%%%%%%%%%%%%%%%%%%%%%%%%%%%%%%%%%%%%%%%%%%%%%%%%%%%%%%%%%%%%%%%%%%%%%%%%%
% Exercise 3.7.10
%%%%%%%%%%%%%%%%%%%%%%%%%%%%%%%%%%%%%%%%%%%%%%%%%%%%%%%%%%%%%%%%%%%%%%%%%%%%%%%%%%%%%%%%%%%%%%%%%%%%%%%%%%%%%%%%%%%%%%%%%%%%%%%%%%%%%%%%
\begin{problem}{3.7.10}
Give an example to demonstrate that the inverse limit \(\varprojlim X_i\) may be empty even if all \(X_i\) are non empty.	
\end{problem}
\begin{solution}
Let \((\mathbb{N},\leq)\) be a set with preorder and \(\underline{\mathbb{N}}\) be the associated category. Consider an inverse system \((X_i)_{i\in \underline{\mathbb{N}}}\) where each \(X_i\:=\mathbb{N}\) is 
a sequence of natrual numbers together with morphisms \(f_{i,i+1}:X_{i+1}\rightarrow X_i\) where \(f_{i,i+1}\) maps every number \(n\) in the sequence \(X_j\) to \(n+1\), still a sequence in natural numbers. And for any \(i\leq j\), 
\(f_{i,j}:=f_{i,i+1}\circ f_{i+1,i+2}\circ \cdots f_{j-1,j}\). This is an inverse system and by construction, the inverse limit 
\[\varprojlim X_i:=\left\{ (x_i)_{i\in \underline{\mathbb{N}}}\in \prod_{i\in \underline{\mathbb{N}}}X_i|f_{i,j}(x_j)=x_i \, \text{for all}\, i\leq j\right\}.\]
So by definition \(x_0=f_{0,1}(x_1)=x_1+1\), similar for any \(n\in \underline{\mathbb{N}}\), \(x_n=f_{n,n+1}(x_{n+1})=x_{n+1}+1\). So \((x_n)_{n\in \mathbb{N}}\) is a infinite strictly decreasing sequence but you cannnot have a infinite strictly decreasing sequence 
in natural numbers because \(\mathbb{N}\) is bounded below by \(0\). So the inverse limit for this inverse system is empty. 
\end{solution}

\noindent\rule{7in}{2.8pt}
%%%%%%%%%%%%%%%%%%%%%%%%%%%%%%%%%%%%%%%%%%%%%%%%%%%%%%%%%%%%%%%%%%%%%%%%%%%%%%%%%%%%%%%%%%%%%%%%%%%%%%%%%%%%%%%%%%%%%%%%%%%%%%%%%%%%%%%%
% Exercise 4.1.5
%%%%%%%%%%%%%%%%%%%%%%%%%%%%%%%%%%%%%%%%%%%%%%%%%%%%%%%%%%%%%%%%%%%%%%%%%%%%%%%%%%%%%%%%%%%%%%%%%%%%%%%%%%%%%%%%%%%%%%%%%%%%%%%%%%%%%%%%
\begin{problem}{4.1.5}
For any vector space \(U\) and \(V\), show that there is a natural isomorphism \(R_{U,V}:U\otimes V\xrightarrow{\sim} V\otimes U\) such that \(u\otimes v\mapsto v\otimes u\) 
for all \(u\in U\), \(v\in V\).
\end{problem}
\begin{solution}
For any pure tensors, \(R_{U,V}\) is an isomorphism and extend by \(\mathbb{F}\)-linearity, we know that \(R_{U,V}\) is an isomorphism between \(U\otimes V\) and \(V\otimes U\). Now consider two linear maps 
\(f:U\rightarrow X\) and \(g:V\rightarrow Y\). We claim the following diagram commutes:
\[\begin{tikzcd}
	{U\otimes V} & {V\otimes U} \\
	{X\otimes Y} & {Y\otimes X}
	\arrow["{R_{U,V}}", from=1-1, to=1-2]
	\arrow["{f\otimes g}"', from=1-1, to=2-1]
	\arrow["{g\otimes f}", from=1-2, to=2-2]
	\arrow["{R_{X,Y}}", from=2-1, to=2-2]
\end{tikzcd}\]
We only need to check on pure tensores. Given \(u\otimes v\in U\otimes V\), we have 
\begin{align*}
	(g\otimes f)(R_{U,V}(u\otimes v)) & =(g\otimes f)(v\otimes u)\\ 
	                                  & =g(v)\otimes f(u)\\ 
									  & =R_{X,Y}(f(u)\otimes g(v))\\ 
									  & =R_{X,Y}(f\otimes g)(u\otimes v)
\end{align*}
This proves that \(R_{U,V}\) is natural with respect to \(U\) and with respect to \(V\).
\end{solution}

\noindent\rule{7in}{2.8pt}
%%%%%%%%%%%%%%%%%%%%%%%%%%%%%%%%%%%%%%%%%%%%%%%%%%%%%%%%%%%%%%%%%%%%%%%%%%%%%%%%%%%%%%%%%%%%%%%%%%%%%%%%%%%%%%%%%%%%%%%%%%%%%%%%%%%%%%%%
% Exercise 4.1.7
%%%%%%%%%%%%%%%%%%%%%%%%%%%%%%%%%%%%%%%%%%%%%%%%%%%%%%%%%%%%%%%%%%%%%%%%%%%%%%%%%%%%%%%%%%%%%%%%%%%%%%%%%%%%%%%%%%%%%%%%%%%%%%%%%%%%%%%%
\begin{problem}{4.1.7}
For any vector spaces \(U\) and \(V\) with \(U\) finite dimensional, show that there is a natural isomorphism of vector spaces
\[\alpha_{U,V}:U\otimes V\xrightarrow{\sim} \hom_{\mathbb{F}}(U^*,V) \]
sending \(u\otimes v\in U\otimes V\) to the linear map \(U^*\rightarrow V,\, f\mapsto f(u)v\). The naturality here means formally that \(\alpha:=(\alpha_{U,v})\) is a 
natural transformation
\[\alpha:-\otimes -\Rightarrow \hom_{\mathbb{F}}(\mathcal{D}(-),-)\]
of bifunctors from \(\mathbf{FVec}(\mathbb{F})\times \mathbf{Vec}(\mathbb{F})\) to \(\mathbf{Vec}(\mathbb{F})\).	
\end{problem}
\begin{solution}
Suppose the finite dimensional vector space \(U\) has a basis \(e_1,\ldots,e_n\), then \(e_1^*,\ldots,e_n^*\) is the dual basis for \(U^*\). We define a linear map 
\(\beta_{U,V}: \hom_{\mathbb{F}}(U^*,V) \rightarrow U\otimes V\). Given a linear map \(T:U^*\rightarrow V\), let \(\beta_{U,V}(T)= \sum_{i=1}^n e_i\otimes T(e_i^*)\). This is obviously an \(\mathbb{F}\)-linear map. 
For any pure tensor \(u\otimes v\in U\otimes V\), write \(u=\sum_{i=1}^{n}u_ie_i\) where \(u_i\in \mathbb{F}\) for \(i=1,2,\ldots,n\), then 
\begin{align*}
	(\beta_{U,V}\circ \alpha_{U,V})(u\otimes v) & = \beta_{U,V}(f\mapsto f(u)v)\\ 
	                                            & =\sum_{i=1}^{n} e_i\otimes e_i^*(u)v\\ 
												& =\sum_{i=1}^{n} e_i\otimes u_i v\\ 
												& =(\sum_{i=1}^{n} u_i e_i)\otimes v\\ 
												& =u\otimes v.
\end{align*} 
On the other hand, for any fixed \(i=1,2,\ldots,n\), given a linear map \(T:U^*\rightarrow V\), we have  
\begin{align*}
(\alpha_{U,V}\circ \beta_{U,V})(T)(e_i^*) & = \alpha_{U,V}(\sum_{j=1}^{n}e_j\otimes T(e_j^*))(e_i^*)\\ 
                                          & =\sum_{j=1}^{n} e_i^*(e_j)\otimes T(e_j^*)\\ 
										  & =1\otimes T(e_i^*)\\ 
										  & =T(e_i^*)
\end{align*}
This proves that \(\alpha\) and \(\beta\) are inverse to each other, so it is an isomorphism of vector space. To show that this isomorphism is natural, 
Given two linear maps \(p:U\rightarrow X\) and \(q:V\rightarrow Y\) where \(U,X\) are finite dimensional vector spaces, the following diagram commutes:
\[\begin{tikzcd}
	{U\otimes V} & {\hom_{\mathbb{F}}(U^*,V)} \\
	{X\otimes Y} & {\hom_\mathbb{F}(X^*,Y)}
	\arrow["{\alpha_{U,V}}", from=1-1, to=1-2]
	\arrow["{p\otimes q}"', from=1-1, to=2-1]
	\arrow["{T\mapsto q\circ T\circ p^*}", from=1-2, to=2-2]
	\arrow["{\alpha_{X,Y}}"', from=2-1, to=2-2]
\end{tikzcd}\]
Given a pure tensor \(u\otimes v\) and \(f\in X^*\), first apply \(\alpha_{U,V}\) we will finally get \(q((p^*f)(u)v)\). Note that \((p^*f)(u)=(f\circ p)(u)\in \mathbb{F}\) by definition of pullback, so this equals to \(f(p(u))\otimes q(v)\). On the 
other hand, \((\alpha_{X,Y}\circ (p\otimes q))(u\otimes v)(f)=\alpha_{X,Y}(p(u)\otimes q(v))(f)=(f(p(u)))\otimes q(v)\). Thus, this diagram does commute.
\end{solution}

\noindent\rule{7in}{2.8pt}
%%%%%%%%%%%%%%%%%%%%%%%%%%%%%%%%%%%%%%%%%%%%%%%%%%%%%%%%%%%%%%%%%%%%%%%%%%%%%%%%%%%%%%%%%%%%%%%%%%%%%%%%%%%%%%%%%%%%%%%%%%%%%%%%%%%%%%%%
% Exercise 4.1.8
%%%%%%%%%%%%%%%%%%%%%%%%%%%%%%%%%%%%%%%%%%%%%%%%%%%%%%%%%%%%%%%%%%%%%%%%%%%%%%%%%%%%%%%%%%%%%%%%%%%%%%%%%%%%%%%%%%%%%%%%%%%%%%%%%%%%%%%%
\begin{problem}{4.1.8}
Is it true that \(\hom_{\mathbb{F}}(U^*,V)\cong \hom_{\mathbb{F}}(V^*,U)\) for infinite dimensional \(U\) and \(V\)? Instead, prove for \textit{any} vector spaces \(U\) and \(V\) that 
the map 
\begin{align*}
	\hom_{\mathbb{F}}(U,V^*) & \rightarrow \hom_{\mathbb{F}}(V,U^*)\\ 
	              	      f  & \mapsto f^*\circ \iota_V
\end{align*}
is an isomorphism where \(\iota_V:V\rightarrow V^{**}\) is the canonical map.
\end{problem}
\begin{solution}
\begin{enumerate}[(1)]
\item In general it is not true that \(\hom_\mathbb{F}(U^*,V)\cong \hom_\mathbb{F}(V^*,U)\). Consider \(\mathbb{F}=\mathbb{F}_2\), \(U\) is infinite dimensional with \(|U|=\dim U=\aleph_0\) and 
      \(V\) is 2 dimensional. Then \(|U^*|=\dim U^*=|\mathbb{F}_2|^{\aleph_0}=2^{\aleph_0}\) and \(|V|=|V^*|=|\mathbb{F}_2|^{\dim V^*}=4\). So 
	  \begin{align*}
		|\hom_\mathbb{F}(U^*,V)|&=|V|^{\dim U^*}=4^{2^{\aleph_0}}\\ 
		|\hom_\mathbb{F}(V^*,U)|&=|U|^{\dim V^*}=\aleph_0^4=\aleph_0
	  \end{align*}
	  Note that \(4^{2^{\aleph_0}}>2^{\aleph_0}>\aleph_0\), so the two hom sets cannot be bijective.
\item We write 
\begin{align*}
	\alpha: \hom_\mathbb{F}(U,V^*) &\rightarrow \hom_\mathbb{F}(V,U^*),\\
	                             f &\mapsto f^*\circ \iota_V.
\end{align*}
Now we define an inverse map 
\begin{align*}
	\beta:\hom_\mathbb{F}(V,U^*) &\rightarrow \hom_\mathbb{F}(U,V^*),\\ 
	                                 g &\mapsto g^*\circ \iota_U.
\end{align*}
We check that this is indeed inverse to each other. Given a linear map \(f:U\rightarrow V^* \) and \(u\in U\), we have 
\begin{align*}
	(\beta\circ \alpha)(f)(u) & =\beta(f^*\circ \iota_V)(u)\\ 
	                          & =((f^*\circ \iota_V)^*\circ \iota_U)(u)\\ 
							  & =(\iota_V^*\circ f^{**}\circ \iota_U)(u)\\ 
							  & =(\iota_V^*\circ f^{**})(ev_u)\\ 
							  & =(ev_u)\circ f^*\circ \iota_V\\ 
							  & =f(u)
\end{align*}
Similarly, we can show that \((\alpha\circ \beta)(g)(v)=g(v)\) for any \(g:V\rightarrow U^*\) and \(v\in V\). This proves that \(\alpha\) and \(\beta\) are inverse to each other so we get an 
isomorphism \(\hom_\mathbb{F}(U,V^*)\cong \hom_\mathbb{F}(V,U^*)\).
\end{enumerate}
\end{solution}

\noindent\rule{7in}{2.8pt}
%%%%%%%%%%%%%%%%%%%%%%%%%%%%%%%%%%%%%%%%%%%%%%%%%%%%%%%%%%%%%%%%%%%%%%%%%%%%%%%%%%%%%%%%%%%%%%%%%%%%%%%%%%%%%%%%%%%%%%%%%%%%%%%%%%%%%%%%
% Exercise 4.1.10
%%%%%%%%%%%%%%%%%%%%%%%%%%%%%%%%%%%%%%%%%%%%%%%%%%%%%%%%%%%%%%%%%%%%%%%%%%%%%%%%%%%%%%%%%%%%%%%%%%%%%%%%%%%%%%%%%%%%%%%%%%%%%%%%%%%%%%%%
\begin{problem}{4.1.10}
(Tensor products of matrices) Let \(V\) and \(W\) be finite dimensional vector spaces with basis \(v_1,\ldots,v_m\) and \(w_1,\ldots,w_n\), respectively. Let \(f:V\rightarrow V\) and 
\(g:W\rightarrow W\) be linear transformations. Let \(A\) be the matrix of \(f\) with respect to the basis \(v_1,\ldots, v_m\), \(B\) be the matrix of \(g\) with respect to the basis 
\(w_1,\ldots,w_n\), and \(A\otimes B\) be the matrix of \(f\otimes g:V\otimes W\rightarrow V\otimes W\) with respect to the basis 
\[v_1\otimes w_1,\ldots,v_1\otimes w_n,v_2\otimes w_1,\ldots,v_2\otimes w_n,\ldots, v_m\otimes w_1,\ldots,v_m\otimes w_n.\]
\begin{enumerate}[(1)]
	\item Describe the entries of the matrix \(A\otimes B\) explicitly in terms of the entries of the matrices \(A\) and \(B\).
	\item Show that \(\text{tr}(A\otimes B)=\text{tr}(A)\text{tr}(B)\).
\end{enumerate}
\end{problem}
\begin{solution}
\begin{enumerate}[(1)]
\item  We know the for \(1\leq i,j\leq m\),the \((i,j)\)-entry of \(A\) is given by \(v_i^*(f(v_j))\) where \(v_i^*\) is the dual of \(v_i\). Similarly, for \(1\leq k,l\leq n\), the \((k,l)\)-entry 
       of the matrix \(B\) is given by \(w_k^*(g(w_l))\). The matrix \(A\otimes B\) is a \(mn\times mn\) matrix, for \(1\leq p,q\leq mn\), the \((p,q)\)-entry of \(A\otimes B\) 
	   is given by \((v_i\otimes w_k)^*((f\otimes g)(v_j\otimes w_l))\) where \(ik=p\) and \(jl=q\). The order of \((i,k)\) is given by 
	   \[(1,1),(1,2),\ldots,(1,n),(2,1),\ldots,(2,n),\ldots,(m,1)\ldots,(m,n)\]
	   similar for the order of \((k,l)\). Note that the \((p,q)\)-entry
	   \[(v_i\otimes w_k)^*((f\otimes g)(v_j\otimes w_l))=(v_i\otimes w_k)^*(f(v_j)\otimes g(w_l))=A_{ij}\cdot B_{kl}.\]
       So the entries of \(A\otimes B\) are given by the multiplication of entries from \(A\) and \(B\)
\item By definition, the trace 
\begin{align*}
	tr(A\otimes B) & = \sum_{i=1}^{m}\sum_{k=1}^{n} (v_i\otimes w_k)^*(f(v_i)\otimes g(w_k))\\ 
	               & =\sum_{i=1}^{m}\sum_{k=1}^{n} v_i^*(f(v_i))\cdot w_k^*(g(w_k))\\ 
				   & =(\sum_{i=1}^{m}v_i^*(f(v_i)))(\sum_{k=1}^{n}w_k^*(g(w_k)))\\ 
				   & =tr(A)\cdot tr(B)
\end{align*}
\end{enumerate}
\end{solution}

\noindent\rule{7in}{2.8pt}
%%%%%%%%%%%%%%%%%%%%%%%%%%%%%%%%%%%%%%%%%%%%%%%%%%%%%%%%%%%%%%%%%%%%%%%%%%%%%%%%%%%%%%%%%%%%%%%%%%%%%%%%%%%%%%%%%%%%%%%%%%%%%%%%%%%%%%%%
% Exercise 4.1.11
%%%%%%%%%%%%%%%%%%%%%%%%%%%%%%%%%%%%%%%%%%%%%%%%%%%%%%%%%%%%%%%%%%%%%%%%%%%%%%%%%%%%%%%%%%%%%%%%%%%%%%%%%%%%%%%%%%%%%%%%%%%%%%%%%%%%%%%%
\begin{problem}{4.1.11}
Let \(U\) and \(V\) be finite dimensional vector spaces with bases \(u_1,\ldots, u_m\) and \(v_1,\ldots, v_n\), respectively. Let \(t=\sum_{i=1}^{m}\sum_{j=1}^{n}a_{i,j}u_i\otimes v_j\in U\otimes V\) be a tensor 
and set \(A:=[a_{i,j}]_{1\leq i\leq m,1\leq j\leq n}\).
\begin{enumerate}[(1)]
	\item Let \(\tau:V^*\rightarrow U\) be the image of \(t\) under the canonical isomorphism \(U\otimes V\cong \hom_{\mathbb{F}}(V^*,U)\). Let \(f_1,\ldots, f_n\) be the basis for \(V^*\) that is dual to the given basis 
	      for \(V\). Show that the matrix of \(\tau\) with respect to \(f_1,\ldots,f_n\) and \(u_1,\ldots, u_m\) is equal to \(A\).
    \item Let \(r\) be the rank of the tensor \(t\). Suppose we are given an expansion \(t=\sum_{i=1}^{r}x_i\otimes y_j\) for vectors \(x_i\in U\) and \(y_j\in V\). Show that the vectors \(x_1,\ldots, x_r\) are linearly independent, 
          as are the vectors \(y_1,\ldots,y_r\). Deduce that \(x_1,\ldots, x_r\) is a basis for \(\text{im}\tau\).
    \item Hence prove that the rank of \(t\) is equal to the rank of the matrix \(A\). In particular, \(t\) is a pure tensor if and only if \(\text{rank}(A)\leq 1 \).
\end{enumerate}
\end{problem}
\begin{solution}
\begin{enumerate}[(1)]
\item The \((i,j)\)-entry of \(\tau\) is given by \(u_i^*(\tau(f_j))\). We know that \(\tau\) is the image of \(t\) under the canonical isomorphism 
      \(U\otimes V\cong \hom_\mathbb{F}(V^*,U)\), so 
	  \[\tau(f_j)=\sum_{k=1}^{n}\sum_{l=1}^{m} a_{l,k} f_j(v_k)u_l=f_j(v_j)\sum_{l=1}^{m}a_{l,j}u_l=\sum_{l=1}^{m}a_{l,j}u_l.\]
	  Now we know that the \((i,j)\)-entry of \(\tau\) is equal to \(u_i^*(\sum_{l=1}^{m}a_{l,j}u_l)=a_{i,j}u_i^*(u_i)=a_{i,j}\). Thus, the matrix of \(\tau\) is equal to \(A\) under this basis.
\item Suppose \(x_1,\ldots,x_r\) are not linearly independent. Without loss of generality, we can write \(x_1=\sum_{i=2}^{r}a_ix_i\) for \(a_2,\ldots,a_r\in \mathbb{F}\). We define 
     \begin{align*}
		y_2'&:=a_2y_1+y_2,\\ 
        y_3'&:=a_3y_1+y_3,\\ 
		&\ldots\\ 
        y_r'&:=a_ry_1+y_r.
	 \end{align*}
	 We have 
	 \begin{align*}
		\sum_{i=2}^{r}x_i\otimes y_i' & =\sum_{i=2}^{r} x_i\otimes (a_iy_1+y_i)\\ 
		                              & =(\sum_{i=2}^{r}a_ix_i\otimes y_1)+(\sum_{i=2}^{r}x_i\otimes y_i)\\ 
									  & =(\sum_{i=2}^{r}a_ix_i)\otimes y_1+\sum_{i=2}^{r}x_i\otimes y_i\\ 
									  & =x_1\otimes y_1+\sum_{i=2}^{r}x_i\otimes y_i\\ 
									  & =t
	 \end{align*}
	 This contradicts that the rank of \(t\) is \(r\). Thus, \(x_1,\ldots,x_r\) are linearly independent. Now use the natural isomorphism \(R_{U,V}:U\otimes V\xrightarrow{\sim} V\otimes U\), the same 
	 method gives us that \(y_1,\ldots,y_r\) are also linearly independent. 
	 \par 
	 To show that \(x_1,\ldots,x_r\) span \(\text{im}\tau\), we need to show that \(\tau(f_1),\ldots,\tau(f_m)\) can be written as the linear combination of \(x_1,\ldots,x_r\). For each \(i=1,2,\ldots,m\), 
	 \(\tau(f_i)=\sum_{j=1}^{r}f_i(y_j)\cdot x_j=f_i(y_i)x_i\). So we are done.
\end{enumerate}
\end{solution}

\noindent\rule{7in}{2.8pt}
%%%%%%%%%%%%%%%%%%%%%%%%%%%%%%%%%%%%%%%%%%%%%%%%%%%%%%%%%%%%%%%%%%%%%%%%%%%%%%%%%%%%%%%%%%%%%%%%%%%%%%%%%%%%%%%%%%%%%%%%%%%%%%%%%%%%%%%%
% Exercise 4.1.16
%%%%%%%%%%%%%%%%%%%%%%%%%%%%%%%%%%%%%%%%%%%%%%%%%%%%%%%%%%%%%%%%%%%%%%%%%%%%%%%%%%%%%%%%%%%%%%%%%%%%%%%%%%%%%%%%%%%%%%%%%%%%%%%%%%%%%%%%
\begin{problem}{4.1.16}
For \(A,B\in \mathbf{CAlg}(\mathbb{F})\), prove that \(A\otimes B\) together with the maps \(A\rightarrow A\otimes B,\, a\mapsto a\otimes 1_B\) and \(B\rightarrow A\otimes B,\, b\mapsto 1_A\otimes b\) 
is a coproduct of \(A\) and \(B\) in \(\mathbf{CAlg}(\mathbb{F})\).
\end{problem}
\begin{solution}
Write the canonical maps \(p:A\rightarrow A\otimes B,\, p(a)=a\otimes 1_B\) and \(q:B\rightarrow A\otimes B,\, q(b)=1_A\otimes B\). Suppose \(C\in \mathbf{CAlg}(\mathbb{F})\) is a commutative algebra with two morphisms \(f:A\rightarrow C\) and \(g:B\rightarrow C\). We have a commutative diagram:
\[\begin{tikzcd}
	{\mathbb{F}} & B \\
	A & C
	\arrow[from=1-1, to=1-2]
	\arrow[from=1-1, to=2-1]
	\arrow["g", from=1-2, to=2-2]
	\arrow["f"', from=2-1, to=2-2]
\end{tikzcd}\]
We define a unqiue \(\mathbb{F}\)-algebra homomorphism \(h:A\otimes B\rightarrow C,\, h(a\otimes b)=f(a)g(b)\). We know that \((h\circ p)(a)=h(a\otimes 1_B)=f(a)g(1_B)=f(a)\) and 
\((h\circ g)(b)=h(1_A\otimes b)=f(1_A)g(b)=g(b)\), thus we have the following commutative diagram:

\[\begin{tikzcd}
	{\mathbb{F}} & B \\
	A & {A\otimes B} \\
	&& C
	\arrow[from=1-1, to=1-2]
	\arrow[from=1-1, to=2-1]
	\arrow["q", from=1-2, to=2-2]
	\arrow["g", curve={height=-12pt}, from=1-2, to=3-3]
	\arrow["p"', from=2-1, to=2-2]
	\arrow["f"', curve={height=12pt}, from=2-1, to=3-3]
	\arrow["{\exist! h}"', dashed, from=2-2, to=3-3]
\end{tikzcd}\]

This proves the universal property of \(A\otimes B\). So \(A\otimes B\) is the pushout of \(A\) and \(B\) in \(\mathbf{CAlg}(\mathbb{F})\).
\end{solution}



\end{document}
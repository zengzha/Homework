\documentclass[a4paper, 12pt]{article}
\usepackage{comment} % enables the use of multi-line comments (\ifx \fi) 
\usepackage{lipsum} %This package just generates Lorem Ipsum filler text. 
\usepackage{fullpage} % changes the margin
\usepackage[a4paper, total={7in, 10in}]{geometry}
\usepackage{amsmath}
\usepackage{amssymb,amsthm}  % assumes amsmath package installed
\newtheorem{theorem}{Theorem}
\newtheorem{corollary}{Corollary}
\usepackage{graphicx}
\usepackage{tikz}
\usepackage{quiver}
\usetikzlibrary{arrows}
\usepackage{verbatim}
\usepackage{setspace}
\usepackage{comment}
\usepackage{float}
\usepackage{tikz-cd}
\usepackage[backend=biber,bibencoding=utf8,style=numeric,sorting=ynt]{biblatex}

    
\usepackage{xcolor}
\usepackage{mdframed}
\usepackage[shortlabels]{enumitem}
\usepackage{indentfirst}
\usepackage{hyperref}
    
\renewcommand{\thesubsection}{\thesection.\alph{subsection}}

\newenvironment{problem}[2][Exercise]
    { \begin{mdframed}[backgroundcolor=gray!20] \textbf{#1 #2} \\}
    {  \end{mdframed}}

% Define solution environment
\newenvironment{solution}
    {\textit{Solution:}}
    {}

%Define the claim environment
\newenvironment{claim}[1]{\par\noindent\underline{Claim:}\space#1}{}
\newenvironment{claimproof}[1]{\par\noindent\underline{Proof:}\space#1}{\hfill $\blacksquare$}

\renewcommand{\qed}{\quad\qedsymbol}
\newcommand{\la}{\langle}
\newcommand{\ra}{\rangle}
%%%%%%%%%%%%%%%%%%%%%%%%%%%%%%%%%%%%%%%%%%%%%%%%%%%%%%%%%%%%%%%%%%%%%%%%%%%%%%%%%%%%%%%%%%%%%%%%%%%%%%%%%%%%%%%%%%%%%%%%%%%%%%%%%%%%%%%%
\begin{document}
%Header-Make sure you update this information!!!!
\noindent
%%%%%%%%%%%%%%%%%%%%%%%%%%%%%%%%%%%%%%%%%%%%%%%%%%%%%%%%%%%%%%%%%%%%%%%%%%%%%%%%%%%%%%%%%%%%%%%%%%%%%%%%%%%%%%%%%%%%%%%%%%%%%%%%%%%%%%%%
\large\textbf{Zhengdong Zhang} \hfill \textbf{Homework - Week 8}   \\
Email: zhengz@uoregon.edu \hfill ID: 952091294 \\
\normalsize Course: MATH 647 - Abstract Algebra  \hfill Term: Fall 2024\\
Instructor: Dr.Victor Ostrik \hfill Due Date: $27^{th}$ November, 2024 \\
\noindent\rule{7in}{2.8pt}
\setstretch{1.1}
%%%%%%%%%%%%%%%%%%%%%%%%%%%%%%%%%%%%%%%%%%%%%%%%%%%%%%%%%%%%%%%%%%%%%%%%%%%%%%%%%%%%%%%%%%%%%%%%%%%%%%%%%%%%%%%%%%%%%%%%%%%%%%%%%%%%%%%%
% Exercise 6.3.5
%%%%%%%%%%%%%%%%%%%%%%%%%%%%%%%%%%%%%%%%%%%%%%%%%%%%%%%%%%%%%%%%%%%%%%%%%%%%%%%%%%%%%%%%%%%%%%%%%%%%%%%%%%%%%%%%%%%%%%%%%%%%%%%%%%%%%%%%
\begin{problem}{6.3.5}
Let \(\sigma\in S_n\) be written as a product of disjoint cycles:
\[\sigma=(a_1\ldots a_s)(b_1\ldots b_t)\cdots\]
\begin{enumerate}[(a)]
\item Write \(\sigma^{-1}\) as a product of disjoint cycles.
\item Deduce that \(\sigma\) and \(\sigma^{-1}\) are conjuagte in \(S_n\).
\item Deduce the stronger statement, that there is \(\tau\in S_n\) with \(\tau(n)=n\) and \(\sigma^{-1}=\tau\sigma\tau^{-1}\).
\end{enumerate}
\end{problem}
\begin{solution}
\begin{enumerate}[(a)]
\item Consider 
\[\sigma^{-1}=(a_s\ a_{s-1}\ldots a_1)(b_t\  b_{t-1}\ldots)\cdots\]
To see that it is indeed the inverse, we only need to show that each of the disjoint cycles is the inverse, namely: for any \(1\leq i\leq s\)(assume \(a_{1-1}=a_s\) and \(a_{s+1}=a_1\)), we know that \((a_1\ldots a_s)(a_s\ldots a_1)\) sends 
\[a_i\xrightarrow{(a_s\ldots a_1)}a_{i-1}\xrightarrow{(a_1\ldots a_s)}a_i.\]
Similar for \((a_s\ldots a_1)(a_1\ldots a_s)\), which sends \(a_i\) to \(a_{i+1}\), then back to \(a_i\).
\item Note that \(\sigma\) and \(\sigma^{-1}\) have the same cycle type, by Theorem 6.3.4, they belong to the same conjugacy class in \(S_n\).
\item We first prove this for disjoint cycles. Without loss of generality, assume \(\sigma=(a_1\ldots a_s)\) and \(a_s=n\). We know that \(\sigma^{-1}=(a_s\ldots a_1)\). Rewrite 
\((a_s\ldots a_1)=(a_{s-1} a_{s-2}\ldots a_1 a_s)\) and consider \(\tau\in S_n\) with \(\tau(a_i)=a_{s-i}\) for \(1\leq i\leq s-1\), \(\tau(a_s)=a_s\) and \(\tau\) fixes any other elements in \(\left\{ 1,2,\ldots,n \right\}\setminus \left\{ a_1,\ldots,a_s\right\}\). By 
Lemma 6.3.3, we have 
\[\tau\sigma\tau^{-1}=(\tau a_1\ldots \tau a_s)=(a_{s-1} a_{s-2}\ldots a_1 a_s)=\sigma^{-1}\]
If \(\sigma\) is a product of disjoint cycles, note that in our construction \(\tau\) only permutes elements in one disjoint cycles, so the above conclusion is also valid for \(\sigma\).
\end{enumerate}
\end{solution}

\noindent\rule{7in}{2.8pt}
%%%%%%%%%%%%%%%%%%%%%%%%%%%%%%%%%%%%%%%%%%%%%%%%%%%%%%%%%%%%%%%%%%%%%%%%%%%%%%%%%%%%%%%%%%%%%%%%%%%%%%%%%%%%%%%%%%%%%%%%%%%%%%%%%%%%%%%%
% Exercise 6.3.6
%%%%%%%%%%%%%%%%%%%%%%%%%%%%%%%%%%%%%%%%%%%%%%%%%%%%%%%%%%%%%%%%%%%%%%%%%%%%%%%%%%%%%%%%%%%%%%%%%%%%%%%%%%%%%%%%%%%%%%%%%%%%%%%%%%%%%%%%
\begin{problem}{6.3.6}
Let \(x\in S_n\) be of cycle type \((\lambda_1,\lambda_2,\ldots,\lambda_l)\). What is the order of \(x\)?
\end{problem}
\begin{solution}
Let \(a_1,a_2,\ldots, a_s\) be distinct elements in \(\left\{ 1,2,\ldots,n \right\}\). Let 
\[\sigma=(a_1\ldots a_s)\]
be a \(n\)-cycle in \(S_n\). 
\begin{claim}
The order of \(\sigma\) in \(S_n\) is equal to \(s\).
\end{claim} 
\begin{claimproof}
We have \(\sigma(a_i)=a_{i+1}\). So we have \(\sigma^s(a_i)=a_{i+s}\). Here we assume for any integer \(k\), \(a_{i+k}=a_j\) if \(i+k\equiv j\)(mod \(s\)) and \(a_0=a_s\). So \(a_{i+s}=a_i\) and 
for any \(1\leq k\leq s-1\), \(\sigma^k(a_i)=a_{i+k}\neq a_i\). 
\end{claimproof} 
Let \(x\in S_n\) be of cycle type \((\lambda_1,\ldots,\lambda_l)\). So the order of \(x\) is the least common multiple \(lcd(\lambda_1,\ldots,\lambda_l)\). 
\end{solution}

\noindent\rule{7in}{2.8pt}
%%%%%%%%%%%%%%%%%%%%%%%%%%%%%%%%%%%%%%%%%%%%%%%%%%%%%%%%%%%%%%%%%%%%%%%%%%%%%%%%%%%%%%%%%%%%%%%%%%%%%%%%%%%%%%%%%%%%%%%%%%%%%%%%%%%%%%%%
% Exercise 6.3.7
%%%%%%%%%%%%%%%%%%%%%%%%%%%%%%%%%%%%%%%%%%%%%%%%%%%%%%%%%%%%%%%%%%%%%%%%%%%%%%%%%%%%%%%%%%%%%%%%%%%%%%%%%%%%%%%%%%%%%%%%%%%%%%%%%%%%%%%%
\begin{problem}{6.3.7}
The center of \(S_n\) is trivial for \(n\geq 3\).
\end{problem}
\begin{solution}
Let \(\sigma\in S_n\) which is not the identity. We want to show that there exists some \(\tau\in S_n\) such that \(\tau\sigma\tau^{-1}\neq \sigma\). Decompose \(\sigma\) into disjoint cycles and first suppose 
this decomposition contains a \(s\)-cycle \((x_1\ldots x_s)\) for \(s\geq 3\), where \(x_1,\ldots,x_s\) are different elements in \(\left\{ 1,2,\ldots,n \right\}\). Consider the transposition \(\tau=(a_1 a_2)\in S_n\). By Lemma 6.3.3, 
\(\tau(x_1\ldots x_s)\tau^{-1}=(x_2 x_1 x_3\ldots x_s)\). Note that \(s\geq 3\), so \((x_1x_2x_3\ldots x_s)\) and \((x_2x_1x_3\ldots x_s)\) are different elements in \(S_n\). This implies that the conjugate of \(\tau\) changes one of the 
disjoint cycles in \(\sigma\), so we have \(\tau\sigma\tau^{-1}\neq \sigma\).

Now assume the decomposition of \(\sigma\) only contains \(2\)-cycles. If \(\sigma=(ij)\) is a transposition, since \(n\geq 3\), there exists \(1\leq k\leq n\) with \(k\neq i\) and \(k\neq j\). Consider \(\tau=(ik)\), we have 
\[\tau\sigma\tau^{-1}=(\tau(i) \tau(j))=(kj)\neq (ij).\]
Now suppose the decomposition of \(\sigma\) contains at least two disjoint \(2\)-cycles \((ij)(kl)\) for differen \(i,j,k,l\). Consider \(\tau=(jk)\). We have 
\[\tau(ij)(kl)\tau^{-1}=(ik)(jl)\neq (ij)(kl).\]
we are done.
\end{solution}

\noindent\rule{7in}{2.8pt}
%%%%%%%%%%%%%%%%%%%%%%%%%%%%%%%%%%%%%%%%%%%%%%%%%%%%%%%%%%%%%%%%%%%%%%%%%%%%%%%%%%%%%%%%%%%%%%%%%%%%%%%%%%%%%%%%%%%%%%%%%%%%%%%%%%%%%%%%
% Exercise 6.4.1
%%%%%%%%%%%%%%%%%%%%%%%%%%%%%%%%%%%%%%%%%%%%%%%%%%%%%%%%%%%%%%%%%%%%%%%%%%%%%%%%%%%%%%%%%%%%%%%%%%%%%%%%%%%%%%%%%%%%%%%%%%%%%%%%%%%%%%%%
\begin{problem}{6.4.1}
The \textit{Klein four-group} 
\[V_4=\left\{ 1,(12)(34),(13)(24),(14)(23) \right\}.\]
Prove that \(V_4\) is a normal subgroup of \(A_4\). In particular, \(A_4\) is not simple.
\end{problem}
\begin{solution}
Write \(a=(12)(34)\), \(b=(13)(23)\) and \(c=(14)(23)\). We have 
\[ab=ba=c,bc=cb=a,ac=ca=b, a^2=b^2=c^2=1.\]
So this is a subgroup of \(S_4\). Moreover, note that sgn\((a)=\)sgn\((b)=\)sgn\((c)=1\), so \(V_4\) is a subgroup of \(A_4\). Given any \(\tau\in A_4\), by Lemma 6.3.3, \(\tau a\tau^{-1}\) has the same cycle type \((2,2)\), and \(V_4\) contains all the elements of cycle type 
\((2,2)\) in \(S_4\), so \(\tau a\tau^{-1}\in V_4\). Similar for \(b\) and \(c\). This shows that \(\tau V_4 \tau^{-1}=V_4\). \(V_4\) is a normal subgroup of \(A_4\). And we know that \(|A_4|=|S_4|/2=12\), so \(A_4\) is not simple.
\end{solution}

\noindent\rule{7in}{2.8pt}
%%%%%%%%%%%%%%%%%%%%%%%%%%%%%%%%%%%%%%%%%%%%%%%%%%%%%%%%%%%%%%%%%%%%%%%%%%%%%%%%%%%%%%%%%%%%%%%%%%%%%%%%%%%%%%%%%%%%%%%%%%%%%%%%%%%%%%%%
% Exercise 6.4.2
%%%%%%%%%%%%%%%%%%%%%%%%%%%%%%%%%%%%%%%%%%%%%%%%%%%%%%%%%%%%%%%%%%%%%%%%%%%%%%%%%%%%%%%%%%%%%%%%%%%%%%%%%%%%%%%%%%%%%%%%%%%%%%%%%%%%%%%%
\begin{problem}{6.4.2}
Show that 
\[S_4>A_4>V_4>C_2>\left\{ 1 \right\}\]
is a Jordan-H\"{o}lder series of \(S_4\). What are the Jordan-H\"{o}lder factors?
\end{problem}
\begin{solution}
Use the same notation for Exercise 6.4.1. We know that \([S_4:A_4]=2\) and any index 2 subroup is normal, so \(A_4\) is normal in \(S_4\). We have proved in Exercise 6.4.1 that \(V_4\) is normal in \(A_4\). Note that \(V_4=\left\{ 1,a,b,c, \right\}\) is abelian and 
\(C_2=\la a\ra=\la b\ra =\la c\ra\) is a subrgoup, so it is automatically normal in \(V_4\). 

Use the presentation of \(V_4\)
\[V_4=\la\la a,b,c\,|\, a^2=b^2=c^2=1, ab=ba=c,ac=ca=b,bc=cb=a\ra\ra.\]
and assume \(C_2=\la a\ra\). The quotient group \(V_4/C_2\) consists of two cosets \(C_2\) and \(bC_2\), thus \(V_4/C_2\cong C_2\) is simple.  
Note that \(|A_4|=12\) and \(|V_4|=4\), so the quotient group \(|A_4/V_4|=\frac{|A_4|}{|V_4|}=3\). The only order 3 group is cyclic group \(C_3\) and it is simple. 
Similarly, we have \(|S_4/A_4|=\frac{|S_4|}{|A_4|}=2\) and the only group of order 2 is the cyclic group \(C_2\), so \(S_4/A_4\cong C_2\) is simple. This proves that 
\[S_4>A_4>V_4>C_2>\left\{ 1 \right\}\]
is a Jordan-H\"{o}lder series of \(S_4\).
\end{solution}

\noindent\rule{7in}{2.8pt}
%%%%%%%%%%%%%%%%%%%%%%%%%%%%%%%%%%%%%%%%%%%%%%%%%%%%%%%%%%%%%%%%%%%%%%%%%%%%%%%%%%%%%%%%%%%%%%%%%%%%%%%%%%%%%%%%%%%%%%%%%%%%%%%%%%%%%%%%
% Exercise 6.4.10
%%%%%%%%%%%%%%%%%%%%%%%%%%%%%%%%%%%%%%%%%%%%%%%%%%%%%%%%%%%%%%%%%%%%%%%%%%%%%%%%%%%%%%%%%%%%%%%%%%%%%%%%%%%%%%%%%%%%%%%%%%%%%%%%%%%%%%%%
\begin{problem}{6.4.10}
Any finite group is isomorphic to a subgroup of \(A_n\) for some \(n\).
\end{problem}
\begin{solution}
By Theorem 6.3.1, any finite group is isomorphic to a subgroup of \(S_n\) for some \(n\). If we could show that any symmetric group \(S_n\) is isomrphic to a 
subgroup of \(A_m\) for some \(m\), then we are done. By Lemma 4.3.11, the symmetric group \(S_n\) is generated by the set 
\[\left\{ (12),(23),\ldots,(n-1\ n) \right\}.\]
Consider a subgroup \(G\) of \(S_{n+2}\) generated by the following elements 
\[\left\{ (12)(n+1\ n+2),(23)(n+1\ n+2),\ldots,(n-1\ n)(n+1\ n+2) \right\}\]
Note that for any \(1\leq i\leq n-1\), \((i\ i+1)\) and \((n+1\ n+2)\) are disjoint, so sgn\(((i\ i+1)(n+1\ n+2))=1\). Thus \(G\) is a subgroup of \(A_{n+2}\) and we have a 
group homomorphism \(f:S_n\rightarrow G\) sending \(\sigma\in S_n\) to \(\sigma (n+1\ n+2)\) if \(\sigma\) is odd and to \(\sigma\) if \(\sigma\) is even. This is also an isomorphism 
because \(f\) is injective and every element in \(G\) is a product of its generating set.
\end{solution}

\noindent\rule{7in}{2.8pt}
%%%%%%%%%%%%%%%%%%%%%%%%%%%%%%%%%%%%%%%%%%%%%%%%%%%%%%%%%%%%%%%%%%%%%%%%%%%%%%%%%%%%%%%%%%%%%%%%%%%%%%%%%%%%%%%%%%%%%%%%%%%%%%%%%%%%%%%%
% Exercise 6.4.11
%%%%%%%%%%%%%%%%%%%%%%%%%%%%%%%%%%%%%%%%%%%%%%%%%%%%%%%%%%%%%%%%%%%%%%%%%%%%%%%%%%%%%%%%%%%%%%%%%%%%%%%%%%%%%%%%%%%%%%%%%%%%%%%%%%%%%%%%
\begin{problem}{6.4.11}
Find the smallest \(n\) such that \(A_n\) contains a subgroup of order \(15\).
\end{problem}
\begin{solution}
\begin{claim}
If a group \(G\) has order 15, then \(G\) must be isomorphic to the cyclic group \(C_{15}\).
\end{claim}
\begin{claimproof}
By Cauchy's theorem, \(G\) must have an element \(a\) of order \(3\) and an element \(b\) of \(5\). Consider the cyclic subgroup generated by \(a\) and \(b\). 
The index of \(\la b\ra\) in \(G\) is \(3\), which is the smallest prime dividing \(15\), so \(\la b\ra\) is normal in \(G\). 
Since \(3\) and \(5\) are coprime, 
\(\la a\ra \cap \la b\ra =\left\{ 1 \right\}\). We know that \(\text{Aut}(\la b\ra)=C_4\). Consider a group homomorphism \(\phi:\la a\ra\cong C_3 \rightarrow C_4\). Since \(3\) and \(4\) are also coprime, 
\(\phi\) can only be the trivial map. \(G\) can only be \(C_3\times C_5\cong C_{15}\).
\end{claimproof}

\(A_n\) having a subrgoup isomorphic to \(C_{15}\) is equivalent to have an element of order 15. Consider \(x\in S_n\) with the cycle type \((5,3)\). It has order \(15\) and is an even permutation, so \(x\in A_8\). The smallest 
possible \(n\) is 8.
\end{solution}

\noindent\rule{7in}{2.8pt}
%%%%%%%%%%%%%%%%%%%%%%%%%%%%%%%%%%%%%%%%%%%%%%%%%%%%%%%%%%%%%%%%%%%%%%%%%%%%%%%%%%%%%%%%%%%%%%%%%%%%%%%%%%%%%%%%%%%%%%%%%%%%%%%%%%%%%%%%
% Exercise 6.5.7
%%%%%%%%%%%%%%%%%%%%%%%%%%%%%%%%%%%%%%%%%%%%%%%%%%%%%%%%%%%%%%%%%%%%%%%%%%%%%%%%%%%%%%%%%%%%%%%%%%%%%%%%%%%%%%%%%%%%%%%%%%%%%%%%%%%%%%%%
\begin{problem}{6.5.7(Isometries)}
Let \(E\) be the Euclidean space \(\mathbb{R}^n\) with the standard scalar product. A distance preserving bijection of \(E\) is called an \textit{isometry} of \(E\). 
\begin{enumerate}
\item The isometries of \(E\) form a group denoted by \(ISO(E)\). 
\item \(AO(E)\) is a subgroup of \(ISO(E)\).
\item If \(f\in ISO(E)\) preserves zero, i.e. \(f(0)=0\), then \(f\) preserves the scalar product, i.e. \((f(v)|f(w))=(v|w)\) for all \(v,w\in E\).
\item An isometry of \(E\) preserving zero is a linear map.
\item \(AO(E)=ISO(E)\).
\end{enumerate}
\end{problem}
\begin{solution}
Write \(d:E\times E\rightarrow \mathbb{R}_{\geq 0},\  d(x,y)=|x-y|\) as the distance function on \(E\). 
\begin{enumerate}
\item Let \(f,g\in ISO(E)\). For any \(a,b\in E\), we have 
\[d((f\circ g)(a),(f\circ g)(b))=d(g(a),g(b))=d(a,b).\]
So \((f\circ g)\in ISO(E)\). The identify function is the identity element in \(ISO(E)\). \(ISO(E)\) is indeed a group.
\item For any \(x\in E\), write \(x\) as a vector and we know that \(|x|^2=x^T\cdot x\). Suppose \(A\in O(E)\) is an orthogonal transformation. We have 
\[(Ax)^T(Ax)=x^T(A^TA)x=|x|^2.\]
This implies that \(d(Ax,0)=d(x,0)\). For any \(x,y\in E\), we have 
\[d(Ax,Ay)=|Ax-Ay|=|A(x-y)|=d(A(x-y),0)=d(x-y,0)=d(x,y).\]
Moreover, for any \(x,y,z\in E\), we have 
\[d(x-z,y-z)=|(x-z)-(y-z)|=|x-y|=d(x,y).\]
So both \(O(E)\) and \(T(E)\) are a subgroup of \(ISO(E)\). We have \(AO(E)=O(E)T(E)<ISO(E)\).
\item For any vector \(v\in E\), we have 
\[|f(v)|=d(f(v),0)=d(f(v),f(0))=d(v,0)=|v|\]
since \(f\in ISO(E)\) is an isometry and \(f(0)=0\). For any \(v,w\in E\), \(f\) is an isometry impiles that 
\begin{align*}
    |f(v)-f(w)|^2&=|v-w|^2\\ 
    (f(v)^T-f(w)^T)\cdot (f(v)-f(w))&=(v^T-w^T)\cdot (v-w)\\ 
    |f(v)|^2+|f(w)|^2-(f(v)^Tf(w)+f(w)^Tf(v))&=|v|^2+|w|^2-(v^Tw+w^Tv)\\ 
    f(v)^Tf(w)+f(w)^Tf(v)&=v^Tw+w^Tv
\end{align*}
Note that \(2(f(v)|f(w))= f(v)^Tf(w)+f(w)^Tf(v)\) and \(2(v|w)=v^Tw+w^Tv\). So we have \((f(v)|f(w))=(v|w)\).
\item Let \(v,w\in E\) and \(c_1,c_2\) be scalars. Then we have 
\begin{align*}
    |f(c_1v+c_2w)-c_1f(v)-c_2f(w)|^2=&|f(c_1v+c_2w)|^2+|c_1f(v)+c_2f(w)|^2\\ 
                                     &-2(f(c_1v+c_2w)|c_1f(v)+c_2f(w))\\ 
                                    =&|c_1v|^2+|c_2w|^2+|c_1|^2|f(v)|^2+|c_2|^2|f(w)|^2+4|c_1c_2|(f(v)|f(w))\\ 
                                     &-2(c_1(c_1v+c_2w)|v)+c_2(c_1v+c_2(w)|w)\\ 
                                    =&2|c_1|^2|v|^2+2|c_2|^2|w|^2+4|c_1c_2|(v|w)\\ 
                                     &-2(|c_1|^2(v|v)^2+|c_2|^2(w|w)^2+2|c_1c_2|(v|w))\\
                                    =&0.
\end{align*}
This shows that 
\[f(c_1v+c_2w)=c_1f(v)+c_2f(w).\]
We can conlude that \(f\) is linear.
\item We have seen in (2) that \(AO(E)\) is a subgroup of \(ISO(E)\). Given \(f\in ISO(E)\), define a translation \(\bar{f}: v\mapsto f(v)-f(0)\). we have 
\(\bar{f}(0)=f(0)-f(0)=0\). From the previous discussion, we know that \(\bar{f}\) is a linear map. Write \(\bar{f}\) as a matrix \(A\). For any \(x\in E\), we have 
\[(Ax)^T(Ax)=x^T(A^TA)x=x^Tx.\]
This shows that \(A^TA=Id\) and \(\bar{f}\in O(E)\). So \(f\) can be written as a composition of a translation and an element in \(O(E)\). This proves that \(ISO(E)\) is contained in \(AO(E)\). We 
can conclude that \(ISO(E)=AO(E)\).
\end{enumerate} 
\end{solution}

\noindent\rule{7in}{2.8pt}
%%%%%%%%%%%%%%%%%%%%%%%%%%%%%%%%%%%%%%%%%%%%%%%%%%%%%%%%%%%%%%%%%%%%%%%%%%%%%%%%%%%%%%%%%%%%%%%%%%%%%%%%%%%%%%%%%%%%%%%%%%%%%%%%%%%%%%%%
% Exercise 6.6.2
%%%%%%%%%%%%%%%%%%%%%%%%%%%%%%%%%%%%%%%%%%%%%%%%%%%%%%%%%%%%%%%%%%%%%%%%%%%%%%%%%%%%%%%%%%%%%%%%%%%%%%%%%%%%%%%%%%%%%%%%%%%%%%%%%%%%%%%%
\begin{problem}{6.6.2(Coxeter presentation of dihedral groups)}
\[D_{2n}\cong \la\la s_1,s_2\,|\, s_1^2=1,s_2^2=1,(s_1s_2)^n=1\ra\ra. \]
\end{problem}
\begin{solution}
In Example 6.6.1, we have already seen that 
\[D_{2n}\cong \la\la a,b\,|\, a^n=1,b^2=1,bab=a^{-1}\ra\ra.\]
Write
\begin{align*}
G_1&=\la\la a,b\,|\, a^n=1,b^2=1,bab=a^{-1}\ra\ra,\\ 
G_2&= \la\la s_1,s_2\,|\, s_1^2=s_2^2=1, (s_1s_2)^n=1\ra\ra.
\end{align*}
We only need to show that \(G_1\cong G_2\).
Consider the following map 
\begin{align*}
    f:G_1&\rightarrow G_2,\\ 
    a&\mapsto s_1s_2,\\ 
    b&\mapsto s_1.
\end{align*}
Note that in \(G_2\), we have 
\begin{align*}
(s_1s_2)(s_2s_1)&=s_1(s_2^2)s_1=s_1^2=1,\\ 
(s_2s_1)(s_1s_2)&=s_2(s_1^2)s_2=s_2^2=1.
\end{align*}
We check \(f\) to be a well-defined group homomorphism. We have 
\begin{align*}
f(a)^n&=(s_1s_2)^n=1=f(1)=f(a^n),\\ 
f(b)^2&=s_1^2=1=f(b^2),\\ 
f(b)f(a)f(b)&=s_1(s_1s_2)s_1=(s_1^2)s_2s_1=(s_1s_2)^{-1}=f(a^{-1}).
\end{align*} 
Moreover, \(f\) is surjective since \(f(b)=s_1\) and \(f(ba)=s_2\). For \(f\) to be an isomorphism, the only thing left to check is that \(|G_2|\geq 2n\). 
\begin{claim}
 The follwoing elements 
 \[1,s_1,s_1s_2,s_1s_2s_1,s_1s_2s_1s_2,\ldots, \underbrace{s_1s_2\cdots s_1s_2}_\text{n-1 times}s_1\]
 are different in \(G_2\).   
\end{claim}
\begin{claimproof}
First we show that none of the nontrivial words as above is equal to \(1\). Suppose \(a=s_1s_2\cdots=1\), if \(a\) ends with \(s_1\), then both left and right multiply with \(s_1\), we have 
\[s_2s_1\cdots s_2=1.\]
Now left and right multiply with \(s_2\). Repeat this and it will give us either \(s_1=1\) or \(s_2=1\). A contradiction. Suppose two words \(a=s_1s_2\cdots\) and \(b=s_1s_2\cdots\) are equal. We are 
going to show that they must have the same length. Write \(a=b\) and left multiply with \(s_1\) and \(s_2\) continously, if \(a\) and \(b\) have different length, then we have a nontrivial word is equal to \(1\). It is 
impossible as we have seen before.  
\end{claimproof}
\end{solution}

\noindent\rule{7in}{2.8pt}
%%%%%%%%%%%%%%%%%%%%%%%%%%%%%%%%%%%%%%%%%%%%%%%%%%%%%%%%%%%%%%%%%%%%%%%%%%%%%%%%%%%%%%%%%%%%%%%%%%%%%%%%%%%%%%%%%%%%%%%%%%%%%%%%%%%%%%%%
% Exercise 6.6.3
%%%%%%%%%%%%%%%%%%%%%%%%%%%%%%%%%%%%%%%%%%%%%%%%%%%%%%%%%%%%%%%%%%%%%%%%%%%%%%%%%%%%%%%%%%%%%%%%%%%%%%%%%%%%%%%%%%%%%%%%%%%%%%%%%%%%%%%%
\begin{problem}{6.6.3}
Prove that the group of upperunitriangular \(3\times 3\) matrices over \(\mathbb{F}_2\) is isomorphic to \(D_8\).
\end{problem}
\begin{solution}
Write the group of upperunitriangular matrices as \(G\) and define 
\[a=\begin{pmatrix}
    1 & 1& 1\\ 
    0 & 1& 1\\ 
    0 & 0& 1
\end{pmatrix}, b=\begin{pmatrix}
    1 & 1& 0\\ 
    0 & 1& 0\\ 
    0 & 0& 1
\end{pmatrix},1=\begin{pmatrix}
    1 & 0& 0\\ 
    0 & 1& 0\\ 
    0 & 0& 1
\end{pmatrix}\]
Note that \(a^4=1,b^2=1\) and \(bab=a^3\). So we have a surjective map \(G\twoheadrightarrow D_8\). Since \(G\) has \(8\) elements, same as \(D_8\). So we have 
\(D_8\cong G\).   
\end{solution}
\end{document}
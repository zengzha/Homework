\documentclass[a4paper, 12pt]{article}

\usepackage{/Users/zhengz/Desktop/Math/Workspace/Homework1/homework}

%%%%%%%%%%%%%%%%%%%%%%%%%%%%%%%%%%%%%%%%%%%%%%%%%%%%%%%%%%%%%%%%%%%%%%%%%%%%%%%%%%%%%%%%%%%%%%%%%%%%%%%%%%%%%%%%%%%%%%%%%%%%%%%%%%%%%%%%
\begin{document}
%Header-Make sure you update this information!!!!
\noindent
%%%%%%%%%%%%%%%%%%%%%%%%%%%%%%%%%%%%%%%%%%%%%%%%%%%%%%%%%%%%%%%%%%%%%%%%%%%%%%%%%%%%%%%%%%%%%%%%%%%%%%%%%%%%%%%%%%%%%%%%%%%%%%%%%%%%%%%%
\large\textbf{Zhengdong Zhang} \hfill \textbf{Homework - Week 3}   \\
Email: zhengz@uoregon.edu \hfill ID: 952091294 \\
\normalsize Course: MATH 649 - Abstract Algebra  \hfill Term: Spring 2025\\
Instructor: Professor Sasha Polishchuk \hfill Due Date: $23^{th}$ April, 2025 \\
\noindent\rule{7in}{2.8pt}
\setstretch{1.1}
%%%%%%%%%%%%%%%%%%%%%%%%%%%%%%%%%%%%%%%%%%%%%%%%%%%%%%%%%%%%%%%%%%%%%%%%%%%%%%%%%%%%%%%%%%%%%%%%%%%%%%%%%%%%%%%%%%%%%%%%%%%%%%%%%%%%%%%%
% Exercise 11.4.2
%%%%%%%%%%%%%%%%%%%%%%%%%%%%%%%%%%%%%%%%%%%%%%%%%%%%%%%%%%%%%%%%%%%%%%%%%%%%%%%%%%%%%%%%%%%%%%%%%%%%%%%%%%%%%%%%%%%%%%%%%%%%%%%%%%%%%%%%



%%%%%%%%%%%%%%%%%%%%%%%%%%%%%%%%%%%%%%%%%%%%%%%%%%%%%%%%%%%%%%%%%%%%%%%%%%%%%%%%%%%%%%%%%%%%%%%%%%%%%%%%%%%%%%%%%%%%%%%%%%%%%%%%%%%%%%%%
% Exercise 11.5.5
%%%%%%%%%%%%%%%%%%%%%%%%%%%%%%%%%%%%%%%%%%%%%%%%%%%%%%%%%%%%%%%%%%%%%%%%%%%%%%%%%%%%%%%%%%%%%%%%%%%%%%%%%%%%%%%%%%%%%%%%%%%%%%%%%%%%%%%%
\begin{problem}{11.5.5}
Let \(\mathbb{K}/\Bbbk\) be a Galois extension, and \(\mathbb{L}\), \(\mathbb{M}\) be intermediate fields. Denote by \(\mathbb{L}\vee \mathbb{M}\) the minimal subfield of \(\mathbb{K}\) containing 
\(\mathbb{L}\) and \(\mathbb{M}\). 
\begin{enumerate}[(a)]
\item \((\mathbb{L}\cap \mathbb{M})^*=\la \mathbb{L}^*,\mathbb{M}^*\ra\).
\item \((\mathbb{L}\vee \mathbb{M})^*=\mathbb{L}^*\cap \mathbb{M}^*\).
\item Assume that \(\mathbb{L}/\Bbbk\) is normal. Then \(\Gal(\mathbb{L}\vee \mathbb{M}/\mathbb{M})\cong \Gal(\mathbb{L}/(\mathbb{L}\cap \mathbb{M}))\).
\end{enumerate}
\end{problem}
\begin{solution}
\begin{enumerate}[(a)]
\item We know that \(L\cap M\subseteq L\), by the Galois correspondence, we have \(L^*\subseteq (L\cap M)^*\). Similarly, we can see that \(M^*\subseteq (L\cap M)^*\). Note that \(\la L^*, M^*\ra\) is the smallest subgroup containing \(L^*\) and \(M^*\). This implies 
\((L\cap M)^*\) contains \(\la L^*,M^*\ra\). On the other hand, suppose \(a\in \mathbb{K}\) is fixed by every element in the group \(\la L^*,M^*\ra\), so \(a\) is invariant under every element in \(L^*\) and \(M^*\). This is the same as \(a\in L\) and \(a\in M\), so \(a\in L\cap M\). This proves 
\(\la L^*,M^*\ra^*\subseteq L\cap M\), by Galois correspondence, we have \((L\cap M)^*\subseteq \la L^*,M^*\ra\). Thus, we can conclude that \((L\cap M)^*=\la L^*,M^*\ra\). 
\item By definition, we know that \(L\vee M\supseteq L\) and \(L\vee M\supseteq M\), by Galois correspondence, we have \((L\vee M)^*\subseteq L^*\) and \((L\vee M)^*\subseteq M^*\), so \((L\vee M)^*\subseteq L^*\cap M^*\). On the other hand, \(L^*\cap M^*\subseteq L^*\) and \(L^*\cap M^*\subseteq M^*\), by Galois correspondence, we have 
\((L^*\cap M^*)^*\supseteq L\) and \((L^*\cap M^*)^*\supseteq M\). Note that \(L\vee M\) is the smallest subfield containing \(L\) and \(M\), so \((L^*\cap M^*)^*\supseteq L\vee M\), by Galois correspondence, we have \(L^*\cap M^*\subseteq (L\vee M)^*\). Thus, we can conclude that \((L\vee M)^*=L^*\cap M^*\). 
\item Consider the field extension \(\mathbb{L}/(\mathbb{L}\cap \mathbb{M})/\Bbbk\). We know \(\mathbb{L}/\Bbbk\) is normal, so \(\mathbb{L}/\mathbb{L}\cap \mathbb{M}\) is also normal. The Galois correspondence and the isomorphisms in (a) and (b) give us two graphs as follows
% https://q.uiver.app/#q=WzAsMTIsWzEsMCwiXFxCYmJrIl0sWzEsMSwiXFxtYXRoYmJ7TH1cXGNhcFxcbWF0aGJie019Il0sWzAsMiwiXFxtYXRoYmJ7TH0iXSxbMiwyLCJcXG1hdGhiYntNfSJdLFsxLDMsIlxcbWF0aGJie0x9XFx2ZWVcXG1hdGhiYntNfSJdLFsxLDQsIlxcbWF0aGJie0t9Il0sWzMsMiwiXFxtYXRoYmJ7TH1eKiJdLFs1LDIsIlxcbWF0aGJie019XioiXSxbNCwxLCJcXGxhbmdsZSBcXG1hdGhiYntMfV4qLFxcbWF0aGJie019XipcXHJhbmdsZSAiXSxbNCwzLCJcXG1hdGhiYntMfV4qXFxjYXAgXFxtYXRoYmJ7TX1eKiJdLFs0LDQsIlxcbGVmdFxce2VcXHJpZ2h0XFx9Il0sWzQsMCwiXFx0ZXh0e0dhbH0oXFxtYXRoYmJ7S30vXFxCYmJrKSJdLFswLDEsIiIsMCx7InN0eWxlIjp7ImhlYWQiOnsibmFtZSI6Im5vbmUifX19XSxbMSwyLCJcXHRleHR7bm9ybWFsfSIsMix7InN0eWxlIjp7ImhlYWQiOnsibmFtZSI6Im5vbmUifX19XSxbMSwzLCIiLDAseyJzdHlsZSI6eyJoZWFkIjp7Im5hbWUiOiJub25lIn19fV0sWzIsNCwiIiwyLHsic3R5bGUiOnsiaGVhZCI6eyJuYW1lIjoibm9uZSJ9fX1dLFszLDQsIiIsMCx7InN0eWxlIjp7ImhlYWQiOnsibmFtZSI6Im5vbmUifX19XSxbNCw1LCIiLDAseyJzdHlsZSI6eyJoZWFkIjp7Im5hbWUiOiJub25lIn19fV0sWzgsNiwiXFx0ZXh0e25vcm1hbH0iLDIseyJzdHlsZSI6eyJoZWFkIjp7Im5hbWUiOiJub25lIn19fV0sWzgsNywiIiwwLHsic3R5bGUiOnsiaGVhZCI6eyJuYW1lIjoibm9uZSJ9fX1dLFs2LDksIiIsMix7InN0eWxlIjp7ImhlYWQiOnsibmFtZSI6Im5vbmUifX19XSxbNyw5LCIiLDAseyJzdHlsZSI6eyJoZWFkIjp7Im5hbWUiOiJub25lIn19fV0sWzExLDgsIiIsMCx7InN0eWxlIjp7ImhlYWQiOnsibmFtZSI6Im5vbmUifX19XSxbOSwxMCwiIiwwLHsic3R5bGUiOnsiaGVhZCI6eyJuYW1lIjoibm9uZSJ9fX1dXQ==
\[\begin{tikzcd}
	& \Bbbk &&& {\text{Gal}(\mathbb{K}/\Bbbk)} \\
	& {\mathbb{L}\cap\mathbb{M}} &&& {\langle \mathbb{L}^*,\mathbb{M}^*\rangle } \\
	{\mathbb{L}} && {\mathbb{M}} & {\mathbb{L}^*} && {\mathbb{M}^*} \\
	& {\mathbb{L}\vee\mathbb{M}} &&& {\mathbb{L}^*\cap \mathbb{M}^*} \\
	& {\mathbb{K}} &&& {\left\{e\right\}}
	\arrow[no head, from=1-2, to=2-2]
	\arrow[no head, from=1-5, to=2-5]
	\arrow["{\text{normal}}"', no head, from=2-2, to=3-1]
	\arrow[no head, from=2-2, to=3-3]
	\arrow["{\text{normal}}"', no head, from=2-5, to=3-4]
	\arrow[no head, from=2-5, to=3-6]
	\arrow[no head, from=3-1, to=4-2]
	\arrow[no head, from=3-3, to=4-2]
	\arrow[no head, from=3-4, to=4-5]
	\arrow[no head, from=3-6, to=4-5]
	\arrow[no head, from=4-2, to=5-2]
	\arrow[no head, from=4-5, to=5-5]
\end{tikzcd}\]
By the second isomorphism theorems in groups, we know that \(\mathbb{L}^*\cap \mathbb{M}^*\) is normal in \(\mathbb{M}^*\) and we have an isomorphism 
\[\la \mathbb{L}^*,\mathbb{M}^*\ra/\mathbb{L}^*\cong \mathbb{M}^*/\mathbb{L}^*\cap \mathbb{M}^*.\]
Apply the Galois correspondence again, and we have 
\[(\mathbb{L}\cap \mathbb{M})^*/\mathbb{L}^*\cong \Gal(\mathbb{L}/\mathbb{L}\cap \mathbb{M})\cong (\mathbb{L}\vee \mathbb{M})^*/\mathbb{M}^*\cong \Gal(\mathbb{L}\vee \mathbb{M}/\mathbb{M}).\]
\end{enumerate}
\end{solution}

\noindent\rule{7in}{2.8pt}
%%%%%%%%%%%%%%%%%%%%%%%%%%%%%%%%%%%%%%%%%%%%%%%%%%%%%%%%%%%%%%%%%%%%%%%%%%%%%%%%%%%%%%%%%%%%%%%%%%%%%%%%%%%%%%%%%%%%%%%%%%%%%%%%%%%%%%%%
% Exercise 11.5.6
%%%%%%%%%%%%%%%%%%%%%%%%%%%%%%%%%%%%%%%%%%%%%%%%%%%%%%%%%%%%%%%%%%%%%%%%%%%%%%%%%%%%%%%%%%%%%%%%%%%%%%%%%%%%%%%%%%%%%%%%%%%%%%%%%%%%%%%%
\begin{problem}{11.5.6}
Let \(\mathbb{K}/\Bbbk\) be a finite Galois extension and \(p\) be a prime number. 
\begin{enumerate}[(a)]
\item \(\mathbb{K}\) has an intermediate subfield \(\mathbb{L}\) such that \([\mathbb{K}:\mathbb{L}]\) is a prime power. 
\item If \(\mathbb{L}_1\) and \(\mathbb{L_2}\) are intermediate subfields with \([\mathbb{K}:\mathbb{L}_1]\), \([\mathbb{K}:\mathbb{L}_2]\) both 
\(p\)-powers, and \([\mathbb{L}_1:\Bbbk]\), \([\mathbb{L}_2:\Bbbk]\) both prime to \(p\), then \(\mathbb{L}_1\) is \(\mathbb{L}_1\) is \(\Bbbk\)-isomorphic to \(\mathbb{L}_2\).
\end{enumerate}
\end{problem}
\begin{solution}
\begin{enumerate}[(a)]
\item Suppose \([\mathbb{K}:\Bbbk]=n\) is finite. We know \(n\) can be written as product of prime powers and suppose \(n=p^km\) for some prime number \(p\) and \((p,m)=1\). The Galois group \(G=\Gal(\mathbb{K}/\Bbbk)\) has order \(n\) and by Sylow's theorem, the 
Sylow \(p\)-subgroup of \(G\) exists and has order \(p^k\). By Galois correspondence, there exists a subfield \(\mathbb{K}/\mathbb{L}/\Bbbk\) such that \([\mathbb{K}:\mathbb{L}]=p^k\). 
\item Under the same assumption of (a), suppose \([\mathbb{K}:\mathbb{L}_1]=[\mathbb{K}:\mathbb{L}_2]=p^k\) and since \([\mathbb{L}_1:\Bbbk]\), \([\mathbb{L}_2]:\Bbbk\) are prime to \(p\), the Galois group \(\Gal(\mathbb{K}/\mathbb{L}_1)\) and \(\Gal(\mathbb{K}/\mathbb{L}_2)\) are Sylow \(p\)-subgroups 
in \(G\), and by Sylow theory, they are conjugate. There exists \(g\in G\) such that \(g \mathbb{L}_1^*g^{-1}=\mathbb{L}_2^*\). By Galois correspondence and the proof of Theorem 11.5.4 (iv), we know that 
\[g \mathbb{L}_1^*g^{-1}=g(\mathbb{L}_1)^*=\mathbb{L}_2^*.\]
So \(g:\mathbb{K}\rightarrow \mathbb{K}\) restricting to \(\mathbb{L}_1\) defines an isomorphism \(\mathbb{L}_1\rightarrow \mathbb{L}_2\) fixing the base field \(\Bbbk\).
\end{enumerate}
\end{solution}

\noindent\rule{7in}{2.8pt}
%%%%%%%%%%%%%%%%%%%%%%%%%%%%%%%%%%%%%%%%%%%%%%%%%%%%%%%%%%%%%%%%%%%%%%%%%%%%%%%%%%%%%%%%%%%%%%%%%%%%%%%%%%%%%%%%%%%%%%%%%%%%%%%%%%%%%%%%
% Exercise 11.5.7
%%%%%%%%%%%%%%%%%%%%%%%%%%%%%%%%%%%%%%%%%%%%%%%%%%%%%%%%%%%%%%%%%%%%%%%%%%%%%%%%%%%%%%%%%%%%%%%%%%%%%%%%%%%%%%%%%%%%%%%%%%%%%%%%%%%%%%%%
\begin{problem}{11.5.7}
Let \(f\in \Bbbk[x]\), \(\mathbb{K}/\Bbbk\) be a splitting field for \(f\) over \(\Bbbk\), and \(G:=\Gal(\mathbb{K}/\Bbbk)\).
\begin{enumerate}
\item \(G\) acts on the set of the roots of \(f\).
\item \(G\) acts transitively if \(f\) is irreducible. 
\item If \(f\) has no multiple roots and \(G\) acts transitively then \(f\) is irreducible.
\end{enumerate}
\end{problem}
\begin{solution}
\begin{enumerate}[(a)]
\item We need to show that for any \(g\in G\) and any \(\alpha\in \mathbb{K}\) is a root of \(f\), \(g(\alpha)\) is also a root of \(f\). Indeed, we know that \(g(\alpha)\) is a root of \(g(f)\) and since 
\(f\in \Bbbk[x]\) and \(g\) fixes every element in \(\Bbbk\), \(g\) fixes the polynomial \(f\), so \(g(f)=f\). Thus, we can conclude that \(G\) acts on the set of roots of \(f\). 
\item By Theorem 11.3.3, \(\mathbb{K}/\Bbbk\) is a normal extension and by Proposition 11.3.9, \(G\) acts transitively if \(f\) is irreducible. 
\item The condition is equivalent to \(\mathbb{K}/\Bbbk\) is a finite Galois extension. Assume \(f\) is not irreducible over \(\Bbbk\) and \(h|f\) for some irreducible polynomial \(h\in \Bbbk[x]\). Suppose \(\alpha_1,\ldots,\alpha_n\in \mathbb{K}\) are roots of \(f\) and \(\alpha_1,\ldots,\alpha_k\) are roots 
of \(h\) for \(1\leq k<n\). Note that for any \(g\in G\), \(g\) fixes \(h\in \Bbbk[x]\) so \(g\) must send a root of \(h\) to another root of \(h\). This means there does not exists \(g\in G\) such that \(g(\alpha_1)=\alpha_n\). This contradicts the assumption that \(G\) acts transitively, so \(f\) is irreducible. 
\end{enumerate}
\end{solution}

\noindent\rule{7in}{2.8pt}
%%%%%%%%%%%%%%%%%%%%%%%%%%%%%%%%%%%%%%%%%%%%%%%%%%%%%%%%%%%%%%%%%%%%%%%%%%%%%%%%%%%%%%%%%%%%%%%%%%%%%%%%%%%%%%%%%%%%%%%%%%%%%%%%%%%%%%%%
% Exercise 11.6.2
%%%%%%%%%%%%%%%%%%%%%%%%%%%%%%%%%%%%%%%%%%%%%%%%%%%%%%%%%%%%%%%%%%%%%%%%%%%%%%%%%%%%%%%%%%%%%%%%%%%%%%%%%%%%%%%%%%%%%%%%%%%%%%%%%%%%%%%%
\begin{problem}{11.6.2}
Let \(\Bbbk\) be a field, \(p(x)\) be an irreducible polynomial in \(\Bbbk[x]\) of degree \(n\), and let \(\mathbb{K}\) be a Galois extension of \(\Bbbk\) containing a root \(\alpha\) of \(p(x)\). Let \(G=\Gal(\mathbb{K}/\Bbbk)\), 
and \(G_\alpha\) be the set of all \(\sigma\in G\) with \(\sigma(\alpha)=\alpha\). Then: 
\begin{enumerate}[(a)]
\item \([G:G_\alpha]=n\);
\item \(G_\alpha^*=\Bbbk(\alpha)\);
\item If \(G_\alpha\) is normal in \(G\) then \(p(x)\) splits in the fixed field of \(G_\alpha\).
\end{enumerate}
\end{problem}
\begin{solution}
\begin{enumerate}[(a)]
\item Suppose \(\alpha=\alpha_1,\alpha_2,\ldots,\alpha_n\in \mathbb{K}\) are roots of \(p(x)\). For all \(1\leq i\leq n\), choose \(\sigma_i\in G\) satisfying \(\sigma_i(\alpha_1)=\alpha_i\). 
\begin{claim}
\(G=\sigma_1G_\alpha\sqcup \cdots\sqcup \sigma_nG_\alpha\) is a coset decomposition of \(G\) with respect to the subgroup \(G_\alpha\).
\end{claim} 
\begin{claimproof}
We first prove the cosets are disjoint. Suppose there exists \(g\in\sigma_iG_\alpha\cap \sigma_jG_\alpha\) for some \(1\leq i,j\leq n\), then \(g=\sigma_ig_1=\sigma_jg_2\) for some \(g_1,g_2\in G_\alpha\). Then 
\[\alpha_i=\sigma_ig_1(\alpha_1)=\sigma_jg_2(\alpha_1)=\alpha_j.\]
This implies \(i=j\). Next, we are going to show that for every \(g\in G\), \(g\) must be in one of the coset. Suppose \(g(\alpha_1)=\alpha_k\) for some \(1\leq k\leq n\). Note that \(\sigma_k^{-1}g(\alpha_1)=\alpha_1\), so \(\sigma_k^{-1}g\in G_\alpha\). There exists \(g' \in G_\alpha\) such that 
\(\sigma_k^{-1}g=g'\), namely \(g=\sigma_kg'\), so \(g\in \sigma_kG_\alpha\).
\end{claimproof}

From the claim, we know that \(G_\alpha\) has \(n\) cosets in \(G\), so by definition \([G:G_\alpha]=n\). 
\item By definition, \(G_\alpha\) fixes every element in \(\Bbbk(\alpha)\), so \(G_\alpha\subseteq \Gal(\Bbbk(\alpha)/\Bbbk)\). By Galois correspondence, this means \(G_\alpha^*\supseteq \Bbbk(\alpha)\). Moreover, by Galois correspondence and (a), we have 
\[[\Bbbk(\alpha):\Bbbk]=|\Gal(\Bbbk(\alpha)/\Bbbk)|=n=[G:G_\alpha]=[G_\alpha^*:\Bbbk].\]
This tells us that \(G_\alpha^*=\Bbbk(\alpha)\).
\item If \(G_\alpha\) is normal in \(G\), by Galois correspondence, \(G_\alpha^*/\Bbbk\) is a normal extension. We know the polynomial \(p(x)\) already has one root \(\alpha\) in \(G_\alpha^*=\Bbbk(\alpha)\), by definition of normal extension, \(p(x)\) splits in \(G_\alpha^*\).
\end{enumerate}
\end{solution}
%%%%%%%%%%%%%%%%%%%%%%%%%%%%%%%%%%%%%%%%%%%%%%%%%%%%%%%%%%%%%%%%%%%%%%%%%%%%%%%%%%%%%%%%%%%%%%%%%%%%%%%%%%%%%%%%%%%%%%%%%%%%%%%%%%%%%%%%
% Exercise 11.6.3
%%%%%%%%%%%%%%%%%%%%%%%%%%%%%%%%%%%%%%%%%%%%%%%%%%%%%%%%%%%%%%%%%%%%%%%%%%%%%%%%%%%%%%%%%%%%%%%%%%%%%%%%%%%%%%%%%%%%%%%%%%%%%%%%%%%%%%%%
\begin{problem}{11.6.3}
Let \(\Bbbk(\alpha)/\Bbbk\) be a field extension obtained by adjoining a root \(\alpha\) of an irreducible separable polynomial \(f\in \Bbbk[x]\). 
Then there exists an intermediate field \(\Bbbk\subsetneq \mathbb{F}\subsetneq \Bbbk(\alpha)\) if and only if \(\Gal(f;\Bbbk)\) is imprimitive (as a permutation group on the roots), in which case \(\mathbb{F}\) can be chosen so that 
\([\mathbb{F}:\Bbbk]\) is equal to the number of imprimitive blocks.
\end{problem}
\begin{solution}
By Theorem 7.1.11 (Primitivity Criterion), \(G=\Gal(f;\Bbbk)\) is primitive if and only if the stabilizer \(G_\beta\) is a maximal subgroup for any root \(\beta\) of the polynomial \(f\). Write \(\mathbb{N}\) as the splitting field of \(f\). Suppose there exists an intermediate field \(\Bbbk\subsetneq \mathbb{F}\subsetneq \Bbbk(\alpha)\), by Galois correspondence, there exists a proper subgroup 
\(\mathbb{F}^*\subsetneq G\) containing the stabilizer \(\Bbbk(\alpha)^*=G_\alpha\). This implies \(G\) is not primitive. Conversely, suppose \(G\) is not primitive. Then there exists a proper subgroup \(H\) satisfying \(G_\alpha\subsetneq H\subsetneq G\). By Galois correspondence, the fixed field \(H^*\) is an intermediate field and \([H^*:\Bbbk]=[G:H]=n\). Write 
\[G=g_1H\sqcup \cdots \sqcup g_nH\]
and define \(X_i:=\left\{ g_ih\cdot \alpha\mid h\in H \right\}\) for \(1\leq i\leq n\). We have proved in the proof of Theorem 7.1.11, \(X_1,\ldots,X_n\) are imprimitivity blocks, so this implies that \(\mathbb{F}=H^*\) can be chosen so that 
\([\mathbb{F}:\Bbbk]\) is equal to the number of imprimitive blocks.
\end{solution}

\noindent\rule{7in}{2.8pt}
%%%%%%%%%%%%%%%%%%%%%%%%%%%%%%%%%%%%%%%%%%%%%%%%%%%%%%%%%%%%%%%%%%%%%%%%%%%%%%%%%%%%%%%%%%%%%%%%%%%%%%%%%%%%%%%%%%%%%%%%%%%%%%%%%%%%%%%%
% Exercise 11.6.6
%%%%%%%%%%%%%%%%%%%%%%%%%%%%%%%%%%%%%%%%%%%%%%%%%%%%%%%%%%%%%%%%%%%%%%%%%%%%%%%%%%%%%%%%%%%%%%%%%%%%%%%%%%%%%%%%%%%%%%%%%%%%%%%%%%%%%%%%
\begin{problem}{11.6.6}
Find all subfields of the splitting field of \(x^3-7\) over \(\mathbb{Q}\). Which of the subfields are normal over \(\mathbb{Q}\)?
\end{problem}
\begin{solution}
Write 
\[x^3-7=(x-\sqrt[3]{7})(x-\sqrt[3]{7}\omega)(x-\sqrt[3]{7}\omega^2)\]
where \(\omega\) is the 3rd primitive root of unit satisfying \(\omega^2+\omega+1=0\). The splitting field of \(x^3-7\) is \(\mathbb{K}=\mathbb{Q}(\sqrt[3]{7},\omega)\). We know that 
\[[\mathbb{K}:\mathbb{Q}]=[\mathbb{K}:\mathbb{Q}(\sqrt[3]{7})][\mathbb{Q}(\sqrt[3]{7}):\mathbb{Q}]=2\cdot 3=6.\]
So the Galois group \(G=\Gal(\mathbb{K}/\mathbb{Q})\) is a group of order 6. Consider the following two field automorphisms \(\sigma,\tau:\mathbb{K}\rightarrow \mathbb{K}\) where \(\sigma\) fixes \(\sqrt[3]{7}\) and permutes \(\omega\) and \(\omega^2\) in \(\mathbb{K}\), \(\tau\) sends \(\sqrt[3]{7}\) to \(\sqrt[3]{7}\omega\), \(\sqrt[3]{7}\omega\) to \(\sqrt[3]{7}\omega^2\) and \(\sqrt[3]{7}\omega^2\) to \(\sqrt[3]{7}\). \(\sigma\in G\) is an element of order \(2\) and \(\tau\in G\) is an element of order \(3\). Note that 
\[\sigma\tau(\sqrt[3]{7})=\sigma(\sqrt[3]{7}\omega)=\sqrt[3]{7}\omega^2\neq \sqrt[3]{7}\omega=\tau\sigma(\sqrt[3]{7}).\]
So \(G\) is not commutative and has to be \(S_3\). The subgroup generated by \(\sigma\) is a subgroup of index \(2\) in \(G\), thus it is the normal subgroup \(\la (123)\ra\), corresponding to the normal extension \(\mathbb{Q}(\omega)/\mathbb{Q}\). The subgroups \(\la(12)\ra\), \(\la(23)\ra\) and \(\la(13)\ra\) are conjugate Sylow \(2\)-group in \(G\) of index \(3\), corresponding to the degree \(3\) subextension \(\mathbb{Q}(\sqrt[3]{7})\), \(\mathbb{Q}(\sqrt[3]{7}\omega)\) and \(\mathbb{Q}(\sqrt[3]{7}\omega^2)\). None of them are normal. These are all the subfields of \(\mathbb{K}\).
\end{solution}

\noindent\rule{7in}{2.8pt}
%%%%%%%%%%%%%%%%%%%%%%%%%%%%%%%%%%%%%%%%%%%%%%%%%%%%%%%%%%%%%%%%%%%%%%%%%%%%%%%%%%%%%%%%%%%%%%%%%%%%%%%%%%%%%%%%%%%%%%%%%%%%%%%%%%%%%%%%
% Exercise 11.6.7
%%%%%%%%%%%%%%%%%%%%%%%%%%%%%%%%%%%%%%%%%%%%%%%%%%%%%%%%%%%%%%%%%%%%%%%%%%%%%%%%%%%%%%%%%%%%%%%%%%%%%%%%%%%%%%%%%%%%%%%%%%%%%%%%%%%%%%%%
\begin{problem}{11.6.7}
Let \(\mathbb{K}\) be a splitting field for \(x^4+6x^2+5\) over \(\mathbb{Q}\). Find subfields of \(\mathbb{K}\).
\end{problem}
\begin{solution}
Write 
\[x^4+6x^2+5=(x+i)(x-i)(x+\sqrt{5}i)(x-\sqrt{5}i).\]
We know that 
\[[\mathbb{K}:\mathbb{Q}]=[\mathbb{Q}:\mathbb{Q}(i)][\mathbb{Q}(i):\mathbb{Q}]=2\cdot 2=4.\]
So the Galois group \(G=\Gal(\mathbb{K}/\mathbb{Q})\) is either the cyclic group \(C_4\) or the direct sum of two cyclic groups \(C_2\oplus C_2\). Note that \(\mathbb{Q}(i)\) and \(\mathbb{Q}(\sqrt{5})\) are two different subfields of \(\mathbb{K}\), but \(C_4\) only has one nontrivial proper subgroup, so \(G=C_2\oplus C_2\). \(G\) has three subgroups of index \(2\), corresponding to the subfields \(\mathbb{Q}(i)\), \(\mathbb{Q}(\sqrt{5})\) and \(\mathbb{Q}(\sqrt{5}i)\). All of them are normal because \(G\) is an abelian group. 
\end{solution}

\end{document}
\documentclass[a4paper, 12pt]{article}

\usepackage{/Users/zhengz/Desktop/Math/Workspace/Homework1/homework}

\begin{document}
\noindent

\large\textbf{Zhengdong Zhang} \hfill \textbf{Homework - Week 4}
Email: zhengz@uoregon.edu \hfill ID: 952091294
\normalsize Course: MATH 649 - Abstract Algebra \hfill Term: Spring 2025
Instructor: Professor Sasha Polishchuk \hfill Due Date: $30^{th}$ April , 2025 
\noindent\rule{7in}{2.8pt}
\setstretch{1.1}

%%%%%%%%%%%%%%%%%%%%%%%%%%%%%%%%%%%%%%%%%%%%%%%%%%%%%%%%%%%%%%%%%%%%%%%%%%%%%%%%%%%%%%%%%%%%%%%%%%%%%%%%%%%%%%%%%%%%%%%%%
% Problem 12.1.6
%%%%%%%%%%%%%%%%%%%%%%%%%%%%%%%%%%%%%%%%%%%%%%%%%%%%%%%%%%%%%%%%%%%%%%%%%%%%%%%%%%%%%%%%%%%%%%%%%%%%%%%%%%%%%%%%%%%%%%%%%%
\begin{problem}{12.16}
Let \(\text{char}\  k\neq 2\) and \(f\in \Bbbk[x]\) be a cubic whose discriminant has a square root in \(\Bbbk\), then \(f\) is either irreducible or splits in \(\Bbbk\).
\end{problem}
\begin{solution}

\end{solution}

\noindent\rule{7in}{2.8pt}
%%%%%%%%%%%%%%%%%%%%%%%%%%%%%%%%%%%%%%%%%%%%%%%%%%%%%%%%%%%%%%%%%%%%%%%%%%%%%%%%%%%%%%%%%%%%%%%%%%%%%%%%%%%%%%%%%%%%%%%%%
% Problem 12.4.9
%%%%%%%%%%%%%%%%%%%%%%%%%%%%%%%%%%%%%%%%%%%%%%%%%%%%%%%%%%%%%%%%%%%%%%%%%%%%%%%%%%%%%%%%%%%%%%%%%%%%%%%%%%%%%%%%%%%%%%%%%%
\begin{problem}{12.4.9}
Let \(\mathbb{K}/\Bbbk\) be a finite Galois extension and \(\alpha\in \mathbb{K}\). Consider the \(\Bbbk\)-linear operator \(A_\alpha:x\mapsto \alpha x\) on the \(\Bbbk\)-vector space \(\mathbb{K}\). Then \(\det A_\alpha=N_{\mathbb{K}/\Bbbk}(\alpha)\) and 
\(\tr A_\alpha=T_{\mathbb{K}/\Bbbk}(\alpha)\).
\end{problem}
\begin{solution}

\end{solution}

\noindent\rule{7in}{2.8pt}
%%%%%%%%%%%%%%%%%%%%%%%%%%%%%%%%%%%%%%%%%%%%%%%%%%%%%%%%%%%%%%%%%%%%%%%%%%%%%%%%%%%%%%%%%%%%%%%%%%%%%%%%%%%%%%%%%%%%%%%%%
% Problem 12.4.11
%%%%%%%%%%%%%%%%%%%%%%%%%%%%%%%%%%%%%%%%%%%%%%%%%%%%%%%%%%%%%%%%%%%%%%%%%%%%%%%%%%%%%%%%%%%%%%%%%%%%%%%%%%%%%%%%%%%%%%%%%%
\begin{problem}{12.4.11}
Let \(a,b\in \mathbb{Q}\). 
\begin{enumerate}[(a)]
\item \(a^2+b^2=1\) is equivalent to \(N_{\mathbb{Q}(i)/\mathbb{Q}}(a+ib)=1\).
\item Use Hilbert's Theorem 90 to prove that the rational solutions of \(a^2+b^2=1\) are of the form \(a=(s^2-t^2)/(s^2+t^2)\) and \(b=2st/(s^2+t^2)\) for \(s,t\in \mathbb{Q}\).
\end{enumerate}
\end{problem}
\begin{solution}

\end{solution}

\noindent\rule{7in}{2.8pt}
%%%%%%%%%%%%%%%%%%%%%%%%%%%%%%%%%%%%%%%%%%%%%%%%%%%%%%%%%%%%%%%%%%%%%%%%%%%%%%%%%%%%%%%%%%%%%%%%%%%%%%%%%%%%%%%%%%%%%%%%%
% Problem 13.2.9
%%%%%%%%%%%%%%%%%%%%%%%%%%%%%%%%%%%%%%%%%%%%%%%%%%%%%%%%%%%%%%%%%%%%%%%%%%%%%%%%%%%%%%%%%%%%%%%%%%%%%%%%%%%%%%%%%%%%%%%%%%
\begin{problem}{13.2.9}
True or false? Let \(\mathbb{K}/\mathbb{F}_q\) be a finite extension, and \(\mathbb{L}\), \(\mathbb{M}\) be two intermediate subfields. Then either \(\mathbb{L}\subseteq \mathbb{M}\) or \(\mathbb{M}\subseteq \mathbb{L}\). 
\end{problem}
\begin{solution}

\end{solution}

\noindent\rule{7in}{2.8pt}
%%%%%%%%%%%%%%%%%%%%%%%%%%%%%%%%%%%%%%%%%%%%%%%%%%%%%%%%%%%%%%%%%%%%%%%%%%%%%%%%%%%%%%%%%%%%%%%%%%%%%%%%%%%%%%%%%%%%%%%%%
% Problem 13.2.12
%%%%%%%%%%%%%%%%%%%%%%%%%%%%%%%%%%%%%%%%%%%%%%%%%%%%%%%%%%%%%%%%%%%%%%%%%%%%%%%%%%%%%%%%%%%%%%%%%%%%%%%%%%%%%%%%%%%%%%%%%%
\begin{problem}{13.2.12}
Let \(p\) be a prime. Then there are exactly \((q^p-q)/p\) monic irreducible polynomials of degree \(p\) in \(\mathbb{F}_q[x]\) (\(q\) is not necessarily a power of \(p\)).
\end{problem}
\begin{solution}

\end{solution}

\noindent\rule{7in}{2.8pt}
%%%%%%%%%%%%%%%%%%%%%%%%%%%%%%%%%%%%%%%%%%%%%%%%%%%%%%%%%%%%%%%%%%%%%%%%%%%%%%%%%%%%%%%%%%%%%%%%%%%%%%%%%%%%%%%%%%%%%%%%%
% Problem 13.2.13
%%%%%%%%%%%%%%%%%%%%%%%%%%%%%%%%%%%%%%%%%%%%%%%%%%%%%%%%%%%%%%%%%%%%%%%%%%%%%%%%%%%%%%%%%%%%%%%%%%%%%%%%%%%%%%%%%%%%%%%%%%
\begin{problem}{13.2.13}
What is \(\sum_A A^{100}\), where the sum is over all \(17\times 17\) matrices \(A\) over \(\mathbb{F}_{17}\)?
\end{problem}
\begin{solution}

\end{solution}

\end{document}
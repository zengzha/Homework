\documentclass[a4paper, 12pt]{article}

\usepackage{/Users/zhengz/Desktop/Math/Workspace/Homework1/homework}

\begin{document}
\noindent

\large\textbf{Zhengdong Zhang} \hfill \textbf{Homework - Week 5} \\
Email: zhengz@uoregon.edu \hfill ID: 952091294 \\
\normalsize Course: MATH 649 - Abstract Algebra \hfill Term: Spring 2025 \\
Instructor: Professor Sasha Polishchuk \hfill Due Date: $7^{th}$ May 2025 \\
\noindent\rule{7in}{2.8pt}
\setstretch{1.1}

%%%%%%%%%%%%%%%%%%%%%%%%%%%%%%%%%%%%%%%%%%%%%%%%%%%%%%%%%%%%%%%%%%%%%%%%%%%%%%%%%%%%%%%%%%%%%%%%%%%%%%%%%%%%%%%%%%%%%%%%%
% Problem 13.3.5
%%%%%%%%%%%%%%%%%%%%%%%%%%%%%%%%%%%%%%%%%%%%%%%%%%%%%%%%%%%%%%%%%%%%%%%%%%%%%%%%%%%%%%%%%%%%%%%%%%%%%%%%%%%%%%%%%%%%%%%%%%
\begin{problem}{13.3.5}
The 5th cyclotomic field \(\mathbb{Q}(\zeta_5)\) contains \(\sqrt{5}\). 
\end{problem}
\begin{solution}
Let \(\zeta_5\) be a 5th primitive root of \(x^5-1\). The cyclotomic field \(\mathbb{Q}(\zeta_5)\) contains all \(5\) roots: \(1,\zeta^5,\zeta_5^2\) 
\end{solution}

\noindent\rule{7in}{2.8pt}
%%%%%%%%%%%%%%%%%%%%%%%%%%%%%%%%%%%%%%%%%%%%%%%%%%%%%%%%%%%%%%%%%%%%%%%%%%%%%%%%%%%%%%%%%%%%%%%%%%%%%%%%%%%%%%%%%%%%%%%%%
% Problem 13.3.7
%%%%%%%%%%%%%%%%%%%%%%%%%%%%%%%%%%%%%%%%%%%%%%%%%%%%%%%%%%%%%%%%%%%%%%%%%%%%%%%%%%%%%%%%%%%%%%%%%%%%%%%%%%%%%%%%%%%%%%%%%%
\begin{problem}{13.3.7}
If \(p\) is a prime then 
\[\Phi_{p^n}(x)=1+x^{p^{n-1}}+x^{2p^{n-1}}+\cdots+x^{(p-1)p^{n-1}}.\]
\end{problem}
\begin{solution}

\end{solution}

\noindent\rule{7in}{2.8pt}
%%%%%%%%%%%%%%%%%%%%%%%%%%%%%%%%%%%%%%%%%%%%%%%%%%%%%%%%%%%%%%%%%%%%%%%%%%%%%%%%%%%%%%%%%%%%%%%%%%%%%%%%%%%%%%%%%%%%%%%%%
% Problem 13.3.9
%%%%%%%%%%%%%%%%%%%%%%%%%%%%%%%%%%%%%%%%%%%%%%%%%%%%%%%%%%%%%%%%%%%%%%%%%%%%%%%%%%%%%%%%%%%%%%%%%%%%%%%%%%%%%%%%%%%%%%%%%%
\begin{problem}{13.3.9}
\(\mathbb{Q}(\sqrt[3]{2})\) is not a subfield of any cyclotomic extension of \(\mathbb{Q}\).
\end{problem}
\begin{solution}

\end{solution}

\noindent\rule{7in}{2.8pt}
%%%%%%%%%%%%%%%%%%%%%%%%%%%%%%%%%%%%%%%%%%%%%%%%%%%%%%%%%%%%%%%%%%%%%%%%%%%%%%%%%%%%%%%%%%%%%%%%%%%%%%%%%%%%%%%%%%%%%%%%%
% Problem 13.5.2
%%%%%%%%%%%%%%%%%%%%%%%%%%%%%%%%%%%%%%%%%%%%%%%%%%%%%%%%%%%%%%%%%%%%%%%%%%%%%%%%%%%%%%%%%%%%%%%%%%%%%%%%%%%%%%%%%%%%%%%%%%
\begin{problem}{13.5.2}
Let \(\mathbb{K}/\Bbbk\) be a field extension. If \(\alpha_1,\ldots,\alpha_n\in \mathbb{K}\) is algebraically independent over \(\Bbbk\), and \(\alpha\notin \Bbbk\) is the element of \(k(\alpha_1,\ldots,\alpha_n)\), then \(\alpha\) is transcendental over \(\Bbbk\). 
\end{problem}
\begin{solution}

\end{solution}

\noindent\rule{7in}{2.8pt}
%%%%%%%%%%%%%%%%%%%%%%%%%%%%%%%%%%%%%%%%%%%%%%%%%%%%%%%%%%%%%%%%%%%%%%%%%%%%%%%%%%%%%%%%%%%%%%%%%%%%%%%%%%%%%%%%%%%%%%%%%
% Problem 13.5.4
%%%%%%%%%%%%%%%%%%%%%%%%%%%%%%%%%%%%%%%%%%%%%%%%%%%%%%%%%%%%%%%%%%%%%%%%%%%%%%%%%%%%%%%%%%%%%%%%%%%%%%%%%%%%%%%%%%%%%%%%%%
\begin{problem}{13.5.4}
If \(\beta\) is algebraic over \(\Bbbk(\alpha)\) and \(\beta\) is transcendental over \(\Bbbk\) then \(\alpha\) is algebraic over \(\Bbbk(\beta)\).
\end{problem}
\begin{solution}

\end{solution}

\noindent\rule{7in}{2.8pt}
%%%%%%%%%%%%%%%%%%%%%%%%%%%%%%%%%%%%%%%%%%%%%%%%%%%%%%%%%%%%%%%%%%%%%%%%%%%%%%%%%%%%%%%%%%%%%%%%%%%%%%%%%%%%%%%%%%%%%%%%%
% Problem 13.5.19
%%%%%%%%%%%%%%%%%%%%%%%%%%%%%%%%%%%%%%%%%%%%%%%%%%%%%%%%%%%%%%%%%%%%%%%%%%%%%%%%%%%%%%%%%%%%%%%%%%%%%%%%%%%%%%%%%%%%%%%%%%
\begin{problem}{13.5.19}
Let \(\Bbbk \subsetneq\mathbb{F}\subseteq\Bbbk(x)\) be field extensions, with \(x\) transcendental over \(\Bbbk\). Then \(\Bbbk(x)/\mathbb{F}\) is finite. 
\end{problem}
\begin{solution}

\end{solution}

\noindent\rule{7in}{2.8pt}
%%%%%%%%%%%%%%%%%%%%%%%%%%%%%%%%%%%%%%%%%%%%%%%%%%%%%%%%%%%%%%%%%%%%%%%%%%%%%%%%%%%%%%%%%%%%%%%%%%%%%%%%%%%%%%%%%%%%%%%%%
% Problem 13.6.5
%%%%%%%%%%%%%%%%%%%%%%%%%%%%%%%%%%%%%%%%%%%%%%%%%%%%%%%%%%%%%%%%%%%%%%%%%%%%%%%%%%%%%%%%%%%%%%%%%%%%%%%%%%%%%%%%%%%%%%%%%%
\begin{problem}{13.6.5 (Newton's identities)}
Let \(x_1,\ldots, x_n\) be variables, and define power sum symmetric functions
\[p_k=p_k(x_1,\ldots,x_n)=x_1^k+\cdots+x_n^k\ \ \ \ (k\in \mathbb{Z}_{>0}).\]
Prove the \textit{Newton identities}:
\[ke_k=\sum_{i=1}^k(-1)^{i-1}e_{k-i}p_i\]
where \(e_k\) are the elementary symmetric functions interpreted \(1\) if \(k=0\) and as \(0\) if \(k>n\). Deduce that every elementary symmetric function \(e_k\) can be written down as a polynomial in \(p_1,\ldots, p_k\) with rational coefficients. Deduce that every symmetric polynomial can be written down as a polynomial in the power sum symmetric functions.
\end{problem}
\begin{solution}
Let \(x_1,\ldots,x_n\) be variables, define the following polynomial 
\[f(x)=(x-x_1)(x-x_2)\cdots (x-x_n).\]
Remove the parentheses, and we can rewrite \(f(x)\) as 
\[f(x)=x^n-e_1x^{n-1}+\cdots+(-1)^{n-1}e_{n-1}x+(-1)^n e_n.\]
We prove Newton's identities in different cases.
\begin{enumerate}[(a)]
\item Suppose \(k=n\).\\ 
We know that for \(1\leq i\leq n\), \(x_i\) is the root of \(f\), so it satisfies the following equation. 
\begin{equation}
    x_i^n-e_1x_i^{n-1}+\cdots+(-1)^{n-1}e_{n-1}x+(-1)^ne_n=0.
\end{equation}
Add all these equations from \(i=1\) to \(i=n\), we obtain 
\[p_n-e_1p_{n-1}+\cdots+(-1)^{n-1}e_{n-1}p_1+(-1)^n n e_n=0.\]
This is the same as 
\begin{align*}
    (-1)^{n-1}n e_n&=\sum_{i=1}^n (-1)^{n-i} e_{n-i}p_i\\ 
              n e_n&=\sum_{i=1}^{n} (-1)^{i-1} e_{n-i}p_i.
\end{align*}
\item Suppose \(k>n\).\\ 
Let \(\alpha_1,\alpha_2,\ldots, \alpha_{k-n}\) be a variable. Consider the polynomial 
\[g(x)=f(x)(x-\alpha_1)(x-\alpha_2)\cdots(x-\alpha_{k-n}).\]
In this case, the generic polynomial is \(e'_j\) and the power sum function is \(p'_j\) for \(1\leq j\leq k\). From the result in (a), we have an equation
\[ke'_k=\sum_{i=1}^{k}(-1)^{i-1}e'_{k-i}p'_i.\]
Let \(\alpha_1=\alpha_2=\cdots=\alpha_{k-n}=0\). Then we have 
\begin{align}
    e'_j&=e_j,&&\iif 1\leq j\leq n,\\
    e'_j&=0,&&\iif n+1\leq j\leq k,\\
    p'_j&=p_j,&&\iif 1\leq j\leq k.
\end{align}
Then the equation can be rewritten as 
\[0=ke_k=\sum_{i=1}^{k}(-1)^{i-1}e_{k-i}p_i.\]
\item Suppose \(k<n\).\\ 
Consider the formal derivative \(f'(x)\) of \(f(x)\), which can be written in two forms:
\begin{align*}
    f'(x)&=\sum_{j=1}^{n}\frac{f(x)}{x-x_j},\\
    f'(x)&=nx^{n-1}-(n-1)e_1x^{n-2}+\cdots+(-1)^{n-1}e_{n-1}.
\end{align*}
For \(0\leq l\leq n-1\), the coefficient in front of \(x^{n-1-l}\) is \((-1)^l (n-l)e_l\). 
For \(1\leq j\leq n\), \(\frac{f(x)}{x-x_j}\) can be written as 
\[\frac{f(x)}{x-x_j}=(x-x_1)\cdots (x-x_{j-1})(x-x_{j+1})\cdots(x-x_n).\]
Remove the parentheses, and we obtain 
\begin{align*}
    \frac{f(x)}{x-x_j}=&x^{n-1}+(-e_1+x_j)x^{n-2}+(e_2-e_1x_j+x_j^2)x^{n-3}\\
                       &+\cdots+((-1)^l e_l+\sum_{m=1}^{l} (-1)^{l+m}e_{l-m}x_j^{m})x^{n-l-1}+\cdots\\
                       &+(-1)^{n-1}e_{n-1}+\sum_{m=1}^{n-1} (-1)^{m+n-1}e_{n-m-1}x_j^{m}.
\end{align*}
Add \(j=1\) to \(j=n\) together and we have 
\[f'(x)=nx^{n-1}+\sum_{l=1}^{n-1}[(-1)^l n e_l+(\sum_{m=1}^{l}(-1)^{l+m}e_{l-m}p_m)]x^{n-l-1}\]
Comparing coefficients, and we have, for \(1\leq l\leq n-1\), 
\[(-1)^l (n-l)e_l=(-1)^l n e_l+\sum_{m=1}^{l} (-1)^{l+m} e_{l-m}p_m.\]
This is equivalent to 
\[l e_l=\sum_{m=1}^{l}(-1)^{m-1} e_{l-m}p_m\]
for all \(1\leq l\leq n-1\).
\end{enumerate}

We have proved Newton's identities for \(k>0\). We have 
\begin{align*}
    e_1&=p_1,\\
    2e_2&=e_1p_1-p_2,\\
    3e_3&=e_2p_1-e_1p_2+p_3,\\ 
    &\cdots 
\end{align*}
for all \(k>0\). From this, we can inductively write \(e_k\) as a polynomial of \(p_1,\ldots,p_k\) with rational coefficients. By Theorem 13.6.1, since every symmetric polynomial can be written down as a polynomial in symmetric functions, then it can also be written as a polynomial in power sum functions. 
\end{solution}


\end{document}
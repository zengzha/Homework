\documentclass[a4paper, 12pt]{article}

\usepackage{/Users/zhengz/Desktop/Math/Workspace/Homework1/homework}
%%%%%%%%%%%%%%%%%%%%%%%%%%%%%%%%%%%%%%%%%%%%%%%%%%%%%%%%%%%%%%%%%%%%%%%%%%%%%%%%%%%%%%%%%%%%%%%%%%%%%%%%%%%%%%%%%%%%%%%%%%%%%%%%%%%%%%%%
\begin{document}
%Header-Make sure you update this information!!!!
\noindent
%%%%%%%%%%%%%%%%%%%%%%%%%%%%%%%%%%%%%%%%%%%%%%%%%%%%%%%%%%%%%%%%%%%%%%%%%%%%%%%%%%%%%%%%%%%%%%%%%%%%%%%%%%%%%%%%%%%%%%%%%%%%%%%%%%%%%%%%
\large\textbf{Zhengdong Zhang} \hfill \textbf{Homework - Week 2}   \\
Email: zhengz@uoregon.edu \hfill ID: 952091294 \\
\normalsize Course: MATH 648 - Abstract Algebra  \hfill Term: Winter 2025\\
Instructor: Professor Arkady Berenstein \hfill Due Date: $22^{nd}$ January, 2025 \\
\noindent\rule{7in}{2.8pt}
\setstretch{1.1}
%%%%%%%%%%%%%%%%%%%%%%%%%%%%%%%%%%%%%%%%%%%%%%%%%%%%%%%%%%%%%%%%%%%%%%%%%%%%%%%%%%%%%%%%%%%%%%%%%%%%%%%%%%%%%%%%%%%%%%%%%%%%%%%%%%%%%%%%
% Exercise 14.4.1
%%%%%%%%%%%%%%%%%%%%%%%%%%%%%%%%%%%%%%%%%%%%%%%%%%%%%%%%%%%%%%%%%%%%%%%%%%%%%%%%%%%%%%%%%%%%%%%%%%%%%%%%%%%%%%%%%%%%%%%%%%%%%%%%%%%%%%%%
\begin{problem}{14.4.1}
Give examples of 
\begin{enumerate}[(1)]
\item a module which is both noetherian and artinian;
\item a module which is noetherian but not artinian;
\item a module which is artinian but not noetherian;
\item a module which is neither artinian nor noetherian.
\end{enumerate}
\end{problem}
\begin{solution}
\begin{enumerate}[(1)]
\item Consider a field \(\mathbb{F}\), viewed as an \(\mathbb{F}\)-module (\(\mathbb{F}\)-vector space). We know \(\mathbb{F}\) only has \(0\) and \(\mathbb{F}\) as its ideal,  so \(\mathbb{F}\) is both artinian and noetherian. 
\item Consider the ring \(\mathbb{Z}\) as a \(\mathbb{Z}\)-module. It is a PID so any ideal has the form \((n)\) for \(n\geq 0\). Note that \((m)\subset (n)\) if and only if \(n|m\). Since \(\mathbb{Z}\) is a UFD, every positive number \(n\) has 
a unique prime decomposition up to reordering, the ascending chain of submodules must stabilize. So \(\mathbb{Z}\) is noetherian. On the other hands, consider the following descending chain of submodules 
\[(2)\supset (2^2)\supset (2^3)\supset \cdots\]
This chain never stabilizes so \(\mathbb{Z}\) is not artinian. 
\item Let \(p\) be a prime number. Consider the \(\mathbb{Z}\)-module \(V=\mathbb{Z}[\frac{1}{p}]/\mathbb{Z}\). \(V\) can be written as 
\[V=\mathbb{Z}[\frac{1}{p}]/\mathbb{Z}=\left\{ \frac{a}{p^n}\mid n\geq 0, 0\leq a\leq p^n\right\}.\]
Every submodule of \(V\) is generated by a single element \(\frac{1}{p^k}\), where \(k\in \mathbb{N}\) is a positive integer. Note that \(\la \frac{1}{p^k}\ra\subset \la \frac{1}{p^e}\ra\) if and only if \(0\leq k\leq e\). So we have an ascending chain of submodules 
\[\la \frac{1}{p}\ra\subset \la \frac{1}{p^2}\ra\subset \cdots\]
which never stabilizes. This means \(V\) is not noetherian. On the other hand, for any submodule \(\la \frac{1}{p^k}\ra\), the descending chain 
\[\la \frac{1}{p^k}\ra \supset \la \frac{1}{p^{k-1}}\ra\supset \cdots \supset \la \frac{1}{p}\ra\supset (0)\]
is the longest possible descending chain, so \(V\) is artinian but not noetherian.
\item Let \(\mathbb{F}\) be a field and \(V\) be an infinite dimensional \(\mathbb{F}\)-vector space with a countable basis \(B=\left\{ v_1,v_2,\ldots,v_n,\ldots \right\}\). For any \(i\geq 1\), let \(V_i\) be the 
finite dimensional subspace generated by \(v_1,v_2,\ldots,v_i\). Then we have an ascending chain of subspaces 
\[V_1\subset V_2\subset \cdots\subset V_n\subset\cdots\]
which never stabilizes since \(V\) is infinite dimensional. So \(V\) is not noetherian. Moreover, consider the descending chain 
\[V\setminus V_1\supset V\setminus V_2\supset \cdots\supset V\setminus V_n\supset \cdots\]
which also never stabilizes because \(V\setminus V_n\) is always infinite dimensional for any \(n\). So \(V\) is neither noetherian nor artinian.
\end{enumerate}
\end{solution}

\noindent\rule{7in}{2.8pt}
%%%%%%%%%%%%%%%%%%%%%%%%%%%%%%%%%%%%%%%%%%%%%%%%%%%%%%%%%%%%%%%%%%%%%%%%%%%%%%%%%%%%%%%%%%%%%%%%%%%%%%%%%%%%%%%%%%%%%%%%%%%%%%%%%%%%%%%%
% Exercise 14.4.2
%%%%%%%%%%%%%%%%%%%%%%%%%%%%%%%%%%%%%%%%%%%%%%%%%%%%%%%%%%%%%%%%%%%%%%%%%%%%%%%%%%%%%%%%%%%%%%%%%%%%%%%%%%%%%%%%%%%%%%%%%%%%%%%%%%%%%%%%
\begin{problem}{14.4.2}
If \(V\) is a noetherian \(R\)-module then any surjective \(R\)-module endomorphism of \(V\) is an isomorphism.
\end{problem}
\begin{solution}
The kernel of an \(R\)-module homomorphism is a submodule, so we have an ascending chain of submodules 
\[\ker f\subset \ker f^2\subset \ker f^3\subset \cdots\]
which stabilizes since \(V\) is noetherian. This means there exists \(N\geq 1\) such that \(\ker f^n=\ker f^{n+1} \) for all \(n\geq N\). Let \(x\in \ker f\). \(f\) being surjective tells us that 
there exists \(y\in V\) such that \(f^n(y)=x\) for some \(n\geq N\). We have \(0=f(x)=f^{n+1}(y)\), so \(y\in \ker f^{n+1}=\ker f^n\). This shows that \(0=f^n(y)=x\). We have proved that 
\(\ker f=0\), namely \(f\) is injective. Thus, \(f\) is an isomorphism. 
\end{solution}

\noindent\rule{7in}{2.8pt}
%%%%%%%%%%%%%%%%%%%%%%%%%%%%%%%%%%%%%%%%%%%%%%%%%%%%%%%%%%%%%%%%%%%%%%%%%%%%%%%%%%%%%%%%%%%%%%%%%%%%%%%%%%%%%%%%%%%%%%%%%%%%%%%%%%%%%%%%
% Exercise 14.4.10
%%%%%%%%%%%%%%%%%%%%%%%%%%%%%%%%%%%%%%%%%%%%%%%%%%%%%%%%%%%%%%%%%%%%%%%%%%%%%%%%%%%%%%%%%%%%%%%%%%%%%%%%%%%%%%%%%%%%%%%%%%%%%%%%%%%%%%%%
\begin{problem}{14.4.10}
Given an exmaple of 
\begin{enumerate}[(1)]
\item A commutative non-noetherian ring.
\item A commutative ring which is noetherian but not artinian.
\end{enumerate}
\end{problem}
\begin{solution}
\begin{enumerate}[(1)]
\item Let \(R=\mathbb{F}[x_1,x_2,\ldots]\) be a polynomial ring over a field \(\mathbb{F}\) with countably many indeterminates. consider the following chain of ideals 
\[(x_1)\subset (x_1,x_2)\subset \cdots (x_1,x_2,\ldots,x_n)\subset \cdots\]
which never stabilizes since we have infinitely many variables. So \(R\) is not a noetherian ring.
\item Let \(R=\mathbb{F}[x]\) be the polynomial ring over a field \(\mathbb{F}\). We know \(R\) is a PID, so it is noetherian. On the other hand, consider the descending chain of ideals 
\[(x)\supset (x^2)\supset (x^3)\supset \cdots\]
which never stabilizes. So \(R\) is not artinian.
\end{enumerate}  
\end{solution}

\noindent\rule{7in}{2.8pt}
%%%%%%%%%%%%%%%%%%%%%%%%%%%%%%%%%%%%%%%%%%%%%%%%%%%%%%%%%%%%%%%%%%%%%%%%%%%%%%%%%%%%%%%%%%%%%%%%%%%%%%%%%%%%%%%%%%%%%%%%%%%%%%%%%%%%%%%%
% Exercise 14.4.11
%%%%%%%%%%%%%%%%%%%%%%%%%%%%%%%%%%%%%%%%%%%%%%%%%%%%%%%%%%%%%%%%%%%%%%%%%%%%%%%%%%%%%%%%%%%%%%%%%%%%%%%%%%%%%%%%%%%%%%%%%%%%%%%%%%%%%%%%
\begin{problem}{14.4.11}
True or false? If \(R\) is artinian, then \(R[x]\) is artinian.
\end{problem}
\begin{solution}
This is false. Consider \(R=\mathbb{F}\) is a field. We know \(R\) is artinian since the only ideals in \(\mathbb{F}\) is the zero ideal \(0\) and \(\mathbb{F}\) itself, so \(R\) is artinian. But \(R[x]\) is not artinian as we have shown in 
Exercise 14.4.10(2). 
\end{solution}

\noindent\rule{7in}{2.8pt}
%%%%%%%%%%%%%%%%%%%%%%%%%%%%%%%%%%%%%%%%%%%%%%%%%%%%%%%%%%%%%%%%%%%%%%%%%%%%%%%%%%%%%%%%%%%%%%%%%%%%%%%%%%%%%%%%%%%%%%%%%%%%%%%%%%%%%%%%
% Exercise 14.4.14
%%%%%%%%%%%%%%%%%%%%%%%%%%%%%%%%%%%%%%%%%%%%%%%%%%%%%%%%%%%%%%%%%%%%%%%%%%%%%%%%%%%%%%%%%%%%%%%%%%%%%%%%%%%%%%%%%%%%%%%%%%%%%%%%%%%%%%%%
\begin{problem}{14.4.14}
Give an example showing that a submodule of a finitely generated module in general does not have to be finitely generated.
\end{problem}
\begin{solution}
Let \(\mathbb{F}\) be a field of characteristic not equal to 2 and \(R=\mathbb{F}[x_1,x_2,\ldots]\) be a polynomial ring over \(\mathbb{F}\) with countably many variables, viewed as a regular module. The principal ideal \((2)\) is finitely generated but 
\((2)\subset (2x_1,2x_2,\ldots)\). Here \((2x_1,2x_2,\ldots)\) is not finitely generated since we have a ascending chain 
\[(2x_1)\subset (2x_1,2x_2)\subset \cdots\]
which never stabilizes.  
\end{solution}

\noindent\rule{7in}{2.8pt}
%%%%%%%%%%%%%%%%%%%%%%%%%%%%%%%%%%%%%%%%%%%%%%%%%%%%%%%%%%%%%%%%%%%%%%%%%%%%%%%%%%%%%%%%%%%%%%%%%%%%%%%%%%%%%%%%%%%%%%%%%%%%%%%%%%%%%%%%
% Exercise 14.4.17
%%%%%%%%%%%%%%%%%%%%%%%%%%%%%%%%%%%%%%%%%%%%%%%%%%%%%%%%%%%%%%%%%%%%%%%%%%%%%%%%%%%%%%%%%%%%%%%%%%%%%%%%%%%%%%%%%%%%%%%%%%%%%%%%%%%%%%%%
\begin{problem}{14.4.17}
Let \(D\) be a division ring and \(R=M_n(D)\). Construct composition series of \(\leftindex_R R\) and \(R_R\) and conclude that \(R\) is left and right artinian.
\end{problem}
\begin{solution}
First we consider the case \(\leftindex_R R\) is a left regular \(R\)-module. For any \(1\leq i\leq n\), define \(V_i\subset V\) to be the set of matrices with the first \(i\) columns being zeros. It is easy to check that they are 
left submodules of \(\leftindex_R R\). Moreover, let \(V_0=V\) and \(V_n=0\), for any \(0\leq i\leq n-1\), the quotient \(V_i/V_{i+1}\cong D^n\) as a column vector, by Example 14.1.22, \(D^n\) is irreducible as an \(R\)-module. So we have a composition 
series 
\[V=V_0\supset V_1\supset V_2\supset \cdots\supset V_n=0\]
This proves that \(R\) is left artinian. 

Now view \(R\) as a right regular module. This time for any \(1\leq i\leq n\), define \(V_i\subset V\) to be the set of all matrices with the first \(i\) rows being zeros. They are right \(R\)-submodules of \(R\), and still we have \(V_i/V_{i+1}\cong D^n\) as a row vector. A 
similar argument as before shows that \(R\) is right artinian.
\end{solution}

\noindent\rule{7in}{2.8pt}
%%%%%%%%%%%%%%%%%%%%%%%%%%%%%%%%%%%%%%%%%%%%%%%%%%%%%%%%%%%%%%%%%%%%%%%%%%%%%%%%%%%%%%%%%%%%%%%%%%%%%%%%%%%%%%%%%%%%%%%%%%%%%%%%%%%%%%%%
% Exercise 14.4.19
%%%%%%%%%%%%%%%%%%%%%%%%%%%%%%%%%%%%%%%%%%%%%%%%%%%%%%%%%%%%%%%%%%%%%%%%%%%%%%%%%%%%%%%%%%%%%%%%%%%%%%%%%%%%%%%%%%%%%%%%%%%%%%%%%%%%%%%%
\newpage 
\begin{problem}{14.4.19}
Let \(\ell(V)\) denote the composition length of an \(R\)-module \(V\). Suppose that \(V\) is an \(R\)-module of finite length and let \(X\) and \(Y\) be submodules of \(V\). Then 
\(\ell(X+Y)+\ell(X\cap Y)=\ell(X)+\ell(Y)\).
\end{problem}
\begin{solution}
We first prove the following claim:
\begin{claim}
Let \(K,V,Q\) be submodules of \(V\) and suppose we have a short exact sequence 
\[0\rightarrow K\xrightarrow{\iota} V\xrightarrow{\pi} Q\rightarrow 0.\]
Then we have 
\[\ell(V)=\ell(K)+\ell(Q).\]
\end{claim}
\begin{claimproof}
V is an \(R\)-module of finite length, so \(V,K,Q\) as submodules also has finite length. These exists a composition series for \(K\) 
\[K=K_0\supset K_1\supset \cdots\supset K_n=0.\]
Apply the map \(\iota\) and we get a sequence 
\[\iota(K)=\iota(K_0)\supset \iota(K_1)\supset \cdots\supset \iota(K_n)=0.\]
Note that because \(\iota\) is injective, so for any \(0\leq i\leq n-1\), \(\iota(K_i)/\iota(K_{i+1})\cong \iota(K_i/K_{i+1})\) is still simple. So this is a 
composition series for \(\iota(K)\). On the other hand, \(V/\iota(K)\cong Q\) is of finite length, so we have a composition series 
\[V/\iota(K)=V_0/\iota(K)\supset V_1/\iota(K)\supset \cdots\supset V_m/\iota(K)=0.\]
By the correspondence theorem for modules, this is equivalent to a sequnce 
\[V=V_0\supset V_1\supset \cdots\supset V_m=K.\]
and by the third isomorphism theorem, for any \(0\leq i\leq m-1\), \(V_i/V_{i+1}\cong \frac{V_i/\iota(K)}{V_{i+1}/\iota(K)}\) is still simple. So we have a composition series for \(V\) 
\[V=V_0\supset V_1\supset \cdots\supset V_m=K=K_0\supset K_1\supset \cdots\supset K_n=0\]
and we can see that \(\ell(V)=m+n=\ell(K)+\ell(Q)\).
\end{claimproof}

Consider the two short exact sequences of \(R\)-modules as follows 
\[\begin{tikzcd}
	0 & {X\cap Y} & X & {X/(X\cap Y)} & 0 \\
	0 & Y & {X+Y} & {(X+Y)/Y} & 0
	\arrow[from=1-1, to=1-2]
	\arrow[from=1-2, to=1-3]
	\arrow[from=1-3, to=1-4]
	\arrow[from=1-4, to=1-5]
	\arrow[from=2-1, to=2-2]
	\arrow[from=2-2, to=2-3]
	\arrow[from=2-3, to=2-4]
	\arrow[from=2-4, to=2-5]
\end{tikzcd}\]
By the previous claim, we have 
\begin{align*}
    \ell(X)&=\ell(X\cap Y)+\ell(X/(X\cap Y)),\\ 
    \ell(X+Y)&=\ell(Y)+\ell((X+Y)/Y).
\end{align*}
By the second isomorphism theorem, we have 
\[X/(X\cap Y)\cong (X+Y)/Y.\]
So we can write 
\[\ell(X)-\ell(X\cap Y)=\ell(X/(X\cap Y))=\ell((X+Y)/Y)=\ell(X+Y)-\ell(Y).\]
This is equivalent to 
\[\ell(X+Y)+\ell(X\cap Y)=\ell(X)+\ell(Y).\]
\end{solution}

\noindent\rule{7in}{2.8pt}
%%%%%%%%%%%%%%%%%%%%%%%%%%%%%%%%%%%%%%%%%%%%%%%%%%%%%%%%%%%%%%%%%%%%%%%%%%%%%%%%%%%%%%%%%%%%%%%%%%%%%%%%%%%%%%%%%%%%%%%%%%%%%%%%%%%%%%%%
% Exercise 14.4.20
%%%%%%%%%%%%%%%%%%%%%%%%%%%%%%%%%%%%%%%%%%%%%%%%%%%%%%%%%%%%%%%%%%%%%%%%%%%%%%%%%%%%%%%%%%%%%%%%%%%%%%%%%%%%%%%%%%%%%%%%%%%%%%%%%%%%%%%%
\begin{problem}{14.4.20}
If \(V_1\) and \(V_2\) are non-isomorphic irreducible \(R\)-modules, then \(V_1\oplus V_2\) has exactly four submodules: \((0)\), \((0)\oplus V_2\), \(V_1\oplus (0)\), and \(V_1\oplus V_2\). 
\end{problem}
\begin{solution}
It is easy to see that \((0),(0)\oplus V_2,V_1\oplus (0),V\) are all submodules of \(V\), we need to show \(V\) does not have any other submodule. Suppose \(W=\la (v_1,v_2)\ra\subset V\) is a submodule of \(V\) and \(W\neq 0\), \(W\neq (0)\oplus V_2\) and \(W\neq V_1\oplus (0)\).  
So \(v_1\in V_1\) and \(v_2\in V_2\), \(v_1\) and \(v_2\) are both nonzero. Note that \(W\cap (V_1\oplus 0)\) is a submodule of \(W\) and by the second isomorphism theorem, we have 
\[W/(W\cap (V_1\oplus 0))\cong (W+(V_1\oplus 0))/(V_1\oplus 0)\subset V/(V_1\oplus 0)\cong 0\oplus V_2.\]
Since \(V_2\) is simple, we know that \(W/(W\cap (V_1\oplus 0))=0\) or \(W\cap (V_1\oplus 0)\cong V_2\). If \(W/(W\cap (V_1\oplus 0))=0\), namely \(W=W\cap (V_1\oplus 0)\), but we know that \(v_2\neq 0\). So 
\(W/(W\cap (V_1\oplus 0))\cong V_2\) is simple and nonzero. We have shown \(V_2\) is a composition factor of \(W\) and similarly using \(W\cap (0\oplus V_2)\), we can show that \(V_1\) is also a composition factor of 
\(W\). By Jordan-H\"{o}lder theorem, \(W\) must be isomorphic to \(V\). Thus, we can conclude that \(V\) does not have any other submodule.
\end{solution}

\noindent\rule{7in}{2.8pt}
%%%%%%%%%%%%%%%%%%%%%%%%%%%%%%%%%%%%%%%%%%%%%%%%%%%%%%%%%%%%%%%%%%%%%%%%%%%%%%%%%%%%%%%%%%%%%%%%%%%%%%%%%%%%%%%%%%%%%%%%%%%%%%%%%%%%%%%%
% Exercise 14.5.14
%%%%%%%%%%%%%%%%%%%%%%%%%%%%%%%%%%%%%%%%%%%%%%%%%%%%%%%%%%%%%%%%%%%%%%%%%%%%%%%%%%%%%%%%%%%%%%%%%%%%%%%%%%%%%%%%%%%%%%%%%%%%%%%%%%%%%%%%
\begin{problem}{14.5.14}
Let \(A\) be the ring of continous functions \(\mathbb{R}\rightarrow \mathbb{R}\). Prove that \(A\) has no non-trivial idempotens, but \(A\) is not local.
\end{problem}
\begin{solution}
Suppose \(f\) is an idempotent in \(A\) and \(f\neq 0\). For any \(x\in \mathbb{R}\) satisfying \(f(x)\neq 0\), we have \((f(x))^2=f(x)\). Divide both sides by \(f(x)\), we have 
\(f(x)=1\). Since \(f\) is continous everywhere on \(\mathbb{R}\), so \(f(x)=1\) for all \(x\in \mathbb{R}\). This proves that \(A\) does not have any idempotent except for \(0\) and \(1\).

Both \(f(x)=1-x^2\) and \(g(x)=x^2\) are non-units in \(A\) but \((f+g)(x)=1\) is a unit. So \(A\) is not a local ring.
\end{solution}

\noindent\rule{7in}{2.8pt}
%%%%%%%%%%%%%%%%%%%%%%%%%%%%%%%%%%%%%%%%%%%%%%%%%%%%%%%%%%%%%%%%%%%%%%%%%%%%%%%%%%%%%%%%%%%%%%%%%%%%%%%%%%%%%%%%%%%%%%%%%%%%%%%%%%%%%%%%
% Exercise 14.5.17
%%%%%%%%%%%%%%%%%%%%%%%%%%%%%%%%%%%%%%%%%%%%%%%%%%%%%%%%%%%%%%%%%%%%%%%%%%%%%%%%%%%%%%%%%%%%%%%%%%%%%%%%%%%%%%%%%%%%%%%%%%%%%%%%%%%%%%%%
\newpage 
\begin{problem}{14.5.17}
Let \(D\) be a division ring, and let \(R\) be the ring of all \(2\times 2\) upper triangular matrices over \(D\). Let \(e_1:=E_{1,1}\) and \(e_2:=E_{2,2}\).
\begin{enumerate}[(1)]
\item \(e_1\) and \(e_2\) are orthogonal idempotents with \(e_1+e_2=1\), so \(\leftindex_R R=Re_1\oplus Re_2\).
\item \(Re_1\) is irreducible but \(Re_2\) is not.
\item \(\text{End}_R(Re_1)^{op}\cong \text{End}_R(Re_2)^{op}\cong D\) as rings. Deduce that \(Re_1\) and \(Re_2\) are indecomposable \(R\)-modules, and the idempotents \(e_1\) and \(e_2\) are primitive.
\item Classify irreducible \(R\)-modules.
\end{enumerate}
\end{problem}
\begin{solution}
\begin{enumerate}[(1)]
\item We can check that \(e_1\) and \(e_2\) are idempotents by direct computations.
\[\begin{pmatrix}
    1&0\\ 
    0&0
\end{pmatrix}\begin{pmatrix}
    1&0\\ 
    0&0
\end{pmatrix}=\begin{pmatrix}
    1&0\\ 
    0&0
\end{pmatrix}.\]
Same for \(e_2\). And we have 
\[\begin{pmatrix}
    1&0\\ 
    0&0
\end{pmatrix}+\begin{pmatrix}
    0&0\\ 
    0&1
\end{pmatrix}=\begin{pmatrix}
    1&0\\ 
    0&1
\end{pmatrix}\]
Note that \(e_1e_2=e_2e_1=0\) and \(\text{End}_R(\leftindex_R R)=R\), by Lemma 14.5.1, we have \(R=Re_1\oplus Re_2\).
\item Given an upper triangular matrix \(A=\begin{pmatrix}
    a&b\\ 
    0&c
\end{pmatrix}\), it is easy to check that \(Ae_1=ae_1\). So \(Re_1\cong D\) and since \(D\) is a division ring, the only ideal is the zero ideal and \(D\) itself, so \(Re_1\) is simple. 
For \(Re_2\), we have 
\[Ae_2=\begin{pmatrix}
    a&b\\ 
    0&c
\end{pmatrix}\begin{pmatrix}
    0&0\\ 
    0&1
\end{pmatrix}=\begin{pmatrix}
    0&b\\ 
    0&c
\end{pmatrix}.\]
Consider the submodule 
\[W=\left\{ \begin{pmatrix}
0&w\\ 
0&0
\end{pmatrix}\mid w\in D \right\}\cong D.\]
This is a proper submodule of \(Re_2\) as 
\[\begin{pmatrix}
    a&b\\ 
    0&c
\end{pmatrix}\begin{pmatrix}
    0&w\\ 
    0&0
\end{pmatrix}=\begin{pmatrix}
    0&aw\\ 
    0&0
\end{pmatrix}.\]
So \(Re_2\) is not irreducible.
\item We know that \(\text{End}_R(\leftindex_R R)\cong R\) and \(e_1,e_2\) are idempotents, by Lemma 14.5.4, we only need to show that 
\[e_1Re_1\cong e_2Re_2\cong D.\]
This can be checked by direct computations, suppose \(\begin{pmatrix}
    a&b\\ 
    0&c
\end{pmatrix}\in R\), we have 
\begin{align*}
    \begin{pmatrix}
        1&0\\ 
        0&0
    \end{pmatrix}\begin{pmatrix}
        a&b\\ 
        0&c
    \end{pmatrix}\begin{pmatrix}
        1&0\\ 
        0&0
    \end{pmatrix}&=\begin{pmatrix}
        a&0\\ 
        0&0
    \end{pmatrix}\\[5pt]
    \begin{pmatrix}
        0&0\\ 
        0&1
    \end{pmatrix}\begin{pmatrix}
        a&b\\ 
        0&c
    \end{pmatrix}\begin{pmatrix}
        0&0\\ 
        0&1
    \end{pmatrix}&=\begin{pmatrix}
        0&0\\ 
        0&c
    \end{pmatrix}
\end{align*}
So we can see that \(e_1Re_1\cong e_2Re_2\cong D\). Moreover, we know by Exercise 14.4.16 that \(R\) is of finite length, and \(D\) is a local ring, so by Proposition 14.5.11 \(Re_1\) 
and \(Re_2\) are both indecomposable, and by Corollary 14.5.5 \(e_1\) and \(e_2\) are primitive idempotents.
\item By Exercise 14.1.23, this is the same as classifying the maximal left ideals of \(R\) since every irreducible \(R\)-module is isomorphic to \(R/I\) for some maximal left ideal \(I\) in \(R\). Suppose \(I\subset R\) is a left ideal. 
\begin{claim}
For \(1\leq i,j\leq 2\), let \(I_{ij}\) be the set of all \((i,j)\)-entries in \(I\). \(I_{ij}\subset D\) is a left ideal in \(D\).
\end{claim}
\begin{claimproof}
Let \(a,r\in D\), we have 
\begin{align*}
\begin{pmatrix}
    r&*\\ 
    0&*
\end{pmatrix}\begin{pmatrix}
    a&*\\ 
    0&*
\end{pmatrix}&=\begin{pmatrix}
    ra&*\\ 
    0&*
\end{pmatrix},\\[5pt]
\begin{pmatrix}
    r&0\\ 
    0&*
\end{pmatrix}\begin{pmatrix}
    *&a\\ 
    0&*
\end{pmatrix}&=\begin{pmatrix}
    *&ra\\ 
    0&*
\end{pmatrix},\\[5pt]
\begin{pmatrix}
    *&*\\ 
    0&r
\end{pmatrix}\begin{pmatrix}
    *&*\\ 
    0&a
\end{pmatrix}&=\begin{pmatrix}
    *&*\\ 
    0&ra
\end{pmatrix}
\end{align*}
From this we can see that \(I_{ij}\) has to be a left ideal in \(D\).
\end{claimproof}

\(D\) being a division ring implies the only left ideals are \((0)\) and \(D\). It is easy to check that \(\left\{ \begin{pmatrix}
*&0\\ 
0&*
\end{pmatrix} \right\}\) is not a left ideal in \(R\). So \(R\) only has two maximal left ideal \(I=\left\{ \begin{pmatrix}
    *&*\\ 
    0&0
\end{pmatrix} \right\}\) and \(J=\left\{ \begin{pmatrix}
    0&*\\ 
    0&*
\end{pmatrix} \right\}\). Note that \(R/I\cong Re_2\) and \(R/J\cong Re_1\). From (b), we know that \(R/I\) is not isomorphic to \(R/J\). Thus, we can 
conclude that we have two nonisomorphic classes of irreducible \(R\)-modules, given by \(Re_1\) and \(Re_2\).
\end{enumerate}
\end{solution}

\end{document}
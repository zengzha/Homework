\documentclass[a4paper, 12pt]{article}

\usepackage{/Users/zhengz/Desktop/Math/Workspace/Homework1/homework}
%%%%%%%%%%%%%%%%%%%%%%%%%%%%%%%%%%%%%%%%%%%%%%%%%%%%%%%%%%%%%%%%%%%%%%%%%%%%%%%%%%%%%%%%%%%%%%%%%%%%%%%%%%%%%%%%%%%%%%%%%%%%%%%%%%%%%%%%
\begin{document}
%Header-Make sure you update this information!!!!
\noindent
%%%%%%%%%%%%%%%%%%%%%%%%%%%%%%%%%%%%%%%%%%%%%%%%%%%%%%%%%%%%%%%%%%%%%%%%%%%%%%%%%%%%%%%%%%%%%%%%%%%%%%%%%%%%%%%%%%%%%%%%%%%%%%%%%%%%%%%%
\large\textbf{Zhengdong Zhang} \hfill \textbf{Homework - Week 4}   \\
Email: zhengz@uoregon.edu \hfill ID: 952091294 \\
\normalsize Course: MATH 648 - Abstract Algebra  \hfill Term: Winter 2025\\
Instructor: Professor Arkady Berenstein \hfill Due Date: $5^{th}$ February, 2025 \\
\noindent\rule{7in}{2.8pt}
\setstretch{1.1}
%%%%%%%%%%%%%%%%%%%%%%%%%%%%%%%%%%%%%%%%%%%%%%%%%%%%%%%%%%%%%%%%%%%%%%%%%%%%%%%%%%%%%%%%%%%%%%%%%%%%%%%%%%%%%%%%%%%%%%%%%%%%%%%%%%%%%%%%
% Exercise 16.1.2
%%%%%%%%%%%%%%%%%%%%%%%%%%%%%%%%%%%%%%%%%%%%%%%%%%%%%%%%%%%%%%%%%%%%%%%%%%%%%%%%%%%%%%%%%%%%%%%%%%%%%%%%%%%%%%%%%%%%%%%%%%%%%%%%%%%%%%%%
\begin{problem}{16.1.2}
True or false? If \(V\) is an \(\mathbb{R}\)-module and \(\End_R(V)\) is a division ring, then \(V\) is irreducible.
\end{problem}
\begin{solution}
This is false. Consider the ring \(\mathbb{Q}\) of rational numbers, viewed as a \(\mathbb{Z}\)-module. 
\begin{claim}
\(\End_{\mathbb{Z}}\mathbb{Q}=\mathbb{Q}\).
\end{claim}
\begin{claimproof}
Given \(\frac{p}{q}\in \mathbb{Q}\), we could define a endomorphism \(\mathbb{Q}\rightarrow \mathbb{Q}\) by sending any rational number \(\frac{t}{s}\in \mathbb{Q}\) to 
\(\frac{pt}{qs}\in \mathbb{Q}\). This is a \(\mathbb{Z}\)-module homomorphism since multiplication in \(\mathbb{Q}\) is commutative. So we have \(\mathbb{Q}\subset \End_{\mathbb{Z}}\mathbb{Q}\). 
On the other hand, given \(\phi\in \End_{\mathbb{Z}}\mathbb{Q}\), for any \(m\in \mathbb{Z}\) and \(n\in \mathbb{Z}-0\), the commutativity of multiplication in \(\mathbb{Q}\) tells us that 
\[\phi(\frac{m}{n})=m\phi(\frac{1}{n})=\frac{m}{n}\cdot n\phi(\frac{1}{n})=\frac{m}{n}\phi(1).\]
This shows that \(\phi\) is completely determined by the image \(\phi(1)\in \mathbb{Q}\), so we have \(\End_{\mathbb{Z}}\mathbb{Q}\subset \mathbb{Q}\). Now we have \(\End_{\mathbb{Z}}\mathbb{Q}=\mathbb{Q}\).
\end{claimproof}

We know that \(\End_{\mathbb{Z}}\mathbb{Q}\cong \mathbb{Q}\) is a division ring, but \(\mathbb{Q}\) is not simple, which has a proper submodule \(\mathbb{Z}\subset \mathbb{Q}\).
\end{solution}

\noindent\rule{7in}{2.8pt}
%%%%%%%%%%%%%%%%%%%%%%%%%%%%%%%%%%%%%%%%%%%%%%%%%%%%%%%%%%%%%%%%%%%%%%%%%%%%%%%%%%%%%%%%%%%%%%%%%%%%%%%%%%%%%%%%%%%%%%%%%%%%%%%%%%%%%%%%
% Exercise 16.1.6
%%%%%%%%%%%%%%%%%%%%%%%%%%%%%%%%%%%%%%%%%%%%%%%%%%%%%%%%%%%%%%%%%%%%%%%%%%%%%%%%%%%%%%%%%%%%%%%%%%%%%%%%%%%%%%%%%%%%%%%%%%%%%%%%%%%%%%%%
\begin{problem}{16.1.6}
True or false? If \(A\) is a commutative algebra over an algebraically closed field then all irreducible \(A\)-modules are 1-dimensional.
\end{problem}
\begin{solution}
This is false. Consider the \(\mathbb{C}\)-algebra \(\mathbb{C}(x)\). Let \(A=\mathbb{C}(x)\) and view \(A\) as a regular left \(A\)-module. \(A\) is a field so it is simple since 
the only submodules are \(0\) and \(A\) itself. But \(A\) is a infinite dimensional \(\mathbb{C}\)-vector space with a basis \(\left\{ 1,x,x^{-1},x^2,x^{-2},\ldots \right\}\).
\end{solution}

\noindent\rule{7in}{2.8pt}
%%%%%%%%%%%%%%%%%%%%%%%%%%%%%%%%%%%%%%%%%%%%%%%%%%%%%%%%%%%%%%%%%%%%%%%%%%%%%%%%%%%%%%%%%%%%%%%%%%%%%%%%%%%%%%%%%%%%%%%%%%%%%%%%%%%%%%%%
% Exercise 16.1.7
%%%%%%%%%%%%%%%%%%%%%%%%%%%%%%%%%%%%%%%%%%%%%%%%%%%%%%%%%%%%%%%%%%%%%%%%%%%%%%%%%%%%%%%%%%%%%%%%%%%%%%%%%%%%%%%%%%%%%%%%%%%%%%%%%%%%%%%%
\begin{problem}{16.1.7}
True or false? If \(A\) is a finite dimensional commutative algebra over a field, then all irreducible \(A\)-modules are 1-dimensional.
\end{problem}
\begin{solution}
This is false. Consider \(A=\mathbb{C}\) as an \(\mathbb{R}\)-algebra. \(\mathbb{C}\) can be viewed as a left regular \(\mathbb{C}\)-module. \(\mathbb{C}\) as a field is simple, but it is a 
2-dimensional \(\mathbb{R}\)-vector space.
\end{solution}

\noindent\rule{7in}{2.8pt}
%%%%%%%%%%%%%%%%%%%%%%%%%%%%%%%%%%%%%%%%%%%%%%%%%%%%%%%%%%%%%%%%%%%%%%%%%%%%%%%%%%%%%%%%%%%%%%%%%%%%%%%%%%%%%%%%%%%%%%%%%%%%%%%%%%%%%%%%
% Exercise 16.1.10
%%%%%%%%%%%%%%%%%%%%%%%%%%%%%%%%%%%%%%%%%%%%%%%%%%%%%%%%%%%%%%%%%%%%%%%%%%%%%%%%%%%%%%%%%%%%%%%%%%%%%%%%%%%%%%%%%%%%%%%%%%%%%%%%%%%%%%%%
\begin{problem}{16.1.10}
If \(D\) is a division ring, then \(M_n(D)\) is a simple ring.
\end{problem}
\begin{solution}
We prove this by induction on \(n\). When \(n=1\), \(M_1(D)\cong D\) as a division ring is simple. When \(n\geq 2\), suppose we have proved \(M_{n-1}(D)\) is a simple ring. Given \(A\in M_n(D)\) is an \(n\times n\) 
matrix with entries in \(D\), we are going to show that the two-sided ideal generated by \(A\) must be the whole ring \(M_n(D)\) or the zero ideal \((0)\). If every entry in \(A\) is zero, then the ideal \((A)\) must be the zero ideal. Suppose there is an entry in \(A\) which is 
not zero. We know we can switch rows and columns in \(A\) by multiplying the elementary matrices on the left or on the right. So we may assume the \((1,1)\)th entry \(a_{11}\) in \(A\) is not zero. Since \(D\) is a division ring, by multiply 
\(\begin{pmatrix}
    a_{11}^{-1}& & & \\
    & 1 & & \\ 
    & & \ddots & \\ 
    & & & 1
\end{pmatrix}\) on the left we can make \(a_{11}=1\). Next, for \(2\leq m\leq n\), we can multiply the first row with \(-a_{m1}\) then add it to the \(m\)th row. This is elementary operations and can be done via multiplying elementary matrices on the left. This makes all 
\(a_{m1}=0\). Do the same for all \(a_{1m}\), and in this case it is just multiplying elementary matrices on the right. Now \(A\) has the form 
\[\begin{pmatrix}
    1 & 0 &\cdots & 0\\ 
    0 &   &       &   \\ 
    \vdots & & B &  \\ 
    0 & & &
\end{pmatrix}\] 
where \(B\in M_{n-1}(D)\). If \(B\) has no nonzero entries, then \(A\) can be viewed as in \(M_{n-1}(D)\) with only one nonzero entry at upper left corner, we have proved this case by assumption on \(n\). Similarly, if \(B\) has at least one nonzero entry, then 
by the assumption there exists \(B_1,B_2\in M_{n-1}(D)\) such that  
\[\begin{pmatrix}
    1 & 0 &\cdots & 0\\ 
    0 &   &       &   \\ 
    \vdots & & B_1 &  \\ 
    0 & & &
\end{pmatrix}\begin{pmatrix}
    1 & 0 &\cdots & 0\\ 
    0 &   &       &   \\ 
    \vdots & & B &  \\ 
    0 & & &
\end{pmatrix}\begin{pmatrix}
    1 & 0 &\cdots & 0\\ 
    0 &   &       &   \\ 
    \vdots & & B_2 &  \\ 
    0 & & &
\end{pmatrix}=\begin{pmatrix}
    1 & 0 &\cdots & 0\\ 
    0 &   &       &   \\ 
    \vdots & & I_{n-1} &  \\ 
    0 & & &
\end{pmatrix}=I_n\]
This shows that the two sided ideal generated by \(A\) must contain \(I_n\), which just means \((A)=M_n(D)\). So \(M_n(D)\) is a simple ring.  
\end{solution}

\noindent\rule{7in}{2.8pt}
%%%%%%%%%%%%%%%%%%%%%%%%%%%%%%%%%%%%%%%%%%%%%%%%%%%%%%%%%%%%%%%%%%%%%%%%%%%%%%%%%%%%%%%%%%%%%%%%%%%%%%%%%%%%%%%%%%%%%%%%%%%%%%%%%%%%%%%%
% Exercise 16.1.11
%%%%%%%%%%%%%%%%%%%%%%%%%%%%%%%%%%%%%%%%%%%%%%%%%%%%%%%%%%%%%%%%%%%%%%%%%%%%%%%%%%%%%%%%%%%%%%%%%%%%%%%%%%%%%%%%%%%%%%%%%%%%%%%%%%%%%%%%
\begin{problem}{16.1.11}
True or false? If \(V\) is a vector space over a field \(\mathbb{F}\), then \(\End_{\mathbb{F}}(V)\) is a simple ring.
\end{problem}
\begin{solution}
This is false. Consider a \(\mathbb{C}\)-vector space \(V\) with a countable ordered basis \(\left\{ v_1,\ldots,v_n,\ldots \right\}\). Let \(S=\oplus_{n=1}^{\infty} M_n(\mathbb{C})\). Each element in 
\(M_n(\mathbb{C})\) can be viewed as a linear transformation on the first \(n\) base vectors and sending the rest to \(0\). \(S\subset \End_{\mathbb{C}}V\) is a two sided ideal since every matrix in \(S\) has only finite rank. Consider \(f:V\rightarrow V\) sends 
\(v_i\) to \(v_{i+1}\) for all \(1\leq i\). \(f\in \End_{\mathbb{C}}V\) but \(f\notin S\) since \(f\) operates on infinitely many base vectors. This proves that \(S\) is a proper two sided ideal in \(\End_{\mathbb{C}}V\), so \(\End_{\mathbb{C}}V\) is not a 
simple ring. 
\end{solution}

\noindent\rule{7in}{2.8pt}
%%%%%%%%%%%%%%%%%%%%%%%%%%%%%%%%%%%%%%%%%%%%%%%%%%%%%%%%%%%%%%%%%%%%%%%%%%%%%%%%%%%%%%%%%%%%%%%%%%%%%%%%%%%%%%%%%%%%%%%%%%%%%%%%%%%%%%%%
% Exercise 16.2.2
%%%%%%%%%%%%%%%%%%%%%%%%%%%%%%%%%%%%%%%%%%%%%%%%%%%%%%%%%%%%%%%%%%%%%%%%%%%%%%%%%%%%%%%%%%%%%%%%%%%%%%%%%%%%%%%%%%%%%%%%%%%%%%%%%%%%%%%%
\newpage
\begin{problem}{16.2.2}
Let \(R\) be a ring. Then \(R\) is left semisimple if and only if every left ideal of \(R\) is generated by an idempotent.
\end{problem}
\begin{solution}
Suppose \(R\) is left semisimple. By Lemma 16.2.1, \(R\) as a left regular \(R\)-module is completely reducible. So for any left ideal \(I\subset R\), there exists a left ideal \(J\subset R\) such that 
\(I\oplus J=R\). This means there exists \(a\in I\) such that \(1-a\in J\). \(I\cap J=0\) implies that \(ra\in I\) for any \(r\in R\). Also, we have 
\[a=a\cdot 1=a(a+(1-a))=a^2+a(1-a).\]
Note that \(a(1-a)\in J\) and both \(a\) and \(a^2\) are in \(I\), so \(a(1-a)=0\). This proves \(a=a^2\) is an idempotent. Moreover, for any \(x\in I\), we have 
\[x=x\cdot 1=xa^2=xa.\]
This proves the left ideal \(I\) is generated by an idempotent \(a\). 

Conversely, suppose every left ideal in \(R\) is generated by an idempotent. Given a left ideal \(I\subset R\), we know that \(I\) is generated by an idempotent \(a\). We know that 
\(a\) and \(1-a\) are orthogonal idempotents and \(a+1-a=1\), by Lemma 14.5.1, \(R=Ra\oplus R(1-a)=I\oplus R(1-a)\). This proves that \(R\) as a left regular \(R\)-module is completely reducible, so \(R\) is a left semisimple ring.  
\end{solution}

\noindent\rule{7in}{2.8pt}
%%%%%%%%%%%%%%%%%%%%%%%%%%%%%%%%%%%%%%%%%%%%%%%%%%%%%%%%%%%%%%%%%%%%%%%%%%%%%%%%%%%%%%%%%%%%%%%%%%%%%%%%%%%%%%%%%%%%%%%%%%%%%%%%%%%%%%%%
% Exercise 16.2.4
%%%%%%%%%%%%%%%%%%%%%%%%%%%%%%%%%%%%%%%%%%%%%%%%%%%%%%%%%%%%%%%%%%%%%%%%%%%%%%%%%%%%%%%%%%%%%%%%%%%%%%%%%%%%%%%%%%%%%%%%%%%%%%%%%%%%%%%%
\begin{problem}{16.2.4}
Find a group \(G\) for which not every finite dimensional \(\mathbb{C}G\)-module is completely reducible.
\end{problem}
\begin{solution}
Consider 
\[G=\left\{ \begin{pmatrix}
1&a\\ 
0&1
\end{pmatrix}\mid a\in \mathbb{C} \right\}\]
and a \(\mathbb{C}G\)-module 
\[V=\left\{ \begin{pmatrix}
m\\ 
n
\end{pmatrix} \right\}\mid m,n\in \mathbb{C}.\]
\(V\) is left \(\mathbb{C}G\)-module and a \(2\)-dimensional \(\mathbb{C}\)-vector space. Consider the submodule \(W\subset V\) where 
\[W=\left\{ \begin{pmatrix}
m\\ 
0
\end{pmatrix}\mid m\in \mathbb{C} \right\}.\]
\(W\) is indeed a submodule of \(V\) since for any \(c\in \mathbb{C}\), we have 
\[c\begin{pmatrix}
    1&a\\ 
    0&1
\end{pmatrix}\begin{pmatrix}
    m\\ 
    0
\end{pmatrix}=\begin{pmatrix}
    cm\\ 
    0
\end{pmatrix}\in W.\]
Suppose \(V\) is completely reducible, then there exists a submodule \(W'\subset V\) such that \(V=W\oplus W'\). Since \(W\cap W'=0\), so \(\begin{pmatrix}
    1\\ 
    -1
\end{pmatrix}\in V\) and \(\begin{pmatrix}
    1\\ 
    1
\end{pmatrix}\in V\) are both in \(W'\), but we have 
\[\begin{pmatrix}
    1\\ 
    -1
\end{pmatrix}+\begin{pmatrix}
    1\\ 
    1
\end{pmatrix}=\begin{pmatrix}
    2\\
    0
\end{pmatrix}\in W.\]
A contradiction. So \(V\) is not completely reducible.
\end{solution}

\noindent\rule{7in}{2.8pt}
%%%%%%%%%%%%%%%%%%%%%%%%%%%%%%%%%%%%%%%%%%%%%%%%%%%%%%%%%%%%%%%%%%%%%%%%%%%%%%%%%%%%%%%%%%%%%%%%%%%%%%%%%%%%%%%%%%%%%%%%%%%%%%%%%%%%%%%%
% Exercise 16.2.17
%%%%%%%%%%%%%%%%%%%%%%%%%%%%%%%%%%%%%%%%%%%%%%%%%%%%%%%%%%%%%%%%%%%%%%%%%%%%%%%%%%%%%%%%%%%%%%%%%%%%%%%%%%%%%%%%%%%%%%%%%%%%%%%%%%%%%%%%
\begin{problem}{16.2.17}
True or false? If \(R\) is a ring with no non-trivial left ideals, then it also has no non-trivial right ideals.
\end{problem}
\begin{solution}
This is true. We prove \(R\) is a division ring, so that \(R\) has no non trivial right ideal. First we prove that \(R\) has no zero divisors. Suppose 
\(ab=0\) and \(0\neq a\in R, 0\neq b\in R\). \(R\) having no left ideals implies there exists \(r\in R\) such that \(ra=1\). So we have 
\[b=(ra)b=r(ab)=r\cdot 0=0.\]
This contradicts the assumption that \(b\neq 0\). So \(R\) has no zero divisors. Now for any \(x\in R\), there exists \(y\in R\) such that \(yx=1\) since \(R\) has no non trivial left 
ideals. So \(y\) is the left inverse of \(x\) in \(R\). Note that we have
\[y=(yx)y=y(xy).\]
This implies that \(y(xy-1)=0\). Since \(R\) has no zero divisors, so \(xy=1\). This shows that \(y\) is also the right inverse of \(x\). Thus, we proved that \(R\) is a division ring.  
\end{solution}

\noindent\rule{7in}{2.8pt}
%%%%%%%%%%%%%%%%%%%%%%%%%%%%%%%%%%%%%%%%%%%%%%%%%%%%%%%%%%%%%%%%%%%%%%%%%%%%%%%%%%%%%%%%%%%%%%%%%%%%%%%%%%%%%%%%%%%%%%%%%%%%%%%%%%%%%%%%
% Exercise 16.2.19
%%%%%%%%%%%%%%%%%%%%%%%%%%%%%%%%%%%%%%%%%%%%%%%%%%%%%%%%%%%%%%%%%%%%%%%%%%%%%%%%%%%%%%%%%%%%%%%%%%%%%%%%%%%%%%%%%%%%%%%%%%%%%%%%%%%%%%%%
\begin{problem}{16.2.19}
True or false? If \(A\) and \(B\) are semisimple complex algebras of dimension \(3\), then \(A\cong B\).
\end{problem}
\begin{solution}
This is true. By Theorem 16.2.18, since \(\mathbb{C}\) is algebraically closed, \(A\) and \(B\) are isomorphic to 
\[M_{n_1}(\mathbb{C})\times M_{n_m}(\mathbb{C}).\]
Note that \(M_2(\mathbb{C})\) has complex dimension \(4\). So \(n_1=\cdots=n_m=1\). This implies \(A\cong B\cong \mathbb{C}\times \mathbb{C}\times \mathbb{C}\cong \mathbb{C}^3\).
\end{solution}

\noindent\rule{7in}{2.8pt}
%%%%%%%%%%%%%%%%%%%%%%%%%%%%%%%%%%%%%%%%%%%%%%%%%%%%%%%%%%%%%%%%%%%%%%%%%%%%%%%%%%%%%%%%%%%%%%%%%%%%%%%%%%%%%%%%%%%%%%%%%%%%%%%%%%%%%%%%
% Exercise 16.2.21
%%%%%%%%%%%%%%%%%%%%%%%%%%%%%%%%%%%%%%%%%%%%%%%%%%%%%%%%%%%%%%%%%%%%%%%%%%%%%%%%%%%%%%%%%%%%%%%%%%%%%%%%%%%%%%%%%%%%%%%%%%%%%%%%%%%%%%%%
\begin{problem}{16.2.21}
Let \(C_n\) be the cyclic group of order \(n\) and let \(\mathbb{F}C_n\) denote its group algebra over a field \(\mathbb{F}\).
\begin{enumerate}[(1)]
\item Prove that \(\mathbb{F}C_n\cong \mathbb{F}[x]/(x^n-1)\). 
\item How many isomorphism classes of irreducible \(\mathbb{C}C_n\)-modules are there? What are their dimensions? 
\item Decompose \(\mathbb{C}C_n\) explicitly as a direct sum of simple algebras. 
\item How many isomorphism classes of irreducible \(\mathbb{Q}C_n\)-modules are there up to isomorphism? What are their dimensions? 
\item Describe the Wedderburn-Artin decomposition of \(\mathbb{Q}C_n\) up to isomorphism.
\end{enumerate}
\end{problem}
\begin{solution}
\begin{enumerate}[(1)]
\item Let \(c\in C_n\) be the generator of \(C_n\). Consider the following map 
\begin{align*}
    f:\mathbb{F}[x]&\rightarrow \mathbb{F}C_n,\\ 
    x&\mapsto c.
\end{align*}
This is a well-defined map of algebras since the group algebra \(\mathbb{F}C_n\) is defined \(\mathbb{F}\)-linearly and the group operation in \(C_n\) is just multiplication by the power of \(c\). \(f\) is also surjective since 
elements in \(\mathbb{F}C_n\) is just \(\mathbb{F}\)-linear combination of powers of \(c\). Consider the ideal \(I=(x^n-1)\subset \mathbb{F}[x]\). Every element \(p\in I\) can be written as 
\(p(x)=(x^n-1)g(x)\) and we have \(f(p)=(c^n-1)g(c)=0\). So \(I\subset \ker f\). Note that \(\ker f\subset \mathbb{F}\) is an ideal in a PID \(\mathbb{F}[x]\), so \(\ker f\) must be generated by one nonzero polynomial \(p\). Since \(I\subset \ker f\), we know that 
\(p|(x^n-1)\). Suppose \(\deg p<n\). We know \(p(c)\neq 0\) in \(\mathbb{F}C_n\), so \(\deg p=n\). This implies \(\ker f=(x^n-1)\). By the first isomorphism theorem, we have 
\[\mathbb{F}[x]/(x^n-1)\cong \mathbb{F}C_n.\]
\item Let \(R=\mathbb{C}C_n\). Note that \(R\) is commutative. By Exercise 14.1.25, the set of isomorphism classes of simple \(R\)-modules is bijective to the set of maximal ideals of \(R\). So we only need to classify maximal ideals in \(R\). By Wedderburn-Artin Theorem for Algebras, 
\(R=\mathbb{C}C_n\cong \mathbb{C}^{\oplus n}\) since \(R\) is commutative and has a basis \(\left\{ e,c,c^2,\ldots,c^{n-1} \right\}\) where \(c\in C_n\) is the generator and \(e\) is the identity element in the group \(C_n\). The maximal ideal in \(\mathbb{C}^n\) is isomorphic to \(\mathbb{C}^{n-1}\), so we 
have \(n\) maximal ideals in \(R\) and there are \(n\) isomorphisms classes of simple \(R\)-modules. Every one of them is isomorphic to \(\mathbb{C}^n/\mathbb{C}^{n-1}\cong \mathbb{C}\) and is \(1\)-dimensional. 
\item Let \(w=e^{\frac{2\pi i}{n}}\). For \(0\leq r\leq n-1\), define 
\[e_r=\sum_{j=0}^{n-1} w^{rj} c^j.\]
Note that \(\frac{1}{n}e_r\) is an idempotent since 
\begin{align*}
    e_r^2&=(1+w^r c+w^{2r} c^2+\cdots+w^{(n-1)r}c^{n-1})^2\\ 
         &=\begin{aligned}[t]
            &1+w^r c+w^{2r} c^2+\cdots+w^{(n-1)r}c^{n-1}\\ 
            &+w^r c+w^{2r} c^2+\cdots +w^{(n-1)r}c^{n-1}+1\\ 
            &+\cdots\\ 
            &+w^{(n-1)r}c^{n-1}+1+w^r c^+\cdots+w^{(n-2)r}c^{n-2}
         \end{aligned}\\
         &=n(1+w^r c+\cdots+w^{(n-1)r}c^{n-1})\\ 
         &=ne_r
\end{align*}
Next, we prove that \(e_0,e_1,\ldots,e_{n-1}\) are orthogonal. For \(0\leq r\neq s\leq n-1\), we have
\[e_r=\sum_{j=0}^{n-1}w^{rj}c^j=\sum_{j=0}^{n-1}w^{r(i+j)}c^{i+j}\] 
for any \(i\) because both \(w\) and \(c\) are \(n\)th root of \(1\). And this only gives a permutation on every summand. 
\begin{align*}
    e_re_s=&(\sum_{j=0}^{n-1}w^{rj}c^j)(\sum_{i=0}^{n-1}w^{si}c^i)\\
          =&\sum_{i,j=0}^{n-1}w^{rj+si}c^{j+i}\\ 
          =&\sum_{i,j=0}^{n-1}w^{r(j+i)-ri+si}c^{j+i}\\ 
          =&\sum_{i,j=0}^{n-1}w^{(s-r)i}w^{rj}c^j\\ 
          =&(\sum_{i=0}^{n-1}w^{(s-r)i})e_r.
\end{align*}
Here \(r-s\neq 0\). If \(r-s\) is coprime with \(n\), then 
\[\sum_{i=0}^{n-1}w^{(r-s)i}=1+w+w^2+\cdots+w^{n-1}=0.\]
If \(1<d=\text{gcd}(|r-s|,n)\) where \(d|n\), then 
\[\sum_{i=0}^{n-1}w^{(r-s)i}=1+w^d+w^{2d}+\cdots+w^{(n-1)d}=0.\]
This proves that \(e_0,e_1,\ldots,e_{n-1}\) are orthogonal. Note that \(e_0+e_1+\cdots +e_{n-1}=n(1)\). By Lemma 14.5.1, we have a decomposition 
\[\mathbb{C}C_n\cong \mathbb{C}C_n e_0\oplus \cdots \mathbb{C}C_ne_{n-1}.\]
We check that for any \(0\leq r\leq n-1\), \(\mathbb{C}C_ne_r\) is a simple ideal. Suppose \(\sum_{i=0}^{n-1}a_ic^i\in \mathbb{C}C_n\), we have 
\begin{align*}
    (\sum_{i=0}^{n-1}a_ic^i)(\sum_{j=0}^{n-1}w^{rj}c^j)&= \sum_{i,j=0}^{n-1}a_iw^{rj}c^{i+j}\\ 
                                                       &=\sum_{i,j=0}^{n-1}a_iw^{-ri}w^{r(j+i)}c^{i+j}\\ 
                                                       &=\sum_{i,j}^{n-1}a_iw^{-ri}w^{rj}c^j\\ 
                                                       &=(\sum_{i=0}^{n-1}a_iw^{-ri})e_r.
\end{align*}
This proves \(\mathbb{C}C_ne_r\cong \mathbb{C}e_r\) is a ideal in \(R\). Moreover, since \(\mathbb{C}\) is a field, this shows that \(\mathbb{C}e_r\) is a simple ideal. So we have written \(R=\mathbb{C}C_n\) as a product of simple ideals 
\[\mathbb{C}C_n\cong \mathbb{C}e_0\oplus \cdots\oplus \mathbb{C}e_{n-1}.\] 

\item Let \(R=\mathbb{Q}C_n\). \(R\) is commutative and by Exercise 14.1.25, we only need to classify the isomorphism classes of maximal ideals in \(R\). Note that by (1), \(R\) is isomorphic to 
\(\mathbb{Q}[x]/(x^n-1)\). Note that on the field \(\mathbb{Q}\), \(x^n-1=\prod_{d|n} \Phi_d(x)\) where \(\Phi_d(x)\) is the cyclotomic polynomials corresponding to the \(d\)th primitive root of unity. For any \(d|n\), by Theorem 13.3.2, \(\Phi_d(x)\) is 
irreducible and by Chinese remainder theorem, \(\mathbb{Q}[x]/(x^n-1)\cong \prod_{d|n}\mathbb{Q}[x]/(\Phi_d(x))\). So each \(\mathbb{Q}[x]/\Phi_d(x)\) is simple as a \(R\)-module since \(\Phi_d(x)\) is maximal in \(\mathbb{Q}[x]\). This gives 
an isomorphism class of simple \(R\)-module. For any \(d|n\), the dimension of \(\mathbb{Q}[x]/\Phi_d(x)\) as a \(\mathbb{Q}\)-vector space is just the degree of \(\Phi_d(x)\), which is \(\varphi(d)\), where \(\varphi\) is the Euler's totient function. 
\item Note that \(\mathbb{Q}C_n\) is semisimple by Maschke's theorem.  Since \(R=\mathbb{Q}C_n\) is commutative, by Wedderburn-Artin theorem for algebras, \(R\) is isomorphic to \(D_1\times D_2\times\cdots\times D_m\) where \(D_i\) is a finite extension of \(\mathbb{Q}\). From (4) and by uniqueness of Wedderburn-Artin, we know each \(D_i\) is isomorphic to 
\(\mathbb{Q}[x]/\Phi_d(x)\) for some \(d|n\). Note that the cyclotomic polynomial \(\Phi_d(x)\) is the minimal polynomial for the \(d\)th cyclotomic field, which can be obtained by adjoining a complex primitive \(d\)th root \(\omega_d\) to \(\mathbb{Q}\). So we have 
\[D_d\cong \mathbb{Q}[x]/\Phi_d(x)\cong \mathbb{Q}(\omega_d).\]
We have a decomposition \(\mathbb{Q}C_n\cong \prod_{d|n} \mathbb{Q}(\omega_d)\).
\end{enumerate}
\end{solution}

\end{document}
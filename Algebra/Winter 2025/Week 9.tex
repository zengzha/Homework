\documentclass[a4paper, 12pt]{article}
\usepackage{comment} % enables the use of multi-line comments (\ifx \fi) 
\usepackage{lipsum} %This package just generates Lorem Ipsum filler text. 
\usepackage{fullpage} % changes the margin
\usepackage[a4paper, total={7in, 10in}]{geometry}
\usepackage{amsmath}
\usepackage{amssymb,amsthm}  % assumes amsmath package installed
\newtheorem{theorem}{Theorem}
\newtheorem{corollary}{Corollary}
\usepackage{graphicx}
\usepackage{tikz}
\usepackage{leftindex}
\usepackage{multicol}
\usepackage{quiver}
\usetikzlibrary{arrows}
\usepackage{verbatim}
\usepackage{setspace}
\usepackage{comment}
\usepackage{float}
\usepackage{tikz-cd}
\usepackage[backend=biber,bibencoding=utf8,style=numeric,sorting=ynt]{biblatex}

    
\usepackage{xcolor}
\usepackage{mdframed}
\usepackage[shortlabels]{enumitem}
\usepackage{indentfirst}
\usepackage{hyperref}
    
\renewcommand{\thesubsection}{\thesection.\alph{subsection}}

\newenvironment{problem}[2][Exercise]
    { \begin{mdframed}[backgroundcolor=gray!20] \textbf{#1 #2} \\}
    {  \end{mdframed}}

% Define solution environment
\newenvironment{solution}
    {\textit{Solution:}}
    {}

%Define the claim environment
\newenvironment{claim}[1]{\par\noindent\underline{Claim:}\space#1}{}
\newenvironment{claimproof}[1]{\par\noindent\underline{Proof:}\space#1}{\hfill $\blacksquare$}

\renewcommand{\qed}{\quad\qedsymbol}
\newcommand{\la}{\langle}
\newcommand{\ra}{\rangle}
\newcommand{\ord}{\text{ord}\,}
\newcommand{\Ann}{\text{Ann}\,}
\newcommand{\im}{\text{im}\,}
\newcommand{\coker}{\text{coker}\,}
\newcommand{\Com}{\text{Com}}
\newcommand{\End}{\text{End}}
\newcommand{\tr}{\text{tr}}
\newcommand{\iif}{\ \ \text{if}\ \ }
\newcommand{\rank}{\text{rank}\,}
\newcommand{\Rad}{\text{Rad}}
\newcommand{\ind}{\text{ind}}
\newcommand{\coind}{\text{coind}}
\newcommand{\res}{\text{res}}
\newcommand{\li}{\leftindex}
%%%%%%%%%%%%%%%%%%%%%%%%%%%%%%%%%%%%%%%%%%%%%%%%%%%%%%%%%%%%%%%%%%%%%%%%%%%%%%%%%%%%%%%%%%%%%%%%%%%%%%%%%%%%%%%%%%%%%%%%%%%%%%%%%%%%%%%%
\begin{document}
%Header-Make sure you update this information!!!!
\noindent
%%%%%%%%%%%%%%%%%%%%%%%%%%%%%%%%%%%%%%%%%%%%%%%%%%%%%%%%%%%%%%%%%%%%%%%%%%%%%%%%%%%%%%%%%%%%%%%%%%%%%%%%%%%%%%%%%%%%%%%%%%%%%%%%%%%%%%%%
\large\textbf{Zhengdong Zhang} \hfill \textbf{Homework - Week 9}   \\
Email: zhengz@uoregon.edu \hfill ID: 952091294 \\
\normalsize Course: MATH 648 - Abstract Algebra  \hfill Term: Winter 2025\\
Instructor: Professor Arkady Berenstein \hfill Due Date: $12^{th}$ March, 2025 \\
\noindent\rule{7in}{2.8pt}
\setstretch{1.1}
%%%%%%%%%%%%%%%%%%%%%%%%%%%%%%%%%%%%%%%%%%%%%%%%%%%%%%%%%%%%%%%%%%%%%%%%%%%%%%%%%%%%%%%%%%%%%%%%%%%%%%%%%%%%%%%%%%%%%%%%%%%%%%%%%%%%%%%%
% Exercise 24.1.4
%%%%%%%%%%%%%%%%%%%%%%%%%%%%%%%%%%%%%%%%%%%%%%%%%%%%%%%%%%%%%%%%%%%%%%%%%%%%%%%%%%%%%%%%%%%%%%%%%%%%%%%%%%%%%%%%%%%%%%%%%%%%%%%%%%%%%%%%
\begin{problem}{24.1.4}
Let \(V\) and \(W\) be finite dimensional \(\mathbb{F}G\)-modules. Define a \(\mathbb{F}G\)-module structure on \(\hom_\mathbb{F}(V,W)\), so that \(\hom_\mathbb{F}(V,W)\cong V^{*}\otimes W\) as 
\(\mathbb{F}G\)-module and \(\hom_{\mathbb{F}G}(V,W)=\hom_\mathbb{F}(V,W)^G\). 
\end{problem}
\begin{solution}
We define a \(\mathbb{F}G\)-module structure on \(\hom_\mathbb{F}(V,W)\). For any \(g\in G\), \(f\in \hom_\mathbb{F}(V,W)\) and \(v\in V\), we define 
\[(g\cdot f)(v)=g\cdot f(g^{-1}v).\]
This is a well-defined \(\mathbb{F}G\)-module structure. Indeed, for any \(h,g\in G\), we have 
\begin{align*}
	(h\cdot (g\cdot f))(v)&=h\cdot (g\cdot f)(h^{-1}v)\\ 
	                      &=h\cdot g\cdot f(g^{-1}h^{-1}v)\\ 
						  &=(hg)\cdot f((hg)^{-1}v)\\ 
						  &=((hg\cdot f))(v).
\end{align*}
This \(\mathbb{F}G\)-module structure is compatible with the isomorphism \(\hom_\mathbb{F}(V,W)\cong V^*\otimes W\). Indeed, consider the isomorphism 
\begin{align*}
	\phi:V^*\otimes W&\rightarrow \hom_\mathbb{F}(V,W),\\ 
	  (l\otimes w)&\mapsto (v\mapsto l(v)w).
\end{align*}
We check that the \(G\)-action we defined above is compatible. For any \(g\in G\), \(v\in V\), \(l\in V^*\) and \(w\in W\), we have 
\begin{align*}
	\phi(g\cdot (l\otimes w))(v)&=\phi((g\cdot l)\otimes gw)(v)\\ 
	                            &=(g\cdot l)(v)gw\\ 
								&=l(g^{-1}v)gw\\ 
								&=g\cdot l(g^{-1}v)w\\ 
								&=(g\cdot (\phi(l\otimes w)))(v).
\end{align*}
So we know that the isomorphism \(\phi\) is also a \(\mathbb{F}G\)-module isomorphism. 

Consider a map between \(\mathbb{F}G\)-modules 
\begin{align*}
	\hom_{\mathbb{F}G}(V,W)&\rightarrow \hom_\mathbb{F}(V,W),\\ 
	\alpha&\mapsto \alpha.
\end{align*}
We first prove that for any \(\alpha\in \hom_{\mathbb{F}G}(V,W)\), if we view \(\alpha\) as an element in \(\hom_\mathbb{F}(V,W)\), then \(\alpha\) must be invariant under the \(G\)-action we defined above. Indeed, for 
any \(g\in G\) and \(v\in V\), we have 
\[(g\cdot \alpha)(v)=g\alpha(g^{-1}v)=\alpha(gg^{-1}v)=\alpha(v).\]
We have a well-defined \(\mathbb{F}G\)-module homomorphism \(\psi:\hom_{\mathbb{F}G}(V,W)\rightarrow \hom_\mathbb{F}(V,W)^G\). We check \(\psi\) is injective. Suppose 
\(\alpha\in \ker \psi\), note that the zero map is invariant under the \(G\)-action, by definition of \(\psi\), \(\alpha=0\). Next, we check that \(\psi\) is surjective. Given a \(G\)-invariant \(\mathbb{F}\)-linear map \(\beta\), for any \(g\in G\) and \(v\in V\), we have 
\[g\beta(v)=g(g^{-1}\cdot \beta)(v)=gg^{-1}\beta(gv)=\beta(gv).\]
This proves that \(\beta\) is also a homomorphism if we view \(V\) and \(W\) as \(\mathbb{F}G\)-modules. Thus, \(\psi\) is an isomorphism. 
\end{solution}

\noindent\rule{7in}{2.8pt}
%%%%%%%%%%%%%%%%%%%%%%%%%%%%%%%%%%%%%%%%%%%%%%%%%%%%%%%%%%%%%%%%%%%%%%%%%%%%%%%%%%%%%%%%%%%%%%%%%%%%%%%%%%%%%%%%%%%%%%%%%%%%%%%%%%%%%%%%
% Exercise 24.1.7
%%%%%%%%%%%%%%%%%%%%%%%%%%%%%%%%%%%%%%%%%%%%%%%%%%%%%%%%%%%%%%%%%%%%%%%%%%%%%%%%%%%%%%%%%%%%%%%%%%%%%%%%%%%%%%%%%%%%%%%%%%%%%%%%%%%%%%%%
\begin{problem}{24.1.7}
Let \(G\) be a finite group and \(H\leq G\). Then each irreducible \(\mathbb{F}G\)-module is a quotient of a module induced from an irreducible \(\mathbb{F}H\)-module. 
\end{problem}
\begin{solution}
Let \(V\) be an irreducible \(\mathbb{F}G\)-module. We know that \(\res^G_HV\) is a finite dimensional \(\mathbb{F}H\)-module, so it is artinian. Every descending chain must stabilizes. Thus, there exists an irreducible \(\mathbb{F}H\) submodule \(W\subseteq \res^G_HV\). Consider the \(\mathbb{F}H\)-module homomorphism 
\[W\rightarrow \res^G_HV\]
defined by inclusion. By adjointness of \(\res^G_H\) and \(\ind^G_H\), we know there exists an \(\mathbb{F}G\)-module homomorphism 
\[f:\ind^G_HW\rightarrow V\]
which is not the zero homomorphism. So there exists \(v\in V\) such that \(f(a)=v\) for some \(a\in \ind^G_HW\). Note that \(V\) is irreducible, so for any \(v'\in V\), there exists \(\sum k_g g\) such that \(\sum k_g gv=v'\). Therefore, we have 
\[f(\sum k_g ga)=\sum k_g gf(a)=\sum k_g gv=v'.\]
This proves that \(f\) is surjective. Thus, we can write 
\[V\cong \ind^G_HW/\ker f\]
where \(W\) is an irreducible \(\mathbb{F}H\)-module.
\end{solution}

\noindent\rule{7in}{2.8pt}
%%%%%%%%%%%%%%%%%%%%%%%%%%%%%%%%%%%%%%%%%%%%%%%%%%%%%%%%%%%%%%%%%%%%%%%%%%%%%%%%%%%%%%%%%%%%%%%%%%%%%%%%%%%%%%%%%%%%%%%%%%%%%%%%%%%%%%%%
% Exercise 24.1.10
%%%%%%%%%%%%%%%%%%%%%%%%%%%%%%%%%%%%%%%%%%%%%%%%%%%%%%%%%%%%%%%%%%%%%%%%%%%%%%%%%%%%%%%%%%%%%%%%%%%%%%%%%%%%%%%%%%%%%%%%%%%%%%%%%%%%%%%%
\begin{problem}{24.1.10}
If \(G\) acts transitively on a set \(X\) with a point stabilizer \(H\), then the permutation module \(\mathbb{F}X\) is isomorphic to the induced module \(\ind^G_H 1_H\).
\end{problem}
\begin{solution}
Let \(H=\text{Stab}(x)\) be the stabilizer for the point \(x\in X\). For \(\ind^G_H1_H=\mathbb{F}G\otimes_{\mathbb{F}H}1_H\), we define a map 
\begin{align*}
	\phi:\ind^G_H1_H&\rightarrow \mathbb{F}X,\\ 
	     g\otimes 1&\mapsto g\cdot x.
\end{align*}
For any \(h\in G\), we have 
\[h\phi(g\otimes 1)=h(g\cdot x)=(hg)\cdot x=\phi(hg\otimes 1).\]
This is a well-defined \(\mathbb{F}G\)-module homomorphism. Moreover, since the \(G\)-action on \(X\) is transive, we know that \(\phi\) is surjective. By theorem 7.1.7, we know that \(X\cong G/H\) is an isomorphism of \(G\)-sets. So we have \(\dim_\mathbb{F}\mathbb{F}X=|G/H|\). Let \(g_1,\ldots,g_k\) be coset 
representatives for \(G/H\). By Lemma 24.1.9, \(\left\{ g_1\otimes 1,\ldots,g_k\otimes 1 \right\}\) is a basis for \(\ind^G_H1_H\). This tells us 
\[\dim_\mathbb{F}\ind^G_H1_H=\dim_\mathbb{F}\mathbb{F}X=|G/H|.\]
So \(\phi\) is an isomorphism by dimension reasons. We have proved \(\ind^G_H1_H\) is isomorphic to \(\mathbb{F}X\) as \(\mathbb{F}G\)-modules.
\end{solution}

\noindent\rule{7in}{2.8pt}
%%%%%%%%%%%%%%%%%%%%%%%%%%%%%%%%%%%%%%%%%%%%%%%%%%%%%%%%%%%%%%%%%%%%%%%%%%%%%%%%%%%%%%%%%%%%%%%%%%%%%%%%%%%%%%%%%%%%%%%%%%%%%%%%%%%%%%%%
% Exercise 24.2.4
%%%%%%%%%%%%%%%%%%%%%%%%%%%%%%%%%%%%%%%%%%%%%%%%%%%%%%%%%%%%%%%%%%%%%%%%%%%%%%%%%%%%%%%%%%%%%%%%%%%%%%%%%%%%%%%%%%%%%%%%%%%%%%%%%%%%%%%%
\begin{problem}{24.2.4}
Show that the map \(g\mapsto \begin{pmatrix}
	0&1\\ 
	-1&-1
\end{pmatrix}\) defines a representation of the cyclic group \(C_3=\la g\ra\). Prove that this representation is irreducible over the field of real numbers.
\end{problem}
\begin{solution}
Let \(A=\begin{pmatrix}
	0&1\\ 
	-1&-1
\end{pmatrix}\). To check that \(\rho:g\rightarrow A\) is a representation of \(C_3\), we need to check that 
\[I_3=\rho(1)=\rho(g^3)=\rho(g)^3=A^3.\]
And indeed we have
\[\begin{pmatrix}
	0&1\\ 
	-1&-1
\end{pmatrix}\begin{pmatrix}
	0&1\\ 
	-1&-1
\end{pmatrix}\begin{pmatrix}
	0&1\\ 
	-1&-1
\end{pmatrix}=\begin{pmatrix}
	1&0\\ 
	0&1
\end{pmatrix}.\]
To show that \(\rho\) is an irreducible \(\mathbb{R}C_3\) representation, suppose the corresponding \(\mathbb{C}C_3\)-module \(V\) is generated by \(v_1,v_2\) as \(\mathbb{C}\)-vector space. If \(V\) 
has a nontrivial submodule, then it must be one dimensional and generated by \(av_1+bv_2\) where \(a,b\in \mathbb{C}\). Since it is a submodule of \(V\), we have 
\[A\begin{pmatrix}
	a\\ 
	b
\end{pmatrix}\in \la av_1+bv_2\ra.\]
So 
\[A\begin{pmatrix}
	a\\ 
	b
\end{pmatrix}=\begin{pmatrix}
	ka\\ 
	kb
\end{pmatrix}\]
for some \(k\in \mathbb{R}\). \(k\) must be an eigenvalue of \(A\) but \(A\) only have two complex eigenvalues. This is a contradiction and thus, \(V\) is irreducible.
\end{solution}

\noindent\rule{7in}{2.8pt}
%%%%%%%%%%%%%%%%%%%%%%%%%%%%%%%%%%%%%%%%%%%%%%%%%%%%%%%%%%%%%%%%%%%%%%%%%%%%%%%%%%%%%%%%%%%%%%%%%%%%%%%%%%%%%%%%%%%%%%%%%%%%%%%%%%%%%%%%
% Exercise 24.2.6
%%%%%%%%%%%%%%%%%%%%%%%%%%%%%%%%%%%%%%%%%%%%%%%%%%%%%%%%%%%%%%%%%%%%%%%%%%%%%%%%%%%%%%%%%%%%%%%%%%%%%%%%%%%%%%%%%%%%%%%%%%%%%%%%%%%%%%%%
\begin{problem}{24.2.6}
True or false? A non-abelian group of order 55 has exactly five one-dimensional complex representations up to isomorphism.
\end{problem}
\begin{solution}
This is true. Let \(G\) be a non-abelian group of order 55. We know that \(|G|=55=5\times 11\). By Sylow's theory, we know that the Sylow 11-subgroup must be unique and is isomorphic to the cyclic group \(C_{11}\). There are two cases for the Sylow 5-subgroup. Either we have 
a unique Sylow \(5\)-subgroup, or we have \(11\) Sylow \(5\)-subgroup, each of them isomorphic to \(C_5\) and conjugate to each other. Suppose \(G\) has a unique Sylow \(5\)-subgroup \(C_5\) and a unique Sylow \(11\)-subgroup \(C_{11}\). By Proposition 7.5.16(2), Both Sylow subgroups are normal and \(G\) is the 
direct product \(G\cong C_5\times C_{11}\cong C_{55}\), which is abelian. So \(G\) must have \(11\) Sylow \(5\)-subgroup. In this case, \(C_{11}\) is the smallest normal subgroup of \(G\) such that \(G/C_{11}\cong C_5\) is abelian. So the commutator subgroup \(G'=C_{11}\). From Exercise 24.2.5, we know that \(G\) has exactly five 1-dimensional complex representations 
up to isomorphism.
\end{solution}

\noindent\rule{7in}{2.8pt}
%%%%%%%%%%%%%%%%%%%%%%%%%%%%%%%%%%%%%%%%%%%%%%%%%%%%%%%%%%%%%%%%%%%%%%%%%%%%%%%%%%%%%%%%%%%%%%%%%%%%%%%%%%%%%%%%%%%%%%%%%%%%%%%%%%%%%%%%
% Exercise 24.2.8
%%%%%%%%%%%%%%%%%%%%%%%%%%%%%%%%%%%%%%%%%%%%%%%%%%%%%%%%%%%%%%%%%%%%%%%%%%%%%%%%%%%%%%%%%%%%%%%%%%%%%%%%%%%%%%%%%%%%%%%%%%%%%%%%%%%%%%%%
\begin{problem}{24.2.8(Irreducible representation of dihedral groups)}
Let 
\[D_{2n}=\la\la a,b\mid a^n=b^2=1, bab^{-1}=a^{-1}\ra\ra\]
be the dihedral group, \(\varepsilon:=e^{2\pi i/n}\), and set 
\[B:=\begin{pmatrix}
	0&1\\ 
	1&0
\end{pmatrix},\ \ \ \ A_j:=\begin{pmatrix}
	\varepsilon^j&0\\ 
	0&\varepsilon^{-j}
\end{pmatrix}\ \ \ \ (1\leq j<n).\]
\begin{enumerate}[(1)]
\item Show that for \(j=1,\ldots,n-1\) there is a matrix representation \(\rho_j:D_{2n}\rightarrow GL_2(\mathbb{C})\) such that \(\rho_j(a)=A_j\) and \(\rho_j(b)=B\). 
\item Use Schur's Lemma to prove that \(\rho_1,\ldots,\rho_{n-1}\) are irreducible unless \(n\) is even and \(j=n/2\). 
\item Use Schur's Lemma to prove that the representation \(\rho_1, \ldots,\rho_{\lfloor (n-1)/2\rfloor}\) are pairwise non-isomorphic. 
\item If \(n=2k\) is even, then \(D_{2n}\) has four non-isomorphic one-dimensional representations, which together with \(\rho_1,\ldots,\rho_{k-1}\) give a complete and irrdundant list of irreducible \(\mathbb{C}D_{2n}\)-modules up to isomorphism. 
\item If \(n=2k+1\) is odd, then \(D_{2n}\) has two non-isomorphic representations, which together with \(\rho_1,\ldots,\rho_k\) give a complete and irredundant list of irreducible \(\mathbb{C}D_{2n}\)-modules up to isomorphism.
\end{enumerate}
\end{problem}
\begin{solution}
\begin{enumerate}[(1)]
\item We need to check that \(B\) and \(A_j\) satisfy the relations for the dihedral group \(D_{2n}\). We have 
\begin{align*}
B^2&=\begin{pmatrix}
	0&1\\ 
	1&0
\end{pmatrix}\begin{pmatrix}
	0&1\\ 
	1&0
\end{pmatrix}=I_2,\\ 
A^n&=\begin{pmatrix}
	\varepsilon^j&0\\ 
	0&\varepsilon^{-j}
\end{pmatrix}^n=\begin{pmatrix}
	\varepsilon^{nj}&0\\ 
	0&\varepsilon^{-nj}
\end{pmatrix}=I_2.\\ 
BAB^{-1}&=\begin{pmatrix}
	0&1\\ 
	1&0
\end{pmatrix}\begin{pmatrix}
	\varepsilon^j&0\\ 
	0&\varepsilon^{-j}
\end{pmatrix}\begin{pmatrix}
	0&1\\ 
	1&0
\end{pmatrix}=\begin{pmatrix}
	\varepsilon^{-j}&0\\ 
	0&\varepsilon^j
\end{pmatrix}=A^{-1}.
\end{align*}
This proves \(\rho_j\) is a matrix representation for all \(1\leq j\leq n-1\).
\item Let \(V_j\) be the corresponding \(\mathbb{C}D_{2n}\)-module with repesct to the representation \(\rho_j\). Suppose it has an \(\mathbb{F}\)-basis \(v_1,v_2\). If \(V_j\) has a non-trivial 1-dimensional submodule \(W\), then \(W\) must be generated by 
\(av_1+bv_2\) for some \(a,b\in \mathbb{C}\). We know that \(W\) is irreducible because it is 1-dimensional, by Schur's lemma, \(A_j\) acts on \(W\) by a scalar \(\lambda\in \mathbb{C}\). We have 
\[A_j\begin{pmatrix}
	a\\ 
	b
\end{pmatrix}=\begin{pmatrix}
\varepsilon^j&0\\ 
0&\varepsilon^{-j}
\end{pmatrix}\begin{pmatrix}
a\\ 
b
\end{pmatrix}=\begin{pmatrix}
\varepsilon^ja\\ 
\varepsilon^{-j}b
\end{pmatrix}=\begin{pmatrix}
	\lambda a\\
	\lambda b
\end{pmatrix}.\]
This means \(\varepsilon^{2j}=1\), namely, \(e^{2\pi i\cdot \frac{2j}{n}}=1\). We have \(2j|n\). Since \(1\leq j\leq n-1\), we have \(2j=n\). Thus, \(V_j\) is not irreducible if and only if \(n\) is even and \(j=n/2\). 
\item Let \(n=2k\) and \(1\leq i<j\leq k-1\). Let \(V_i,V_j\) be the \(\mathbb{C}D_{2n}\)-modules corresponding to the irreducible representation \(\rho_i\) and \(\rho_j\). Let \(\phi:V_i\rightarrow V_j\) be an isomorphism of \(\mathbb{C}D_{2n}\)-modules, represented by 
an invertible matrix \(M\in GL_2(\mathbb{C})\). For any \(v\in V_1\), \(M\) being compatible with \(D_{2n}\)-action tells us that 
\[A_j(Mv)=M(A_iv).\]
This implies \(A_jM-MA_i\in \Ann V_i\). Because \(V_i\) is a two dimensional \(\mathbb{C}\)-vector space, so \(\Ann V_i=0\). We have 
\[A_i=M^{-1}A_jM.\]
Thus, \(\tr(A_i)=\tr(A_j)\). This is a contradiction because \(1\leq i<j\leq k-1\) and \(\varepsilon^{i}+\varepsilon^{-i}\neq \varepsilon^{j}+\varepsilon^{-j}\). 
\item We calculate the commutator subgroup \(D'_{2n}\). Note that \(aba^{-1}b=a^2\in D'_{2n}\). So the cyclic subgroup generated by \(a^2\) is contained in \(D'_{2n}\). From a claim below by calculating conjugacy classes we know that \(\la a^2\ra\) is normal in \(D_{2n}\). Consider the quotient group \(D_{2n}/\la a^2\ra\), we can check that \(D_{2n}/\la a^2\ra\) is an abelian group by direct computation. Since the commutator subgroup 
is the smallest subgroup such that \(D_{2n}/D'_{2n}\) is abelian. This proves that \(D'_{2n}=\la a^2\ra\). When \(n=2k\), we have 
\[|D_{4k}/D'_{4k}|=4k/k=4.\]
By Exercise 24.2.5(3), there exists four isomorphism classes of 1-dimensional complex representation of \(D_{2n}\). To show that this is a complete list of irreducible representations for \(\mathbb{C}D_{2n}\), by Theorem 24.2.2, we need to calculate the conjugacy classes for \(D_{2n}\). 
\begin{claim}
If \(n=2k\) is even, then the conjuagcy classes of \(D_{2n}\) is given by 
\[\left\{ 1 \right\},\left\{ a,a^{2k-1} \right\},\ldots,\left\{ a^{k-1},a^{k+1} \right\},\left\{ a^k \right\},\left\{ b,a^2b,\ldots,a^{2k-2}b \right\},\left\{ ab,a^3b,\ldots,a^{2k-1}b \right\}.\]
If \(n=2k+1\) is odd, then the conjugacy classes of \(D_{2n}\) is given by 
\[\left\{ 1 \right\},\left\{ a,a^{2k} \right\},\left\{ a^2,a^{2k-1} \right\}\ldots,\left\{ a^k,a^{k+1} \right\},\left\{ b,ab,\ldots,a^{2k}b \right\}.\]
\end{claim}
\begin{claimproof}
Assume \(n=2k\) is even, we first show \(\left\{ a^i,a^{2k-i} \right\}\) is a conjugate class for \(1\leq i\leq k\). Note that all elements in \(D_{2n}\) can be written as \(a^pb^q\), it is easy to see the conjugate action of \(a^p\) sends \(a^i\) to \(a^i\), so we only need to consider the conjugate action by \(a^p b\) for \(0\leq p\leq 2k-1\). We have 
\[(a^p b) a^i (a^pb)^{-1}=a^pba^iba^{2k-p}=a^{p-i+2k-p}=a^{2k-i}.\]
Similarly, we have 
\[(a^pb)a^{2k-i}(a^pb)^{-1}=a^{2k-2k+i}=a^i.\] 
This proves \(\left\{ a^i,a^{2k-i} \right\}\) is a conjugate class for \(i=0,1,\ldots,k\). We need to show the rest two is also conjugate classes. Consider the conjugate class of \(b\), we have 
\begin{align*}
	aba^{-1}&=aab=a^2b,\\ 
	a(a^2b)a^{-1}&=aa^2ab=a^4b,\\ 
	&\cdots\\ 
	a(a^{2k-2}b)a^{-1}&=a^{2k}b=b.
\end{align*}
And for any \(0\leq i\leq k-1\), we have 
\[b(a^{2i}b)b=ba^{2i}=a^{2k-2i}b.\]
This calculates all elements insider the conjuagcy class of \(b\). Lastly, for the conjugacy class of \(ab\), we have 
\begin{align*}
	a(ab)a^{-1}&=a^3b,\\ 
	a(a^3b)a^{-1}&=a^5b,\\ 
	&\cdots\\ 
	a(a^{2k-1}b)a^{-1}&=ab.
\end{align*}
Also, for \(0\leq i\leq k-1\), we have 
\[b(a^{2i+1}b)b=a^{2k-2i-1}b=a^{2(k-1-i)+1}.\]
This calculates the last conjugate class. 

Now assume \(n=2k+1\) is odd. A similar calculation shows that we have \(k+1\) conjugacy classes 
\[\left\{ a^i,a^{2k+1-i} \right\}\]
for \(i=0,1,\ldots,k\). We need to show that rest elements are in the same conjugacy class. Note that 
\begin{align*}
	aba^{-1}&=a^2b,\\ 
	&\cdots\\ 
	a(a^{2k}b)a^{-1}&=a^{2k+2}b=ab,\\ 
	&\cdots\\ 
	a(a^{2k-1}b)a^{-1}&=a^{2k+1}b=b.
\end{align*}
This proves the rest elements are in the same conjugacy class.
\end{claimproof}

From the claim we know \(D_{4k}\) has \(k+3\) conjuagcy classes, so this is a complete and irredundant list of all the \(\mathbb{C}D_{2n}\)-modules up to isomorphism.
\item Similar to what we have discussed above, the commutator group is generated by \(\la a^2\ra\). When \(n=2k+1\) is odd, note that \(a\) has odd order, so in this case \(a^2\) can generate the element \(a=a^{2k+2}\), so the commutator subgroup   
\[|D'_{2n}|=2k+1.\]
So \(D_{4k+2}\) has exactly two 1-dimensional complex representations up to isomorphism. From the claim above, we know that it has \(k+2\) conjugacy classes, so this is a complete and irredundant list of irreducible complex representations.
\end{enumerate}
\end{solution}

\noindent\rule{7in}{2.8pt}
%%%%%%%%%%%%%%%%%%%%%%%%%%%%%%%%%%%%%%%%%%%%%%%%%%%%%%%%%%%%%%%%%%%%%%%%%%%%%%%%%%%%%%%%%%%%%%%%%%%%%%%%%%%%%%%%%%%%%%%%%%%%%%%%%%%%%%%%
% Exercise 24.3.3
%%%%%%%%%%%%%%%%%%%%%%%%%%%%%%%%%%%%%%%%%%%%%%%%%%%%%%%%%%%%%%%%%%%%%%%%%%%%%%%%%%%%%%%%%%%%%%%%%%%%%%%%%%%%%%%%%%%%%%%%%%%%%%%%%%%%%%%%
\begin{problem}{24.3.3}
Describe the character of the two-dimensional irreducible \(\mathbb{C}S_3\)-module.
\end{problem}
\begin{solution}
We can calculate directly from Example 24.2.7, suppose \(V\) is the irreducible 2-dimensional \(\mathbb{C}S_3\)-module, then we know that 
\begin{align*}
	\chi_V((1))=\tr\begin{pmatrix}
		1&0\\ 
		0&1
	\end{pmatrix}=2,\ \ \chi_V((12))=\tr\begin{pmatrix}
		-1&1\\ 
		0&1
	\end{pmatrix}=0,\\ 
	\chi_V((23))=\tr\begin{pmatrix}
        1&0\\ 
		1&-1
	\end{pmatrix}=0,\ \ \chi_V((13))=\tr\begin{pmatrix}
	     0&-1\\ 
		-1&0
	\end{pmatrix}=0,\\ 
	\chi_V((123))=\tr\begin{pmatrix}
         0&-1\\ 
		 1&-1
	\end{pmatrix}=-1,\ \ \chi_V((1))=\tr\begin{pmatrix}
	     -1&1\\ 
		 -1&0
	\end{pmatrix}=-1.
\end{align*}
\end{solution}

\noindent\rule{7in}{2.8pt}
%%%%%%%%%%%%%%%%%%%%%%%%%%%%%%%%%%%%%%%%%%%%%%%%%%%%%%%%%%%%%%%%%%%%%%%%%%%%%%%%%%%%%%%%%%%%%%%%%%%%%%%%%%%%%%%%%%%%%%%%%%%%%%%%%%%%%%%%
% Exercise 24.3.7
%%%%%%%%%%%%%%%%%%%%%%%%%%%%%%%%%%%%%%%%%%%%%%%%%%%%%%%%%%%%%%%%%%%%%%%%%%%%%%%%%%%%%%%%%%%%%%%%%%%%%%%%%%%%%%%%%%%%%%%%%%%%%%%%%%%%%%%%
\begin{problem}{24.3.7}
If \(g\in G\) is an involution then \(\chi(g)\in \mathbb{Z}\) and \(\chi(g)\equiv \chi(1)\) (mod 2) for any character \(\chi\).
\end{problem}
\begin{solution}
Let \(d\) be the dimension of the representation and \(A=\rho(g)\) be the corresponding matrix representation for \(g\in G\). \(g\) being an involution tells us that 
\(A^2-I=0\). So the minimal of polynomial of \(A\) is a factor of \(x^2-1\) with roots \(1\) and \(-1\). We know that the characteistic polynomial and the minimal polynomial have the same roots and the trace of 
\(A\) is just the sum of all the roots of the characteistic polynomial, counting multiplicity. This means \(\tr(A)\) is the sum of \(1\) and \(-1\), so \(\chi(g)=\tr(A)\) is an integer. Suppose the minimal polynomial of \(A\) is 
\(x-1\) or \(x+1\). In this case \(A=I\) or \(A=-I\), so they have the same parity. Now assume the minimal polynomial is \(x^2-1\). When \(d=2\), the characteistic polynomial of \(A\) is also \(x^2-1\), so \(2\) and \(\tr(A)=0\) has the same parity. Now suppose \(d\geq 3\). We can 
write the characteistic polynomial as \((x^2-1)p(x)\) where \(p(x)\) is a product of \(x-1\) and \(x+1\) and \(\deg p(x)=d-2\). Note that when the dimension \(d\) increase \(1\), we add \(1\) or \(-1\) to the trace, so the 
parity of the trace and the parity of the dimension is always the same. This proves 
\[\chi(g)\equiv \chi(1)=d\ \  (\text{mod}\  2).\] 
\end{solution}

\noindent\rule{7in}{2.8pt}
%%%%%%%%%%%%%%%%%%%%%%%%%%%%%%%%%%%%%%%%%%%%%%%%%%%%%%%%%%%%%%%%%%%%%%%%%%%%%%%%%%%%%%%%%%%%%%%%%%%%%%%%%%%%%%%%%%%%%%%%%%%%%%%%%%%%%%%%
% Exercise 24.3.12
%%%%%%%%%%%%%%%%%%%%%%%%%%%%%%%%%%%%%%%%%%%%%%%%%%%%%%%%%%%%%%%%%%%%%%%%%%%%%%%%%%%%%%%%%%%%%%%%%%%%%%%%%%%%%%%%%%%%%%%%%%%%%%%%%%%%%%%%
\begin{problem}{24.3.12}
Let \(C_n\) be the cyclic group of order \(n\). Write dow explicit formulas for the central idempotents \(e_1,\ldots,e_n\in \mathbb{C}C_n\).
\end{problem}
\begin{solution}
\(C_n=\la g\ra\) is an abelian group, so every irreducible \(\mathbb{C}C_n\) module has dimensional \(1\). From Example 24.2.3, let \(\xi\) be the \(n\)th primitive root of \(1\), then the irreducible 
\(C_n\) representation \(\rho_i\) is given by \(\rho_i(g)=\xi^i\). Since every \(\rho_i\) is 1-dimensional, for any \(0\leq k\leq n-1\), we have \(\chi_i(g^k)=\rho_i(g^k)=\xi^{ik}\). And 
\[\chi_i(g^{-k})=\chi_i(g^{n-k})=\xi^{i(n-k)}.\]
By Lemma 24.3.11, we know that the central idempotent 
\[e_i=\frac{1}{n}\sum_{k=0}^{n-1} \chi_i(g^{n-k})g^k=\frac{1}{n}\sum_{k=0}^{n-1}\xi^{i(n-k)}g^k.\] 	
\end{solution}

\noindent\rule{7in}{2.8pt}
%%%%%%%%%%%%%%%%%%%%%%%%%%%%%%%%%%%%%%%%%%%%%%%%%%%%%%%%%%%%%%%%%%%%%%%%%%%%%%%%%%%%%%%%%%%%%%%%%%%%%%%%%%%%%%%%%%%%%%%%%%%%%%%%%%%%%%%%
% Exercise 24.3.18
%%%%%%%%%%%%%%%%%%%%%%%%%%%%%%%%%%%%%%%%%%%%%%%%%%%%%%%%%%%%%%%%%%%%%%%%%%%%%%%%%%%%%%%%%%%%%%%%%%%%%%%%%%%%%%%%%%%%%%%%%%%%%%%%%%%%%%%%
\begin{problem}{24.3.18}
For any finite dimensional \(\mathbb{C}G\)-modules \(V\) and \(W\), we have 
\[\dim \hom_{\mathbb{C}G}(V,W)=(\chi_V,\chi_W).\]
\end{problem}
\begin{solution}
By the Wedderburn-Artin theorem, every finite dimenisonal \(\mathbb{C}G\)-module is semisimple and 
\[\mathbb{C}G\cong M_{n_1}(\mathbb{C})\times \cdots\times M_{n_r}(\mathbb{C}).\]
Let \(\left\{ L_1,\cdots,L_r \right\}\) be a complete and irrdundant set of irreducible \(\mathbb{C}G\)-modules. We know both \(V\) and \(W\) are semisimple, assume 
\(V=\oplus_i V_i\) and \(W=\oplus_j W_j\) where \(V_i,W_j\in \left\{ L_1,\ldots,L_r \right\}\) are irreducible \(\mathbb{C}G\)-modules for any \(i,j\). On the left hand side, we have 
\[\hom_{\mathbb{C}G}(V,W)=\hom_{\mathbb{C}G}(\oplus_i V_i,\oplus_j W_j)=\oplus_{i,j}\hom_{\mathbb{C}G}(V_i,W_j).\]
On the left hand side, by Exercise 23.3.4, we have 
\[(\chi_V,\chi_W)=(\chi_{\oplus_i V_i},\chi_{\oplus_j W_j})=(\sum_i \chi_{V_i},\sum_j \chi_{W_j})=\sum_{i,j}(\chi_{V_i},\chi_{W_j})\]
since \((-,-)\) is an inner product. We only need to show that for two irreducible \(\mathbb{C}G\)-module \(L_i,L_j\) and the corresponding character \(\chi_i,\chi_j\), we have 
\[\dim \hom_{\mathbb{C}G}(L_i,L_j)=(\chi_i,\chi_j).\]
This is true. By Schur's lemma, we know that 
\[\hom_{\mathbb{C}G}(L_i,L_j)=\begin{cases}
	\mathbb{C},&\iif i=j,\\ 
	0,&\iif i\neq j.
\end{cases}\] 
By Theorem 24.3.16, we know that \(\chi_1,\ldots,\chi_r\) are orthonormal with respect to the inner product \((-,-)\), so we also have 
\[(\chi_i,\chi_j)=\begin{cases}
	1,&\iif i=j,\\ 
	0,&\iif i\neq j.
\end{cases}\]
This proves that 
\[\dim \hom_{\mathbb{C}G}(L_i,L_j)=(\chi_i,\chi_j).\]
\end{solution}

\end{document}
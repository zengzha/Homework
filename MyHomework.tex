\documentclass[a4paper, 12pt]{article}
\usepackage{comment} % enables the use of multi-line comments (\ifx \fi) 
\usepackage{lipsum} %This package just generates Lorem Ipsum filler text. 
\usepackage{fullpage} % changes the margin
\usepackage[a4paper, total={7in, 10in}]{geometry}
\usepackage{amsmath}
\usepackage{amssymb,amsthm}  % assumes amsmath package installed
\newtheorem{theorem}{Theorem}
\newtheorem{corollary}{Corollary}
\usepackage{graphicx}
\usepackage{tikz}
\usepackage{leftindex}
\usepackage{multicol}
\usepackage{quiver}
\usetikzlibrary{arrows}
\usepackage{verbatim}
\usepackage{setspace}
\usepackage{comment}
\usepackage{float}
\usepackage{tikz-cd}
\usepackage[backend=biber,bibencoding=utf8,style=numeric,sorting=ynt]{biblatex}

    
\usepackage{xcolor}
\usepackage{mdframed}
\usepackage[shortlabels]{enumitem}
\usepackage{indentfirst}
\usepackage{hyperref}
    
\renewcommand{\thesubsection}{\thesection.\alph{subsection}}

\newenvironment{problem}[2][Exercise]
    { \begin{mdframed}[backgroundcolor=gray!20] \textbf{#1 #2} \\}
    {  \end{mdframed}}

% Define solution environment
\newenvironment{solution}
    {\textit{Solution:}}
    {}

%Define the claim environment
\newenvironment{claim}[1]{\par\noindent\underline{Claim:}\space#1}{}
\newenvironment{claimproof}[1]{\par\noindent\underline{Proof:}\space#1}{\hfill $\blacksquare$}

\renewcommand{\qed}{\quad\qedsymbol}
\newcommand{\la}{\langle}
\newcommand{\ra}{\rangle}
\newcommand{\ord}{\text{ord}\,}
\newcommand{\Ann}{\text{Ann}\,}
\newcommand{\im}{\text{im}\,}
\newcommand{\coker}{\text{coker}\,}
\newcommand{\Com}{\text{Com}}
\newcommand{\End}{\text{End}}
\newcommand{\tr}{\text{tr}}
\newcommand{\iif}{\ \ \text{if}\ \ }
\newcommand{\rank}{\text{rank}\,}
\newcommand{\Rad}{\text{Rad}}
\newcommand{\ind}{\text{ind}}
\newcommand{\coind}{\text{coind}}
\newcommand{\res}{\text{res}}
\newcommand{\li}{\leftindex}
\newcommand{\GCD}{\text{GCD}}
\newcommand{\irr}{\text{irr}}
\newcommand{\Nm}{\text{Nm}}
\newcommand{\Gal}{\text{Gal}}
\newcommand{\Vect}{\text{Vect}}
\newcommand{\cRe}{\text{Re}}
\newcommand{\cIm}{\text{Im}}
%%%%%%%%%%%%%%%%%%%%%%%%%%%%%%%%%%%%%%%%%%%%%%%%%%%%%%%%%%%%%%%%%%%%%%%%%%%%%%%%%%%%%%%%%%%%%%%%%%%%%%%%%%%%%%%%%%%%%%%%%%%%%%%%%%%%%%%%
\begin{document}
%Header-Make sure you update this information!!!!
\noindent
%%%%%%%%%%%%%%%%%%%%%%%%%%%%%%%%%%%%%%%%%%%%%%%%%%%%%%%%%%%%%%%%%%%%%%%%%%%%%%%%%%%%%%%%%%%%%%%%%%%%%%%%%%%%%%%%%%%%%%%%%%%%%%%%%%%%%%%%
\large\textbf{Zhengdong Zhang} \hfill \textbf{Homework - Week 3}   \\
Email: zhengz@uoregon.edu \hfill ID: 952091294 \\
\normalsize Course: MATH 649 - Abstract Algebra  \hfill Term: Spring 2025\\
Instructor: Professor Sasha Polishchuk \hfill Due Date: $23^{th}$ April, 2025 \\
\noindent\rule{7in}{2.8pt}
\setstretch{1.1}
%%%%%%%%%%%%%%%%%%%%%%%%%%%%%%%%%%%%%%%%%%%%%%%%%%%%%%%%%%%%%%%%%%%%%%%%%%%%%%%%%%%%%%%%%%%%%%%%%%%%%%%%%%%%%%%%%%%%%%%%%%%%%%%%%%%%%%%%
% Exercise 11.4.2
%%%%%%%%%%%%%%%%%%%%%%%%%%%%%%%%%%%%%%%%%%%%%%%%%%%%%%%%%%%%%%%%%%%%%%%%%%%%%%%%%%%%%%%%%%%%%%%%%%%%%%%%%%%%%%%%%%%%%%%%%%%%%%%%%%%%%%%%



%%%%%%%%%%%%%%%%%%%%%%%%%%%%%%%%%%%%%%%%%%%%%%%%%%%%%%%%%%%%%%%%%%%%%%%%%%%%%%%%%%%%%%%%%%%%%%%%%%%%%%%%%%%%%%%%%%%%%%%%%%%%%%%%%%%%%%%%
% Exercise 11.5.5
%%%%%%%%%%%%%%%%%%%%%%%%%%%%%%%%%%%%%%%%%%%%%%%%%%%%%%%%%%%%%%%%%%%%%%%%%%%%%%%%%%%%%%%%%%%%%%%%%%%%%%%%%%%%%%%%%%%%%%%%%%%%%%%%%%%%%%%%
\begin{problem}{11.5.5}
Let \(\mathbb{K}/\Bbbk\) be a Galois extension, and \(\mathbb{L}\), \(\mathbb{M}\) be intermediate fields. Denote by \(\mathbb{L}\vee \mathbb{M}\) the minimal subfield of \(\mathbb{K}\) containing 
\(\mathbb{L}\) and \(\mathbb{M}\). 
\begin{enumerate}[(a)]
\item \((\mathbb{L}\cap \mathbb{M})^*=\la \mathbb{L}^*,\mathbb{M}^*\ra\).
\item \((\mathbb{L}\vee \mathbb{M})^*=\mathbb{L}^*\cap \mathbb{M}^*\).
\item Assume that \(\mathbb{L}/\Bbbk\) is normal. Then \(\Gal(\mathbb{L}\vee \mathbb{M}/\mathbb{M})\cong \Gal(\mathbb{L}/(\mathbb{L}\cap \mathbb{M}))\).
\end{enumerate}
\end{problem}
\begin{solution}
\begin{enumerate}[(a)]
\item We know that \(L\cap M\subseteq L\), by the Galois correspondence, we have \(L^*\subseteq (L\cap M)^*\). Similarly, we can see that \(M^*\subseteq (L\cap M)^*\). Note that \(\)
\end{enumerate}
\end{solution}

\noindent\rule{7in}{2.8pt}
%%%%%%%%%%%%%%%%%%%%%%%%%%%%%%%%%%%%%%%%%%%%%%%%%%%%%%%%%%%%%%%%%%%%%%%%%%%%%%%%%%%%%%%%%%%%%%%%%%%%%%%%%%%%%%%%%%%%%%%%%%%%%%%%%%%%%%%%
% Exercise 11.5.6
%%%%%%%%%%%%%%%%%%%%%%%%%%%%%%%%%%%%%%%%%%%%%%%%%%%%%%%%%%%%%%%%%%%%%%%%%%%%%%%%%%%%%%%%%%%%%%%%%%%%%%%%%%%%%%%%%%%%%%%%%%%%%%%%%%%%%%%%
\begin{problem}{11.5.6}
Let \(\mathbb{K}/\Bbbk\) be a finite Galois extension and \(p\) be a prime number. 
\begin{enumerate}[(a)]
\item \(\mathbb{K}\) has an intermediate subfield \(\mathbb{L}\) such that \([\mathbb{K}:\mathbb{L}]\) is a prime power. 
\item If \(\mathbb{L}_1\) and \(\mathbb{L_2}\) are intermediate subfields with \([\mathbb{K}:\mathbb{L}_1]\), \([\mathbb{K}:\mathbb{L}_2]\) both 
\(p\)-powers, and \([\mathbb{L}_1:\Bbbk]\), \([\mathbb{L}_2:\Bbbk]\) both prime to \(p\), then \(\mathbb{L}_1\) is \(\mathbb{L}_1\) is \(\Bbbk\)-isomorphic to \(\mathbb{L}_2\).
\end{enumerate}
\end{problem}
\begin{solution}
    
\end{solution}

\noindent\rule{7in}{2.8pt}
%%%%%%%%%%%%%%%%%%%%%%%%%%%%%%%%%%%%%%%%%%%%%%%%%%%%%%%%%%%%%%%%%%%%%%%%%%%%%%%%%%%%%%%%%%%%%%%%%%%%%%%%%%%%%%%%%%%%%%%%%%%%%%%%%%%%%%%%
% Exercise 11.5.7
%%%%%%%%%%%%%%%%%%%%%%%%%%%%%%%%%%%%%%%%%%%%%%%%%%%%%%%%%%%%%%%%%%%%%%%%%%%%%%%%%%%%%%%%%%%%%%%%%%%%%%%%%%%%%%%%%%%%%%%%%%%%%%%%%%%%%%%%
\begin{problem}{11.5.7}
Let \(f\in \Bbbk[x]\), \(\mathbb{K}/\Bbbk\) be a splitting field for \(f\) over \(\Bbbk\), and \(G:=\Gal(\mathbb{K}/\Bbbk)\).
\begin{enumerate}
\item \(G\) acts on the set of the roots of \(f\).
\item \(G\) acts transitively if \(f\) is irreducible. 
\item If \(f\) has no multiple roots and \(G\) acts transitively then \(f\) is irreducible.
\end{enumerate}
\end{problem}
\begin{solution}
    
\end{solution}

\noindent\rule{7in}{2.8pt}
%%%%%%%%%%%%%%%%%%%%%%%%%%%%%%%%%%%%%%%%%%%%%%%%%%%%%%%%%%%%%%%%%%%%%%%%%%%%%%%%%%%%%%%%%%%%%%%%%%%%%%%%%%%%%%%%%%%%%%%%%%%%%%%%%%%%%%%%
% Exercise 11.6.2
%%%%%%%%%%%%%%%%%%%%%%%%%%%%%%%%%%%%%%%%%%%%%%%%%%%%%%%%%%%%%%%%%%%%%%%%%%%%%%%%%%%%%%%%%%%%%%%%%%%%%%%%%%%%%%%%%%%%%%%%%%%%%%%%%%%%%%%%
\begin{problem}{11.6.2}
Let \(\Bbbk\) be a field, \(p(x)\) be an irreducible polynomial in \(\Bbbk[x]\) of degree \(n\), and let \(\mathbb{K}\) be a Galois extension of \(\Bbbk\) containing a root \(\alpha\) of \(p(x)\). Let \(G=\Gal(\mathbb{K}/\Bbbk)\), 
and \(G_\alpha\) be the set of all \(\sigma\in G\) with \(\sigma(\alpha)=\alpha\). Then: 
\begin{enumerate}[(a)]
\item \([G:G_\alpha]=n\);
\item \(G_\alpha^*=\Bbbk(\alpha)\);
\item If \(G_\alpha\) is normal in \(G\) then \(p(x)\) splits in the fixed field of \(G_\alpha\).
\end{enumerate}
\end{problem}
\begin{solution}
    
\end{solution}
%%%%%%%%%%%%%%%%%%%%%%%%%%%%%%%%%%%%%%%%%%%%%%%%%%%%%%%%%%%%%%%%%%%%%%%%%%%%%%%%%%%%%%%%%%%%%%%%%%%%%%%%%%%%%%%%%%%%%%%%%%%%%%%%%%%%%%%%
% Exercise 11.6.3
%%%%%%%%%%%%%%%%%%%%%%%%%%%%%%%%%%%%%%%%%%%%%%%%%%%%%%%%%%%%%%%%%%%%%%%%%%%%%%%%%%%%%%%%%%%%%%%%%%%%%%%%%%%%%%%%%%%%%%%%%%%%%%%%%%%%%%%%
\begin{problem}{11.6.3}
Let \(\Bbbk(\alpha)/\Bbbk\) be a field extension obtained by adjoining a root \(\alpha\) of an irreducible separable polynomial \(f\in \Bbbk[x]\). 
Then there exists an intermediate field \(\Bbbk\subseteq \mathbb{F}\subseteq \Bbbk(\alpha)\) if and only if \(\Gal(f;\Bbbk)\) is imprimitive (as a permutation group on the roots), in which case \(\mathbb{F}\) can be chosen so that 
\([\mathbb{F}:\Bbbk]\) is equal to the number of imprimitive blocks.
\end{problem}
\begin{solution}

\end{solution}

\noindent\rule{7in}{2.8pt}
%%%%%%%%%%%%%%%%%%%%%%%%%%%%%%%%%%%%%%%%%%%%%%%%%%%%%%%%%%%%%%%%%%%%%%%%%%%%%%%%%%%%%%%%%%%%%%%%%%%%%%%%%%%%%%%%%%%%%%%%%%%%%%%%%%%%%%%%
% Exercise 11.6.6
%%%%%%%%%%%%%%%%%%%%%%%%%%%%%%%%%%%%%%%%%%%%%%%%%%%%%%%%%%%%%%%%%%%%%%%%%%%%%%%%%%%%%%%%%%%%%%%%%%%%%%%%%%%%%%%%%%%%%%%%%%%%%%%%%%%%%%%%
\begin{problem}{11.6.6}
Find all subfields of the splitting field of \(x^3-7\) over \(\mathbb{Q}\). Which of the subfields are normal over \(\mathbb{Q}\)?
\end{problem}
\begin{solution}
    
\end{solution}

\noindent\rule{7in}{2.8pt}
%%%%%%%%%%%%%%%%%%%%%%%%%%%%%%%%%%%%%%%%%%%%%%%%%%%%%%%%%%%%%%%%%%%%%%%%%%%%%%%%%%%%%%%%%%%%%%%%%%%%%%%%%%%%%%%%%%%%%%%%%%%%%%%%%%%%%%%%
% Exercise 11.6.7
%%%%%%%%%%%%%%%%%%%%%%%%%%%%%%%%%%%%%%%%%%%%%%%%%%%%%%%%%%%%%%%%%%%%%%%%%%%%%%%%%%%%%%%%%%%%%%%%%%%%%%%%%%%%%%%%%%%%%%%%%%%%%%%%%%%%%%%%
\begin{problem}{11.6.7}
Let \(\mathbb{K}\) be a splitting field for \(x^4+6x^2+5\) over \(\mathbb{Q}\). Find subfields of \(\mathbb{K}\).
\end{problem}
\begin{solution}
    
\end{solution}

\end{document}
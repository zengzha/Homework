\documentclass[a4paper, 11pt]{article}

\usepackage{/Users/zhengz/Desktop/Math/Workspace/Homework1/homework}
%%%%%%%%%%%%%%%%%%%%%%%%%%%%%%%%%%%%%%%%%%%%%%%%%%%%%%%%%%%%%%%%%%%%%%%%%%%%%%%%%%%%%%%%%%%%%%%%%%%%%%%%%%%%%%%%%%%%%%%%%%%%%%%%%%%%%%%%
\begin{document}
%Header-Make sure you update this information!!!!
\noindent
%%%%%%%%%%%%%%%%%%%%%%%%%%%%%%%%%%%%%%%%%%%%%%%%%%%%%%%%%%%%%%%%%%%%%%%%%%%%%%%%%%%%%%%%%%%%%%%%%%%%%%%%%%%%%%%%%%%%%%%%%%%%%%%%%%%%%%%%
\large\textbf{Zhengdong Zhang} \hfill \textbf{Homework - Week 3}   \\
Email: zhengz@uoregon.edu \hfill ID: 952091294 \\
\normalsize Course: MATH 634 - Algebraic Topology  \hfill Term: Fall 2024\\
Instructor: Dr.Patricia Hersh \hfill Due Date: $22^{nd}$ October, 2024 \\
\noindent\rule{7in}{2.8pt}
%%%%%%%%%%%%%%%%%%%%%%%%%%%%%%%%%%%%%%%%%%%%%%%%%%%%%%%%%%%%%%%%%%%%%%%%%%%%%%%%%%%%%%%%%%%%%%%%%%%%%%%%%%%%%%%%%%%%%%%%%%%%%%%%%%%%%%%%
% Exercise 2.1.11
%%%%%%%%%%%%%%%%%%%%%%%%%%%%%%%%%%%%%%%%%%%%%%%%%%%%%%%%%%%%%%%%%%%%%%%%%%%%%%%%%%%%%%%%%%%%%%%%%%%%%%%%%%%%%%%%%%%%%%%%%%%%%%%%%%%%%%%%
\begin{problem}{2.1.11.}
Show that if \(A\) is a retract of \(X\) then the map \(H_n(A)\rightarrow H_n(X)\) induced by the inclusion \(A\subset X\) is injective.
\end{problem}

\begin{solution}
Let \(i:A\hookrightarrow X\) be the inclusion map. Since \(A\subset X\) is a retract, \(i\) has a left inverse \(r:X\rightarrow A\) such that \(r\circ i=\text{id}_A\). We have 
an induced map on \(n\)th singular homology group:
$$\begin{tikzcd}
	{H_n(A)} & {H_n(X)} & {H_n(A)}
	\arrow["{i_*}", from=1-1, to=1-2]
	\arrow["id"', curve={height=24pt}, from=1-1, to=1-3]
	\arrow["{r_*}", from=1-2, to=1-3]
\end{tikzcd}$$
Suppose \(a\in \text{ker}i_*\), we have \(id(a)=(r_*\circ i_*) (a)=r_*(i_*a)=r_*(0)=0\). This means \(a=0\). Thus \(\text{ker}i_*=0\) and \(i_*\) is injective.
\end{solution}
\\ 
\noindent\rule{7in}{2.8pt}
%%%%%%%%%%%%%%%%%%%%%%%%%%%%%%%%%%%%%%%%%%%%%%%%%%%%%%%%%%%%%%%%%%%%%%%%%%%%%%%%%%%%%%%%%%%%%%%%%%%%%%%%%%%%%%%%%%%%%%%%%%%%%%%%%%%%%%%%
% Exercise 2.1.12
%%%%%%%%%%%%%%%%%%%%%%%%%%%%%%%%%%%%%%%%%%%%%%%%%%%%%%%%%%%%%%%%%%%%%%%%%%%%%%%%%%%%%%%%%%%%%%%%%%%%%%%%%%%%%%%%%%%%%%%%%%%%%%%%%%%%%%%%
\begin{problem}{2.1.12}
Show that chain homotopy of chain maps is an equivalence relation.
\end{problem}
\begin{solution}
We need to show that chain homotopy is reflexive, symmetric and transitive. Suppose \(f,g,h:(C_\bullet,\partial^C) \rightarrow (D_\bullet,\partial^D)\) are chain maps between chain complexes 
\(C_\bullet\) and \(D_\bullet\). 
\\ 
1.(reflexivity) Consider a collection of zero maps 
$$\left\{ 0:C_n\rightarrow D_{n+1} | n\in \mathbb{N} \right\}.$$
We have that for all \(n\in \mathbb{N}\), 
$$0=f_n-f_n=\partial^D\circ 0+0\circ \partial^C.$$
This proves that chain homotopy is reflexive.
\\ 
2. (symmetry) Suppose \(f\) is chain homotopic to \(g\). We have a collection of group homomorphisms 
$$\left\{ \psi_n:C_n\rightarrow D_{n+1}|n\in \mathbb{N} \right\}$$
such that 
$$f_n-g_n=\partial^D\circ \psi_n+\psi_{n-1}\circ \partial^C.$$
Consider the collection of group homomorphisms
$$\left\{ -\psi_n:C_n\rightarrow D_{n+1}|n\in \mathbb{N} \right\}$$
and they satisfy 
$$g_n-f_n=\partial^D\circ (-\psi_n)+(-\psi_{n-1})\circ \partial^C.$$
This proves that chain homotopy is symmetric.
\\ 
3. (transitivity) Suppose \(f\) is chain homotopic to \(g\) with the collection of group homomorphisms 
$$\left\{ \psi_n:C_n\rightarrow D_{n+1}|n\in \mathbb{N} \right\}$$
and \(g\) is chain homotopic to \(h\) with the collection of group homomorphisms 
$$\left\{ \phi_n:C_n\rightarrow D_{n+1}|n\in \mathbb{N} \right\}.$$
Consider the collection of group homomorphisms
$$\left\{ \psi_n+\phi_n:C_n\rightarrow D_{n+1}|n\in \mathbb{N} \right\}$$
and we have 
\begin{align*}
    f_n-h_n = &\, f_n-g_n+g_n-h_n\\ 
            = &\, \partial^D\circ \psi_n+\psi_{n-1}\circ \partial^C\\ 
              & +\partial^D\circ \phi_n+\phi_{n-1}\circ \partial^C\\ 
            = &\, \partial^D\circ (\psi_n+\phi_n)+(\psi_{n-1}+\phi_{n-1})\circ \partial^C.
\end{align*}
Thus \(f\) is chain homotopic to \(h\). This proves that chain homotopy is transitive.
\end{solution}
\\  
\noindent\rule{7in}{2.8pt}
%%%%%%%%%%%%%%%%%%%%%%%%%%%%%%%%%%%%%%%%%%%%%%%%%%%%%%%%%%%%%%%%%%%%%%%%%%%%%%%%%%%%%%%%%%%%%%%%%%%%%%%%%%%%%%%%%%%%%%%%%%%%%%%%%%%%%%%%
% Exercise 2.1.13
%%%%%%%%%%%%%%%%%%%%%%%%%%%%%%%%%%%%%%%%%%%%%%%%%%%%%%%%%%%%%%%%%%%%%%%%%%%%%%%%%%%%%%%%%%%%%%%%%%%%%%%%%%%%%%%%%%%%%%%%%%%%%%%%%%%%%%%%
\begin{problem}{2.1.13}
Verify that \(f\simeq g\) implies \(f_*=g_*\) for induced homomorphisms of reduced homology groups.
\end{problem}
\begin{solution}
Let \(f,g:X\rightarrow Y\) be homotopic maps between topological spaces \(X\) and \(Y\). For \(n\geq 1\), by Theorem 2.10 and by definition \(\tilde{H}_n(X)=H_n(X)\) and 
\(\tilde{H}_n(Y)=H_n(Y)\), we have:
$$f_*=g_*:\tilde{H}_n(X)\rightarrow \tilde{H}_n(Y).$$
Now we need to show that \(f,g\) induce the same map between \(\tilde{H}_0(X)\) and \(\tilde{H_0(Y)}\). According to the proof of Theorem 2.10 in the book, we only need to show that 
there exists a chain homotopic map \(\psi_{-1}:\) in degree 0 such that 
$$f_\sharp-g_\sharp=\partial_1^D\circ \psi_0+\psi_{-1}\circ \varepsilon.$$
as shown in the following diagram of augmented chain complex:
$$\begin{tikzcd}
	{C_1(X)} & {C_0(X)} & {\mathbb{Z}} & 0 \\
	{D_1(Y)} & {D_0(Y)} & {\mathbb{Z}} & 0
	\arrow[from=1-1, to=1-2]
	\arrow["\varepsilon", from=1-2, to=1-3]
	\arrow["{\psi_0}", from=1-2, to=2-1]
	\arrow["{f_\sharp}"', shift right, from=1-2, to=2-2]
	\arrow["{g_\sharp}", shift left, from=1-2, to=2-2]
	\arrow[from=1-3, to=1-4]
	\arrow["{\psi_{-1}}", from=1-3, to=2-2]
	\arrow["{\partial_1^D}"', from=2-1, to=2-2]
	\arrow["\varepsilon"', from=2-2, to=2-3]
	\arrow[from=2-3, to=2-4]
\end{tikzcd}$$
Recall that \(f_\sharp\) is chain homotopic to \(g_\sharp\) in the usual singular chain comlex, we have 
$$f_\sharp-g_\sharp=\partial_1^D\circ \psi_0+0\circ \partial_0^C.$$
Take \(\psi_{-1}\) to be the zero map and we have a chain homotopy. This shows that \(f_*=g_*\) is also true for \(0\)th reduced homology groups.

\end{solution}
\noindent\rule{7in}{2.8pt}
%%%%%%%%%%%%%%%%%%%%%%%%%%%%%%%%%%%%%%%%%%%%%%%%%%%%%%%%%%%%%%%%%%%%%%%%%%%%%%%%%%%%%%%%%%%%%%%%%%%%%%%%%%%%%%%%%%%%%%%%%%%%%%%%%%%%%%%%
% Exercise 2.1.14
%%%%%%%%%%%%%%%%%%%%%%%%%%%%%%%%%%%%%%%%%%%%%%%%%%%%%%%%%%%%%%%%%%%%%%%%%%%%%%%%%%%%%%%%%%%%%%%%%%%%%%%%%%%%%%%%%%%%%%%%%%%%%%%%%%%%%%%%
\begin{problem}{2.1.14}
Determine whether there exists a short exact sequence \(0\rightarrow \mathbb{Z}\rightarrow \mathbb{Z}_8\oplus \oplus \mathbb{Z}_2\rightarrow \mathbb{Z}_4\rightarrow 0\). 
More generally, determine which abelian groups \(A\) fit into a short exact sequence \(0\rightarrow \mathbb{Z}_{p^m}\rightarrow A\rightarrow \mathbb{Z}_{p^n}\rightarrow 0\) with 
\(p\) prime. What about the case of short exact sequencs \(0\rightarrow \mathbb{Z}\rightarrow A\rightarrow \mathbb{Z}_n\rightarrow 0\)?
\end{problem}
\begin{solution}
Suppose we have a short exact sequence:
$$\begin{tikzcd}
	0 & {\mathbb{Z}_4} & {\mathbb{Z}_8\oplus\mathbb{Z}_2} & {\mathbb{Z}_4} & 0
	\arrow[from=1-1, to=1-2]
	\arrow["f", from=1-2, to=1-3]
	\arrow["g", from=1-3, to=1-4]
	\arrow[from=1-4, to=1-5]
\end{tikzcd}$$
where \(f(1)=(2,1)\) and \(g(1,0)=1,g(0,1)=2\). It is obvious that \(f\) is injective and \(g\) is surjective. Now to prove this sequence is exact, we need to show that \(\text{im}f=\text{ker}g\). 
Let \(0\leq a\leq 7\) and \(0\leq b\leq 1\) be integers. Any element \((a,b)\in \mathbb{Z}_8\oplus \mathbb{Z}_2\) can be written as \(a(1,0)+b(0,1)\). So \(f(a,b)=af(1,0)+bf(0,1)=a+2b\). The condition 
\((a,b)\in \text{ker}g\) is equivalent to \(4|(a+2b)\). Test all the possibilities and we could find out that 
$$\text{ker}g=\left\{ (0,0),(2,1),(4,0),(6,1) \right\}=\text{im}f.$$
This shows that the sequence is exact.
\par 
Now consider a short exact sequence of abelian groups:
$$\begin{tikzcd}
	0 & {\mathbb{Z}_{p^m}} & A & {\mathbb{Z}_{p^n}} & 0
	\arrow[from=1-1, to=1-2]
	\arrow["f",from=1-2, to=1-3]
	\arrow["g",from=1-3, to=1-4]
	\arrow[from=1-4, to=1-5]
\end{tikzcd}$$
\(f\) is an injective map and view it as a normal subgroup of \(A\). We can see that \(A/\mathbb{Z}_{p^m}\cong \mathbb{Z}_{p^n}\). The group order of \(A\) must be \(|A|=p^{m+n}\).
\\ 
\begin{enumerate}
    \item Assume \(A\) is generated by one element, i.e. \(A\) is a cyclic group of order \(p^{m+n}\). In this case, \(A\) is isomorphic to \(\mathbb{Z}_{p^{m+n}}\). And we define 
          \(f(1)=p^n\in \mathbb{Z}_{p^{m+n}}\). Identify \(\mathbb{Z}_{p^n}\) with \(\mathbb{Z}_{p^{m+n}}/\mathbb{Z}_{p^m}\) and we have the short exact sequence. 
    \item Assume \(A\) is generated by 2 elements. Since \(|A|=p^{m+n}\), we could write \(A=\mathbb{Z}_{p^k}\oplus \mathbb{Z}_{p^{m+n-k}}\). Without loss of generality, assume \(k\geq m+n-k\). Since \(f\) is injective, 
          \(\mathbb{Z}_{p^k}\) must have a subgroup isomorphic to \(\mathbb{Z}_{p^m}\). This implies \(k\geq m\). Thus, \(m+n-k\) must be smaller than \(n\). So to have a surjective map \(A\rightarrow \mathbb{Z}_{p^n}\), \(k\) 
          must also be larger than \(n\). So we have \(k\geq \max\left\{ m,n \right\}\). Now we define \(f(1)=(p^{k-m},1)\) and \(g\) is the canonical quotient map. We need to show that the quotient group \(A/\mathbb{Z}_{p^m}\) is 
          isomorphic to \(\mathbb{Z}_{p^n}\). Note that \(\mathbb{Z}_{p^k}\oplus \mathbb{Z}_{p^{m+n-k}}\) is generated by \((1,0)\) and \((0,1)\), now identify the image of \(\mathbb{Z}_{p^m}\) in \(A\) as 
          the cyclic group generated by \((p^{k-m},1)\). So the quotient group is generated by \((1,0)+\langle (p^{k-m},1)\rangle\) and \((0,1)+\langle (p^{k-m},1)\rangle\). But \((0,1)+p^{m+n-k}(p^{k-m},1)=(p^n,0)\in \mathbb{Z}_{p^k}\oplus \mathbb{Z}_{p^{m+n-k}}\), 
          this shows that the group generated by \((0,1)+\langle (p^{k-m},1)\rangle\) is contained in the group generated by \((1,0)+\langle (p^{k-m},1)\rangle\). So the quotient group is a cyclic group generated by \((1,0)+\langle (p^{k-m},1)\rangle\), which has 
          order \(p^n\), so it is isomorphic to \(\mathbb{Z}_{p^n}\).
     \item Assume \(A\) has 3 generators. Let \(f(1)=(a,b,c)\). We know that the folloing three elements must generated the quotient group:
          \begin{align*}
            A=(1,0,0)+\mathbb{Z}_{p^m},\\ 
            B=(0,1,0)+\mathbb{Z}_{p^m},\\ 
            C=(0,0,1)+\mathbb{Z}_{p^m},
          \end{align*}
          Since the quotient group is cyclic, \(B\) and \(C\) must be contained in \(\langle A\rangle\). Note that \(p\geq 2\), so either \(b\) or \(c\) must be 0. But in this case, \(\ker g\) will be larger than 
          \(\text{im} f\). A contradiction. Similar arguments show that \(A\) cannot be generated by more than 2 elements.
\end{enumerate}
\par 
Now consider we have a short exact sequence:
$$\begin{tikzcd}
	0 & {\mathbb{Z}} & A & {\mathbb{Z}_n} & 0
	\arrow[from=1-1, to=1-2]
	\arrow["f", from=1-2, to=1-3]
	\arrow["g", from=1-3, to=1-4]
	\arrow[from=1-4, to=1-5]
\end{tikzcd}$$
\(f\) being injective means that \(A\) must contain \(\mathbb{Z}\) as a subgroup. A similar argument as 3. above shows that \(A\) has at most two generators. If \(A\) is cyclic and it has \(\mathbb{Z}\) as a subgroup, then \(A=\mathbb{Z}\). In this case \(f(1)=n\) and \(g\) is the quotient map. If 
\(A\) is generated by 2 elements. Note that the torsion part cannot be kill by quotient, so \(A\) must both contain \(\mathbb{Z}\) and \(\mathbb{Z}_n\) as a subgroup. So \(A=\mathbb{Z}\oplus \mathbb{Z}_n\).
\end{solution}
\\ 
\noindent\rule{7in}{2.8pt}
%%%%%%%%%%%%%%%%%%%%%%%%%%%%%%%%%%%%%%%%%%%%%%%%%%%%%%%%%%%%%%%%%%%%%%%%%%%%%%%%%%%%%%%%%%%%%%%%%%%%%%%%%%%%%%%%%%%%%%%%%%%%%%%%%%%%%%%%
% Exercise 5
%%%%%%%%%%%%%%%%%%%%%%%%%%%%%%%%%%%%%%%%%%%%%%%%%%%%%%%%%%%%%%%%%%%%%%%%%%%%%%%%%%%%%%%%%%%%%%%%%%%%%%%%%%%%%%%%%%%%%%%%%%%%%%%%%%%%%%%%
\begin{problem}{5}
Calculate the homology with integer coefficients for the chain complex 
$$\begin{tikzcd}
	0 & {\mathbb{Z}} & {\mathbb{Z}^2} & {\mathbb{Z}} & 0
	\arrow["{\partial_3}", from=1-1, to=1-2]
	\arrow["{\partial_2}", from=1-2, to=1-3]
	\arrow["{\partial_1}", from=1-3, to=1-4]
	\arrow["{\partial_0}", from=1-4, to=1-5]
\end{tikzcd}$$
with boundary maps \(\partial_3=\partial_0=0\) and \(\partial_2=\begin{pmatrix}
     8\\
    -4
\end{pmatrix}\) and \(\partial_1=\begin{pmatrix}
    4 & 8
\end{pmatrix}\).
\end{problem}
\begin{solution}
Suppose for free abelian groups, 
$$C_2=\mathbb{Z}=\langle F\rangle, C_1=\mathbb{Z}^2=\langle L_1,L_2\rangle, C_0=\mathbb{Z}=\langle v\rangle.$$
We have \(\partial_2 F=8L_1-4L_2\) and \(\partial_1 L_1=4v, \partial_1 L_2=8v\). Then \(\text{ker}\partial_1=\langle L_2-2L_1\rangle\) and \(\text{im}\partial_1=\langle 4v\rangle\). 
Similarly, \(\text{ker}\partial_2=0\) and \(\text{im}\partial_2=\langle 8L_1-4L_2\rangle\). Now we can calculate the homology group.
\begin{align*}
    H_0(C_\bullet) & = \text{ker}\partial_0/\text{im}\partial_1\\ 
                   & = \langle v\rangle / \langle 4v\rangle \\ 
                   & = \mathbb{Z}/4 \mathbb{Z}.
\end{align*}
\begin{align*}
    H_1(C_\bullet) & = \text{ker}\partial_1/\text{im}\partial_2\\ 
                   & = \langle L_2-2L_1\rangle /\langle 8L_1-4L_2\rangle \\ 
                   & = \langle 2L_1-L_2\rangle /\langle 4(2L_1-L_2)\rangle \\ 
                   & = \mathbb{Z}/4 \mathbb{Z}.
\end{align*}
\begin{align*}
    H_2(C_\bullet) & = \text{ker}\partial_2/\text{im}\partial_3 \\ 
                   & =\text{ker}\partial_2\\ 
                   & =0.
\end{align*}
In conclusion, we have 
$$H_n(C_\bullet)=\begin{cases}
    \mathbb{Z}/4 \mathbb{Z} & n=0\ \text{or} \ n=1,\\ 
     0. & \text{otherwise} 
\end{cases}$$
\end{solution}
\noindent\rule{7in}{2.8pt}
%%%%%%%%%%%%%%%%%%%%%%%%%%%%%%%%%%%%%%%%%%%%%%%%%%%%%%%%%%%%%%%%%%%%%%%%%%%%%%%%%%%%%%%%%%%%%%%%%%%%%%%%%%%%%%%%%%%%%%%%%%%%%%%%%%%%%%%%
% Exercise 6
%%%%%%%%%%%%%%%%%%%%%%%%%%%%%%%%%%%%%%%%%%%%%%%%%%%%%%%%%%%%%%%%%%%%%%%%%%%%%%%%%%%%%%%%%%%%%%%%%%%%%%%%%%%%%%%%%%%%%%%%%%%%%%%%%%%%%%%%
\begin{problem}{6}
Calculate the homology with integer coefficients for the chain complex 
$$\begin{tikzcd}
	0 & {\mathbb{Z}} & {\mathbb{Z}^3} & {\mathbb{Z}^3} & {\mathbb{Z}} & 0
	\arrow["{\partial_4}", from=1-1, to=1-2]
	\arrow["{\partial_3}", from=1-2, to=1-3]
	\arrow["{\partial_2}", from=1-3, to=1-4]
	\arrow["{\partial_1}", from=1-4, to=1-5]
	\arrow["{\partial_0}", from=1-5, to=1-6]
\end{tikzcd}$$
with boundary maps \(\partial_0=\partial_4=0\) and \(\partial_3=\begin{pmatrix}
    4\\ 
    -6\\ 
    10
\end{pmatrix}\) and \(\partial_2=\begin{pmatrix}
    -2 & -23 & -13\\ 
    2  & 23 & 13\\ 
    4 & 16 & 8
\end{pmatrix}\) and \(\partial_1=\begin{pmatrix}
    2 & 2& 0
\end{pmatrix}\).
\end{problem}
\begin{solution}
First write down the generators:
\begin{align*}
    C_0 & =\mathbb{Z}  =\langle v\rangle \\ 
    C_1 & =\mathbb{Z}^3  =\langle L_1,L_2,L_3\rangle\\ 
    C_2 & =\mathbb{Z}^3  =\langle F_1,F_2,F_3\rangle\\
    C_4 & =\mathbb{Z}  =\langle T\rangle  
\end{align*}
We have already know \(\partial_0=\partial_4=0\). And \(\partial_3 T=4F_1-6F_2+10F_3\), so 
$$\text{ker}\partial_3=0,\ \text{im}\partial_3=\langle 4F_1-6F_2+10F_3\rangle.$$
And we have \(\partial_1 L_1=2v\), \(\partial_1 L_2=2v\) and \(\partial_1 L_3=0\), so 
$$\text{ker}\partial_1=\langle L_2-L_1, L_3\rangle,\ \text{im}\partial_1=\langle 2v\rangle.$$
Finally, for \(\partial_2\), we have 
\begin{align*}
    \partial_2 F_1 & =-2L_1+2L_2+4L_3\\ 
    \partial_2 F_2 & =-23L_1+23L_2+16L_3\\ 
    \partial_2 F_3 & =-13L_1+13L_2+8L_3
\end{align*}
Now we do a base change. The base \(L_1,L_2,L_3\) is changed into \(L_2-L_1,L1,L_3\) and the base \(F_1,F_2,F_3\) is change into \(F_1,F_2-4F_1,F_3-2F_1\). Now 
we have 
\begin{align*}
    \partial_2 F_1 & =2(L_2-L_1)+4L_3\\ 
    \partial_2 F_2-4F_1 & =15(L_2-L_1)\\ 
    \partial_2 F_3-2F_1 & =9(L_2-L_1)
\end{align*}
So we can calculate:
\begin{align*}
    \text{ker}\partial_2 & =\langle 3(F_2-4F_1)-5(F_3-2F_1)\rangle\\ 
                         & =\langle -2F_1+3F_2-5F_3\rangle 
\end{align*}
and 
\begin{align*}
    \text{im}\partial_2 & = \langle 3(L_2-L_1), 2(L_2-L_1)+4L_3\rangle\\ 
                        & = \langle 3(L_2-L_1), (L_2-L_1)-4L_3\rangle
\end{align*}
Now we calculate the homology of the chain complex:
\begin{align*}
    H_0(C_\bullet) & = \text{ker}{\partial_0}/\text{im}\partial_1\\ 
                   & = \langle v \rangle/\langle 2v\rangle\\ 
                   & = \mathbb{Z}/2 \mathbb{Z}.
\end{align*}
\begin{align*}
    H_1(C_\bullet) & = \text{ker}{\partial_1}/\text{im}\partial_2\\
                   & = \frac{\langle L_2-L_1,L_3\rangle}{\langle 3(L_2-L_1), (L_2-L_1)-4L_3\rangle}\\ 
                   & = \frac{\langle (L_2-L_1)-4L_3,L_3\rangle}{\langle 3[(L_2-L_1)-4L_3]+12L_3, (L_2-L_1)-4L_3\rangle}\\ 
                   & = \langle L_3\rangle/\langle 12L_3\rangle \\ 
                   & =\mathbb{Z}/ 12 \mathbb{Z}.
\end{align*}
\begin{align*}
    H_2(C_\bullet) & = \text{ker}{\partial_2}/\text{im}\partial_3\\
                   & = \frac{\langle -2F_1+3F_2-5F_3\rangle}{\langle 4F_1-6F_2+10F_3\rangle}\\ 
                   & =\mathbb{Z}/2 \mathbb{Z}.
\end{align*}
\begin{align*}
    H_3(C_\bullet) & = \text{ker}{\partial_3}/\text{im}\partial_4\\
                   & = 0
\end{align*}
In conclusion, we have 
$$H_n(C_\bullet)=\begin{cases}
    \mathbb{Z}/2 \mathbb{Z} & n=0\ \text{or} \ n=2,\\ 
    \mathbb{Z}/12 \mathbb{Z} & n=1,\\ 
    0. & \text{otherwise} 
\end{cases}$$

\end{solution}


\end{document}
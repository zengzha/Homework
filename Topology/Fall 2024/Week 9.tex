\documentclass[a4paper, 12pt]{article}
\usepackage{comment} % enables the use of multi-line comments (\ifx \fi) 
\usepackage{lipsum} %This package just generates Lorem Ipsum filler text. 
\usepackage{fullpage} % changes the margin
\usepackage[a4paper, total={7in, 10in}]{geometry}
\usepackage{amsmath}
\usepackage{amssymb,amsthm}  % assumes amsmath package installed
\newtheorem{theorem}{Theorem}
\newtheorem{corollary}{Corollary}
\usepackage{graphicx}
\usepackage{tikz}
\usepackage{quiver}
\usepackage{setspace}
%\usepackage{enumitem}
\usetikzlibrary{arrows}
\usepackage{verbatim}
\usepackage[shortlabels]{enumitem}

\usepackage{float}
\usepackage{tikz-cd}


    
\usepackage{xcolor}
\usepackage{mdframed}
\usepackage[shortlabels]{enumitem}
%\usepackage{indentfirst}
\usepackage{hyperref}
    
\renewcommand{\thesubsection}{\thesection.\alph{subsection}}


\newenvironment{problem}[2][Exercise]
    { \begin{mdframed}[backgroundcolor=gray!20] \textbf{#1 #2} \\}
    {  \end{mdframed}}

% Define solution environment
\newenvironment{solution}
    {\textit{Solution:}}
    {}

%Define the claim environment
\newenvironment{claim}[1]{\par\noindent\underline{Claim:}\space#1}{}
\newenvironment{claimproof}[1]{\par\noindent\underline{Proof:}\space#1}{\hfill $\blacksquare$}

\renewcommand{\qed}{\quad\qedsymbol}
\newcommand{\rank}{\text{rank}\,}
\newcommand{\im}{\text{Im}\,}
%%%%%%%%%%%%%%%%%%%%%%%%%%%%%%%%%%%%%%%%%%%%%%%%%%%%%%%%%%%%%%%%%%%%%%%%%%%%%%%%%%%%%%%%%%%%%%%%%%%%%%%%%%%%%%%%%%%%%%%%%%%%%%%%%%%%%%%%
\begin{document}
%Header-Make sure you update this information!!!!
\noindent
%%%%%%%%%%%%%%%%%%%%%%%%%%%%%%%%%%%%%%%%%%%%%%%%%%%%%%%%%%%%%%%%%%%%%%%%%%%%%%%%%%%%%%%%%%%%%%%%%%%%%%%%%%%%%%%%%%%%%%%%%%%%%%%%%%%%%%%%
\large\textbf{Zhengdong Zhang} \hfill \textbf{Homework - Week 9}   \\
Email: zhengz@uoregon.edu \hfill ID: 952091294 \\
\normalsize Course: MATH 634 - Algebraic Topology  \hfill Term: Fall 2024\\
Instructor: Dr.Patricia Hersh \hfill Due Date: $5^{th}$ December, 2024 \\
\noindent\rule{7in}{2.8pt}
\setstretch{1.1}
%%%%%%%%%%%%%%%%%%%%%%%%%%%%%%%%%%%%%%%%%%%%%%%%%%%%%%%%%%%%%%%%%%%%%%%%%%%%%%%%%%%%%%%%%%%%%%%%%%%%%%%%%%%%%%%%%%%%%%%%%%%%%%%%%%%%%%%%
% Exercise 2.2.16
%%%%%%%%%%%%%%%%%%%%%%%%%%%%%%%%%%%%%%%%%%%%%%%%%%%%%%%%%%%%%%%%%%%%%%%%%%%%%%%%%%%%%%%%%%%%%%%%%%%%%%%%%%%%%%%%%%%%%%%%%%%%%%%%%%%%%%%%
\begin{problem}{2.2.16}
Let \(\Delta^n=[v_0,\ldots,v_n]\) have its natural \(\Delta\)-complex structure with \(k\)-simplices \([v_{i_0},\ldots,v_{i_k}]\) for \(i_0<\cdots<i_k\). Compute the ranks of the 
simplicial (or cellular) chain groups \(\Delta_i(\Delta^n)\) and the subgroups of cycles and boundaries. [Hint:Pascal's triangle.] Apply this to show that the \(k\)-skeleton of \(\Delta^n\) has 
homology groups \(\tilde{H}_i((\Delta^n)^k)\) equal to 0 for \(i<k\) and free of rank \(\binom{n}{k+1}\) for \(i=k\).
\end{problem}
\begin{solution}
Let \(C_k\) denote the \(k\)th simplicial chain group and we have a chain complex of abelian groups 
\[\begin{tikzcd}
	0 & {C_n} & \cdots & {C_k} & {C_{k-1}} & \cdots & {C_1} & {C_0} & 0
	\arrow["{d_{n+1}}", from=1-1, to=1-2]
	\arrow["{d_n}", from=1-2, to=1-3]
	\arrow["{d_{k+1}}", from=1-3, to=1-4]
	\arrow["{d_k}", from=1-4, to=1-5]
	\arrow["{d_{k-1}}", from=1-5, to=1-6]
	\arrow["{d_2}", from=1-6, to=1-7]
	\arrow["{d_1}", from=1-7, to=1-8]
	\arrow["{d_0}", from=1-8, to=1-9]
\end{tikzcd}\]
For \(0\leq k\leq n\), \(C_k\) is generated by \(k\)-simplices in a standard \(n\)-simplex \(\Delta^n\), choosing a \(k\)-simplex is the same as choosing \((k+1)\) vertices, so \(\rank C_k=\binom{n+1}{k+1}\). Now write 
\(Z_k=\ker d_k\subset C_k\) as the subgroup of \(k\)-cycles and \(B_k=\im d_{k+1}\subset C_k\) as the subgroup of \(k\)-boundaries. The \(H_k=Z_k/B_k\) is the \(k\)-th simplicial homology group of \(\Delta_n\). We have two short exact sequences 
\[\begin{tikzcd}
	0 & {Z_k} & {C_k} & {B_{k-1}} & 0 \\
	0 & {B_k} & {Z_k} & {H_k} & 0
	\arrow[from=1-1, to=1-2]
	\arrow[from=1-2, to=1-3]
	\arrow[from=1-3, to=1-4]
	\arrow[from=1-4, to=1-5]
	\arrow[from=2-1, to=2-2]
	\arrow[from=2-2, to=2-3]
	\arrow[from=2-3, to=2-4]
	\arrow[from=2-4, to=2-5]
\end{tikzcd}\]
This gives us 
\begin{align*}
    \rank Z_k&=\rank B_k+\rank H_k,\\ 
    \rank C_k&=\rank Z_k+\rank B_{k-1}.
\end{align*}
Note that \(\Delta^n\) is contractible so \(\rank H_0=1\) and \(\rank H_k=0\) for \(k\neq 0\). Moreover, \(B_{-1}=\im d_0=0\), so 
\begin{align*}
    \rank Z_0&=\rank C_0=\binom{n+1}{1}=n+1,\\
    \rank B_0&=\rank Z_0-\rank H_0=n+1-1=n.
\end{align*}
For \(k>0\), we can inductively calculate 
\begin{align*}
    \rank Z_k&=\rank C_k-\rank B_{k-1},\\
    \rank B_k&=\rank Z_k-\rank H_k=\rank Z_k
\end{align*}
Using the law of Pascal's triangle, we can see that for \(1\leq k\leq n\),  
\begin{align*}
    \rank Z_k&=\binom{n+1}{k+1}-\binom{n}{k}=\binom{n}{k+1},\\ 
    \rank B_k&=\rank Z_k=\binom{n}{k+1}.
\end{align*}
The simplicial chain complex of the \(k\)-skeleton \((\Delta^n)^k\) is the tuncated version 
\[\begin{tikzcd}
	0 & {C_k} & {C_{k-1}} & \cdots & {C_1} & {C_0} & 0
	\arrow["0", from=1-1, to=1-2]
	\arrow["{d_k}", from=1-2, to=1-3]
	\arrow["{d_{k-1}}", from=1-3, to=1-4]
	\arrow["{d_2}", from=1-4, to=1-5]
	\arrow["{d_1}", from=1-5, to=1-6]
	\arrow["{d_0}", from=1-6, to=1-7]
\end{tikzcd}\]
So for \(0\leq i\leq k-1\), we have 
\[\tilde{H}_i((\Delta^n)^k)=\tilde{H}_i(\Delta^n)=0.\]
and 
\[\tilde{H}_k((\Delta^n)^k)=\ker d_k\]
is free abelian and has rank \(\binom{n}{k+1}\).
\end{solution}

\noindent\rule{7in}{2.8pt}
%%%%%%%%%%%%%%%%%%%%%%%%%%%%%%%%%%%%%%%%%%%%%%%%%%%%%%%%%%%%%%%%%%%%%%%%%%%%%%%%%%%%%%%%%%%%%%%%%%%%%%%%%%%%%%%%%%%%%%%%%%%%%%%%%%%%%%%%
% Exercise 2.2.17
%%%%%%%%%%%%%%%%%%%%%%%%%%%%%%%%%%%%%%%%%%%%%%%%%%%%%%%%%%%%%%%%%%%%%%%%%%%%%%%%%%%%%%%%%%%%%%%%%%%%%%%%%%%%%%%%%%%%%%%%%%%%%%%%%%%%%%%%
\begin{problem}{2.2.17}
Show the isomorphism between cellular and singular homology is natural in the following sense: A map \(f:X\rightarrow Y\) that is cellular, satisfying \(f(X^n)\subset Y^n\) for all \(n\), induces a chain 
map \(f_\sharp\) between the cellular chain complexes of \(X\) and \(Y\), and the map \(f_*:H_n^{CW}(X)\rightarrow H_n^{CW}(Y)\) induced by this chain map corresponds to \(f_*:H_n(X)\rightarrow H_n(Y)\) under the 
isomorphism \(H_n^{CW}\approx H_n\).
\end{problem}
\begin{solution}
For every \(n\), we have a map of pairs:
\[f:(X^{n+1},X^n)\rightarrow (Y^{n+1},Y^n)\]
which induces the following commutative diagram:
\[\begin{tikzcd}
	0 & {C_n(X^n)} & {C_n(X^{n+1})} & {C_n(X^n,X^{n+1})} & 0 \\
	0 & {C_n(Y^n)} & {C_n(Y^{n+1})} & {C_n(Y^n,Y^{n+1})} & 0
	\arrow[from=1-1, to=1-2]
	\arrow[from=1-1, to=2-1]
	\arrow[from=1-2, to=1-3]
	\arrow[from=1-2, to=2-2]
	\arrow[from=1-3, to=1-4]
	\arrow[from=1-3, to=2-3]
	\arrow[from=1-4, to=1-5]
	\arrow[from=1-4, to=2-4]
	\arrow[from=1-5, to=2-5]
	\arrow[from=2-1, to=2-2]
	\arrow[from=2-2, to=2-3]
	\arrow[from=2-3, to=2-4]
	\arrow[from=2-4, to=2-5]
\end{tikzcd}\]
The naturality of the induced long exact sequence gives us 
\[\begin{tikzcd}
	0 & {H_n(X^n)} & {H_n(X^n,X^{n-1})} & {H_{n-1}(X^{n-1})} & \cdots \\
	0 & {H_n(Y^n)} & {H_n(Y^n,Y^{n-1})} & {H_{n-1}(Y^{n-1})} & \cdots
	\arrow[from=1-1, to=1-2]
	\arrow[from=1-1, to=2-1]
	\arrow["{j_n}", from=1-2, to=1-3]
	\arrow["{f'_n}", from=1-2, to=2-2]
	\arrow["{\partial_n}", from=1-3, to=1-4]
	\arrow["{f_n}", from=1-3, to=2-3]
	\arrow[from=1-4, to=1-5]
	\arrow["{f'_{n-1}}", from=1-4, to=2-4]
	\arrow[from=2-1, to=2-2]
	\arrow["{j'_n}", from=2-2, to=2-3]
	\arrow["{\partial'_n}", from=2-3, to=2-4]
	\arrow[from=2-4, to=2-5]
\end{tikzcd}\]
Using \(f_n\circ j_n=j'_n\circ f'_n\) and \(f'_{n-1}\circ \partial_n=\partial'_n\circ f_n\), we claim that we have a chain map \(f_\sharp\) between cellular chain complex:
\[\begin{tikzcd}
	\cdots & {H_{n+1}(X^{n+1},X^n)} & {H_n(X^n,X^{n-1})} & {H_{n-1}(X^{n-1},X^{n-2})} & \cdots \\
	\cdots & {H_{n+1}(Y^{n+1},Y^n)} & {H_n(Y^n,Y^{n-1})} & {H_{n-1}(Y^{n-1},Y^{n-2})} & \cdots
	\arrow[from=1-1, to=1-2]
	\arrow["{d_{n+1}}", from=1-2, to=1-3]
	\arrow["{f_{n+1}}", from=1-2, to=2-2]
	\arrow["{d_n}", from=1-3, to=1-4]
	\arrow["{f_n}", from=1-3, to=2-3]
	\arrow[from=1-4, to=1-5]
	\arrow["{f_{n-1}}", from=1-4, to=2-4]
	\arrow[from=2-1, to=2-2]
	\arrow["{d'_{n+1}}", from=2-2, to=2-3]
	\arrow["{d'_n}", from=2-3, to=2-4]
	\arrow[from=2-4, to=2-5]
\end{tikzcd}\]
To check this diagram indeed commutes, we can see that by the definition of the boundary map
\begin{align*}
    f_n\circ d_{n+1}&=f_n\circ (j_n\circ \partial_{n+1})\\ 
                    &=(f_n\circ j_n)\circ \partial_{n+1}\\
                    &=(j'_n\circ f'_n)\circ \partial_{n+1}\\ 
                    &=j'_n\circ (f'_n\circ \partial_{n+1})\\ 
                    &=j'_n\circ (\partial'_{n+1}\circ f_{n+1})\\ 
                    &=(j'_n\circ \partial'_{n+1})\circ f_{n+1}\\ 
                    &=d'_{n+1}\circ f_{n+1}
\end{align*}
This chain map induces a map \(f_*:H_n^{CW}(X)\rightarrow H_n^{CW}(Y)\) for every \(n\). 

To see that this map corresponds to the map \(f_*:H_n(X)\rightarrow H_n(Y)\) under the isomorphism \(H_n^{CW}\cong H_n\), recall from Theorem 2.35, \(H_n(X)\) is identified 
with \(H_n(X)/\text{Im}\, \partial_{n+1}\), using the naturality of induced long exact sequence, we have 
\[\begin{tikzcd}
	\cdots & {H_{n+1}(X^{n+1},X^n)} & {H_n(X^n)} & {H_n(X^{n+1})} & 0 \\
	\cdots & {H_{n+1}(Y^{n+1},Y^n)} & {H_n(Y^n)} & {H_n(Y^{n+1})} & 0
	\arrow[from=1-1, to=1-2]
	\arrow["{\partial_{n+1}}", from=1-2, to=1-3]
	\arrow["{f_{n+1}}", from=1-2, to=2-2]
	\arrow[from=1-3, to=1-4]
	\arrow[from=1-3, to=2-3]
	\arrow[from=1-4, to=1-5]
	\arrow[from=1-4, to=2-4]
	\arrow[from=2-1, to=2-2]
	\arrow["{\partial'_{n+1}}", from=2-2, to=2-3]
	\arrow[from=2-3, to=2-4]
	\arrow[from=2-4, to=2-5]
\end{tikzcd}\]
This shows that the map \(f_*:H_n(X)\rightarrow H_n(Y)\) is equivalent to the map 
\[f_*:H_n(X^n)/\text{Im}\, \partial_{n+1}\rightarrow H_n(Y^n)/\text{Im}\, \partial'_{n+1}.\]
We know that \(j_n:H_n(X^n)\xrightarrow{\sim} H_n(X^n,X^{n-1})\) induces an isomorphism \(j_{n,*}:H_n(X)\xrightarrow{\sim} H_n^{CW}(X)\), the following diagram commutes: 
\[\begin{tikzcd}
	{H_n(X^n)/\text{Im}\, \partial_{n+1}} & {\ker d_n/\text{Im}\, d_{n+1}} \\
	{H_n(Y^n)/\text{Im}\,\partial'_{n+1}} & {\ker d'_n/\text{Im}\, d'_{n+1}}
	\arrow["{j_{n,*}}", from=1-1, to=1-2]
	\arrow[from=1-1, to=2-1]
	\arrow[from=1-2, to=2-2]
	\arrow["{j'_{n,*}}", from=2-1, to=2-2]
\end{tikzcd}\]
And the commutativity comes from the fact that \(j_n\) commutes with \(f_*\) in the induced long exact sequence.
\end{solution}

\noindent\rule{7in}{2.8pt}
%%%%%%%%%%%%%%%%%%%%%%%%%%%%%%%%%%%%%%%%%%%%%%%%%%%%%%%%%%%%%%%%%%%%%%%%%%%%%%%%%%%%%%%%%%%%%%%%%%%%%%%%%%%%%%%%%%%%%%%%%%%%%%%%%%%%%%%%
% Exercise 2.2.18
%%%%%%%%%%%%%%%%%%%%%%%%%%%%%%%%%%%%%%%%%%%%%%%%%%%%%%%%%%%%%%%%%%%%%%%%%%%%%%%%%%%%%%%%%%%%%%%%%%%%%%%%%%%%%%%%%%%%%%%%%%%%%%%%%%%%%%%%
\begin{problem}{2.2.18}
For a CW pair \((X,A)\) show there is a relative cellular chain complex formed by the groups \(H_i(X^i,X^{i-1}\cup A^i)\), having homology groups isomorphic to \(H_n(X,A)\).
\end{problem}
\begin{solution}
We first establish some preliminary facts similar to lemma 2.34 in the book.
\begin{claim}
\begin{enumerate}[(1)]
\item \(H_k(X^n,X^{n-1}\cup A^n)=0\) for \(k>n\) and is free abelian for \(k=n\), with a basis in one-to-one correspondence with \(n\)-cells in \(X\) excluding the \(n\)-cells in \(A\). 
\item \(H_k(X^n\cup A^{n+1},A^{n+1})\cong H_k(X^n,A^n)=0\) for \(k>n\). If \(X\) is finite dimensional, then \(H_k(X,A)=0\) for \(k>\dim X\).
\item The map \(H_k(X^n,A^n)\rightarrow H_k(X,A)\) induced by the inclusion of pairs \((X^n,A^n)\hookrightarrow (X,A)\) is an isomorphism for \(k<n\) and surjective for \(k=n\).
\end{enumerate}
\end{claim}
\begin{claimproof}
\begin{enumerate}[(1)]
\item Since \((X,A)\) is a CW pair, \((X^n,X^{n-1}\cup A^n)\) is also a good pair. We have a isomorphism 
\[\tilde{H}_k(X^n,X^{n-1}\cup A^n)\xrightarrow{\sim} \tilde{H}_k(X^n/(X^{n-1}\cup A^n)).\]
Note that the \((n-1)\)-skeleton along with any \(n\)-cells in \(A\) collasped into a point in the quotient space \(X^n/(X^{n-1}\cup A^n)\). This space is a wedge sum of \(n\)-spheres corresponding each 
\(n\)-cells in \(X\) excluding the \(n\)-cells in \(A\). So \(H_k(X^n,X^{n-1}\cup A^n)=0\) if \(k>n\) and is free abelain if \(k=n\).
\item Note that \(A\) is a subcomplex of \(X\), so we have \(A^{n+1}\cap X^n=A^n\). The isomorphism \(H_k(X^n\cup A^{n+1},A^{n+1})\cong H_k(X^n,A^n)\) is given by the excision. Using the fact that \((X^n,A^n)\) is a good pair, 
we have \(H_k(X^n,A^n)\cong \tilde{H}_k(X^n/A^n)\). The quotient space \(X^n/A^n\) inhabits a natural CW complex structure of dimensional \(n\), so 
\[H_k(X^n,A^n)\cong \tilde{H}_k(X^n/A^n)=0\]
if \(k>n\).
\item The natural inclusions \(A^n\rightarrow A\) and \(X^n\rightarrow X\) give a map between long exact sequences.
\[\begin{tikzcd}
	{H_k(A^n)} & {H_k(X^n)} & {H_k(X^n,A^n)} & {H_{k-1}(A^n)} & {H_{k-1}(X^n)} \\
	{H_k(A)} & {H_k(X)} & {H_k(X,A)} & {H_{k-1}(A)} & {H_{k-1}(X)}
	\arrow[from=1-1, to=1-2]
	\arrow[from=1-1, to=2-1]
	\arrow[from=1-2, to=1-3]
	\arrow[from=1-2, to=2-2]
	\arrow[from=1-3, to=1-4]
	\arrow[from=1-3, to=2-3]
	\arrow[from=1-4, to=1-5]
	\arrow[from=1-4, to=2-4]
	\arrow[from=1-5, to=2-5]
	\arrow[from=2-1, to=2-2]
	\arrow[from=2-2, to=2-3]
	\arrow[from=2-3, to=2-4]
	\arrow[from=2-4, to=2-5]
\end{tikzcd}\]
By Lemma 2.34, the outer four maps are isomorphism when \(k<n\), by 5 lemma, the middle map is also an isomorphism.
\end{enumerate}
\end{claimproof}

Consider the triple \(A^n\subset X^{n-1}\cup A^n\subset X^n\), which induces a long exact sequence of relative homology groups 
\[\cdots\rightarrow H_k(X^{n-1}\cup A^n,A^n)\rightarrow H_k(X^n,A^n)\rightarrow H_k(X^n,X^{n-1}\cup A^n)\rightarrow H_{k-1}(X^{n-1}\cup A^n,A^n)\rightarrow \cdots.\]
Consider the following diagram 
\[\begin{tikzcd}
	{H_n(X^{n+1},X^n\cup A^{n+1})=0} \\
	{H_n(X^{n+1},A^{n+1})\cong H_n(X,A)} & {H_n(X^{n-1}\cup A^n,A^n)=0} \\
	{H_n(X^n\cup A^{n+1},A^{n+1})} & {H_n(X^n,A^n)} \\
	{H_{n+1}(X^{n+1},X^n\cup A^{n+1})} & {H_n(X^n,X^{n-1}\cup A^n)} & {H_{n-1}(X^{n-1},X^{n-2}\cup A^{n-1})} \\
	& {H_{n-1}(X^{n-1}\cup A^n,A^n)} & {H_{n-1}(X^{n-1},A^{n-1})} \\
	&& {H_{n-1}(X^{n-2}\cup A^{n-1},A^{n-1})=0}
	\arrow[from=2-1, to=1-1]
	\arrow[from=2-2, to=3-2]
	\arrow[two heads, from=3-1, to=2-1]
	\arrow["{e_n}"',"\text{excision}", from=3-1, to=3-2]
	\arrow["{j_n}", hook, from=3-2, to=4-2]
	\arrow["{\partial_{n+1}}", from=4-1, to=3-1]
	\arrow["{d_{n+1}}", from=4-1, to=4-2]
	\arrow["{d_n}", from=4-2, to=4-3]
	\arrow["{\partial_n}", from=4-2, to=5-2]
	\arrow["{e_{n-1}}"', "\text{excision}",from=5-2, to=5-3]
	\arrow["{j_{n-1}}"', hook', from=5-3, to=4-3]
	\arrow[from=6-3, to=5-3]
\end{tikzcd}\]
Note that the vertical column in the above diagram is exact. We define the \(n\)-th chain group as \(H_n(X^n,X^{n-1}\cup A^n)\), which is free abelian with generators corresponding to \(n\)-cells in \(X\) but not in \(A\). The boundary map \(d_n\) is 
defined to be the cocomposition \(d_n=j_{n-1}\circ e_{n-1}\circ \partial_n\). It is easy to see that we have \(d_n\circ d_{n+1}=0\) since 
\[d_n\circ d_{n+1}=j_{n-1}\circ e_{n-1} \circ (\partial_n\circ j_n)\circ e_n\circ \partial_{n+1}=0.\]

Finally we are going to show that the homology of the above chain complex is isomorphic to the relative homology group \(H_\bullet(X,A)\). Because \(j_{n-1}\) is injective and \(e_{n-1}\) is an isomorphism, we have 
\[\ker d_n=\ker \partial_n=\text{Im}\, j_n\cong H_n(X^n,A^n)\cong H_n(X^n\cup A^{n+1},A^{n+1}).\]
Again \(j_n\) is injective implies that \(\text{Im}\, d_{n+1}=\text{Im}\, \partial_{n+1}\), so 
\[\ker d_n/\text{Im}\, d_{n+1}\cong H_n(X^n\cup A^{n+1},A^{n+1})/\text{Im}\, \partial_{n+1}\cong H_n(X^{n+1},A^{n+1})\cong H_n(X,A).\]
\end{solution}

\noindent\rule{7in}{2.8pt}
%%%%%%%%%%%%%%%%%%%%%%%%%%%%%%%%%%%%%%%%%%%%%%%%%%%%%%%%%%%%%%%%%%%%%%%%%%%%%%%%%%%%%%%%%%%%%%%%%%%%%%%%%%%%%%%%%%%%%%%%%%%%%%%%%%%%%%%%
% Exercise 2.2.20
%%%%%%%%%%%%%%%%%%%%%%%%%%%%%%%%%%%%%%%%%%%%%%%%%%%%%%%%%%%%%%%%%%%%%%%%%%%%%%%%%%%%%%%%%%%%%%%%%%%%%%%%%%%%%%%%%%%%%%%%%%%%%%%%%%%%%%%%
\begin{problem}{2.2.20}
For finite CW complexes \(X\) and \(Y\), show that \(\chi(X\times Y)=\chi(X)\chi(Y)\).
\end{problem}
\begin{solution}
Suppose \(X\) has dimension \(m\) and in the dimension \(0\leq i\leq m\), the number of \(i\)-cells is denoted by \(a_i\). Similarly, \(Y\) has dimension \(n\) and the number of \(j\)-cells is denoted by \(b_j\). So by definition of Euler
characteristics, we have 
\begin{align*}
    \chi(X)\chi(Y)&=(\sum_{i=0}^{m} (-1)^i a_i)(\sum_{j=0}^{n}(-1)^j  b_j)\\ 
                  &=\sum_{i=0}^{m}\sum_{j=0}^{n}(-1)^{i+j} a_ib_j
\end{align*}
On the other hand, we know that the product \(X\times Y\) is a CW complex of dimension \(mn\), and it has \(\sum_{i+j=k} a_ib_j\) \(k\)-cells for each \(0\leq k\leq mn\). So by definition 
\begin{align*}
    \chi (X\times Y)&=\sum_{k=0}^{mn} (-1)^k (\sum_{i+j=k} a_ib_j)\\ 
                    &=\sum_{k=0}^{mn} \sum_{i+j=k} (-1)^{i+j} a_ib_j
\end{align*}
reordering the summation and we can see that they are equal.
\end{solution}

\noindent\rule{7in}{2.8pt}
%%%%%%%%%%%%%%%%%%%%%%%%%%%%%%%%%%%%%%%%%%%%%%%%%%%%%%%%%%%%%%%%%%%%%%%%%%%%%%%%%%%%%%%%%%%%%%%%%%%%%%%%%%%%%%%%%%%%%%%%%%%%%%%%%%%%%%%%
% Exercise 2.2.27
%%%%%%%%%%%%%%%%%%%%%%%%%%%%%%%%%%%%%%%%%%%%%%%%%%%%%%%%%%%%%%%%%%%%%%%%%%%%%%%%%%%%%%%%%%%%%%%%%%%%%%%%%%%%%%%%%%%%%%%%%%%%%%%%%%%%%%%%
\begin{problem}{2.2.27}
The short exact sequence 
\[0\rightarrow C_n(A)\rightarrow C_n(X)\rightarrow C_n(X,A)\rightarrow 0\]
always split, but why does this not always yield splittings 
\[H_n(X)\approx H_n(A)\oplus H_n(X,A).\]
\end{problem}
\begin{solution}
Recall that \(C_n(X,A)=C_n(X)/C_n(A)\) is generated by the maps \(\Delta^n\rightarrow X\) whose image is not completely in \(A\). So it is a free abelian group and the short exact sequence 
\[0\rightarrow C_n(A)\rightarrow C_n(X)\rightarrow C_n(X,A)\rightarrow 0\]
splits. However, there is no reason that the for a general pair \((X,A)\), we have 
\[H_n(X)\approx H_n(A)\oplus H_n(X,A).\]
Consider the following case \(X\) is a closed 2-disk \(D^2\) and \(A\subset X\) is its boundary \(\partial D^2=S^1\). This is a good pair and by Proposition 2.22, \(H_2(X,A)\) is isomorphic to 
\(\tilde{H}_2(X/A)\cong \tilde{H}_2(S^2)=\mathbb{Z}\). But \(X\) is contractible and \(H_2(X)=0\).
\end{solution}

\noindent\rule{7in}{2.8pt}
%%%%%%%%%%%%%%%%%%%%%%%%%%%%%%%%%%%%%%%%%%%%%%%%%%%%%%%%%%%%%%%%%%%%%%%%%%%%%%%%%%%%%%%%%%%%%%%%%%%%%%%%%%%%%%%%%%%%%%%%%%%%%%%%%%%%%%%%
% Exercise 2.3.2
%%%%%%%%%%%%%%%%%%%%%%%%%%%%%%%%%%%%%%%%%%%%%%%%%%%%%%%%%%%%%%%%%%%%%%%%%%%%%%%%%%%%%%%%%%%%%%%%%%%%%%%%%%%%%%%%%%%%%%%%%%%%%%%%%%%%%%%%
\begin{problem}{2.3.2}
Define a candidate for a reduced homology theory on CW complexes by \(\tilde{h}_n(X)=\prod_i \tilde{H}_i(X)/\oplus_i \tilde{H}_i(X)\). Thus \(\tilde{h}_n(X)\) is independent of \(n\) and is zero if \(X\) is finite 
dimensional, but is not identically zero, for example for \(X=\vee_iS^i\). Show that the axiom for a homology theory are satisfied except that the wedge axiom fails.
\end{problem}
\begin{solution}
We check the first two axioms and give a counter example for the third axiom.
\begin{enumerate}[(1)]
\item Let \(X\) and \(Y\) be two homotopy equivalent spaces. For every i, we have \(\tilde{H}_i(X)\cong \tilde{H}_i(Y)\) by the homotopy invariance of reduced singular homology. Then 
\[\tilde{h}_n(X)=\prod_i \tilde{H}_i(X)/\oplus_i \tilde{H}_i(X)\cong \prod_i \tilde{H}_i(Y)/\oplus_i \tilde{H}_i(Y)=\tilde{h}_n(Y)\]
for all \(n\).
\item Let \((X,A)\) be a CW pair. For each \(i\), we have a connecting homomorphsim \(\partial_i:\tilde{H}_i(X/A)\rightarrow \tilde{H}_{i-1}(A)\). Consider a homomorphsim 
\[\partial:\prod_i \tilde{H}_i(X/A)\rightarrow \prod_i \tilde{H}_{i-1}(A)\]
where on each component \(\tilde{H}_i(X/A)\), it is just \(\partial_i\). The exactness is preseved in the long exact sequence. Same as the naturality.
\item Consider \(X=\vee_i S^i\). For every \(n\), by Corollary 2.25, we have 
\[\tilde{H}_k(\vee_i S^i)\cong \oplus_i \tilde{H}_k(S^i).\]
Note that \(\tilde{H}_k(S^i)=\mathbb{Z}\) for \(i=k\) and equal to \(0\) otherwise. So we have \(\tilde{H}_k(X)=\mathbb{Z}\) for every \(k\geq 0\). This implies that 
\[\tilde{h}_n(X)=(\prod_k \mathbb{Z})/(\oplus_k \mathbb{Z})\]
is non trivial. On the other hand, \(\tilde{h}_n(S^i)=\mathbb{Z}/\mathbb{Z}=0\) is trivial. So we have 
\[\oplus_i \tilde{h}_n(S^i)\neq \tilde{h}_n(\vee_i S^i).\]
\end{enumerate}
\end{solution}

\noindent\rule{7in}{2.8pt}
%%%%%%%%%%%%%%%%%%%%%%%%%%%%%%%%%%%%%%%%%%%%%%%%%%%%%%%%%%%%%%%%%%%%%%%%%%%%%%%%%%%%%%%%%%%%%%%%%%%%%%%%%%%%%%%%%%%%%%%%%%%%%%%%%%%%%%%%
% Exercise 2.3.3
%%%%%%%%%%%%%%%%%%%%%%%%%%%%%%%%%%%%%%%%%%%%%%%%%%%%%%%%%%%%%%%%%%%%%%%%%%%%%%%%%%%%%%%%%%%%%%%%%%%%%%%%%%%%%%%%%%%%%%%%%%%%%%%%%%%%%%%%
\begin{problem}{2.3.3}
Show that if \(\tilde{h}\) is a reduced homology theory, then \(\tilde{h}_n(point)=0\) for all \(n\). Deduce that there are suspension isomorphism \(\tilde{h}_n(X)\approx \tilde{h}_{n+1}(SX)\) for all \(n\).
\end{problem}
\begin{solution}
Let \(X\) be a CW complex and consider the identity map \(id:X\rightarrow X\). This gives a CW pair \((X,X)\) and note that \(X/X=\left\{ point \right\}\). By the second axiom of homology theory we have a long exact sequence 
\[\cdots\xrightarrow{\partial}\tilde{h}_n(X)\xrightarrow{id_*}\tilde{h}_n(X)\xrightarrow{q_*}\tilde{h}_n(point)\xrightarrow{\partial}\tilde{h}_{n-1}(X)\rightarrow \cdots\]
We know that \(id_*\) is an isomorphism and using exactness, we can see that \(\tilde{h}_n(point)=0\) for all \(n\). 

Let \(CX\) denote the cone of \(X\) and we know that it is homotopy equivalent to a point. We have \(\tilde{h}_n(CX)\cong \tilde{h}_n(point)=0\). Consider the quotient space \(CX/X\) and by the second axiom, we have a long exact sequence 
\[\cdots\rightarrow \tilde{h}_{n+1}(CX)\rightarrow \tilde{h}_{n+1}(CX/X)\rightarrow \tilde{h}_n(X)\rightarrow \tilde{h}_n(CX)\rightarrow \cdots\]
By exactness we have \(\tilde{h}_{n+1}(CX/X)\cong \tilde{h}_n(X)\). Now Consider the suspension \(SX\) and two cones \(A\cong B\cong CX\) whose intersection is homeomorphic to \(A\cap B=X\). We have a homeomorphism \(SX/CX\cong CX/X\). Moreover, we apply the second 
axiom to the quotient space \(SX/CX\)
\[\cdots\rightarrow \tilde{h}_{n+1}(CX)\rightarrow \tilde{h}_{n+1}(SX)\rightarrow \tilde{h}_{n+1}(SX/CX)\rightarrow \tilde{h}_n(CS)\rightarrow \cdots\]
By exactness we have 
\(\tilde{h}_{n+1}(SX)\cong \tilde{h}_{n+1}(SX/CX)\cong \tilde{h}_n(X)\) for all \(n\).
\end{solution}

\end{document}
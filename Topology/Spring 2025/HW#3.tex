\documentclass[a4paper, 12pt]{article}

\usepackage{/Users/zhengz/Desktop/Math/Workspace/Homework1/homework}

%%%%%%%%%%%%%%%%%%%%%%%%%%%%%%%%%%%%%%%%%%%%%%%%%%%%%%%%%%%%%%%%%%%%%%%%%%%%%%%%%%%%%%%%%%%%%%%%%%%%%%%%%%%%%%%%%%%%%%%%%%%%%%%%%%%%%%%%
\begin{document}
%Header-Make sure you update this information!!!!
\noindent
%%%%%%%%%%%%%%%%%%%%%%%%%%%%%%%%%%%%%%%%%%%%%%%%%%%%%%%%%%%%%%%%%%%%%%%%%%%%%%%%%%%%%%%%%%%%%%%%%%%%%%%%%%%%%%%%%%%%%%%%%%%%%%%%%%%%%%%%
\large\textbf{Zhengdong Zhang} \hfill \textbf{Homework 3}   \\
Email: zhengz@uoregon.edu \hfill ID: 952091294 \\
\normalsize Course: MATH 636 - Algebraic Topology III \hfill Term: Spring 2025\\
Instructor: Dr.Daniel Dugger \hfill Due Date: $^{24th}$ April, 2025 \\
\noindent\rule{7in}{2.8pt}
\setstretch{1.1}

%%%%%%%%%%%%%%%%%%%%%%%%%%%%%%%%%%%%%%%%%%%%%%%%%%%%%%%%%%%%%%%%%%%%%%%%%%%%%%%%%%%%%%%%%%%%%%%%%%%%%%%%%%%%%%%%%%%%%%%%%%%%%%%%%%%%%%%%
%Probelm 1 
%%%%%%%%%%%%%%%%%%%%%%%%%%%%%%%%%%%%%%%%%%%%%%%%%%%%%%%%%%%%%%%%%%%%%%%%%%%%%%%%%%%%%%%%%%%%%%%%%%%%%%%%%%%%%%%%%%%%%%%%%%%%%%%%%%%%%%%%
\begin{problem}{1}
Compute both \(\Tor_i(A,B)\) and \(\Ext^i(A,B)\) for all \(i\) in the following cases: 
\begin{enumerate}[(a)]
\item \(A=\mathbb{Z}/9\) and \(B=\mathbb{Z}/6\).
\item \(A=\mathbb{Z}/9\) and \(B=\mathbb{Z}\).
\item \(A=\mathbb{Z}^2\oplus \mathbb{Z}/4\oplus \mathbb{Z}/5\oplus \mathbb{Z}/10\) and \(B=\mathbb{Z}\oplus \mathbb{Z}/3\oplus \mathbb{Z}/4\oplus \mathbb{Z}/6\).
\end{enumerate}
\end{problem}
\begin{solution}

\end{solution}

\noindent\rule{7in}{2.8pt}
%%%%%%%%%%%%%%%%%%%%%%%%%%%%%%%%%%%%%%%%%%%%%%%%%%%%%%%%%%%%%%%%%%%%%%%%%%%%%%%%%%%%%%%%%%%%%%%%%%%%%%%%%%%%%%%%%%%%%%%%%%%%%%%%%%%%%%%%
%Probelm 2
%%%%%%%%%%%%%%%%%%%%%%%%%%%%%%%%%%%%%%%%%%%%%%%%%%%%%%%%%%%%%%%%%%%%%%%%%%%%%%%%%%%%%%%%%%%%%%%%%%%%%%%%%%%%%%%%%%%%%%%%%%%%%%%%%%%%%%%%
\begin{problem}{2}
Let \(A\) be an abelian group and let \(G_*\rightarrow A\rightarrow 0\) be a free resolution. Let \(B\) be another abelian group and let \(J_*\rightarrow B\rightarrow 0\) be a free resolution. 
\begin{enumerate}[(a)]
\item Given a map \(f:A\rightarrow B\), prove that there are maps \(F_i:G_i\rightarrow J_i\) making all squares commute, we call this chain map \(\left\{ F:G_*\rightarrow J_* \right\}\) a lifting of the map \(f\). 
\item Prove that if \(\left\{ F':G_*\rightarrow J_* \right\}\) is another lifting of \(f\) then the chain map \(F\) and \(F'\) are chain homotopic. 
\item If \(C\) is another abelian group one gets an induced map \(F\otimes id:G_*\otimes C\rightarrow J_*\otimes C\) and therefore an induced map on homology groups \(f_*:\Tor_i(A,C)\rightarrow \Tor_i(B,C)\). Since 
any two choices of \(F\) are homotopic, this \(f_*\) is well-defined.\\ 
Use the above procedure to calculate the maps 
\begin{align*}
    j_*:&\Tor_1(\mathbb{Z}/2,\mathbb{Z}/2)\rightarrow \Tor_1(\mathbb{Z}/4,\mathbb{Z}/2),\\ 
    k_*:&\Tor_1(\mathbb{Z}/4,\mathbb{Z}/2)\rightarrow \Tor_1(\mathbb{Z}/2,\mathbb{Z}/2).
\end{align*}
induced by the map \(j:\mathbb{Z}/2\hookrightarrow \mathbb{Z}/4\) (sending 1 to 2) and \(k:\mathbb{Z}/4\rightarrow \mathbb{Z}/2\) (sending 1 to 1).
\end{enumerate}
\end{problem}
\begin{solution}
    
\end{solution}

\noindent\rule{7in}{2.8pt}
%%%%%%%%%%%%%%%%%%%%%%%%%%%%%%%%%%%%%%%%%%%%%%%%%%%%%%%%%%%%%%%%%%%%%%%%%%%%%%%%%%%%%%%%%%%%%%%%%%%%%%%%%%%%%%%%%%%%%%%%%%%%%%%%%%%%%%%%
%Probelm 3.7
%%%%%%%%%%%%%%%%%%%%%%%%%%%%%%%%%%%%%%%%%%%%%%%%%%%%%%%%%%%%%%%%%%%%%%%%%%%%%%%%%%%%%%%%%%%%%%%%%%%%%%%%%%%%%%%%%%%%%%%%%%%%%%%%%%%%%%%%
\begin{problem}{3.7}
If \(F\) is a finitely-generated free abelian group then there is a canonical isomorphism 
\[\hom(\hom(F,\mathbb{Z}),\mathbb{Z})\cong F.\]
So if \(C\) is a chain complex consisting of finitely generated, free abelian groups, one gets an induced isomorphism 
\[\hom(\hom(C,\mathbb{Z}),\mathbb{Z})\cong C.\]
Using this, derive a universal coefficient theorem which lets you predict \(H_*(C)\) if you know \(H^*(\hom(C,\mathbb{X}))\).
\end{problem}
\begin{solution}
    
\end{solution}

\noindent\rule{7in}{2.8pt}
%%%%%%%%%%%%%%%%%%%%%%%%%%%%%%%%%%%%%%%%%%%%%%%%%%%%%%%%%%%%%%%%%%%%%%%%%%%%%%%%%%%%%%%%%%%%%%%%%%%%%%%%%%%%%%%%%%%%%%%%%%%%%%%%%%%%%%%%
%Probelm 3.8
%%%%%%%%%%%%%%%%%%%%%%%%%%%%%%%%%%%%%%%%%%%%%%%%%%%%%%%%%%%%%%%%%%%%%%%%%%%%%%%%%%%%%%%%%%%%%%%%%%%%%%%%%%%%%%%%%%%%%%%%%%%%%%%%%%%%%%%%
\begin{problem}{3.8}
In this problem we'll use the abbreviations \(H^i(\hom(C,\mathcal{A}))=H^i(C;\mathcal{A})\) and \(H^i(C)=H^i(C;\mathbb{Z})\). 
\par 
If \(F\) is a finitely -generated free abelian group then there is a canonical isomorphism 
\[\hom(F,\mathcal{A})\cong \hom(F,\mathbb{Z})\otimes \mathbb{Z}.\]
So if \(C\) is a chain complex consistin of finitely generated free abelian groups, we have an isomorphism 
\[\hom(C,\mathcal{A})\cong \hom(C,\mathbb{Z})\otimes \mathcal{A}.\]
Using this, derive a universal coefficient theorem which lets you predict \(H^*(C;\mathcal{A})\) if you know \(H^*(C)\). The formula should look like 
\[H^i(C;\mathcal{A})\cong [H^?(C)\otimes \mathcal{A}]\oplus [\Tor_1(H^?(C),\mathcal{A})]\]
where you determine the indices marked "?".
\end{problem}
\begin{solution}
    
\end{solution}


\noindent\rule{7in}{2.8pt}
%%%%%%%%%%%%%%%%%%%%%%%%%%%%%%%%%%%%%%%%%%%%%%%%%%%%%%%%%%%%%%%%%%%%%%%%%%%%%%%%%%%%%%%%%%%%%%%%%%%%%%%%%%%%%%%%%%%%%%%%%%%%%%%%%%%%%%%%
%Probelm 4
%%%%%%%%%%%%%%%%%%%%%%%%%%%%%%%%%%%%%%%%%%%%%%%%%%%%%%%%%%%%%%%%%%%%%%%%%%%%%%%%%%%%%%%%%%%%%%%%%%%%%%%%%%%%%%%%%%%%%%%%%%%%%%%%%%%%%%%%
\begin{problem}{4}
Suppose \(X\) is a finite CW complex for which 
\[H_0(X;\mathbb{Z}/2)=\mathbb{Z}/2, H_1(X;\mathbb{Z}/2)=(\mathbb{Z}/2)^3, H_2(X;\mathbb{Z}/2)=0, H_3(X;\mathbb{Z}/2)=H_4(X;\mathbb{Z}/2)=\mathbb{Z}/2\] 
and \(H_i(X;\mathbb{Z}/2)=0\) for all \(i\geq 5\).
\begin{enumerate}[(a)]
\item Determine as much as you can about \(H_*(X;\mathbb{Z})\).
\item Suppose you are also told that \(H_2(X;\mathbb{Z}/3)=\mathbb{Z}/3\) and \(H_3(X;\mathbb{Z}/3)=0\). What else can you say about \(H_*(X;\mathbb{Z})\) now? 
\item Suppose \(Y\) is a space with finitely-generated homology groups and you are told \(H_i(Y;\mathbb{Z}/p)=0\) for a specific prime \(p\). What can you deduce about \(H_i(Y)\) and \(H_{i-1}(Y)\)? 
\end{enumerate}
\end{problem}
\begin{solution}

\end{solution}

\noindent\rule{7in}{2.8pt}
 %%%%%%%%%%%%%%%%%%%%%%%%%%%%%%%%%%%%%%%%%%%%%%%%%%%%%%%%%%%%%%%%%%%%%%%%%%%%%%%%%%%%%%%%%%%%%%%%%%%%%%%%%%%%%%%%%%%%%%%%%%%%%%%%%%%%%%%%
%Probelm 5
%%%%%%%%%%%%%%%%%%%%%%%%%%%%%%%%%%%%%%%%%%%%%%%%%%%%%%%%%%%%%%%%%%%%%%%%%%%%%%%%%%%%%%%%%%%%%%%%%%%%%%%%%%%%%%%%%%%%%%%%%%%%%%%%%%%%%%%%
\begin{problem}{5}
\begin{enumerate}[(a)]
\item 
\end{enumerate}
\end{problem}
 %%%%%%%%%%%%%%%%%%%%%%%%%%%%%%%%%%%%%%%%%%%%%%%%%%%%%%%%%%%%%%%%%%%%%%%%%%%%%%%%%%%%%%%%%%%%%%%%%%%%%%%%%%%%%%%%%%%%%%%%%%%%%%%%%%%%%%%%
%Probelm 6
%%%%%%%%%%%%%%%%%%%%%%%%%%%%%%%%%%%%%%%%%%%%%%%%%%%%%%%%%%%%%%%%%%%%%%%%%%%%%%%%%%%%%%%%%%%%%%%%%%%%%%%%%%%%%%%%%%%%%%%%%%%%%%%%%%%%%%%%



\end{document}
\documentclass[letterpaper, 12pt]{article}

\usepackage{/Users/zhengz/Desktop/Math/Workspace/Homework1/homework}

\begin{document}
\noindent

\large\textbf{Zhengdong Zhang} \hfill \textbf{Homework 4} \\
Email: zhengz@uoregon.edu \hfill ID: 952091294 \\
\normalsize Course: MATH 636 - Algebraic Topology III \hfill Term: Spring 2025 \\
Instructor: Dr. Daniel Dugger \hfill Due Date: $02^{nd}$ May 2025 \\ 
\noindent\rule{7in}{2.8pt}
\setstretch{1.1}

%%%%%%%%%%%%%%%%%%%%%%%%%%%%%%%%%%%%%%%%%%%%%%%%%%%%%%%%%%%%%%%%%%%%%%%%%%%%%%%%%%%%%%%%%%%%%%%%%%%%%%%%%%%%%%%%%%%%%%%%%
% Problem 1
%%%%%%%%%%%%%%%%%%%%%%%%%%%%%%%%%%%%%%%%%%%%%%%%%%%%%%%%%%%%%%%%%%%%%%%%%%%%%%%%%%%%%%%%%%%%%%%%%%%%%%%%%%%%%%%%%%%%%%%%%%
\begin{problem}{1}
Compute all the homology groups for the spaces in parts (a) and (b) below, and use your calculations to show that 
\begin{enumerate}[(a)]
\item \(\mathbb{R}P^2\times S^3\) and \(\mathbb{R}P^3\times S^2\) have isomorphic homotopy groups (in all dimensions), but non-isomorphic homology groups. 
\item \(S^4\times S^2\) and \(\mathbb{C}P^3\) have isomorphic homology groups but non-isomorphic homotopy groups. 
\end{enumerate}
\end{problem}
\begin{solution}
\begin{enumerate}[(a)]
\item Let \(X=\mathbb{R}P^2\times S^3\) and \(Y=\mathbb{R}P^3\times S^2\). It is easy to see that both \(X\) and \(Y\) are path-connected, so \(\pi_0(X)=\pi_0(Y)=*\). By direct calculation, we have 
\begin{align*}
	\pi_1(X)&=\pi_1(\mathbb{R}P^2)\times \pi_1(S^3)=\mathbb{Z}/2,\\ 
	\pi_1(Y)&=\pi_1(\mathbb{R}P^3)\times \pi_1(S^2)=\mathbb{Z}/2.
\end{align*}
This implies \(\pi_1(X)\cong \pi_1(Y)\). Recall that for all \(n\geq 2\), the universal covering space of \(\mathbb{R}P^n\) is \(S^n\). So the universal covering space of \(X\) and \(Y\) are both isomorphic to 
\(S^2\times S^3\cong S^3\times S^2\). The long exact sequence in homotopy groups tells us that 
\[\pi_n(X)\cong \pi_n(Y)\cong \pi_n(S^3\times S^2)\]
for all \(n\geq 2\). Thus, we can conclude that \(X\) and \(Y\) have the same homotopy groups. 

For the homology groups, note that the homology groups of \(S^3\) and \(S^2\) are all free. By Künneth theorem, we have
\begin{align*}
	H_n(X)&=\bigoplus_{p+q=n}H_p(\mathbb{R}P^2)\otimes H_q(S^3),\\ 
	H_n(Y)&=\bigoplus_{p+q=n}H_p(\mathbb{R}P^3)\otimes H_q(S^2).
\end{align*} 
The homology groups of each space is listed below:
% Please add the following required packages to your document preamble:
% \usepackage{graphicx}
\begin{table}[h]
	\centering
	\resizebox{0.6\columnwidth}{!}{%
	\(\begin{array}{l|c|c|ll|c|c|c|}
	\cline{2-3} \cline{6-8}
							& H_*(\mathbb{R}P^2) & H_*(S^3)   &  &  &   & H_*(\mathbb{R}P^3) & H_*(S^2)   \\ \cline{1-3} \cline{6-8} 
	\multicolumn{1}{|l|}{3} & 0                  & \mathbb{Z} &  &  & 3 & \mathbb{Z}         & 0          \\ \cline{1-3} \cline{6-8} 
	\multicolumn{1}{|l|}{2} & 0                  & 0          &  &  & 2 & 0                  & \mathbb{Z} \\ \cline{1-3} \cline{6-8} 
	\multicolumn{1}{|l|}{1} & \mathbb{Z}/2       & 0          &  &  & 1 & \mathbb{Z}/2       & 0          \\ \cline{1-3} \cline{6-8} 
	\multicolumn{1}{|l|}{0} & \mathbb{Z}         & \mathbb{Z} &  &  & 0 & \mathbb{Z}         & \mathbb{Z} \\ \cline{1-3} \cline{6-8} 
	\end{array}\)%
	}
	\end{table}
We can obtain the of \(X\) and \(Y\) by tensoring at each degree, this gives us 
\begin{multicols}{2}
\noindent
\[H_i(X)=\begin{cases}
	\mathbb{Z},&\iif i=0,3;\\ 
	\mathbb{Z}/2,&\iif i=1,4;\\
	0,&\otherwise.
\end{cases}\]
\noindent
\[H_i(Y)=\begin{cases}
	\mathbb{Z},&\iif i=0,2,5;\\
    \mathbb{Z}/2,&\iif i=1;\\ 
	\mathbb{Z}\oplus \mathbb{Z}/2,&\iif i=3;\\ 
	0,&\otherwise.
\end{cases}\]
\end{multicols}
From this we can see that \(X\) and \(Y\) have non-isomorphic homology groups.
\item We know that \(\mathbb{C}P^3\) has a cellular structure with one \(0\)-cell, one \(2\)-cell, one \(4\)-cell, and one \(6\)-cell. The boundary maps in the cellular chain complex are all zero, so \(H_i(\mathbb{C}P^3)=\mathbb{Z}\) for \(i=0,2,4,6\) and \(0\) otherwise. For the space \(S^4\times S^2\), use Künneth theorem and note that \(S^2\) does not have torsion in homology, so \(H_i(S^4\times S^2)=\mathbb{Z}\) for \(i=0,2,4,6\) and \(0\) otherwise. This shows that \(S^4\times S^2\) and \(\mathbb{C}P^3\) have isomorphic homology groups. 

Recall that we have a fibration \(S^1\rightarrow S^7\rightarrow \mathbb{C}P^3\). This induces a long exact sequence in homotopy groups
\[\cdots\rightarrow \pi_3(S^1) \rightarrow \pi_3(S^7)\rightarrow \pi_3(\mathbb{C}P^3)\rightarrow \pi_2(S^1)\rightarrow \cdots\]
Note that \(\pi_3(S^1)=\pi_2(S^1)\) is trivial. This implies that \(\pi_3(\mathbb{C}P^3)\cong \pi_3(S^7)=\left\{ 1 \right\}\) is also trivial. On the other hand, we know that 
\[\pi_3(S^4\times S^2)\cong \pi_3(S^4)\times \pi_3(S^2)=\mathbb{Z}.\]
This implies that \(S^4\times S^2\) and \(\mathbb{C}P^3\) have non-isomorphic homotopy groups.
\end{enumerate}
\end{solution}

\noindent\rule{7in}{2.8pt}
%%%%%%%%%%%%%%%%%%%%%%%%%%%%%%%%%%%%%%%%%%%%%%%%%%%%%%%%%%%%%%%%%%%%%%%%%%%%%%%%%%%%%%%%%%%%%%%%%%%%%%%%%%%%%%%%%%%%%%%%%
% Problem 2
%%%%%%%%%%%%%%%%%%%%%%%%%%%%%%%%%%%%%%%%%%%%%%%%%%%%%%%%%%%%%%%%%%%%%%%%%%%%%%%%%%%%%%%%%%%%%%%%%%%%%%%%%%%%%%%%%%%%%%%%%%
\begin{problem}{2}
Let \(I_*\) be the chain complex concentrated in degree \(0\) and \(1\) with \(I_1=\mathbb{Z}\la e\ra\), \(I_0=\mathbb{Z}\la a,b\ra\), and \(d(e)=b-a\). Note that this is the 
simplicial chain complex for \(\Delta_1\). Let \(C_*\) and \(D_*\) be chain complexes. 
\begin{enumerate}[(a)]
\item Describe the chain complex \(I_*\otimes C_*\) by giving the groups in each degree as well as the boundary maps.
\item Let \(F:I_*\otimes C_*\rightarrow D_*\) be a chain map. Define \(f,g:C_*\rightarrow D_*\) by \(f(x)=F(a\otimes x)\) and \(g(x)=F(b\otimes x)\). Likewise, define \(s_n:C_n\rightarrow D_{n+1}\) by \(s_n:C_n\rightarrow D_{n+1}\) by \(s_n(x)=F(e\otimes x)\). Prove 
that \(f\) and \(g\) are chain maps and the collection \(\left\{ s_n \right\}\) is a chain homotopy between \(f\) and \(g\). 
\end{enumerate}
\end{problem}
\begin{solution}
\begin{enumerate}[(a)]
\item We denote both the boundary map in \(C_*\) by \(d_C\). Consider the double complex \(I_*\otimes C_*\) first.
% https://q.uiver.app/#q=WzAsMTIsWzAsMywiSV8wXFxvdGltZXMgQ18wIl0sWzAsMiwiSV8wXFxvdGltZXMgQ18xIl0sWzAsMSwiSV8wXFxvdGltZXMgQ18yIl0sWzEsMywiSV8xXFxvdGltZXMgQ18wIl0sWzEsMiwiSV8xXFxvdGltZXMgQ18xIl0sWzEsMSwiSV8xXFxvdGltZXMgQ18yIl0sWzAsMCwiXFx2ZG90cyJdLFsxLDAsIlxcdmRvdHMiXSxbMiwzLCIwIl0sWzIsMiwiMCJdLFsyLDEsIjAiXSxbMiwwLCJcXHZkb3RzIl0sWzYsMl0sWzIsMV0sWzEsMF0sWzUsMl0sWzQsMV0sWzMsMF0sWzcsNV0sWzEwLDVdLFs5LDRdLFs1LDRdLFs0LDNdLFs4LDNdLFsxMCw5XSxbOSw4XSxbMTEsMTBdXQ==
\[\begin{tikzcd}
	\vdots & \vdots & \vdots \\
	{I_0\otimes C_2} & {I_1\otimes C_2} & 0 \\
	{I_0\otimes C_1} & {I_1\otimes C_1} & 0 \\
	{I_0\otimes C_0} & {I_1\otimes C_0} & 0
	\arrow[from=1-1, to=2-1]
	\arrow[from=1-2, to=2-2]
	\arrow[from=1-3, to=2-3]
	\arrow[from=2-1, to=3-1]
	\arrow[from=2-2, to=2-1]
	\arrow[from=2-2, to=3-2]
	\arrow[from=2-3, to=2-2]
	\arrow[from=2-3, to=3-3]
	\arrow[from=3-1, to=4-1]
	\arrow[from=3-2, to=3-1]
	\arrow[from=3-2, to=4-2]
	\arrow[from=3-3, to=3-2]
	\arrow[from=3-3, to=4-3]
	\arrow[from=4-2, to=4-1]
	\arrow[from=4-3, to=4-2]
\end{tikzcd}\]
The vertical boundary map \(d_v\) is \(id\otimes d_C\) and the horizontal boundary map \(d_h\) is \(d\otimes id\). Let \(T_*\) be the total complex of this double complex, then in each degree we have 
\[T_n=I_0\otimes C_n\oplus I_1\otimes C_{n-1}.\]
We know that \(I_0=\mathbb{Z}\la a,b\ra\), so \(I_0\times C_n\) is isomorphic to \((C_n)^2\) where the isomorphism is given by sending \(a\otimes x\) to \(x\) and \(b\otimes y\) to \(y\) for all \(x,y\in C_n\). Similarly, \(I_1=\mathbb{Z}\la e\ra\), so \(I_1\otimes C_{n-1}\) is isomorphic to \(C_{n-1}\) where the isomorphism is given by sending \(e\otimes z\) to \(z\) for all \(z\in C_{n-1}\). The boundary map in the total complex is given by \(d_t(x)=d_h(x)+(-1)^p d_v(x)\) for \(x\in I_p\otimes C_q\). For \(a\otimes x\), \(b\otimes y\) in \(I_0\otimes C_n\) and \(e\otimes z\in I_1\otimes C_{n-1}\), we have 
\begin{align*}
	d_t(a\otimes x)&=d_C(x),\\ 
	d_t(b\otimes y)&=d_C(y),\\
	d_t(e\otimes z)&=(b-a)\otimes x-e\otimes d_C(z).
\end{align*}
\item Write the boundary maps in \(C_*\) as \(d_C\) and boundary maps in \(D_*\) as \(d_D\). For any \(n\in \mathbb{Z}\), we know \(F\) is a chain map, so we have a commutative diagram 
% https://q.uiver.app/#q=WzAsNCxbMCwwLCJJXzBcXG90aW1lcyBDX24iXSxbMSwwLCJJXzBcXG90aW1lcyBDX3tuLTF9Il0sWzAsMSwiRF9uIl0sWzEsMSwiRF97bi0xfSJdLFswLDEsImlkXFxvdGltZXMgZF9DIl0sWzAsMiwiRiIsMl0sWzIsMywiZF9EIiwyXSxbMSwzLCJGIl1d
\[\begin{tikzcd}
	{I_0\otimes C_n} & {I_0\otimes C_{n-1}} \\
	{D_n} & {D_{n-1}}
	\arrow["{id\otimes d_C}", from=1-1, to=1-2]
	\arrow["F"', from=1-1, to=2-1]
	\arrow["F", from=1-2, to=2-2]
	\arrow["{d_D}"', from=2-1, to=2-2]
\end{tikzcd}\]
For any \(x\in C_n\), we have  
\[(d_D\circ F)(a\otimes x)=[F\circ (id\otimes d_C)(a\otimes x)]=F(a\otimes d_C(x)).\]
By definition this is equivalent to 
\[(d_D\circ f)(x)=(f\circ d_C)(x).\]
Namely, we have a commutative diagram 
% https://q.uiver.app/#q=WzAsNCxbMCwwLCJDX24iXSxbMSwwLCJDX3tuLTF9Il0sWzAsMSwiRF9uIl0sWzEsMSwiRF97bi0xfSJdLFswLDEsImRfQyJdLFsyLDMsImRfRCIsMl0sWzAsMiwiZiIsMl0sWzEsMywiZiJdXQ==
\[\begin{tikzcd}
	{C_n} & {C_{n-1}} \\
	{D_n} & {D_{n-1}}
	\arrow["{d_C}", from=1-1, to=1-2]
	\arrow["f"', from=1-1, to=2-1]
	\arrow["f", from=1-2, to=2-2]
	\arrow["{d_D}"', from=2-1, to=2-2]
\end{tikzcd}\]
This proves \(f\) is a chain map. By a similar argument, \(g\) is also a chain map. 

Next, to show that \(s_n\) defines a chain homotopy between \(f\) and \(g\), we need to show for any \(n\), there exists a commutative diagram 
% https://q.uiver.app/#q=WzAsMTAsWzAsMCwiXFxjZG90cyJdLFsxLDAsIkNfe24rMX0iXSxbMywwLCJDX24iXSxbNSwwLCJDX3tuLTF9Il0sWzYsMCwiXFxjZG90cyJdLFswLDEsIlxcY2RvdHMiXSxbMSwxLCJEX3tuKzF9Il0sWzMsMSwiRF9uIl0sWzUsMSwiRF97bi0xfSJdLFs2LDEsIlxcY2RvdHMiXSxbMCwxXSxbMSwyLCJkX0MiXSxbMiwzLCJkX0MiXSxbMyw0XSxbNSw2XSxbNiw3LCJkX0QiLDJdLFs3LDgsImRfRCIsMl0sWzgsOV0sWzEsNiwiZiIsMix7Im9mZnNldCI6MX1dLFsxLDYsImciLDAseyJvZmZzZXQiOi0xfV0sWzIsNywiZiIsMix7Im9mZnNldCI6MX1dLFsyLDcsImciLDAseyJvZmZzZXQiOi0xfV0sWzMsOCwiZiIsMix7Im9mZnNldCI6MX1dLFszLDgsImciLDAseyJvZmZzZXQiOi0xfV0sWzIsNiwic19uIiwxLHsibGFiZWxfcG9zaXRpb24iOjQwfV0sWzMsNywic197bi0xfSIsMV1d
\[\begin{tikzcd}
	\cdots & {C_{n+1}} && {C_n} && {C_{n-1}} & \cdots \\
	\cdots & {D_{n+1}} && {D_n} && {D_{n-1}} & \cdots
	\arrow[from=1-1, to=1-2]
	\arrow["{d_C}", from=1-2, to=1-4]
	\arrow["f"', shift right, from=1-2, to=2-2]
	\arrow["g", shift left, from=1-2, to=2-2]
	\arrow["{d_C}", from=1-4, to=1-6]
	\arrow["{s_n}"{description, pos=0.4}, from=1-4, to=2-2]
	\arrow["f"', shift right, from=1-4, to=2-4]
	\arrow["g", shift left, from=1-4, to=2-4]
	\arrow[from=1-6, to=1-7]
	\arrow["{s_{n-1}}"{description}, from=1-6, to=2-4]
	\arrow["f"', shift right, from=1-6, to=2-6]
	\arrow["g", shift left, from=1-6, to=2-6]
	\arrow[from=2-1, to=2-2]
	\arrow["{d_D}"', from=2-2, to=2-4]
	\arrow["{d_D}"', from=2-4, to=2-6]
	\arrow[from=2-6, to=2-7]
\end{tikzcd}\]
For any \(x\in C_n\), we have 
\[g(x)-f(x)=F(b\otimes x)-F(a\otimes x)=F((b-a)\otimes x).\]
On the other hand, use the fact that \(F\) is a chain map, we have 
\begin{align*}
	(d_D\circ s_n)(x)+(s_{n-1}\circ d_C)(x)&=(d_D\circ F)(e\otimes x)+F(e\otimes d_C(x))\\ 
	                                       &=(F\circ d_C)(e\otimes x)+F(e\otimes d_C(x))\\ 
										   &=F((b-a)\otimes x)-F(e\otimes d_C(x))+F(e\otimes d_C(x))\\
										   &=F((b-a)\otimes x).
\end{align*}
This proves that 
\[g-f=d_D\circ s_n+s_{n-1}\circ d_C.\]
The collection of \(s_n\) is a chain homotopy between \(f\) and \(g\).
\end{enumerate}
\end{solution}

\noindent\rule{7in}{2.8pt}
%%%%%%%%%%%%%%%%%%%%%%%%%%%%%%%%%%%%%%%%%%%%%%%%%%%%%%%%%%%%%%%%%%%%%%%%%%%%%%%%%%%%%%%%%%%%%%%%%%%%%%%%%%%%%%%%%%%%%%%%%
% Problem 3
%%%%%%%%%%%%%%%%%%%%%%%%%%%%%%%%%%%%%%%%%%%%%%%%%%%%%%%%%%%%%%%%%%%%%%%%%%%%%%%%%%%%%%%%%%%%%%%%%%%%%%%%%%%%%%%%%%%%%%%%%%
\begin{problem}{3}
Let \(Y\) be the space obtained by starting with \(S^3\) and attaching a \(4\)-cell via a map of degree \(5:Y=S^3\cup_f e^4\) where \(f:\partial (e^4)\rightarrow S^3\) has degree \(5\). 
Write down the cellular chain complex for \(\mathbb{R}P^3\times Y\); in particular, specify the rank of each chain group and identify the boundary maps. Compute the homotopy groups of specify the rank of each chain group and 
identify the boundary maps. Compute the homology groups of \(\mathbb{R}P^3\times Y\).
\end{problem}
\begin{solution}
The space \(Y\) has a cellular chain complex 
\[0\rightarrow \mathbb{Z}\la e^4\ra\xrightarrow{5}\mathbb{Z}\la e^3\ra\rightarrow 0\rightarrow 0\rightarrow \mathbb{Z}\la e^0\ra\rightarrow 0.\]
where \(e^i\) are cells in \(Y\) for \(i=0,3,4\). The real projective space \(\mathbb{R}P^3\) has the following cellular chain complex 
\[0\rightarrow \mathbb{Z}\la f^3\ra\xrightarrow{0}\mathbb{Z}\la f^2\ra\xrightarrow{2}\mathbb{Z}\la f^1\ra\xrightarrow{0}\mathbb{Z}\la f^0\ra\rightarrow 0.\]
The tensor product of these two chain complex is the double complex 
% https://q.uiver.app/#q=WzAsMjAsWzAsNCwiXFxtYXRoYmJ7Wn1cXGxhIGZeMFxcb3RpbWVzIGVeMFxccmEiXSxbMCwzLCIwIl0sWzEsNCwiXFxtYXRoYmJ7Wn1cXGxhIGZeMVxcb3RpbWVzIGVeMFxccmEiXSxbMiw0LCJcXG1hdGhiYntafVxcbGEgZl4yXFxvdGltZXMgZV4wXFxyYSAiXSxbMyw0LCJcXG1hdGhiYntafVxcbGEgZl4zXFxvdGltZXMgZV4wXFxyYSJdLFswLDIsIjAiXSxbMSwzLCIwIl0sWzEsMiwiMCJdLFsyLDMsIjAiXSxbMiwyLCIwIl0sWzMsMywiMCJdLFszLDIsIjAiXSxbMCwxLCJcXG1hdGhiYntafVxcbGEgZl4wXFxvdGltZXMgZV4zXFxyYSJdLFsxLDEsIlxcbWF0aGJie1p9XFxsYSBmXjFcXG90aW1lcyBlXjNcXHJhIl0sWzIsMSwiXFxtYXRoYmJ7Wn1cXGxhIGZeMlxcb3RpbWVzIGVeM1xccmEiXSxbMywxLCJcXG1hdGhiYntafVxcbGEgZl4zXFxvdGltZXMgZV4zXFxyYSJdLFswLDAsIlxcbWF0aGJie1p9XFxsYSBmXjBcXG90aW1lcyBlXjRcXHJhIl0sWzEsMCwiXFxtYXRoYmJ7Wn1cXGxhIGZeMVxcb3RpbWVzIGVeNFxccmEiXSxbMiwwLCJcXG1hdGhiYntafVxcbGEgZl4yXFxvdGltZXMgZV40XFxyYSJdLFszLDAsIlxcbWF0aGJie1p9XFxsYSBmXjNcXG90aW1lcyBlXjRcXHJhIl0sWzIsMCwiMCIsMl0sWzMsMiwiMlxcb3RpbWVzIGlkIiwyXSxbNCwzLCIwIiwyXSxbMTMsMTIsIjAiLDJdLFsxNCwxMywiMlxcb3RpbWVzIGlkIiwyXSxbMTUsMTQsIjAiLDJdLFsxNywxNiwiMCIsMl0sWzE4LDE3LCIyXFxvdGltZXMgaWQiLDJdLFsxOSwxOCwiMCIsMl0sWzEsMF0sWzUsMV0sWzEyLDVdLFsxNiwxMiwiaWRcXG90aW1lcyA1IiwyXSxbMTcsMTMsImlkXFxvdGltZXMgNSIsMl0sWzE4LDE0LCJpZFxcb3RpbWVzIDUiLDJdLFsxOSwxNSwiaWRcXG90aW1lcyA1IiwyXSxbMTMsN10sWzcsNl0sWzYsMl0sWzE0LDldLFs5LDhdLFs4LDNdLFsxNSwxMV0sWzExLDEwXSxbMTAsNF0sWzExLDldLFs5LDddLFs3LDVdLFs2LDFdLFs4LDZdLFsxMCw4XV0=
\[\begin{tikzcd}
	{\mathbb{Z}\la f^0\otimes e^4\ra} & {\mathbb{Z}\la f^1\otimes e^4\ra} & {\mathbb{Z}\la f^2\otimes e^4\ra} & {\mathbb{Z}\la f^3\otimes e^4\ra} \\
	{\mathbb{Z}\la f^0\otimes e^3\ra} & {\mathbb{Z}\la f^1\otimes e^3\ra} & {\mathbb{Z}\la f^2\otimes e^3\ra} & {\mathbb{Z}\la f^3\otimes e^3\ra} \\
	0 & 0 & 0 & 0 \\
	0 & 0 & 0 & 0 \\
	{\mathbb{Z}\la f^0\otimes e^0\ra} & {\mathbb{Z}\la f^1\otimes e^0\ra} & {\mathbb{Z}\la f^2\otimes e^0\ra } & {\mathbb{Z}\la f^3\otimes e^0\ra}
	\arrow["{id\otimes 5}"', from=1-1, to=2-1]
	\arrow["0"', from=1-2, to=1-1]
	\arrow["{id\otimes 5}"', from=1-2, to=2-2]
	\arrow["{2\otimes id}"', from=1-3, to=1-2]
	\arrow["{id\otimes 5}"', from=1-3, to=2-3]
	\arrow["0"', from=1-4, to=1-3]
	\arrow["{id\otimes 5}"', from=1-4, to=2-4]
	\arrow[from=2-1, to=3-1]
	\arrow["0"', from=2-2, to=2-1]
	\arrow[from=2-2, to=3-2]
	\arrow["{2\otimes id}"', from=2-3, to=2-2]
	\arrow[from=2-3, to=3-3]
	\arrow["0"', from=2-4, to=2-3]
	\arrow[from=2-4, to=3-4]
	\arrow[from=3-1, to=4-1]
	\arrow[from=3-2, to=3-1]
	\arrow[from=3-2, to=4-2]
	\arrow[from=3-3, to=3-2]
	\arrow[from=3-3, to=4-3]
	\arrow[from=3-4, to=3-3]
	\arrow[from=3-4, to=4-4]
	\arrow[from=4-1, to=5-1]
	\arrow[from=4-2, to=4-1]
	\arrow[from=4-2, to=5-2]
	\arrow[from=4-3, to=4-2]
	\arrow[from=4-3, to=5-3]
	\arrow[from=4-4, to=4-3]
	\arrow[from=4-4, to=5-4]
	\arrow["0"', from=5-2, to=5-1]
	\arrow["{2\otimes id}"', from=5-3, to=5-2]
	\arrow["0"', from=5-4, to=5-3]
\end{tikzcd}\]
Denote the total chain complex by \((T_n,d_n)\). we have 
\[T_n=\begin{cases}
	\mathbb{Z}\la f^0\otimes e^0\ra\cong \mathbb{Z},&\iif n=0;\\ 
	\mathbb{Z}\la f^1\otimes e^0\ra\cong \mathbb{Z},&\iif n=1;\\
	\mathbb{Z}\la f^2\otimes e^0\ra\cong \mathbb{Z},&\iif n=2;\\ 
	\mathbb{Z}\la f^0\otimes e^3, f^3\otimes e^0\ra\cong \mathbb{Z}^2,&\iif n=3;\\ 
	\mathbb{Z}\la f^0\otimes e^4,f^1\otimes e^3\ra\cong \mathbb{Z}^2,&\iif n=4;\\
	\mathbb{Z}\la f^1\otimes e^4,f^2\otimes e^3\ra\cong \mathbb{Z}^2,&\iif n=5;\\ 
	\mathbb{Z}\la f^2\otimes e^4,f^3\otimes e^3\ra\cong \mathbb{Z}^2,&\iif n=6;\\ 
	\mathbb{Z}\la f^3\otimes e^4\ra\cong \mathbb{Z},&\iif n=7. 
\end{cases}\]
For \(1\leq n\leq 7\), the boundary map \(d_n\) is given by the formula 
\[d_n(f^i\otimes e^j)=d(f^i)\otimes e^j+(-1)^i f^i\otimes d(e^j).\]
The boundary map \(d_n\) is given by the following table:
% Please add the following required packages to your document preamble:
% \usepackage{graphicx}
\begin{table}[h]
	\centering
	\resizebox{0.15\columnwidth}{!}{%
	\(\begin{array}{|c|c|}
	\hline
	i & d_i                                                                                          \\[5pt] \hline
	1 & 0                                                                                            \\[5pt]  \hline
	2 & 2                                                                                            \\ [5pt] \hline
	3 & \begin{pmatrix}     
		0&0
	    \end{pmatrix}        \\[5pt]  \hline
	4 & \begin{pmatrix}    
		5&0\\    
		0&0 
	    \end{pmatrix}         \\[5pt]  \hline
	5 & \begin{pmatrix}
		    0&2\\    
			-5&0
		 \end{pmatrix}         \\[5pt]  \hline
	6 &  \begin{pmatrix}
		    2&0\\    
			5&0 
		\end{pmatrix}           \\[5pt]  \hline
	7 & \begin{pmatrix}
		 0\\ 
		 -5
		 \end{pmatrix}           \\[5pt]  \hline
	\end{array}\)%
	}
	\end{table}
To calculate the homology groups of \(\mathbb{R}P^3\times Y\), we first write down the homology groups of \(\mathbb{R}P^3\) and \(Y\). 
% Please add the following required packages to your document preamble:
% \usepackage{graphicx}
\begin{table}[h]
	\centering
	\resizebox{0.3\columnwidth}{!}{%
	\(\begin{array}{|c|c|c|ll}
	\cline{1-3}
	  & H_*(\mathbb{R}P^3) & H_*(Y)       &  &  \\ \cline{1-3}
	0 & \mathbb{Z}         & \mathbb{Z}   &  &  \\ \cline{1-3}
	1 & \mathbb{Z}/2       & 0            &  &  \\ \cline{1-3}
	2 & 0                  & 0            &  &  \\ \cline{1-3}
	3 & \mathbb{Z}         & \mathbb{Z}/5 &  &  \\ \cline{1-3}
	4 & 0                  & 0            &  &  \\ \cline{1-3}
	\end{array}\)%
	}
	\end{table}

We calculate their tensor products and \(\Tor_1\) respectively. Note that \(\Tor_1(\mathbb{Z}/5,\mathbb{Z}/2)=0\), so we do not have any terms coming from \(\Tor_1\). The homology groups of \(\mathbb{R}P^3\times Y\) can be summarized as follows 
\[H_i(\mathbb{R}P^3\times Y)=\begin{cases}
	\mathbb{Z},&\iif i=0;\\
	\mathbb{Z}/2,&\iif i=1;\\ 
	\mathbb{Z}\oplus \mathbb{Z}/5,&\iif i=3;\\
	\mathbb{Z}/5,&\iif i=6;\\
	0,&\otherwise.
\end{cases}\]


\end{solution}

\noindent\rule{7in}{2.8pt}
%%%%%%%%%%%%%%%%%%%%%%%%%%%%%%%%%%%%%%%%%%%%%%%%%%%%%%%%%%%%%%%%%%%%%%%%%%%%%%%%%%%%%%%%%%%%%%%%%%%%%%%%%%%%%%%%%%%%%%%%%
% Problem 4
%%%%%%%%%%%%%%%%%%%%%%%%%%%%%%%%%%%%%%%%%%%%%%%%%%%%%%%%%%%%%%%%%%%%%%%%%%%%%%%%%%%%%%%%%%%%%%%%%%%%%%%%%%%%%%%%%%%%%%%%%%
\begin{problem}{4}
Compute both the homology and cohomology groups of the following spaces, both with integral and \(\mathbb{Z}/2\) coefficients. Heck, do it with \(\mathbb{Z}/3\) coefficients as well. 
\begin{enumerate}[(a)]
\item \(K\times K\), where \(K\) is the Klein bottle. 
\item \(K\times T^g\), where \(T^g\) is the genus \(g\) torus and \(K\) is the Klein bottle. 
\item \(K\times \mathbb{R}P^n\).
\end{enumerate}
\end{problem}
\begin{solution}
\begin{enumerate}[(a)]
\item We can use UCT for homology to calculate the homology groups of \(K\) in different coefficients, and we summarized them as follows. 
% Please add the following required packages to your document preamble:
% \usepackage{graphicx}
% Please add the following required packages to your document preamble:
% \usepackage{graphicx}
% Please add the following required packages to your document preamble:
% \usepackage{graphicx}
\begin{table}[h]
	\centering
	\resizebox{0.5\columnwidth}{!}{%
	\(\begin{array}{|c|c|c|c|}
	\hline
	  & H_*(K)                        & H_*(K;\mathbb{Z}/2)             & H_*(K:\mathbb{Z}/3) \\ \hline
	0 & \mathbb{Z}                    & \mathbb{Z}/2                    & \mathbb{Z}/3        \\ \hline
	1 & \mathbb{Z}\oplus \mathbb{Z}/2 & \mathbb{Z}/2\oplus \mathbb{Z}/2 & \mathbb{Z}/3        \\ \hline
	2 & 0                             & \mathbb{Z}/2                    & 0                   \\ \hline
	\end{array}\)%
	}
	\end{table}

From this we know that the tensor product is 
\[H_p(K)\otimes H_q(K)=\begin{cases}
	\mathbb{Z},&\iif p+q=0;\\ 
	\mathbb{Z}^2\oplus (\mathbb{Z}/2)^2,&\iif p+q=1;\\ 
	\mathbb{Z}\oplus (\mathbb{Z}/2)^3,&\iif p+q=2;\\ 
	0,&\otherwise.
\end{cases}\]
The only non-trivial \(\Tor_1\) is given by \(\Tor_1(H_1(K),H_1(K))=\mathbb{Z}/2\). Thus, the homology groups of \(K\) is
\[H_i(K\times K;\mathbb{Z})=\begin{cases}
	\mathbb{Z},&\iif i=0;\\ 
	\mathbb{Z}^2\oplus (\mathbb{Z}/2)^2,&\iif i=1;\\
	\mathbb{Z}\oplus (\mathbb{Z}/2)^3,&\iif i=2;\\ 
	\mathbb{Z}/2,&\iif i=3;\\
	0,&\otherwise.
\end{cases}\]
Next, we use UCT for homology to calculate the homology groups with \(\mathbb{Z}/2\) and \(\mathbb{Z}/3\) coefficients. 
\begin{multicols}{2}

\[H_i(K\times K;\mathbb{Z}/2)=\begin{cases}
	\mathbb{Z}/2,&\iif i=0;\\
	(\mathbb{Z}/2)^4,&\iif i=1;\\
	(\mathbb{Z}/2)^6,&\iif i=2;\\
	(\mathbb{Z}/2)^4,&\iif i=3;\\
	(\mathbb{Z}/2),&\iif i=4;\\
	0,&\otherwise.
\end{cases}\]

\columnbreak
\null \vfill
\[H_i(K\times K;\mathbb{Z}/3)=\begin{cases}
	\mathbb{Z}/3,&\iif i=0;\\
	(\mathbb{Z}/3)^2,&\iif i=1;\\
	\mathbb{Z}/3,&\iif i=2;\\
	0,&\otherwise.
\end{cases}\]
\vfill \null
\end{multicols}

Lastly, we use UCT for cohomology to calculate the cohomology groups. 
% Please add the following required packages to your document preamble:
% \usepackage{graphicx}
\begin{table}[h]
	\centering
	\resizebox{0.7\columnwidth}{!}{%
	\(\begin{array}{|c|c|c|c|}
	\hline
	  & H^*(K\times K)                    & H^*(K\times K;\mathbb{Z}/2) & H^*(K\times K;\mathbb{Z}/3) \\ \hline
	0 & \mathbb{Z}                        & \mathbb{Z}/2                & \mathbb{Z}/3                \\ \hline
	1 & \mathbb{Z}^2                      & (\mathbb{Z}/2)^4            & (\mathbb{Z}/3)^2            \\ \hline
	2 & \mathbb{Z}\oplus (\mathbb{Z}/2)^2 & (\mathbb{Z}/2)^6            & \mathbb{Z}/3                \\ \hline
	3 & (\mathbb{Z}/2)^3                  & (\mathbb{Z}/2)^4            & 0                           \\ \hline
	4 & \mathbb{Z}/2                      & \mathbb{Z}/2                & 0                           \\ \hline
	\end{array}\)%
	}
	\end{table}
\item The homology of \(K\) and \(T^g\) are as follows:
% Please add the following required packages to your document preamble:
% \usepackage{graphicx}
\begin{table}[h]
	\centering
	\resizebox{0.3\columnwidth}{!}{%
	\(\begin{array}{|c|c|c|}
	\hline
	  & H_*(K)                        & H_*(T^g)        \\ \hline
	0 & \mathbb{Z}                    & \mathbb{Z}      \\ \hline
	1 & \mathbb{Z}\oplus \mathbb{Z}/2 & \mathbb{Z}^{2g} \\ \hline
	2 & 0                             & \mathbb{Z}      \\ \hline
	\end{array}\)%
	}
	\end{table}

Note that \(H_*(T^g)\) are all free, so by Künneth theorem, we have 
\[H_*(K\times T^g)\cong \bigoplus_{i+j=n}H_i(K)\otimes H_j(T^g). \]
Thus, we conclude \(H_*(K\times T^g)\) as follows:
\[H_i(K\times T^g)=\begin{cases}
	\mathbb{Z},&\iif i=0;\\
	\mathbb{Z}^{2g+1}\oplus \mathbb{Z}/2,&\iif i=1;\\
	\mathbb{Z}^{2g+1}\oplus (\mathbb{Z}/2)^{2g},&\iif i=2;\\
	\mathbb{Z}\oplus \mathbb{Z}/2,&\iif i=3;\\
	0,&\otherwise.
\end{cases}\]
Next, we use UCT for homology to calculate different coefficients.
\begin{multicols}{2}

\[H_i(K\times T^g;\mathbb{Z}/2)=\begin{cases}
	\mathbb{Z}/2,&\iif i=0;\\
	(\mathbb{Z}/2)^{2g+2},&\iif i=1;\\
	(\mathbb{Z}/2)^{4g+2},&\iif i=2;\\
	(\mathbb{Z}/2)^{2g+2},&\iif i=3;\\
	\mathbb{Z}/2,&\iif i=4;\\
	0,&\otherwise.
\end{cases}\]

\columnbreak

\null \vfill
\[H_i(K\times T^g;\mathbb{Z}/3)=\begin{cases}
	\mathbb{Z}/3,&\iif i=0;\\
	(\mathbb{Z}/3)^{2g+1},&\iif i=1;\\
	(\mathbb{Z}/3)^{2g+1},&\iif i=2;\\
	\mathbb{Z}/3,&\iif i=3;\\
	0,&\otherwise.
\end{cases}\]
\vfill \null 
\end{multicols}

Lastly, we use UCT to calculate the cohomology groups.
% Please add the following required packages to your document preamble:
% \usepackage{graphicx}
\begin{table}[h]
	\centering
	\resizebox{0.7\columnwidth}{!}{%
	\(\begin{array}{|c|c|c|c|}
	\hline
	  & H^*(K\times T^g)                     & H^*(K\times T^g;\mathbb{Z}/2) & H^*(K\times T^g;\mathbb{Z}/3) \\ \hline
	0 & \mathbb{Z}                           & \mathbb{Z}/2                  & \mathbb{Z}/3                  \\ \hline
	1 & \mathbb{Z}^{2g+1}                    & (\mathbb{Z}/2)^{2g+2}         & (\mathbb{Z}/3)^{2g+1}         \\ \hline
	2 & \mathbb{Z}^{2g+1}\oplus \mathbb{Z}/2 & (\mathbb{Z}/2)^{4g+2}         & (\mathbb{Z}/3)^{2g+1}         \\ \hline
	3 & \mathbb{Z}\oplus (\mathbb{Z}/2)^{2g} & (\mathbb{Z}/2)^{2g+2}         & \mathbb{Z}/3                  \\ \hline
	4 & \mathbb{Z}/2                         & \mathbb{Z}/2                  & 0                             \\ \hline
	\end{array}\)%
	}
	\end{table}

\item \(\mathbb{R}P^n\) has different homology groups when \(n\) is odd or even. 
\begin{enumerate}[(1)]
\item Suppose \(n\geq 2\) is even.\\ 
The homology groups of \(\mathbb{R}P^n\) can be summarized as follows:
\begin{itemize}
	\item \(H_0(\mathbb{R}P^n)=\mathbb{Z}\).
	\item \(H_i(\mathbb{R}P^n)=\mathbb{Z}/2\) if \(i\geq 0\) and \(i\) is odd.
	\item \(0\) otherwise.
\end{itemize}
Use Künneth theorem, we can calculate the homology groups of \(K\times \mathbb{R}P^n\):
\[H_i(K\times \mathbb{R}P^n)=\begin{cases}
	\mathbb{Z},&\iif i=0;\\
	\mathbb{Z}\oplus (\mathbb{Z}/2)^2,&\iif i=1;\\
	(\mathbb{Z}/2)^2,&\iif 2\leq i\leq n;\\
	\mathbb{Z}/2,&\iif i=n+1;\\
	0,&\otherwise.
\end{cases}\]
Next, we use UCT for homology to calculate the homology groups with other coefficients. 

\[H_i(K\times \mathbb{R}P^n;\mathbb{Z}/2)=\begin{cases}
	\mathbb{Z}/2,&\iif i=0;\\
	(\mathbb{Z}/2)^3,&\iif i=1;\\
	(\mathbb{Z}/2)^4,&\iif 2\leq i\leq n;\\
	(\mathbb{Z}/2)^3,&\iif i=n+1;\\
	\mathbb{Z}/2,&\iif i=n+2;\\
	0,&\otherwise.
\end{cases}\]

\[H_i(K\times \mathbb{R}P^n;\mathbb{Z}/3)=\begin{cases}
	\mathbb{Z}/3,&\iif i=0,1;\\
	0,&\otherwise.
\end{cases}\] 

Lastly, we use UCT for cohomology to calculate the cohomology groups.
% Please add the following required packages to your document preamble:
% \usepackage{graphicx}
\begin{table}[h]
	\centering
	\resizebox{0.7\columnwidth}{!}{%
	\(\begin{array}{|c|c|c|c|}
	\hline
		   & H^*(K\times \mathbb{R}P^n) & H^*(K\times \mathbb{R}P^n;\mathbb{Z}/2) & H^*(K\times \mathbb{R}P^n;\mathbb{Z}/3) \\ \hline
	0      & \mathbb{Z}                 & \mathbb{Z}/2                            & \mathbb{Z}/3                            \\ \hline
	1      & \mathbb{Z}                 & (\mathbb{Z}/2)^3                        & \mathbb{Z}/3                            \\ \hline
	2      & (\mathbb{Z}/2)^2           & (\mathbb{Z}/2)^4                        & 0                                       \\ \hline
	\vdots & (\mathbb{Z}/2)^2           & (\mathbb{Z}/2)^4                        & 0                                       \\ \hline
	n      & (\mathbb{Z}/2)^2           & (\mathbb{Z}/2)^4                        & 0                                       \\ \hline
	n+1    & (\mathbb{Z}/2)^2           & (\mathbb{Z}/2)^3                        & 0                                       \\ \hline
	n+2    & \mathbb{Z}/2               & \mathbb{Z}/2                            & 0                                       \\ \hline
	\end{array}\)%
	}
	\end{table}

\item Suppose \(n\geq 3\) is odd.\\
The homology groups of \(\mathbb{R}P^n\) can be summarized as follows:
\begin{itemize}
	\item \(H_0(\mathbb{R}P^n)=H_n(\mathbb{R}P^n)=\mathbb{Z}\).
	\item \(H_i(\mathbb{R}P^n)=\mathbb{Z}/2\) for \(1\leq i\leq n-1\) if \(i\) is odd.
	\item \(0\) otherwise.
\end{itemize}
Use Künneth theorem, we can calculate the homology groups of \(K\times \mathbb{R}P^n\):
\[H_i(K\times \mathbb{R}P^n)=\begin{cases}
	\mathbb{Z},&\iif i=0;\\
	\mathbb{Z}\oplus (\mathbb{Z}/2)^2,&\iif i=1;\\
	(\mathbb{Z}/2)^2,&\iif 2\leq i\leq n-1;\\
	\mathbb{Z}\oplus \mathbb{Z}/2,&\iif i=n;\\
	\mathbb{Z}\oplus \mathbb{Z}/2,&\iif i=n+1;\\
	0,&\otherwise.
\end{cases}\]
Next, we use UCT for homology to calculate the homology groups with coefficients:
\[H_i(K\times \mathbb{R}P^n;\mathbb{Z}/2)=\begin{cases}
	\mathbb{Z}/2,&\iif i=0;\\
	(\mathbb{Z}/2)^3,&\iif i=1;\\
	(\mathbb{Z}/2)^4,&\iif 2\leq i\leq n;\\
    (\mathbb{Z}/2)^3,&\iif i=n+1;\\
	\mathbb{Z}/2,&\iif i=n+2;\\
	0,&\otherwise.
\end{cases}\]
\[H_i(K\times \mathbb{R}P^n;\mathbb{Z}/3)=\begin{cases}
	\mathbb{Z}/3,&\iif i=0,1,n,n+1;\\
	0,&\otherwise.
\end{cases}\]

Lastly, we use UCT for cohomology to calculate the cohomology groups.
% Please add the following required packages to your document preamble:
% \usepackage{graphicx}
\begin{table}[h]
	\centering
	\resizebox{0.7\columnwidth}{!}{%
	\(\begin{array}{|c|c|c|c|}
	\hline
		   & H^*(K\times \mathbb{R}P^n)        & H^*(K\times \mathbb{R}P^n;\mathbb{Z}/2) & H^*(K\times \mathbb{R}P^n;\mathbb{Z}/3) \\ \hline
	0      & \mathbb{Z}                        & \mathbb{Z}/2                            & \mathbb{Z}/3                            \\ \hline
	1      & \mathbb{Z}                        & (\mathbb{Z}/2)^3                        & \mathbb{Z}/3                            \\ \hline
	2      & (\mathbb{Z}/2)^2                  & (\mathbb{Z}/2)^4                        & 0                                       \\ \hline
	\vdots & (\mathbb{Z}/2)^2                  & (\mathbb{Z}/2)^4                        & 0                                       \\ \hline
	n-1    & (\mathbb{Z}/2)^2                  & (\mathbb{Z}/2)^4                        & 0                                       \\ \hline
	n      & \mathbb{Z}\oplus (\mathbb{Z}/2)^2 & (\mathbb{Z}/2)^4                        & \mathbb{Z}/3                            \\ \hline
	n+1    & \mathbb{Z}\oplus \mathbb{Z}/2     & (\mathbb{Z}/2)^3                        & \mathbb{Z}/3                            \\ \hline
	n+2    & \mathbb{Z}/2                      & \mathbb{Z}/2                            & 0                                       \\ \hline
	\end{array}\)%
	}
	\end{table}

\item The last case is \(n=1\). We have \(\mathbb{R}P^1\cong S^1\). The homology of \(S^1\) is free, so we have 
\[H_i(K\times \mathbb{R}P^1)=\bigoplus_{p+q=i}H_p(K)\otimes H_q(S^1)=\begin{cases}
	\mathbb{Z},&\iif i=0;\\
	\mathbb{Z}^2\oplus \mathbb{Z}/2,&\iif i=1;\\
	\mathbb{Z}\oplus \mathbb{Z}/2,&\iif i=2;\\
	0,&\otherwise.
\end{cases}\]
Use UCT for homology we can calculate the homology groups with different coefficients:
\begin{multicols}{2}
	
\[H_i(K\times  \mathbb{R}P^1;\mathbb{Z}/2)=\begin{cases}
	\mathbb{Z}/2,&\iif i=0;\\
	(\mathbb{Z}/2)^3,&\iif i=1;\\
	(\mathbb{Z}/2)^3,&\iif i=2;\\
	\mathbb{Z}/2,&\iif i=3;\\
	0,&\otherwise.
\end{cases}\]

\columnbreak 
\null \vfill 
\[H_i(K\times \mathbb{R}P^1;\mathbb{Z}/3)=\begin{cases}
	\mathbb{Z}/3,&\iif i=0;\\
	(\mathbb{Z}/3)^2,&\iif i=1;\\
	\mathbb{Z}/3,&\iif i=2;\\
	0,&\otherwise.
\end{cases}\]
\vfill \null
\end{multicols}

Lastly, we use UCT for cohmology to calculate the cohomology groups. 
% Please add the following required packages to your document preamble:
% \usepackage{graphicx}
\begin{table}[h]
	\centering
	\resizebox{0.7\columnwidth}{!}{%
	\(\begin{array}{|c|c|c|c|}
	\hline
	  & H^*(K\times \mathbb{R}P^1)    & H^*(K\times \mathbb{R}P^1;\mathbb{Z}/2) & H^*(K\times \mathbb{R}P^1;\mathbb{Z}/3) \\ \hline
	0 & \mathbb{Z}                    & \mathbb{Z}/2                            & \mathbb{Z}/3                            \\ \hline
	1 & \mathbb{Z}^2                  & (\mathbb{Z}/2)^3                        & (\mathbb{Z}/3)^2                        \\ \hline
	2 & \mathbb{Z}\oplus \mathbb{Z}/2 & (\mathbb{Z}/2)^3                        & \mathbb{Z}/3                            \\ \hline
	3 & \mathbb{Z}/2                  & \mathbb{Z}/2                            & 0                                       \\ \hline
	\end{array}\)%
	}
	\end{table}
\end{enumerate}
\end{enumerate}
\end{solution}

\noindent\rule{7in}{2.8pt}
%%%%%%%%%%%%%%%%%%%%%%%%%%%%%%%%%%%%%%%%%%%%%%%%%%%%%%%%%%%%%%%%%%%%%%%%%%%%%%%%%%%%%%%%%%%%%%%%%%%%%%%%%%%%%%%%%%%%%%%%%
% Problem 5
%%%%%%%%%%%%%%%%%%%%%%%%%%%%%%%%%%%%%%%%%%%%%%%%%%%%%%%%%%%%%%%%%%%%%%%%%%%%%%%%%%%%%%%%%%%%%%%%%%%%%%%%%%%%%%%%%%%%%%%%%%
\begin{problem}{5}
Let \(f:A_*\rightarrow B_*\) be a map of chain complexes. We can regard this as forming a double complex 
% https://q.uiver.app/#q=WzAsOCxbMCwwLCJcXHZkb3RzIl0sWzEsMCwiXFx2ZG90cyJdLFswLDEsIkFfMiJdLFsxLDEsIkJfMiJdLFswLDIsIkFfMSJdLFsxLDIsIkJfMSJdLFswLDMsIkFfMCJdLFsxLDMsIkJfMCJdLFswLDJdLFsxLDNdLFsyLDNdLFsyLDRdLFszLDVdLFs0LDZdLFs2LDddLFs1LDddLFs0LDVdXQ==
\[\begin{tikzcd}
	\vdots & \vdots \\
	{A_2} & {B_2} \\
	{A_1} & {B_1} \\
	{A_0} & {B_0}
	\arrow[from=1-1, to=2-1]
	\arrow[from=1-2, to=2-2]
	\arrow[from=2-1, to=2-2]
	\arrow[from=2-1, to=3-1]
	\arrow[from=2-2, to=3-2]
	\arrow[from=3-1, to=3-2]
	\arrow[from=3-1, to=4-1]
	\arrow[from=3-2, to=4-2]
	\arrow[from=4-1, to=4-2]
\end{tikzcd}\]
by putting zeros in all the "empty" spots. The total complex of this double complex is called the \textbf{algebraic mapping cone} of \(f\), denoted \(Cf\). Specifically, we set \((Cf)_n=A_{n-1}\oplus B_n\) and define \(d:(Cf)_n\rightarrow (Cf)_{n-1}\) by 
\[d(a,b)=(d_A(a), (-1)^{n-1}f(a)+d_B(b))\]
\begin{enumerate}[(a)]
\item Explain why there is a short exact sequence of chain complexes 
\[0\rightarrow B_*\hookrightarrow C(f)\rightarrow \Sigma A_*\rightarrow 0,\]
where \(\Sigma A_*\) is the evident chain complex having \((\Sigma A)_n=A_{n-1}\). 
\item The short exact sequence from (a) gives rise to a long exact sequence in homology groups. This has the form 
\[\cdots \rightarrow H_i(B)\rightarrow H_i(Cf)\rightarrow H_i(\Sigma A)\xrightarrow{\partial}H_{i-1}(B)\rightarrow \cdots\]
Verify that the connecting homomorphism is really just the map \(f_*:H_{i-1}(A)\rightarrow H_{i-1}(B)\), possibly up to a sign. 
\end{enumerate}
\end{problem}
\begin{solution}
\begin{enumerate}[(a)]
\item We need to prove that for any \(n\geq 0\), we have the following commutative diagrams where the top row and bottom row is exact.
% https://q.uiver.app/#q=WzAsMTAsWzAsMCwiMCJdLFsxLDAsIkJfbiJdLFsyLDAsIkFfe24tMX1cXG9wbHVzIEJfbiJdLFszLDAsIkFfe24tMX0iXSxbNCwwLCIwIl0sWzAsMSwiMCJdLFsxLDEsIkJfe24tMX0iXSxbMiwxLCJBX3tuLTJ9XFxvcGx1cyBCX3tuLTF9Il0sWzMsMSwiQV97bi0yfSJdLFs0LDEsIjAiXSxbMCwxXSxbMSwyLCJpX24iXSxbMiwzLCJwX24iXSxbMyw0XSxbNSw2XSxbNiw3LCJpX3tuLTF9IiwyXSxbNyw4LCJwX3tuLTF9IiwyXSxbOCw5XSxbMSw2LCJkX0IiLDJdLFszLDgsImRfQSJdLFsyLDcsImQiXV0=
\[\begin{tikzcd}
	0 & {B_n} & {A_{n-1}\oplus B_n} & {A_{n-1}} & 0 \\
	0 & {B_{n-1}} & {A_{n-2}\oplus B_{n-1}} & {A_{n-2}} & 0
	\arrow[from=1-1, to=1-2]
	\arrow["{i_n}", from=1-2, to=1-3]
	\arrow["{d_B}"', from=1-2, to=2-2]
	\arrow["{p_n}", from=1-3, to=1-4]
	\arrow["d", from=1-3, to=2-3]
	\arrow[from=1-4, to=1-5]
	\arrow["{d_A}", from=1-4, to=2-4]
	\arrow[from=2-1, to=2-2]
	\arrow["{i_{n-1}}"', from=2-2, to=2-3]
	\arrow["{p_{n-1}}"', from=2-3, to=2-4]
	\arrow[from=2-4, to=2-5]
\end{tikzcd}\]
We choose \(i_n:B_n\rightarrow A_{n-1}\oplus B_n\) as the inclusion \(b\mapsto (0,b)\) and \(p_n:A_{n-1}\oplus B_n\rightarrow A_{n-1}\) as the projection \((a,b)\mapsto a\). It is easy to see the top row and the bottom row is exact. For any \((a,b)\in A_{n-1}\oplus B_n\), we have 
\begin{align*}
	(p_{n-1}\circ d)(a,b)&=p_{n-1}(d_A(a),(-1)^{n-1}f(a)+d_B(b))\\ 
	                     &=d_A(a)\\ 
						 &=(d_A\circ p_n)(a,b).
\end{align*} 
This proves the right square commutes. Moreover, for any \(b\in B_n\), we have 
\begin{align*}
	(d\circ i_n)(b)&=d(0,b)\\
	               &=(0,0+d_B(b))\\
				   &=(i_{n-1}\circ d_B)(b).
\end{align*}
This proves the left square commutes. Thus, we have a short exact sequence of chain complex 
\[0\rightarrow B_*\rightarrow (Cf)_*\rightarrow \Sigma A_*\rightarrow 0\]
where \((Cf)_n=A_{n-1}\oplus B_n\) and \((\Sigma A)_n=A_{n-1}\) for all \(n\).
\item By the snake lemma, we have a long exact sequence of homology groups deduced from the short exact sequence of chain complexes 
\[0\rightarrow B_*\rightarrow (Cf)_*\rightarrow \Sigma A_*\rightarrow 0.\]
Take \(a\in \ker d_A\subseteq A_{n-1}\), we specify how to define \(\partial a\in B_{n-1}\) from the snake lemma. We take the preimage \((a,0)\in (Cf)_n\), send it to \(d(a,0)=(0,(-1)^{n-1}f(a))\in (Cf)_{n-1}\), lastly we take the preimage \((-1)^{n-1}f(a)\in B_{n-1}\). Thus, we can conclude that the map 
\begin{align*}
	\partial: H_{n-1}(A)&\rightarrow H_{n-1}(B),\\
	          [a]&\mapsto [(-1)^{n-1}f(a)]
\end{align*}
This implies the connecting homomorphism is just the map induced by \(f\)
\[f_*:H_{n-1}(A)\rightarrow H_{n-1}(B)\]
up to a sign.  
\end{enumerate}
\end{solution}

\noindent\rule{7in}{2.8pt}
%%%%%%%%%%%%%%%%%%%%%%%%%%%%%%%%%%%%%%%%%%%%%%%%%%%%%%%%%%%%%%%%%%%%%%%%%%%%%%%%%%%%%%%%%%%%%%%%%%%%%%%%%%%%%%%%%%%%%%%%%
% Problem 6
%%%%%%%%%%%%%%%%%%%%%%%%%%%%%%%%%%%%%%%%%%%%%%%%%%%%%%%%%%%%%%%%%%%%%%%%%%%%%%%%%%%%%%%%%%%%%%%%%%%%%%%%%%%%%%%%%%%%%%%%%%
\begin{problem}{6}
Let \(k\) be a field, and let \(\mathcal{V}\) denote the category of vector spaces over \(k\). Let \(I\) be any (small) category, and let \(\mathcal{V}^I\) be the category whose objects are functors \(I\rightarrow \mathcal{V}\) and whose morphisms are 
natural transformations. We call \(\mathcal{V}^I\) the category of "\(I\)-shaped diagram in \(\mathcal{V}\)". \\ 
In this problem we will focus on the case where \(I\) is the pushout category 
\[1\leftarrow 0\rightarrow 2\]
with three objects and two non-identity maps (as shown above). An object of \(\mathcal{V}^I\) is then just a diagram of vector spaces \(V_1\leftarrow V_0\rightarrow V_2\). A map from \([V_1\leftarrow V_0\rightarrow V_2]\) to \([W_1\leftarrow W_0\rightarrow W_2]\) 
is a commutative diagram 
% https://q.uiver.app/#q=WzAsNixbMCwwLCJWXzEiXSxbMSwwLCJWXzAiXSxbMiwwLCJWXzIiXSxbMCwxLCJXXzEiXSxbMSwxLCJXXzAiXSxbMiwxLCJXXzIiXSxbMCwzXSxbMSw0XSxbMiw1XSxbMSwwXSxbMSwyXSxbNCw1XSxbNCwzXV0=
\[\begin{tikzcd}
	{V_1} & {V_0} & {V_2} \\
	{W_1} & {W_0} & {W_2}
	\arrow[from=1-1, to=2-1]
	\arrow[from=1-2, to=1-1]
	\arrow[from=1-2, to=1-3]
	\arrow[from=1-2, to=2-2]
	\arrow[from=1-3, to=2-3]
	\arrow[from=2-2, to=2-1]
	\arrow[from=2-2, to=2-3]
\end{tikzcd}\]
Let \(P:\mathcal{V}^I\rightarrow \mathcal{V}\) be the pushout functor. \(P\) assigns each diagram its pushout.
\begin{enumerate}[(a)]
\item Let \(F_1\), \(F_0\) and \(F_2\) be the three diagrams 
\[F_1:[k\leftarrow 0\rightarrow 0]\ \ \ F_0=[k\leftarrow k\rightarrow k]\ \ \ F_2=[0\leftarrow 0\rightarrow k]\]
where in \(F_0\) the maps are the identities. These diagrams are "free" in a certain sense: namely, if \(D\) is an object of \(\mathcal{V}^I\) then morphisms \(F_i\rightarrow D\) are in bijective correspondence with elements of \(D_i\). 
Convince yourself that this is true. 
\item Let \(D=[0\leftarrow k\rightarrow 0]\) and \(E=[0\leftarrow k\rightarrow k]\), where in \(E\) the nontrivial map is the identity. Determine free resolutions for \(D\) and \(E\). 
\item Apply the functor \(P\) to your resolution, to produce a chain complex of vector spaces. Compute the homology groups, which are the groups \((L_iP)(D)\) and \((L_iP)(E)\). These are the derived functor of the pushout functor \(P\). Confirm in your example that \(L_0P=P\). 
\item Now let \(I\) be the category with one object \(0\) and one non-identity map \(t:0\rightarrow 0\) such that \(t^2=id\). Objects of \(\mathcal{V}^I\) are then pairs \((W,t)\) consisting of a vector space \(W\) and an endomorphism \(t:W\rightarrow W\) such that \(t^2=id\). 
In \(\mathcal{V}^I\) the basic "free" object is \((k^2,\begin{pmatrix}
	0&1\\ 
	1&0
\end{pmatrix})\); this can also be thought of as the vector space \(k\la g,tg\ra\) where \(t(tg)=g\). Let \(P:\mathcal{V}^I\rightarrow \mathcal{V}\) be the colimit functor, sending an object \((W,t)\) to \(W/\left\{ x-tx\mid x\in W \right\}\). 
Find the free resolution of the object \((k,id)\) and compute \((L_iP)(k,id)\) for all \(i\geq 0\).
\end{enumerate}
\end{problem}
\begin{solution}
\begin{enumerate}[(a)]
\item 
\item Consider the following sequence 
\[0\rightarrow F_1\oplus F_2\rightarrow F_0\rightarrow D\rightarrow 0.\]
Note that \(F_1\oplus F_2\) is the following diagram \([k\leftarrow 0\rightarrow k]\) Namely the following diagram 
% https://q.uiver.app/#q=WzAsMjAsWzEsMywiMCJdLFsyLDMsImsiXSxbMywzLCIwIl0sWzEsMiwiayJdLFsyLDIsImsiXSxbMywyLCJrIl0sWzEsMSwiayJdLFszLDEsImsiXSxbMiwxLCIwIl0sWzAsMywiRCJdLFswLDIsIkZfMCJdLFswLDEsIkZfMVxcb3BsdXMgRl8yIl0sWzAsMCwiMCJdLFswLDQsIjAiXSxbMSw0LCIwIl0sWzIsNCwiMCJdLFszLDQsIjAiXSxbMSwwLCIwIl0sWzIsMCwiMCJdLFszLDAsIjAiXSxbOCw2XSxbOCw3XSxbNCwzLCJpZCIsMl0sWzQsNSwiaWQiXSxbNiwzLCJpZCJdLFs3LDUsImlkIiwyXSxbOCw0XSxbMywwXSxbNCwxLCJpZCJdLFs1LDJdLFswLDFdLFsxLDJdLFsxMSwxMF0sWzEwLDldLFsxMiwxMV0sWzksMTNdLFsxNyw2XSxbMTgsOF0sWzE5LDddLFsxOCwxN10sWzE4LDE5XSxbMCwxNF0sWzEsMTVdLFsyLDE2XSxbMTUsMTRdLFsxNSwxNl1d
\[\begin{tikzcd}
	0 & 0 & 0 & 0 \\
	{F_1\oplus F_2} & k & 0 & k \\
	{F_0} & k & k & k \\
	D & 0 & k & 0 \\
	0 & 0 & 0 & 0
	\arrow[from=1-1, to=2-1]
	\arrow[from=1-2, to=2-2]
	\arrow[from=1-3, to=1-2]
	\arrow[from=1-3, to=1-4]
	\arrow[from=1-3, to=2-3]
	\arrow[from=1-4, to=2-4]
	\arrow[from=2-1, to=3-1]
	\arrow["id", from=2-2, to=3-2]
	\arrow[from=2-3, to=2-2]
	\arrow[from=2-3, to=2-4]
	\arrow[from=2-3, to=3-3]
	\arrow["id"', from=2-4, to=3-4]
	\arrow[from=3-1, to=4-1]
	\arrow[from=3-2, to=4-2]
	\arrow["id"', from=3-3, to=3-2]
	\arrow["id", from=3-3, to=3-4]
	\arrow["id", from=3-3, to=4-3]
	\arrow[from=3-4, to=4-4]
	\arrow[from=4-1, to=5-1]
	\arrow[from=4-3, to=4-2]
	\arrow[from=4-2, to=5-2]
	\arrow[from=4-3, to=4-4]
	\arrow[from=4-3, to=5-3]
	\arrow[from=4-4, to=5-4]
	\arrow[from=5-3, to=5-2]
	\arrow[from=5-3, to=5-4]
\end{tikzcd}\]
The vertical columns are exact because we only have isomorphisms. 
For \(E\), consider the following sequence
\[0\rightarrow F_1\rightarrow F_0\rightarrow E\rightarrow 0.\]
This can be written as the diagram 
% https://q.uiver.app/#q=WzAsMjAsWzEsMywiMCJdLFsyLDMsImsiXSxbMywzLCJrIl0sWzEsMiwiayJdLFsyLDIsImsiXSxbMywyLCJrIl0sWzEsMSwiayJdLFszLDEsIjAiXSxbMiwxLCIwIl0sWzAsMywiRSJdLFswLDIsIkZfMCJdLFswLDEsIkZfMSJdLFswLDAsIjAiXSxbMCw0LCIwIl0sWzEsNCwiMCJdLFsyLDQsIjAiXSxbMyw0LCIwIl0sWzEsMCwiMCJdLFsyLDAsIjAiXSxbMywwLCIwIl0sWzgsNl0sWzgsN10sWzQsMywiaWQiLDJdLFs0LDUsImlkIl0sWzYsMywiaWQiXSxbNyw1XSxbOCw0XSxbMywwXSxbNCwxLCJpZCJdLFs1LDIsImlkIl0sWzEsMiwiaWQiXSxbMTEsMTBdLFsxMCw5XSxbMTIsMTFdLFs5LDEzXSxbMTcsNl0sWzE4LDhdLFsxOSw3XSxbMTgsMTddLFsxOCwxOV0sWzAsMTRdLFsxLDE1XSxbMiwxNl0sWzE1LDE0XSxbMTUsMTZdLFsxLDBdXQ==
\[\begin{tikzcd}
	0 & 0 & 0 & 0 \\
	{F_1} & k & 0 & 0 \\
	{F_0} & k & k & k \\
	E & 0 & k & k \\
	0 & 0 & 0 & 0
	\arrow[from=1-1, to=2-1]
	\arrow[from=1-2, to=2-2]
	\arrow[from=1-3, to=1-2]
	\arrow[from=1-3, to=1-4]
	\arrow[from=1-3, to=2-3]
	\arrow[from=1-4, to=2-4]
	\arrow[from=2-1, to=3-1]
	\arrow["id", from=2-2, to=3-2]
	\arrow[from=2-3, to=2-2]
	\arrow[from=2-3, to=2-4]
	\arrow[from=2-3, to=3-3]
	\arrow[from=2-4, to=3-4]
	\arrow[from=3-1, to=4-1]
	\arrow[from=3-2, to=4-2]
	\arrow["id"', from=3-3, to=3-2]
	\arrow["id", from=3-3, to=3-4]
	\arrow["id", from=3-3, to=4-3]
	\arrow["id", from=3-4, to=4-4]
	\arrow[from=4-1, to=5-1]
	\arrow[from=4-2, to=5-2]
	\arrow[from=4-3, to=4-2]
	\arrow["id", from=4-3, to=4-4]
	\arrow[from=4-3, to=5-3]
	\arrow[from=4-4, to=5-4]
	\arrow[from=5-3, to=5-2]
	\arrow[from=5-3, to=5-4]
\end{tikzcd}\]
This is a free resolution for \(E\). 
\item Apply the pushout functor to the free resolution
\[0\rightarrow F_1\oplus F_2\rightarrow F_0\rightarrow 0.\]
The pushout of \(F_1\oplus F_2\) is \(k^2\) and the pushout of \(F_0\) is \(k\). The map \(F_1\oplus F_2\rightarrow F_0\) induces a map \(p\) between pushouts
% https://q.uiver.app/#q=WzAsOCxbMCwwLCJrIl0sWzIsMCwiayJdLFsxLDAsIjAiXSxbMSwyLCJrIl0sWzAsMiwiayJdLFsyLDIsImsiXSxbMSwxLCJrXjIiXSxbMSwzLCJrIl0sWzAsNCwiaWQiLDJdLFsxLDUsImlkIl0sWzAsNiwiIiwxLHsic3R5bGUiOnsidGFpbCI6eyJuYW1lIjoibW9ubyJ9fX1dLFsxLDYsIiIsMSx7InN0eWxlIjp7InRhaWwiOnsibmFtZSI6Im1vbm8ifX19XSxbMiwwXSxbMyw0XSxbMyw1XSxbMiwxXSxbNCw3LCJpZCIsMl0sWzUsNywiaWQiXSxbNiw3LCJwIiwyLHsibGFiZWxfcG9zaXRpb24iOjMwLCJjdXJ2ZSI6Mn1dXQ==
\[\begin{tikzcd}
	k & 0 & k \\
	& {k^2} \\
	k & k & k \\
	& k
	\arrow[tail, from=1-1, to=2-2]
	\arrow["id"', from=1-1, to=3-1]
	\arrow[from=1-2, to=1-1]
	\arrow[from=1-2, to=1-3]
	\arrow[tail, from=1-3, to=2-2]
	\arrow["id", from=1-3, to=3-3]
	\arrow["p"'{pos=0.3}, curve={height=12pt}, from=2-2, to=4-2]
	\arrow["id"', from=3-1, to=4-2]
	\arrow[from=3-2, to=3-1]
	\arrow[from=3-2, to=3-3]
	\arrow["id", from=3-3, to=4-2]
\end{tikzcd}\]
We can see from the diagram that \(p=(id,id)\), so \(p\) is surjective, so we have 
\((L_0P)(D)=P(D)=0\) and \((L_1P)(D)=k\).

Apply the pushout functor to the free resolution 
\[0\rightarrow F_1\rightarrow F_0\rightarrow E\rightarrow 0.\]
The map \(F_1\rightarrow F_0\) induces a map between pushouts 
% https://q.uiver.app/#q=WzAsOCxbMCwwLCJrIl0sWzIsMCwiMCJdLFsxLDAsIjAiXSxbMSwyLCJrIl0sWzAsMiwiayJdLFsyLDIsImsiXSxbMSwxLCJrIl0sWzEsMywiayJdLFswLDQsImlkIiwyXSxbMSw1XSxbMCw2LCIiLDEseyJzdHlsZSI6eyJ0YWlsIjp7Im5hbWUiOiJtb25vIn19fV0sWzEsNl0sWzIsMF0sWzMsNF0sWzMsNV0sWzIsMV0sWzQsNywiaWQiLDJdLFs1LDcsImlkIl0sWzYsNywiaWQiLDIseyJsYWJlbF9wb3NpdGlvbiI6MzAsImN1cnZlIjoyfV1d
\[\begin{tikzcd}
	k & 0 & 0 \\
	& k \\
	k & k & k \\
	& k
	\arrow[tail, from=1-1, to=2-2]
	\arrow["id"', from=1-1, to=3-1]
	\arrow[from=1-2, to=1-1]
	\arrow[from=1-2, to=1-3]
	\arrow[from=1-3, to=2-2]
	\arrow[from=1-3, to=3-3]
	\arrow["id"'{pos=0.3}, curve={height=12pt}, from=2-2, to=4-2]
	\arrow["id"', from=3-1, to=4-2]
	\arrow[from=3-2, to=3-1]
	\arrow[from=3-2, to=3-3]
	\arrow["id", from=3-3, to=4-2]
\end{tikzcd}\]
This map must be identity, so we have 
\[(L_1P)(E)=(L_0P)(E)=P(E)=0.\]
\item Consider the following free resolution of \(k\xrightarrow{id} k\):
% https://q.uiver.app/#q=WzAsNyxbNSwwLCJrIl0sWzYsMCwiMCJdLFs0LDAsImteMiJdLFszLDAsImteMiJdLFsyLDAsImteMiJdLFsxLDAsImteMiJdLFswLDAsIlxcY2RvdHMiXSxbNSw1LCJ0Il0sWzQsNCwidCJdLFszLDMsInQiXSxbMiwyLCJ0Il0sWzAsMCwiaWQiXSxbNSw0LCJcXGJlZ2lue3BtYXRyaXh9MSYtMVxcXFwgLTEmMVxcZW5ke3BtYXRyaXh9IiwyXSxbMiwwLCJcXGJlZ2lue3BtYXRyaXh9MSYxXFxlbmR7cG1hdHJpeH0iLDIseyJzdHlsZSI6eyJoZWFkIjp7Im5hbWUiOiJlcGkifX19XSxbMCwxXSxbMywyLCJcXGJlZ2lue3BtYXRyaXh9MSYtMVxcXFwgLTEmMVxcZW5ke3BtYXRyaXh9IiwyXSxbNCwzLCJcXGJlZ2lue3BtYXRyaXh9MSYxXFxcXCAxJjFcXGVuZHtwbWF0cml4fSIsMl0sWzYsNSwiXFxiZWdpbntwbWF0cml4fTEmMVxcXFwgMSYxXFxlbmR7cG1hdHJpeH0iLDJdXQ==
\[\begin{tikzcd}[ampersand replacement=\&,column sep=4em]
	\cdots \& {k^2} \& {k^2} \& {k^2} \& {k^2} \& k \& 0
	\arrow["{\begin{pmatrix}1&1\\ 1&1\end{pmatrix}}"', from=1-1, to=1-2]
	\arrow["t", from=1-2, to=1-2, loop, in=55, out=125, distance=10mm]
	\arrow["{\begin{pmatrix}1&-1\\ -1&1\end{pmatrix}}"', from=1-2, to=1-3]
	\arrow["t", from=1-3, to=1-3, loop, in=55, out=125, distance=10mm]
	\arrow["{\begin{pmatrix}1&1\\ 1&1\end{pmatrix}}"', from=1-3, to=1-4]
	\arrow["t", from=1-4, to=1-4, loop, in=55, out=125, distance=10mm]
	\arrow["{\begin{pmatrix}1&-1\\ -1&1\end{pmatrix}}"', from=1-4, to=1-5]
	\arrow["t", from=1-5, to=1-5, loop, in=55, out=125, distance=10mm]
	\arrow["{\begin{pmatrix}1&1\end{pmatrix}}"', two heads, from=1-5, to=1-6]
	\arrow["id", from=1-6, to=1-6, loop, in=55, out=125, distance=10mm]
	\arrow[from=1-6, to=1-7]
\end{tikzcd}\]
Let \(A=\begin{pmatrix}
   1&-1\\
   -1&1
\end{pmatrix}\) and \(B=\begin{pmatrix}
   1&1\\
   1&1
\end{pmatrix}\). Both \(A\) and \(B\) are compatible with the map \(t\) because 
\begin{align*}
	At&=\begin{pmatrix}
	   1&-1\\
	   -1&1
	\end{pmatrix}\begin{pmatrix}
	   0&1\\
	   1&0
	\end{pmatrix}=\begin{pmatrix}
	   -1&1\\
	   1&-1
	\end{pmatrix}=\begin{pmatrix}
	   1&-1\\
	   -1&1
	\end{pmatrix}\begin{pmatrix}
	   0&1\\
	   1&0
	\end{pmatrix}=tA,\\[5pt]
	Bt&=\begin{pmatrix}
	   0&1\\
	   1&0
	\end{pmatrix}\begin{pmatrix}
	   1&1\\
	   1&1
	\end{pmatrix}=\begin{pmatrix}
	   1&1\\
	   1&1
	\end{pmatrix}=\begin{pmatrix}
	   1&1\\
	   1&1
	\end{pmatrix}\begin{pmatrix}
	   0&1\\
	   1&0
	\end{pmatrix}=Bt.
\end{align*}
Moreover, the sequence is exact at every spot. Apply the colimit functor \(P\), the map \(A=\begin{pmatrix}
   1&-1\\
   -1&1	
\end{pmatrix}:k^2\rightarrow k^2\) will give you the following diagram 
% https://q.uiver.app/#q=WzAsNixbMCwwLCJrXjIiXSxbMiwwLCJrXjIiXSxbMSwxLCJrIl0sWzAsMiwia14yIl0sWzIsMiwia14yIl0sWzEsMywiayJdLFswLDEsInQiXSxbMSwyLCJcXGJlZ2lue3BtYXRyaXh9MSYxXFxlbmR7cG1hdHJpeH0iXSxbMCwyLCJcXGJlZ2lue3BtYXRyaXh9MSYxXFxlbmR7cG1hdHJpeH0iLDJdLFswLDMsIlxcYmVnaW57cG1hdHJpeH0xJi0xXFxcXCAtMSYxXFxlbmR7cG1hdHJpeH0iLDJdLFszLDQsInQiLDJdLFszLDUsIlxcYmVnaW57cG1hdHJpeH0xJjFcXGVuZHtwbWF0cml4fSIsMl0sWzQsNSwiXFxiZWdpbntwbWF0cml4fTEmMVxcZW5ke3BtYXRyaXh9Il0sWzEsNCwiXFxiZWdpbntwbWF0cml4fTEmLTFcXFxcIC0xJjFcXGVuZHtwbWF0cml4fSJdLFsyLDUsIjAiLDIseyJsYWJlbF9wb3NpdGlvbiI6MzAsImN1cnZlIjoyfV1d
\[\begin{tikzcd}[ampersand replacement=\&,column sep=5em,row sep=3em]
	{k^2} \&\& {k^2} \\
	\& k \\
	{k^2} \&\& {k^2} \\
	\& k
	\arrow["t", from=1-1, to=1-3]
	\arrow["{\begin{pmatrix}1&1\end{pmatrix}}"', from=1-1, to=2-2]
	\arrow["{\begin{pmatrix}1&-1\\ -1&1\end{pmatrix}}"', from=1-1, to=3-1]
	\arrow["{\begin{pmatrix}1&1\end{pmatrix}}", from=1-3, to=2-2]
	\arrow["{\begin{pmatrix}1&-1\\ -1&1\end{pmatrix}}", from=1-3, to=3-3]
	\arrow["0"'{pos=0.3}, curve={height=12pt}, from=2-2, to=4-2]
	\arrow["t"', from=3-1, to=3-3]
	\arrow["{\begin{pmatrix}1&1\end{pmatrix}}"', from=3-1, to=4-2]
	\arrow["{\begin{pmatrix}1&1\end{pmatrix}}", from=3-3, to=4-2]
\end{tikzcd}\]
The map \(P(A)\) is the zero map. Similarly, we apply the colimit functor \(P\) to the map \(B=\begin{pmatrix}
   1&1\\
   1&1
\end{pmatrix}:k^2\rightarrow k^2\):
% https://q.uiver.app/#q=WzAsNixbMCwwLCJrXjIiXSxbMiwwLCJrXjIiXSxbMSwxLCJrIl0sWzAsMiwia14yIl0sWzIsMiwia14yIl0sWzEsMywiayJdLFswLDEsInQiXSxbMSwyLCJcXGJlZ2lue3BtYXRyaXh9MSYxXFxlbmR7cG1hdHJpeH0iXSxbMCwyLCJcXGJlZ2lue3BtYXRyaXh9MSYxXFxlbmR7cG1hdHJpeH0iLDJdLFswLDMsIlxcYmVnaW57cG1hdHJpeH0xJjFcXFxcIDEmMVxcZW5ke3BtYXRyaXh9IiwyXSxbMyw0LCJ0IiwyXSxbMyw1LCJcXGJlZ2lue3BtYXRyaXh9MSYxXFxlbmR7cG1hdHJpeH0iLDJdLFs0LDUsIlxcYmVnaW57cG1hdHJpeH0xJjFcXGVuZHtwbWF0cml4fSJdLFsxLDQsIlxcYmVnaW57cG1hdHJpeH0xJjFcXFxcIDEmMVxcZW5ke3BtYXRyaXh9Il0sWzIsNSwiMiIsMix7ImxhYmVsX3Bvc2l0aW9uIjozMCwiY3VydmUiOjJ9XV0=
\[\begin{tikzcd}[ampersand replacement=\&, column sep=5em,row sep=3em]
	{k^2} \&\& {k^2} \\
	\& k \\
	{k^2} \&\& {k^2} \\
	\& k
	\arrow["t", from=1-1, to=1-3]
	\arrow["{\begin{pmatrix}1&1\end{pmatrix}}"', from=1-1, to=2-2]
	\arrow["{\begin{pmatrix}1&1\\ 1&1\end{pmatrix}}"', from=1-1, to=3-1]
	\arrow["{\begin{pmatrix}1&1\end{pmatrix}}", from=1-3, to=2-2]
	\arrow["{\begin{pmatrix}1&1\\ 1&1\end{pmatrix}}", from=1-3, to=3-3]
	\arrow["2"'{pos=0.3}, curve={height=12pt}, from=2-2, to=4-2]
	\arrow["t"', from=3-1, to=3-3]
	\arrow["{\begin{pmatrix}1&1\end{pmatrix}}"', from=3-1, to=4-2]
	\arrow["{\begin{pmatrix}1&1\end{pmatrix}}", from=3-3, to=4-2]
\end{tikzcd}\]
Thus, apply the colimit functor \(P\) to the free resolution, and we obtain a chain complex
\[\cdots\xrightarrow{2}k\xrightarrow{0}k\xrightarrow{2}k\xrightarrow{0}k\rightarrow0\]
\begin{itemize}
	\item When the characteristic of \(k\) is 2, all the boundary maps all zero, we have \((L_iP)(k,id)=k\) for all \(i\geq 0\).
	\item When the characteristic of \(k\) is not equal to \(2\), this means \(2\) is invertible in \(k\), so all \(2\) maps are isomorphisms. In this case we have \((L_iP)(k,id)=0\) for all \(i\geq 1\), and \((L_0P)(k,id)=P(k,id)=k\).
\end{itemize}
\end{enumerate}
\end{solution}


\end{document}
\documentclass[a4paper, 12pt]{article}

\usepackage{/Users/zhengz/Desktop/Math/Workspace/Homework1/homework}

\begin{document}
\noindent

\large\textbf{Zhengdong Zhang} \hfill \textbf{Homework 4}
Email: zhengz@uoregon.edu \hfill ID: 952091294
\normalsize Course: MATH 636 - Algebraic Topology III \hfill Term: Spring 2025
Instructor: Dr.Daniel Dugger \hfill Due Date: $02^{nd}$ May, 2025 
\noindent\rule{7in}{2.8pt}
\setstretch{1.1}

%%%%%%%%%%%%%%%%%%%%%%%%%%%%%%%%%%%%%%%%%%%%%%%%%%%%%%%%%%%%%%%%%%%%%%%%%%%%%%%%%%%%%%%%%%%%%%%%%%%%%%%%%%%%%%%%%%%%%%%%%
% Problem 1
%%%%%%%%%%%%%%%%%%%%%%%%%%%%%%%%%%%%%%%%%%%%%%%%%%%%%%%%%%%%%%%%%%%%%%%%%%%%%%%%%%%%%%%%%%%%%%%%%%%%%%%%%%%%%%%%%%%%%%%%%%
\begin{problem}{1}
Compute all of the homology groups for the spaces in parts (a) and (b) below, and use your calculations to show that 
\begin{enumerate}[(a)]
\item \(\mathbb{R}P^2\times S^3\) and \(\mathbb{R}P^3\times S^2\) have isomorphic homotopy groups (in all dimensions), but non-isomorphic homology groups. 
\item \(S^4\times S^2\) and \(\mathbb{C}P^3\) have isomorphic homology groups but non-isomorphic homotopy groups. 
\end{enumerate}
\end{problem}

%%%%%%%%%%%%%%%%%%%%%%%%%%%%%%%%%%%%%%%%%%%%%%%%%%%%%%%%%%%%%%%%%%%%%%%%%%%%%%%%%%%%%%%%%%%%%%%%%%%%%%%%%%%%%%%%%%%%%%%%%
% Problem 2
%%%%%%%%%%%%%%%%%%%%%%%%%%%%%%%%%%%%%%%%%%%%%%%%%%%%%%%%%%%%%%%%%%%%%%%%%%%%%%%%%%%%%%%%%%%%%%%%%%%%%%%%%%%%%%%%%%%%%%%%%%
\begin{problem}{2}
Let \(I_*\) be the chain complex concentrated in degree \(0\) and \(1\) with \(I_1=\mathbb{Z}\la e\ra\), \(I_0=\mathbb{Z}\la a,b\ra\), and \(d(e)=b-a\). Note that this is the 
simplicial chain complex for \(\Delta_1\). Let \(C_*\) and \(D_*\) be chain complexes. 
\begin{enumerate}[(a)]
\item Describe the chain complex \(I_*\otimes C_*\) by giving the groups in each degree as well as the boundary maps.
\item Let \(F:I_*\otimes C_*\rightarrow D_*\) be a chain map. Define \(f,g:C_*\rightarrow D_*\) by \(f(x)=F(a\otimes x)\) and \(g(x)=F(b\otimes x)\). Likewise, define \(s_n:C_n\rightarrow D_{n+1}\) by \(s_n:C_n\rightarrow D_{n+1}\) by \(s_n(x)=F(e\otimes x)\). Prove 
that \(f\) and \(g\) are chain maps and the collection \(\left\{ s_n \right\}\) is a chain homotopy between \(f\) and \(g\). 
\end{enumerate}
\end{problem}
\begin{solution}

\end{solution}

\noindent\rule{7in}{2.8pt}
%%%%%%%%%%%%%%%%%%%%%%%%%%%%%%%%%%%%%%%%%%%%%%%%%%%%%%%%%%%%%%%%%%%%%%%%%%%%%%%%%%%%%%%%%%%%%%%%%%%%%%%%%%%%%%%%%%%%%%%%%
% Problem 3
%%%%%%%%%%%%%%%%%%%%%%%%%%%%%%%%%%%%%%%%%%%%%%%%%%%%%%%%%%%%%%%%%%%%%%%%%%%%%%%%%%%%%%%%%%%%%%%%%%%%%%%%%%%%%%%%%%%%%%%%%%
\begin{problem}{3}
Let \(Y\) be the space obtained by starting with \(S^3\) and attaching a \(4\)-cell via a map of degree \(5:Y=S^3\cup_f e^4\) where \(f:\partial (e^4)\rightarrow S^3\) has degree \(5\). 
Write down the cellular chain complex for \(\mathbb{R}p^3\otimes Y\); in particular, specify the rank of each chain group and identify the boundary maps. Compute the homotopy groups of specify the rank of each chain group and 
identify the boundary maps. Compute the homology groups of \(\mathbb{R}P^3\otimes Y\).
\end{problem}
\begin{solution}

\end{solution}

\noindent\rule{7in}{2.8pt}
%%%%%%%%%%%%%%%%%%%%%%%%%%%%%%%%%%%%%%%%%%%%%%%%%%%%%%%%%%%%%%%%%%%%%%%%%%%%%%%%%%%%%%%%%%%%%%%%%%%%%%%%%%%%%%%%%%%%%%%%%
% Problem 4
%%%%%%%%%%%%%%%%%%%%%%%%%%%%%%%%%%%%%%%%%%%%%%%%%%%%%%%%%%%%%%%%%%%%%%%%%%%%%%%%%%%%%%%%%%%%%%%%%%%%%%%%%%%%%%%%%%%%%%%%%%
\begin{problem}{4}
Compute both the homology and cohomology groups of the following spaces, both with integral and \(\mathbb{Z}/2\) coefficients. Heck, do it with \(\mathbb{Z}/3\) coefficients as well. 
\begin{enumerate}[(a)]
\item \(K\times K\), where \(K\) is the Klein bottle. 
\item \(K\times T^g\), where \(T^g\) is the genus \(g\) torus and \(K\) is the Klein bottle. 
\item \(K\times \mathbb{R}P^n\).
\end{enumerate}
\end{problem}
\begin{solution}

\end{solution}

\noindent\rule{7in}{2.8pt}
%%%%%%%%%%%%%%%%%%%%%%%%%%%%%%%%%%%%%%%%%%%%%%%%%%%%%%%%%%%%%%%%%%%%%%%%%%%%%%%%%%%%%%%%%%%%%%%%%%%%%%%%%%%%%%%%%%%%%%%%%
% Problem 5
%%%%%%%%%%%%%%%%%%%%%%%%%%%%%%%%%%%%%%%%%%%%%%%%%%%%%%%%%%%%%%%%%%%%%%%%%%%%%%%%%%%%%%%%%%%%%%%%%%%%%%%%%%%%%%%%%%%%%%%%%%
\begin{problem}{5}
Let \(f:A_*\rightarrow B_*\) be a map of chain complexes. We can regard this as forming a double complex 
% https://q.uiver.app/#q=WzAsOCxbMCwwLCJcXHZkb3RzIl0sWzEsMCwiXFx2ZG90cyJdLFswLDEsIkFfMiJdLFsxLDEsIkJfMiJdLFswLDIsIkFfMSJdLFsxLDIsIkJfMSJdLFswLDMsIkFfMCJdLFsxLDMsIkJfMCJdLFswLDJdLFsxLDNdLFsyLDNdLFsyLDRdLFszLDVdLFs0LDZdLFs2LDddLFs1LDddLFs0LDVdXQ==
\[\begin{tikzcd}
	\vdots & \vdots \\
	{A_2} & {B_2} \\
	{A_1} & {B_1} \\
	{A_0} & {B_0}
	\arrow[from=1-1, to=2-1]
	\arrow[from=1-2, to=2-2]
	\arrow[from=2-1, to=2-2]
	\arrow[from=2-1, to=3-1]
	\arrow[from=2-2, to=3-2]
	\arrow[from=3-1, to=3-2]
	\arrow[from=3-1, to=4-1]
	\arrow[from=3-2, to=4-2]
	\arrow[from=4-1, to=4-2]
\end{tikzcd}\]
by putting zeros in all the "empty" spots. The total complex of this double complex is called the \textbf{algebraic mapping cone} of \(f\), denoted \(Cf\). Specifically, we set \((Cf)_n=A_{n-1}\oplus B_n\) and define \(d:(Cf)_n\rightarrow (Cf)_{n-1}\) by 
\[d(a,b)=(d_A(a), (-1)^{n-1}f(a)+d_B(b))\]
\begin{enumerate}[(a)]
\item Explain why there is a short exact sequence of chain complexes 
\[0\rightarrow B_*\hookrightarrow C(f)\rightarrow \Sigma A_*\rightarrow 0,\]
where \(\Sigma A_*\) is the evident chain complex having \((\Sigma A)_n=A_{n-1}\). 
\item The short exact sequence from (a) gives rise to a long exact sequence in homology groups. This has the form 
\[\cdots \rightarrow H_i(B)\rightarrow H_i(Cf)\rightarrow H_i(\Sigma A)\xrightarrow{\partial}H_{i-1}(B)\rightarrow \cdots\]
Verify that the connecting homomorphism is really just the map \(f_*:H_{i-1}(A)\rightarrow H_{i-1}(B)\), possibly up to a sign. 
\end{enumerate}
\end{problem}
\begin{solution}

\end{solution}

\noindent\rule{7in}{2.8pt}
%%%%%%%%%%%%%%%%%%%%%%%%%%%%%%%%%%%%%%%%%%%%%%%%%%%%%%%%%%%%%%%%%%%%%%%%%%%%%%%%%%%%%%%%%%%%%%%%%%%%%%%%%%%%%%%%%%%%%%%%%
% Problem 6
%%%%%%%%%%%%%%%%%%%%%%%%%%%%%%%%%%%%%%%%%%%%%%%%%%%%%%%%%%%%%%%%%%%%%%%%%%%%%%%%%%%%%%%%%%%%%%%%%%%%%%%%%%%%%%%%%%%%%%%%%%
\begin{problem}{6}
Let \(k\) be a field, and let \(\mathcal{V}\) denote the category of vector spaces over \(k\). Let \(I\) be any (small) category, and let \(\mathcal{V}^I\) be the category whose objects are functors \(I\rightarrow \mathcal{V}\) and whose morphisms are 
natural transformations. We call \(\mathcal{V}^I\) the category of "\(I\)-shaped diagram in \(\mathcal{V}\)". \\ 
In this problem we will focus on the case where \(I\) is the pushout category 
\[1\leftarrow 0\rightarrow 2\]
with three objects and two non-identity maps (as shown above). An object of \(\mathcal{V}^I\) is then just a diagram of vector spaces \(V_1\leftarrow V_0\rightarrow V_2\). A map from \([V_1\leftarrow V_0\rightarrow V_2]\) to \([W_1\leftarrow W_0\rightarrow W_2]\) 
is a commutative diagram 
% https://q.uiver.app/#q=WzAsNixbMCwwLCJWXzEiXSxbMSwwLCJWXzAiXSxbMiwwLCJWXzIiXSxbMCwxLCJXXzEiXSxbMSwxLCJXXzAiXSxbMiwxLCJXXzIiXSxbMCwzXSxbMSw0XSxbMiw1XSxbMSwwXSxbMSwyXSxbNCw1XSxbNCwzXV0=
\[\begin{tikzcd}
	{V_1} & {V_0} & {V_2} \\
	{W_1} & {W_0} & {W_2}
	\arrow[from=1-1, to=2-1]
	\arrow[from=1-2, to=1-1]
	\arrow[from=1-2, to=1-3]
	\arrow[from=1-2, to=2-2]
	\arrow[from=1-3, to=2-3]
	\arrow[from=2-2, to=2-1]
	\arrow[from=2-2, to=2-3]
\end{tikzcd}\]
Let \(P:\mathcal{V}^I\rightarrow \mathcal{V}\) be the pushout functor. \(P\) assigns each diagram its pushout.
\begin{enumerate}[(a)]
\item Let \(F_1\), \(F_0\) and \(F_2\) be the three diagrams 
\[F_1:[k\leftarrow 0\rightarrow 0]\ \ \ F_0=[k\leftarrow k\rightarrow k]\ \ \ F_2=[0\leftarrow 0\rightarrow 0]\]
where in \(F_0\) the maps are the identities. These diagrams are "free" in a certain sense: namely, if \(D\) is an object of \(\mathcal{V}^I\) then morphisms \(F_i\rightarrow D\) are in bijective correspondence with elements of \(D_i\). 
Convince yourself that this is true. 
\item Let \(D=[0\leftarrow k\rightarrow 0]\) and \(E=[0\leftarrow k\rightarrow k]\), where in \(E\) the nontrivial map is the identity. Determine free resolutions for \(D\) and \(E\). 
\item Apply the functor \(P\) to your resolution, to produce a chain complex of vector spaces. Compute the homology groups, which are the groups \((L_iP)(D)\) and \((L_iP)(E)\). These are the derived functor of the psuhout functor \(P\). Confirm in your example that \(L_0P=P\). 
\item Now let \(I\) be the category with one object \(0\) and one non-identity map \(t:0\rightarrow 0\) such that \(t^2=id\). Objects of \(\mathcal{V}^I\) are then pairs \((W,t)\) consisting of a vector space \(W\) and an endomorphism \(t:W\rightarrow W\) such that \(t^2=id\). 
In \(\mathcal{V}^I\) the basic "free" object is \((k^2,\begin{pmatrix}
	0&1\\ 
	1&0
\end{pmatrix})\); this can also be thought of as the vector space \(k\la g,tg\ra\) where \(t(tg)=g\). Let \(P:\mathcal{V}^I\rightarrow \mathcal{V}\) be the colimit functor, sending an object \((W,t)\) to \(W/\left\{ x-tx\mid x\in W \right\}\). 
Find the free resolution of the object \((k,id)\) and compute \((L_iP)(k,id)\) for all \(i\geq 0\).
\end{enumerate}
\end{problem}
\begin{solution}

\end{solution}


\end{document}
\documentclass[letterpaper, 12pt]{article}

\usepackage{/Users/zhengz/Desktop/Math/Workspace/Homework1/homework}

\begin{document}
\noindent
\large\textbf{Zhengdong Zhang} \hfill \textbf{Homework 5}  \\
Email: zhengz@uoregon.edu \hfill ID: 952091294  \\
\normalsize Course: MATH 636 - Algebraic Topology III \hfill Term: Spring 2025 \\
Instructor: Dr.Daniel Dugger \hfill Due Date: $16^{th}$ May,2025  \\
\noindent\rule{7in}{2.8pt}
\setstretch{1.1}
%%%%%%%%%%%%%%%%%%%%%%%%%%%%%%%%%%%%%%%%%%%%%%%%%%%%%%%%%%%%%%%%%%%%%%%%%%%%%%%%%%%%%%%%%%%%%%%%%%%%%%%%%%%%%%%%%%%%%%%%%
% Problem 1
%%%%%%%%%%%%%%%%%%%%%%%%%%%%%%%%%%%%%%%%%%%%%%%%%%%%%%%%%%%%%%%%%%%%%%%%%%%%%%%%%%%%%%%%%%%%%%%%%%%%%%%%%%%%%%%%%%%%%%%%%%
\begin{problem}{1}
The space \(S^2\times S^3\) and \(S^2\vee S^3\vee S^5\) have isomorphic homology groups. Use the cup product to prove that they are not homotopy equivalent.
\end{problem}
\begin{solution}

\end{solution}

\noindent\rule{7in}{2.8pt}
%%%%%%%%%%%%%%%%%%%%%%%%%%%%%%%%%%%%%%%%%%%%%%%%%%%%%%%%%%%%%%%%%%%%%%%%%%%%%%%%%%%%%%%%%%%%%%%%%%%%%%%%%%%%%%%%%%%%%%%%%
% Problem 2
%%%%%%%%%%%%%%%%%%%%%%%%%%%%%%%%%%%%%%%%%%%%%%%%%%%%%%%%%%%%%%%%%%%%%%%%%%%%%%%%%%%%%%%%%%%%%%%%%%%%%%%%%%%%%%%%%%%%%%%%%%
\begin{problem}{2}
\begin{enumerate}[(a)]
\item Let \(M\) be an \(n\)-dimensional manifold-with-boundary (boundary points have neighborhoods that look like \(\left\{ \underline{x}\in \mathbb{R}^n\mid x_n\geq 0 \right\}\)). If \(x\) is on the boundary of \(M\), prove that \(H_n(M,M-x)=0\).
\item Let \(M\) be a compact, connected, orientable \(n\)-manifold. If \(U\) is a Euclidean open disk in \(M\), prove that \(H_i(M-U)\rightarrow H_i(M)\) is an isomorphism when \(i<n\)  and prove that \(H_n(M-U)=0\). Also, if \(A=\partial (M-U)\) prove that the connecting homomorphism \(\partial:H_n(M-U,A)\rightarrow H_{n-1}(A)\) is an isomorphism. 
\end{enumerate}
\end{problem}
\begin{solution}

\end{solution}

\noindent\rule{7in}{2.8pt}
%%%%%%%%%%%%%%%%%%%%%%%%%%%%%%%%%%%%%%%%%%%%%%%%%%%%%%%%%%%%%%%%%%%%%%%%%%%%%%%%%%%%%%%%%%%%%%%%%%%%%%%%%%%%%%%%%%%%%%%%%
% Problem 3
%%%%%%%%%%%%%%%%%%%%%%%%%%%%%%%%%%%%%%%%%%%%%%%%%%%%%%%%%%%%%%%%%%%%%%%%%%%%%%%%%%%%%%%%%%%%%%%%%%%%%%%%%%%%%%%%%%%%%%%%%%
\begin{problem}{3}
Let \(M\) and \(N\) be compact, connected \(n\)-manifolds, \(n\geq 2\). Prove the following:
\begin{enumerate}[(a)]
\item If \(M\) and \(N\) are orientable, then so is \(M\#N\). 
\item If \(M\) and \(N\) are non-orientable, then so is \(M\#N\).
\item What happens when \(M\) is orientable and \(N\) is not? Justify your answer.
\end{enumerate}
\end{problem}
\begin{solution}

\end{solution}

\noindent\rule{7in}{2.8pt}
%%%%%%%%%%%%%%%%%%%%%%%%%%%%%%%%%%%%%%%%%%%%%%%%%%%%%%%%%%%%%%%%%%%%%%%%%%%%%%%%%%%%%%%%%%%%%%%%%%%%%%%%%%%%%%%%%%%%%%%%%
% Problem 4
%%%%%%%%%%%%%%%%%%%%%%%%%%%%%%%%%%%%%%%%%%%%%%%%%%%%%%%%%%%%%%%%%%%%%%%%%%%%%%%%%%%%%%%%%%%%%%%%%%%%%%%%%%%%%%%%%%%%%%%%%%
\begin{problem}{4}
Suppose \(X=Y=S^n\). \(H^*(X)\) has no torsion then the Künneth Theorem gives an isomorphism of rings 
\[H^*(X)\otimes H^*(Y)\rightarrow H^*(X\times Y).\]
For \(a\in H^*(X)\) and \(b\in H^*(Y)\), the map sends \(a\otimes b\) to \(\pi_1^*(a)\cup \pi_2^*(b)\). The latter expression is sometimes denoted \(a\times b\). 

Suppose that \(S^n\) has a continuous unital multiplication \(\mu:S^n\times S^n\rightarrow S^n\). So there is a unit element \(e\in S^n\) with property that \(\mu(e,x)=x=\mu(x,e)\) for all \(x\in S^n\). Said differently, the following diagram is commutative:
% https://q.uiver.app/#q=WzAsNCxbMCwwLCJTXm5cXHRpbWVzIFxcbGVmdFxceypcXHJpZ2h0XFx9Il0sWzEsMSwiU15uXFx0aW1lcyBTXm4iXSxbMiwxLCJTXm4iXSxbMCwyLCJcXGxlZnRcXHsqXFxyaWdodFxcfVxcdGltZXMgU15uIl0sWzAsMSwiaWRcXHRpbWVzIGoiLDJdLFszLDEsImpcXHRpbWVzIGlkIl0sWzAsMiwiaWQiLDAseyJjdXJ2ZSI6LTJ9XSxbMywyLCJpZCIsMix7ImN1cnZlIjoyfV0sWzEsMiwiXFxtdSJdXQ==
\[\begin{tikzcd}
	{S^n\times \left\{*\right\}} \\
	& {S^n\times S^n} & {S^n} \\
	{\left\{*\right\}\times S^n}
	\arrow["{id\times j}"', from=1-1, to=2-2]
	\arrow["id", curve={height=-12pt}, from=1-1, to=2-3]
	\arrow["\mu", from=2-2, to=2-3]
	\arrow["{j\times id}", from=3-1, to=2-2]
	\arrow["id"', curve={height=12pt}, from=3-1, to=2-3]
\end{tikzcd}\]
where \(j:*\hookrightarrow S^n\) sends the point to \(e\).
\begin{enumerate}[(a)]
\item Let \(z\) be a generator for \(H^n(S^n)\). Use the above diagram to prove that \(\mu^*(z)=z\otimes 1+1\otimes z\).
\item Use the fact that \(\mu^*\) is a ring homomorphism, together with your knowledge of the ring structure on \(H^*(S^n\times S^n)\), to conclude that \(n\) must be odd. 
\item Consider a multiplication \(\mathbb{R}^3\times \mathbb{R}^3\rightarrow \mathbb{R}^3\) which was unital and had no zero-divisors (that is, \(xy=0\Rightarrow (x=0\ \text{or}\ y=0)\)). Use (b) to prove that no such multiplication exists, assuming that the identity element is nonzero. 
\end{enumerate}
\end{problem}
\begin{solution}

\end{solution}

\noindent\rule{7in}{2.8pt}
%%%%%%%%%%%%%%%%%%%%%%%%%%%%%%%%%%%%%%%%%%%%%%%%%%%%%%%%%%%%%%%%%%%%%%%%%%%%%%%%%%%%%%%%%%%%%%%%%%%%%%%%%%%%%%%%%%%%%%%%%
% Problem 5
%%%%%%%%%%%%%%%%%%%%%%%%%%%%%%%%%%%%%%%%%%%%%%%%%%%%%%%%%%%%%%%%%%%%%%%%%%%%%%%%%%%%%%%%%%%%%%%%%%%%%%%%%%%%%%%%%%%%%%%%%%
\begin{problem}{5}
The relative form of the Künneth Theorem is that there is a natural short exact sequence 
\begin{multline*}
    0\rightarrow \bigoplus_{p+q=n}H_p(X,A)\otimes H_q(Y,B)\rightarrow H_n(X\times Y, X\times B\cup A\times Y)\\
    \rightarrow \bigoplus_{p+q=n-1}\Tor_1(H_p(X,A),H_q(Y,B))\rightarrow 0.
\end{multline*}

\begin{enumerate}[(a)]
\item Suppose that \(M\) is an \(n\)-manifold and \(N\) is a \(k\)-manifold, and that we are given local orientations \(\mu_M\in H_n(M,M-x)\) and \(\mu_N\in H_k(N,N-y)\), for some \(x\in M\) and \(y\in N\). Explain how to get an induced local orientation for \(M\times N\) at \((x,y)\).
\item Explain how to get an interesting continuous map \(\tilde{M}\times \tilde{N}\rightarrow \tilde{M\times N}\), where \(\tilde{M}\) is the space of pairs \((m,\mu_m)\) such that \(m\in M\) and \(\mu_m\in H_n(M,M-m)\) is a generator (and similarly for N, etc.). How many points are in each fiber?
\item Prove that if \(M\) and \(N\) are orientable then so is \(M\times N\) (we are not assuming compactness here). 
\end{enumerate}
\end{problem}
\begin{solution}

\end{solution}

\noindent\rule{7in}{2.8pt}
%%%%%%%%%%%%%%%%%%%%%%%%%%%%%%%%%%%%%%%%%%%%%%%%%%%%%%%%%%%%%%%%%%%%%%%%%%%%%%%%%%%%%%%%%%%%%%%%%%%%%%%%%%%%%%%%%%%%%%%%%
% Problem 6
%%%%%%%%%%%%%%%%%%%%%%%%%%%%%%%%%%%%%%%%%%%%%%%%%%%%%%%%%%%%%%%%%%%%%%%%%%%%%%%%%%%%%%%%%%%%%%%%%%%%%%%%%%%%%%%%%%%%%%%%%%
\begin{problem}{6}
Consider the space \(X=\mathbb{R}P^2\#\mathbb{R}P^2\#\mathbb{R}P^2\), made into a \(\Delta\)-complex as follows.

Recall that \(H^0(X;\mathbb{Z}/2)=H^2(X;\mathbb{Z}/2)=\mathbb{Z}/2\) and \(H^1(X;\mathbb{Z}/2)=(\mathbb{Z}/2)^3\). The Universal Coefficients Theorem implies that the standard maps
\[\phi_i:H^i(X;\mathbb{Z}/2)\rightarrow \hom(H_i(X;\mathbb{Z}/2),\mathbb{Z}/2)\]
are isomorphisms.
\begin{enumerate}[(a)]
\item Write down explicit \(1\)-cocycles \(\alpha\), \(\beta\) and \(\gamma\) (with \(\mathbb{Z}/2\) coefficients) that map to \(\hat{a}\), \(\hat{b}\) and \(\hat{c}\) under \(\phi\). 
\item Given a \(2\)-cochain \(\Theta\) (with \(\mathbb{Z}/2\) coefficients), how can one easily determine if \(\Theta\) is a generator for \(H^2(X;\mathbb{Z}/2)\)?
\item Determine a class \(u\in H^1(X;\mathbb{Z}/2)\) such that \(\alpha\cup u\) is a generator for \(H^2(X;\mathbb{Z}/2)\). Then do the same for \(\beta\) and \(\gamma\).
\end{enumerate}
\end{problem}
\begin{solution}

\end{solution}


\end{document}
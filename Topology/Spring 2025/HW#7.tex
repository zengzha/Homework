\documentclass[letterpaper, 12pt]{article}

\usepackage{/Users/zhengz/Desktop/Math/Workspace/Homework1/homework}

\begin{document}
\noindent
\large\textbf{Zhengdong Zhang} \hfill \textbf{Homework 7}  \\
Email: zhengz@uoregon.edu \hfill ID: 952091294  \\
\normalsize Course: MATH 636 - Algebraic Topology III \hfill Term: Spring 2025 \\
Instructor: Dr.Daniel Dugger \hfill Due Date: $30^{th}$ May , 2025  \\
\noindent\rule{7in}{2.8pt}
\setstretch{1.1}
%%%%%%%%%%%%%%%%%%%%%%%%%%%%%%%%%%%%%%%%%%%%%%%%%%%%%%%%%%%%%%%%%%%%%%%%%%%%%%%%%%%%%%%%%%%%%%%%%%%%%%%%%%%%%%%%%%%%%%%%%
% Problem 1
%%%%%%%%%%%%%%%%%%%%%%%%%%%%%%%%%%%%%%%%%%%%%%%%%%%%%%%%%%%%%%%%%%%%%%%%%%%%%%%%%%%%%%%%%%%%%%%%%%%%%%%%%%%%%%%%%%%%%%%%%%
\begin{problem}{1}
If \(M\) and \(N\) are compact, oriented \(d\)-manifold, then the \textbf{degree} of a map \(f:M\rightarrow N\) is defined to be the integer \(\deg(f)\) such that \(f_*([M])=\deg f\cdot [N]\).
\begin{enumerate}[(a)]
\item Suppose that \(f\) is not surjective--i.e., there is a point \(x\in N\) such that \(x\) is not in the image of \(f\). Prove that the degree of \(f\) is zero. 
\item Explain how \(\deg(f)\) relates to the map \(f^*:H^d(N)\rightarrow H^d(M)\). 
\item Prove that any map \(S^4\rightarrow \mathbb{C}P^2\) must have degree 0.
\end{enumerate}
\end{problem}
\begin{solution}
\begin{enumerate}[(a)]
\item \(f\) being not surjective means that there exists a point \(y\in N\) such that \(f(M)\subseteq N-y\). This implies we have a commutative diagram 
% https://q.uiver.app/#q=WzAsMyxbMCwwLCJNIl0sWzEsMCwiTi15Il0sWzEsMSwiTiJdLFswLDEsImYiXSxbMCwyLCJmIiwyXSxbMSwyLCJJIiwwLHsic3R5bGUiOnsidGFpbCI6eyJuYW1lIjoibW9ubyJ9fX1dXQ==
\[\begin{tikzcd}
	M & {N-y} \\
	& N
	\arrow["f", from=1-1, to=1-2]
	\arrow["f"', from=1-1, to=2-2]
	\arrow["i", tail, from=1-2, to=2-2]
\end{tikzcd}\]
where \(i:N-y\hookrightarrow N\) is the inclusion map. This induces a commutative diagram in homology groups 
% https://q.uiver.app/#q=WzAsNCxbMCwwLCJIX2QoTSkiXSxbMSwwLCJIX2QoTi15KSJdLFsxLDEsIkhfZChOKSJdLFsxLDIsIkhfZChOLE4teSkiXSxbMCwxLCJmXyoiXSxbMCwyLCJmXyoiLDJdLFsxLDIsImlfKiIsMCx7InN0eWxlIjp7InRhaWwiOnsibmFtZSI6Im1vbm8ifX19XSxbMiwzLCJqXyoiXV0=
\[\begin{tikzcd}
	{H_d(M)} & {H_d(N-y)} \\
	& {H_d(N)} \\
	& {H_d(N,N-y)}
	\arrow["{f_*}", from=1-1, to=1-2]
	\arrow["{f_*}"', from=1-1, to=2-2]
	\arrow["{i_*}", tail, from=1-2, to=2-2]
	\arrow["{j_*}", from=2-2, to=3-2]
\end{tikzcd}\]
The map \(j_*:H_d(N)\rightarrow H_d(N,N-y)\) is an isomorphism because \(N\) is compact and oriented. We know that \(j_*\circ i_*\) is the zero map because of the exactness on the vertical map. This implies that \(f_*\) is the zero map, so \(\deg f=0\).
\item By UCT, we have a commutative diagram 
% https://q.uiver.app/#q=WzAsNCxbMCwwLCJIXmQoTikiXSxbMSwwLCJcXGhvbShIX2QoTiksXFxtYXRoYmJ7Wn0pIl0sWzEsMSwiXFxob20oSF9kKE0pLFxcbWF0aGJie1p9KSJdLFswLDEsIkheZChNKSJdLFswLDMsImZeKiIsMl0sWzAsMSwiIiwwLHsic3R5bGUiOnsiaGVhZCI6eyJuYW1lIjoiZXBpIn19fV0sWzMsMiwiIiwyLHsic3R5bGUiOnsiaGVhZCI6eyJuYW1lIjoiZXBpIn19fV0sWzEsMiwiXFxkZWcgZiJdXQ==
\[\begin{tikzcd}
	{H^d(N)} & {\hom(H_d(N),\mathbb{Z})} \\
	{H^d(M)} & {\hom(H_d(M),\mathbb{Z})}
	\arrow[two heads, from=1-1, to=1-2]
	\arrow["{f^*}"', from=1-1, to=2-1]
	\arrow["{\deg f}", from=1-2, to=2-2]
	\arrow[two heads, from=2-1, to=2-2]
\end{tikzcd}\]
Both \(M\) and \(N\) are compact and oriented, so \(H_d(M)\cong H_d(N)\cong \mathbb{Z}\). The right vertical map is induced by the map \(f_*:H_d(M)\rightarrow H_d(N)\), so it is also the multiplication by \(\deg f\). By Poincaré duality, we know that 
\[H^d(M)\cong H_0(M)\cong \mathbb{Z},\ \ \ H^d(N)\cong H_0(N)\cong \mathbb{Z}.\]
This implies the top and bottom horizontal maps in the commutative diagram is isomorphisms. Therefore, if we choose \(\widehat{[M]}\) to be the generator of \(H^d(M)\cong \mathbb{Z}\) and \(\widehat{[N]}\) to be the generator of \(H^d(N)\cong \mathbb{Z}\), then the map \(f^*:H^d(N)\rightarrow H^d(M)\) is sending \(\widehat{[N]}\) to \(\deg f\cdot \widehat{[M]}\).
\item Consider a map \(f:S^4\rightarrow \mathbb{C}P^2\), which induces a map between cohomology rings 
\[f^*:H^*(\mathbb{C}P^2)\rightarrow H^*(S^4).\]
We know that \(H^*(\mathbb{C}P^2)\cong \mathbb{Z}[x]/(x^3)\) where \(x\) is a degree 2 element, then when \(*=4\), we have 
\[f^*(x^2)=f^*(x)\cup f^*(x)=0.\]
because \(H^*(S^4)\) has no degree 2 element. This means that 
\[f^*:H^4(\mathbb{C}P^2)\rightarrow H^4(S^4)\]
is the zero map. We know that both \(\mathbb{C}P^2\) and \(S^4\) are compact and orientable. From the discussion in (b), any map \(f:S^4\rightarrow \mathbb{C}P^4\) has degree \(0\). 
\end{enumerate}
\end{solution}

\noindent\rule{7in}{2.8pt}
%%%%%%%%%%%%%%%%%%%%%%%%%%%%%%%%%%%%%%%%%%%%%%%%%%%%%%%%%%%%%%%%%%%%%%%%%%%%%%%%%%%%%%%%%%%%%%%%%%%%%%%%%%%%%%%%%%%%%%%%%
% Problem 2
%%%%%%%%%%%%%%%%%%%%%%%%%%%%%%%%%%%%%%%%%%%%%%%%%%%%%%%%%%%%%%%%%%%%%%%%%%%%%%%%%%%%%%%%%%%%%%%%%%%%%%%%%%%%%%%%%%%%%%%%%%
\begin{problem}{2}
A topological space is said to be of \textbf{finite type} if \(H_i(X)=0\) for all but finitely many values of \(i\), and each nonzero \(H_i(X)\) is a finitely-generated abelian group. Recall that the Euler characteristic is then defined to be 
\[\chi(X)=\sum_{i=1}^{\infty}(-1)^i\rank H_i(X).\]
Prove that if \(X\) and \(Y\) are CW-complexes of finite type then so is \(X\times Y\), and \(\chi(X\times Y)=\chi(X)\cdot \chi(Y)\).
\end{problem}
\begin{solution}
By Künneth Theorem, for all \(i\), we have 
\[H_i(X\times Y)\cong \sum_{p+q=i} H_p(X)\otimes H_q(Y)\oplus \sum_{p+q=i-1} \Tor_1(H_p(X),H_q(Y)).\]
Since both \(X\) and \(Y\) are CW-complexes of finite type, only finitely many \(H_p(X)\) and \(H_q(X)\) are non-zero, this implies \(H_i(X\times Y)=0\) except for finitely many \(i\). Moreover, if Abelian groups \(A\) and \(B\) are finitely generated, then we know that \(A\otimes B\) and \(\Tor_1(A,B)\) are also finitely generated. This means for all \(i\), \(H_i(X\times Y)\) is finitely generated Abelian group. 

We know that for any Abelian group \(A\), 
\[\rank A=\dim_\mathbb{Q} (A\otimes \mathbb{Q}).\]
Suppose \(H_i(X)=0\) for \(i\geq n+1\) and \(H_j(Y)=0\) for \(j\geq m+1\). Because \(\mathbb{Q}\) is a field, for the space \(X\times Y\), by Künneth Theorem, we have 
\begin{align*}
    \chi(X\times Y)=&\sum_{i=1}^{m+n}(-1)^i\rank H_i(X\times Y)\\ 
            =&\sum_{i=1}^{m+n}(-1)^i\dim_\mathbb{Q} H_i(X\times Y;\mathbb{Q})\\ 
            =&\sum_{i=1}^{m+n}(-1)^i\sum_{p+q=i}\dim_\mathbb{Q} H_p(X;\mathbb{Q})\otimes H_q(Y;\mathbb{Q})\\ 
            =&\sum_{i=1}^{m+n}(-1)^i\sum_{p+q=i}(\dim_\mathbb{Q} H_p(X;\mathbb{Q})\cdot \dim_\mathbb{Q} H_q(Y;\mathbb{Q}))\\ 
            =&\sum_{i=1}^{m+n}\sum_{p+q=i}(-1)^p\dim_\mathbb{Q} H_p(X;\mathbb{Q})\cdot (-1)^q\dim_\mathbb{Q} H_q(Y;\mathbb{Q})\\ 
            =&(\sum_{p=1}^{n}(-1)^p\dim_\mathbb{Q} H_p(X;\mathbb{Q}))\cdot (\sum_{q=1}^{m}(-1)^q \dim_\mathbb{Q} H_q(Y;\mathbb{Q}))\\ 
            =&\chi(X)\cdot \chi(Y)
\end{align*}
\end{solution}

\noindent\rule{7in}{2.8pt}
%%%%%%%%%%%%%%%%%%%%%%%%%%%%%%%%%%%%%%%%%%%%%%%%%%%%%%%%%%%%%%%%%%%%%%%%%%%%%%%%%%%%%%%%%%%%%%%%%%%%%%%%%%%%%%%%%%%%%%%%%
% Problem 3
%%%%%%%%%%%%%%%%%%%%%%%%%%%%%%%%%%%%%%%%%%%%%%%%%%%%%%%%%%%%%%%%%%%%%%%%%%%%%%%%%%%%%%%%%%%%%%%%%%%%%%%%%%%%%%%%%%%%%%%%%%
\begin{problem}{3}
Prove that \(\mathbb{C}P^{n-1}\) is not a retract of \(\mathbb{C}P^n\).
\end{problem}
\begin{solution}
Suppose there exists a retract \(r:\mathbb{C}P^n\rightarrow \mathbb{C}P^{n-1}\) such that the composition 
\[\mathbb{C}P^{n-1}\xrightarrow{i} \mathbb{C}P^n\xrightarrow{r}\mathbb{C}P^{n-1}\]
is the identity map, where \(i:\mathbb{C}P^{n-1}\hookrightarrow \mathbb{C}P^{n-1}\) is the inclusion map. This induces maps between cohomology rings 
\[H^*(\mathbb{C}P^{n-1})\xrightarrow{r^*}H^*(\mathbb{C}P^n)\xrightarrow{i^*}H^*(\mathbb{C}P^{n-1})\]
where \(i^*\circ r^*=id\). Note that \(H^*(\mathbb{C}P^{n-1})\cong \mathbb{Z}[x]/(x^n)\) and \(H^*(\mathbb{C}P^n)\cong \mathbb{Z}[y]/(y^{n+1})\). Suppose \(r^*(x)=ky\in H^2(\mathbb{C}P^n)\) for some \(k\in \mathbb{Z}\). We have 
\[0=r^*(x^n)=r^*(x)^n=(ky)^n=k^ny^n.\]
We know that \(0\neq y^n\) is the generator of \(H^{2n}(\mathbb{C}P^n)\cong \mathbb{Z}\). Thus, \(k=0\). This means \(r^*\) is the zero map, which contradicts the assumption that \(i^*\circ r^*=id\). Such retract \(r\) does not exist. 
\end{solution}

\noindent\rule{7in}{2.8pt}
%%%%%%%%%%%%%%%%%%%%%%%%%%%%%%%%%%%%%%%%%%%%%%%%%%%%%%%%%%%%%%%%%%%%%%%%%%%%%%%%%%%%%%%%%%%%%%%%%%%%%%%%%%%%%%%%%%%%%%%%%
% Problem  4
%%%%%%%%%%%%%%%%%%%%%%%%%%%%%%%%%%%%%%%%%%%%%%%%%%%%%%%%%%%%%%%%%%%%%%%%%%%%%%%%%%%%%%%%%%%%%%%%%%%%%%%%%%%%%%%%%%%%%%%%%%
\begin{problem}{4}
Prove that there is no self-homeomorphism \(\mathbb{C}P^{2n}\rightarrow \mathbb{C}P^{2n}\) that reverses the orientation. 
\end{problem}
\begin{solution}
Suppose \(f:\mathbb{C}P^{2n}\rightarrow \mathbb{C}P^{2n}\) is a homeomorphism and \(f\) induces a map between cohomology rings 
\[f^*:H^*(\mathbb{C}P^{2n})\rightarrow H^*(\mathbb{C}P^{2n})\]
which reverses the orientation. We know that \(H^(\mathbb{C}P^{2n})\cong \mathbb{Z}[x]/(x^{2n+1})\), and \(x^{2n}\) generates the group \(H^{4n}(\mathbb{C}P^{2n})\). \(f\) reversing the orientation means \(f^*(x^{2n})=-x^{2n}\). Assume \(f^*(x)=kx\in H^2(\mathbb{C}P^{2n})\) for some \(k\in \mathbb{Z}\). Then 
\[-x^{2n}=f^*(x^{2n})=f^*(x)^{2n}=(kx)^{2n}=k^{2n}x^{2n}.\]
This implies \(k^2n=-1\). No such \(k\) exists in \(\mathbb{Z}\). Thus, such homeomorphism \(f\) does not exist. 
\end{solution}

\noindent\rule{7in}{2.8pt}
%%%%%%%%%%%%%%%%%%%%%%%%%%%%%%%%%%%%%%%%%%%%%%%%%%%%%%%%%%%%%%%%%%%%%%%%%%%%%%%%%%%%%%%%%%%%%%%%%%%%%%%%%%%%%%%%%%%%%%%%%
% Problem 5
%%%%%%%%%%%%%%%%%%%%%%%%%%%%%%%%%%%%%%%%%%%%%%%%%%%%%%%%%%%%%%%%%%%%%%%%%%%%%%%%%%%%%%%%%%%%%%%%%%%%%%%%%%%%%%%%%%%%%%%%%%
There is an algebraic formula 
\begin{equation}\label{eq1}
    (x_1^2+x_2^2)\cdot (y_1^2+y_2^2)=(x_1y_1-x_2y_2)^2+(x_1y_2+x_2y_1)^2
\end{equation}
which is true for indeterminates \(x_1,x_2,y_1,y_2\) over \(\mathbb{R}\). By a \textbf{sumsof-squares formula} of type \([r,s,n]\) we mean an identity of the form 
\[(x_1^2+x_2^2+\cdots+x_r^2)\cdot (y_1^2+y_2^2+\cdots+y_s^2)=z_1^2+\cdots+z_n^2.\]
where each \(z_i\) is a bilinear expression in the \(x\)'s and \(y\)'s. The identity (\ref{eq1}) was a formula of type \([2,2,2]\). Here is a formula of type \([4,4,4]\):
\begin{align*}
    (x_1^2+x_2^2+x_3^2+x_4^2)\cdot (y_1^2+y_2^2+y_3^2+y_4^2)=&\quad (x_1y_1-x_2y_2-x_3y_3-x_4y_4)^2\\ 
                                                            =&+(x_1y_2+x_2y_1-x_3y_4+x_4y_3)^2\\ 
                                                            =&+(x_1y_3-x_2y_4+x_3y_1+x_4y_2)^2\\ 
                                                            =&+(-x_1y_4+x_2y_3+x_3y_2+x_4y_1)^2.
\end{align*}
If you try to generalize these examples you will find a formula of type \([8,8,8]\), but not one of type \([16,16,16]\).

\begin{problem}{5}
If we have a sums-of-squares formula of type \([r,s,n]\) then we get a bilinear map \(\phi:\mathbb{R}^r\times \mathbb{R}^s\rightarrow \mathbb{R}^n\) such that \(\| \phi(x,y)\|^2=\|x\|^2\cdot \|y\|^2\) by defining 
\[\phi(x_1,\ldots,x_r,y_1,\ldots,y_s)=(z_1,\ldots,z_n)\]
using the bilinear expression \(z_i\).
\begin{enumerate}[(a)]
\item Explain why \(\phi\) restricts to a map \(S^{r-1}\times S^{s-1}\rightarrow S^{n-1}\), and then induces a map 
\[F:\mathbb{R}P^{r-1}\times \mathbb{R}P^{s-1}\rightarrow \mathbb{R}P^{n-1}.\]
\item Use singular cohomology to prove that if an \([r,s,n]\) formula exists then \(n \choose i\) must be even for \(n-r<i<s\). 
\item With some trouble one can discover a sums-of-squares formula of type \([10,10,16]\). Does there exist a better formula of type \([10,10,15]\)>
\end{enumerate}
\end{problem}
\begin{solution}
\begin{enumerate}[(a)]
\item Suppose \(x\in S^{r-1}\subseteq \mathbb{R}^r\) and \(y\in S^{s-1}\subseteq \mathbb{R}^s\). This implies that \(\|x\|=\|y\|=1\). Then 
\[\|\phi(x,y)\|=\|x\|\cdot \|y\|=1\cdot 1=1.\]
So \(\phi(x,y)\in S^{n-1}\). This means the map \(\phi\) can be restricted to a map 
\[\phi:S^{r-1}\times S^{s-1}\rightarrow S^{n-1}.\]
Moreover, for any point \((x,y)\in S^{r-1}\times S^{s-1}\), we have 
\[\phi(-x,y)=\phi(x,-y)=-\phi(x,y)\]
because \(\phi\) is bilinear. This means we can identify the antipodal points in each sphere, and obtain a map 
\[F:\mathbb{R}P^{r-1}\times \mathbb{R}P^{s-1}\rightarrow \mathbb{R}P^{n-1}.\]
The map \(F\) is continuous because the bilinear map \(\phi\) is continuous and \(F\) is induced by the quotient map from sphere to the real projective space. 
\item The map \(F\) induces a map between cohomology rings with \(\mathbb{Z}_2\)-coefficients. Using Künneth Theorem, we have a map 
\[F^*:H^*(\mathbb{R}P^{n-1};\mathbb{Z}_2)\rightarrow H^*(\mathbb{R}P^{r-1};\mathbb{Z}_2)\otimes H^*(\mathbb{R}P^{s-1};\mathbb{Z}_2).\]
This is a map between \(\mathbb{Z}/2\)-algebras 
\[F^*:\mathbb{Z}_2[x]/(x^n)\rightarrow \mathbb{Z}_2[y]/(y^r)\otimes \mathbb{Z}_2[z]/(z^s)\]
sending the generator \(x\) to \(k(y\otimes 1)+l(1\otimes z)\) for some \(k,l\in \mathbb{Z}_2\). 
\begin{claim}
\(k\neq 0\) and \(l\neq 0\), namely \(k=l=1\).
\end{claim}
\begin{claimproof}
Choose a point \(a\in S^{s-1}\) and consider the inclusion map \(i:\mathbb{R}^r\hookrightarrow \mathbb{R}^r\times \left\{ a \right\}\subseteq \mathbb{R}^r\times \mathbb{R}^s\). The composition 
\[\mathbb{R}^r\xrightarrow{i}\mathbb{R}^r\times \mathbb{R}^s\xrightarrow{\phi}\mathbb{R}^n\]
is an \(\mathbb{R}\)-linear map and for all \(x\in \mathbb{R}^r\), we have 
\[\|(\phi\circ i)(x)\|=\|\phi(x,a)\|=\|x\|^2\cdot \|a\|^2=\|x\|^2.\]
meaning that it preserves the norm. This implies \(r\leq n\), otherwise the kernel of the map \(\phi\circ i\) must be non-zero, and a non-zero element will be mapped to \(0\), which contradicts the fact that \(\phi\circ i\) preserves the norm. Write \(g:=\phi\circ i\). \(g\) can be restrected to a map 
\(S^{r-1}\rightarrow S^{n-1}\) and since \(g\) is \(\mathbb{R}\)-linear, it induces a map \(G:\mathbb{R}P^{r-1}\rightarrow \mathbb{R}P^{n-1}\). To prove \(k\neq 0\), it is the same as proving the map induced by \(g\) between cohomology rings is not the zero map. 

We know \(g:\mathbb{R}^r\rightarrow \mathbb{R}^n\) with \(r\leq n\) is a injective \(\mathbb{R}\)-linear map since \(\ker g=0\). There exists an invertible matrix \(T\in GL_n(\mathbb{R})\) such that the composition
\[\mathbb{R}^r\xrightarrow{g}\mathbb{R}^n\xrightarrow{T}\mathbb{R}^n\]
is an inclusion map, namely \(\mathbb{R}^r\) is mapped into the first \(r\) coordinates in \(\mathbb{R}^n\). Note that every map here is \(\mathbb{R}\)-linear and injective, this induces a map between real projective spaces 
\[\mathbb{R}P^{r-1}\xrightarrow{G}\mathbb{R}P^{n-1}\xrightarrow{t}\mathbb{R}P^{n-1}\]
where \(t\) is a homeomorphism as it is induced from an invertible matrix. The composition \(t\circ G\) is the inclusion of \((r-1)\)-skeleton inside \(\mathbb{R}P^{n-1}\) because it is induced from the embedding \(T\circ g:\mathbb{R}^r\hookrightarrow \mathbb{R}^n\). Now we have maps between cohomology rings 
\[H^*(\mathbb{R}P^{n-1};\mathbb{Z}_2)\xrightarrow{t^*}H^*(\mathbb{R}P^{n-1};\mathbb{Z}_2)\xrightarrow{G^*}H^*(\mathbb{R}P^{r-1};\mathbb{Z}_2).\]
We know here \(t^*\) is the identity map between cohomology rings as \(t\) is a homeomorphism, and \(G^*\circ t^*\) is surjective because \(t\circ G:\mathbb{R}P^{r-1}\hookrightarrow \mathbb{R}P^{n-1}\) is the inclusion of \((r-1)\)-skeleton. This implies \(k\neq 0\) and \(k=1\) since we are working \(\mathbb{Z}_2\)-coefficients. A similar argument implies \(l=1\).
\end{claimproof}

Going back to the map 
\[F^*:\mathbb{Z}_2[x]/(x^n)\rightarrow \mathbb{Z}_2[y]/(y^r)\otimes \mathbb{Z}_2[z]/(z^s)\]
sending \(x\) to \(y\otimes 1+1\otimes z\). We have a relation 
\[0=F^*(x^n)=F^*(x)^n=(y\otimes 1+1\otimes z)^n=\sum_{i=0}^{n}\binom{n}{i} y^{n-i}\otimes z^i.\]
in the field \(\mathbb{Z}_2\). If \(n-r<i<s\), we have \(1\leq i<s\) and \(1\leq n-i<r\), this means \(y^{n-i}\otimes z^i\neq 0\) and \(\binom{n}{i}=0\) in \(\mathbb{Z}_2\). So if such a formula \([r,s,n]\) exists, then \(n \choose i\) must be even for \(n-r<i<s\). 
\item If a formula of type \([10,10,15]\) exists, then for \(15-10<6<10\), we have that \(15 \choose 6\) equals to \(5005\), which is an odd number, so such a formula does not exist.
\end{enumerate}
\end{solution}

\noindent\rule{7in}{2.8pt}
%%%%%%%%%%%%%%%%%%%%%%%%%%%%%%%%%%%%%%%%%%%%%%%%%%%%%%%%%%%%%%%%%%%%%%%%%%%%%%%%%%%%%%%%%%%%%%%%%%%%%%%%%%%%%%%%%%%%%%%%%
% Problem 6
%%%%%%%%%%%%%%%%%%%%%%%%%%%%%%%%%%%%%%%%%%%%%%%%%%%%%%%%%%%%%%%%%%%%%%%%%%%%%%%%%%%%%%%%%%%%%%%%%%%%%%%%%%%%%%%%%%%%%%%%%%
\begin{problem}{6}
Suppose \(p(x)\) is an irreducible polynomial over \(\mathbb{C}\) of degree \(n\), where \(n>1\). Let \(E=\mathbb{C}[x]/(p(x))\), which is an algebraic field extension of \(\mathbb{C}\) of degree \(n\). Choose a vector space isomorphism \(\mathbb{C}^n\cong E\), so that the multiplication on \(E\) becomes a bilinear map \(\mathbb{C}^n\times \mathbb{C}^n\rightarrow \mathbb{C}^n\).

Using singular cohomology rings of appropriate topological spaces, derive a contradiction. 
\end{problem}
\begin{solution}
The multiplication \(\mu:\mathbb{C}^n\times \mathbb{C}^n\rightarrow \mathbb{C}^n\) is non-degenerate because it is coming from a multiplication in a field \(E\). Since the map \(\mu\) is \(\mathbb{C}\)-bilinear, for any \(\lambda\in \mathbb{C}^*\) and \((x,y)\in \mathbb{C}^n\times \mathbb{C}^n\), we have 
\[\mu(\lambda x,y)=\mu(x,\lambda y)=\lambda \mu(x,y).\]
This means for any two complex lines \(l_1,l_2\subseteq \mathbb{C}^n\) passing through the origin, they are sent to another line passing through the origin under the map \(\mu\). So \(\mu\) induces a map between complex projective spaces 
\[f:\mathbb{C}P^{n-1}\times \mathbb{C}P^{n-1}\rightarrow \mathbb{C}P^{n-1}.\]
By Künneth Theorem, this further induces a map between cohomology rings 
\[f^*:H^*(\mathbb{C}P^{n-1})\rightarrow H^*(\mathbb{C}P^{n-1})\otimes H^*(\mathbb{C}P^{n-1}).\]
We know that \(H^*(\mathbb{C}P^{n-1})\cong \mathbb{Z}[x]/(x^n)\), so \(f^*\) is a map between \(\mathbb{Z}\)-algebras 
\[f^*:\mathbb{Z}[x]/(x^n)\rightarrow \mathbb{Z}[y]/(y^n)\otimes \mathbb{Z}[z]/(z^n)\]
sending \(x\) to \(k(y\otimes 1)+l(1\otimes z)\) for some \(k,l\in \mathbb{Z}\). A similar argument with Problem 5(b) implies that \(k\) and \(l\) are not zero. So we have 
\[0=f^*(x^n)=f^*(x)^n=(k(y\otimes 1)+l(1\otimes z))^n=\sum_{i=0}^{n}\binom{n}{i} k^i l^{n-i} y^i\otimes z^{n-i}.\]
Since \(k,l\) is non-zero, this implies \(n\choose i\) needs to be \(0\) for \(1\leq i\leq n-1\), so \(n=1\). This contradicts the assumption that the degree of \(p(x)\) is larger than 1. 
\end{solution}

\end{document}